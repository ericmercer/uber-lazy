\section{Introduction}

In recent years symbolic execution has provided the basis for various
software testing and analysis techniques. Symbolic execution
systematically explores the program execution space using symbolic
input values, and for each explored path, computes constraints on the
symbolic inputs to create a \emph{ path condition}.  The path
conditions computed by symbolic execution characterize the observed
program execution behaviors and have been used as an enabling
technology for various applications, e.g., regression
analysis~\cite{backes:2012,Godefroid:SAS11,Person:FSE08,person:pldi2011,Ramos:2011,Yang:ISSTA12},
data structure repair~\cite{KhurshidETAL05RepairingStructurally},
dynamic discovery of
invariants~\cite{CsallnerETAL08DySy,Zhang:ISSTA14}, and
debugging~\cite{Ma:2011}.

Initial work on symbolic execution largely focused on checking
properties of programs with primitive
types~\cite{clarke76TSE,King:76}.  With the advent of object-oriented
languages, recent work has generalized the core ideas of symbolic
execution to enable analysis of programs containing complex data
structures with unbounded domains, i.e., data stored on the
heap~\cite{Kiasan06,Kiasan07,GSE03}.  These \emph{Generalized Symbolic
  Execution} (GSE) techniques construct the heap in a lazy manner,
deferring materialization of objects on the concrete heap until they
are needed for the analysis to proceed. The materialization creates
additional non-deterministic choice points in the symbolic execution
tree by representing the feasible heap configurations as (i) null,
(ii) an instance of a new reference of a compatible class, and (iii)
an alias to a previously initialized symbolic reference.  Symbolic
execution then follows concrete program semantics for materialized
heap locations. Although this approach enables the analysis of heap
manipulating programs, a large number of feasible concrete heap
configurations are created leading to a path explosion.


The goal of this work is to mitigate the path explosion problem in GSE
by grouping multiple heaps together and only partitioning the heaps at
points of divergence in the control flow graph. Our inspiration is
found in the domain of static analysis that uses sets of constraints
over heap locations to encode multiple heaps in a single
representation. These sets, sometimes known as \emph{value sets} or
\emph{value summaries}, allow multiple heaps to be represented
simultaneously with a higher degree of precision than afforded by
traditional techniques for shape analysis. Some of these previous
attempts, however, are unable to handle aliasing in heaps due to a
recursive definition of objects~\cite{Xie:2005}, while others require
a set of heaps as input and are unable to initialize heaps
on-the-fly~\cite{Dillig:2011,Tillmann:2008}.  As is typical in static
analysis, over approximation of the heap representations alleviates
some of these limitations but often leads to infeasible heaps. Also,
most static analysis techniques for heap updates require a rewrite of
the constraint system making it prohibitive for use in the context of
symbolic execution.

In this work, to effectively represent multiple heaps simultaneously
in the context of symbolic execution, we define a novel heap
representation and an algorithm to initialize and update the heap
on-the-fly. In essence, our summary heap approach supports aliasing,
does not require constraint rewriting on heap updates, and reduces
non-determinism in the search space during symbolic execution.

We represent the heap as a bipartite graph consisting of references
and locations. Each reference is able to point to multiple locations,
where each edge is predicated on constraints over aliasing between
references. Each location points to a single reference for a given
field. The use of a bipartite graph affords two key advantages: (i) it
allows for a non-recursive definition of objects which enables us to
support aliasing, and (ii) it does not introduce auxiliary variables
or require rewriting of non-local constraints during updates to the
heap.

The summary heap algorithm defines how the bipartite graph is updated
during lazy initialization. Unlike GSE, however, the summary heap
algorithm introduces non-determinism in the search only at points of
divergence in the control flow graph. These points of non-determinism
are at field accesses that lead to null-pointer exceptions and at
comparisons of references. The former represents a divergence due to
exceptional control flow while the latter is due to program structure.
The combination of the heap representation and the summary heap
algorithm enables us to improve the efficiency of symbolic execution
of heap manipulating programs over state-of-the-art techniques.

The summary heap algorithm is sound and complete with respect to
properties that are provable using GSE. This proof is accomplished by
showing the existence of a bisimulation between states in GSE and
states in the summary heap algorithm.  A preliminary implementation in
Java Pathfinder (JPF) shows that in general, the heap summary
algorithm improves over other state-of-the-art techniques, and in some
instances, the improvement is remarkable: two-orders of magnitude
reduction in running time. More importantly though, the heap summary
enables the initialization of larger more complex heaps than
previously possible as shown in our results.
\noindent{This paper makes the following contributions:}

\begin{compactdesc}

\item\textbf{-} A non-approximate, morphologically arbitrary, and efficient 
summary heap representation.

\item\textbf{-} An algorithm to dynamically initialize, summarize, and
  update the summary heap during symbolic execution.

\item\textbf{-} A bisimulation proof establishing the soundness and 
completeness of the heap summary approach with respect to
properties provable by GSE.

\item\textbf{-} A proof-of-concept implementation and empirical study 
demonstrating the scalability of the summary heap approach
compared to other GSE approaches.

\end{compactdesc}
