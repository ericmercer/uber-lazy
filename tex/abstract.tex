One of the fundamental challenges of using symbolic execution to
analyze software has been the treatment of dynamically allocated data.
%, i.e., programs with complex data structures with
%unbounded domains.
State-of-the-art symbolic execution techniques 
have addressed this challenge by constructing the heap \emph{lazily},
materializing objects on the concrete heap ``as needed'' and
using non-deterministic choice points to explore each feasible
concrete heap configuration. Because analysis of the materialized
heap locations relies on concrete program semantics, the lazy
initialization approach exacerbates the state space explosion
problem that limits the scalability of symbolic execution.
In this work we present a novel approach for lazy symbolic execution
of heap manipulating software which utilizes 
a fully symbolic heap constructed on-the-fly during symbolic execution.
Our approach is 1) \emph{scalable} -- it does not
create the additional points of non-determinism
introduced by existing lazy initialization techniques and it explores
each execution path only once for any given set of isomorphic heaps,
2) \emph{precise} -- at any given point during symbolic execution, 
the symbolic heap represents the exact set of feasible
concrete heap structures for the program under analysis, and
3) \emph{expressive} -- the symbolic heap can represent recursive data structures 
and heaps resulting from loops and recursive control structures in the code. 
We report on a case-study of an implementation of our technique in the
Symbolic PathFinder tool to illustrate its scalability, precision
and expressiveness. We also discuss how test case generation -- a common
 use for symbolic execution results -- can benefit from symbolic
 execution which uses a fully symbolic heap.
