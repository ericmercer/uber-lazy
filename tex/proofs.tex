\section{Proofs}
\label{sec:proofs}
\subsection{Lemmas}

The goal of this section is to prove that the representation relation,
$\sqsubset$, is a bisimulation.  A bisimulation is a relation over pairs of states such that whenever two states, $\gsest$ and
$\symst$, are related in the bisimulation, $\gsest \sqsubset
\symst$, every successor, from either state, $\gsest
\rgse \gsest^\prime$ or $\symst \rsym
\symst^\prime$, has a corresponding mutual successor in the other
state such that both of those successors are also related in the
bisimulation: $\gsest^\prime \sqsubset \symst^\prime$.

If $\rsym$ is considered to be a model of the
$\rgse$ (i.e., a representation of that machine), then the
representation relation, as a bisimulation, ensures that the
$\rsym$ is complete in that any property that can be
shown in states related on $\rgse$ can also be shown in the
$\rsym$; and further, the representation relation as a
bisimulation ensures that the $\rsym$ is also sound in
that any property that can be shown to hold in states related by $\rsym$
can also be shown to hold in the $\rgse$.

The proof of the representation relation as a bisimulation reasons
over individual rules in $\rsym$ to show that for
each rule, the representation relation, $\sqsubset$, exists. The heart
of the representation relation is the homomorphism that maps locations
in one heap to locations in the other heap. The proof reasons over of
each rule and as mentioned previously, and derives from the current
homomorphism in the representation relation for the current states, a
new homomorphism that is sufficient to use in the a new representation
relation that includes the successor states. The proof is constructive
in that it shows how given a valid homomorphism for the current
states, it is possible to derive a new homomorphism that includes
successor states. With such a new homomorphism, it is possible to
state that for $\rsym$, when restricted to a specific
rule (i.e., that is the only rule available in the relation), the
representation relation is a bisimulation for the restricted
$\rsym$. If such a bisimulation exists for all the
individual rules, then it exists for the rules collectively.

There are two slight complications in the proof: first, the field
access rule relies on $\rsum^*$ that operates on a different
state than $\rsym$; and second, constructing the new
homomorphism in the equals-reference rule relies on the incoming heap
being deterministic. The relation $\rsum^*$ effectively produces
an intermediary state that is the state where uninitialized references
are initialized before the field access actually takes place. In
essence, a state on the left of $\rsym$ that is a
field access, undergoes a transition in which its heap is changed to
initialize fields. Once the fields are initialized, then the actual
field access takes place. The state with the heap that has the
initialized fields is the intermediary state between the state on the
left of $\rsym$ and the state on the right of $\rsym$. The proof reasons separately about this
intermediary state to prove that it too is exact.

The equals-reference proof must show that any given reference in the
heap is not able to point to two distinct locations at the same
time. If the incoming heap is able to point to two valid locations at
a given reference at the same time, then the help is
non-deterministic, and it is not possible to construct a valid
homomorphism from the existing homomorphism: which location should be
used in the map? As such, the proof first establishes that
$\rsym$ preserves determinism when the incoming heap is
deterministic. Once that is shown, the exactness of the equals
reference rule is given.

The final statement that $\sqsubset$ is a bisimulation is in \thref{th:bisim}.

\begin{theorem}
If $R \subseteq F_\sim(R)$, then $R$ is a bisimulation.
\label{lem:bisim}
\end{theorem}
The proof can be found in any good text on bisimulation.
%For the proof, we refer the reader to \cite{Sangiorgi:2011}

%\begin{theorem}
%If exists state $\symst$ and set of states $\Gsest$ such that \eqnref{eqn:BisimulationForwards} and \eqnref{eqn:BisimulationBackwards} are satisfied, then:
%$$(\Gsest \ \symst) \in F_\sim(\sqsubset)$$
%\label{}
%\end{theorem}


%\begin{lemma}[$\sqsubset$ is a bisimulation for $\rinit$ and $\rsum$]
%\label{lem:init}
%If $\symst \cong \mathbb{FS}(\rightarrow_{\phi},s_0,\pi_n)$ for symbolic state $\symst = \lp \cfgnt{L}_\sym\ \cfgnt{R}_{s}\ \phi_\sym\ \eta\ \cfgnt{r}\ \lp \cfgt{*}\ \cfgt{\$}\ \cfgnt{f} \rightarrow \cfgnt{k}\rp \rp$, initial state $s_0$, and control flow path $\pi_n$, and if there exists some intermediate state state $\symst^\prime$ such that $(\cfgnt{L}_\sym\ \cfgnt{R}_{s}\ \cfgnt{r}\ \cfgnt{f} \ \cfgnt{C}) \rsum^* (\cfgnt{L}_{s^\prime}\ \cfgnt{R}_{s^\prime}\ \phi_{s^\prime}\ \cfgnt{r}\ \cfgnt{f} \ \cfgnt{C})$, then:
% $$\symst^\prime \cong \{\forall \gsest^\prime | \exists \gsest (\gsest \sqsubset \symst \wedge (\gsest \rinit^* \gsest^\prime) ) \}$$
%\end{lemma}

\begin{lemma}[Existence of Equivalent Initialized States]
\label{lem:init}
If we have state $q$, and set of states $P$ such that $q \in 2^{S_\mathit{FA}}$ and $P \sqsubset q$, then there exists state $q^+$ and set of states $P^+$ such that $P \sim P^+$, $q \sim q^+$, $P^+ \sqsubset q^+$, and all the relevant fields are initialized.
\end{lemma}

\begin{proof}
Our proof strategy is to take arbitrary $q$ and $P$, and then construct $q^+$ and $P^+$ such that $P \sim P^+$, $q \sim q^+$, and all the relevant fields are initialized. We then prove that $P^+ \sqsubset q^+$.

Let $q$ be any state such that $q \in 2^{S_\mathit{FA}}$, and let $P$ be a set of states such that $P \sqsubset q$. By Definition~\ref{representation}, $q = \lp \cfgnt{L}_q\ \cfgnt{R}_q\ \phi_q\ \eta\ \cfgnt{r}\ \lp \cfgt{*}\ \cfgt{\$}\ \cfgnt{f} \rightarrow \cfgnt{k}\rp \rp$. Let $q^+$ be the state such that $q^+ = \lp \cfgnt{L}_{q}^\prime\ \cfgnt{R}_{q}^\prime\ \phi_{q}\ \eta\ \cfgnt{r}\ \lp \cfgt{*}\ \cfgt{\$}\ \cfgnt{f} \rightarrow \cfgnt{k}\rp \rp$ where $\cfgnt{L}_{q}^\prime$ and  $\cfgnt{R}_{q}^\prime$ are as defined in the $\rsym$ field access rule. Although the $\rsum$ relation does not relate states, we use the notation ${q} \rsum^* q^+$ to show this type of relationship between pairs of states. Let $p$ be any state such that $p \in P$. Define state $p^+$ in a similar fashion as $q^+$, except using the $\rgse$ field access rule. Define $P^+$ to be the set of all possible states like $p^+$: $P^+ = \{\forall {p}^+ | \exists {p} ({p} \sqsubset {q} \wedge ({p} \rinit^* {p}^+) ) \}$, and define $P^{q}$ to be the set of all states represented by $q^+$: $P^{q} = \{\forall {p}^\prime | {p}^\prime \sqsubset {q}^\prime \}$. Since $P^+$ and $q^+$ are simply pre-initialized versions of $P$ and $q$, it follows immediately that $P \sim P^+$ and $q \sim q^+$, and that all the relevant fields are initialized.

It remains to show that $P^+ \sqsubset q^+$. We will do so by proving that both $P^+ \subset P^q$ and $P^q \subset P^+$. From $p$, take any location, field pair $\lp l_p\  \cfgnt{f}\rp$ such that $\lp l_p\  \cfgnt{f}\rp \in \cfgnt{R}_{p}^\leftarrow$, and let $l_q = h(l_p)$. We may classify $l_p$ into one of three ways, based on the values of the $\cfgnt{R}$ function in each of the states $p$, $p^+$, $q$, and $q^+$, and we may define a function $h^\prime: \mathcal{L} \mapsto \mathcal{L}$ based on that classification.

Class 1: $\cfgnt{R}_{p} ( l_{p},  \cfgnt{f} ) = \cfgnt{R}_{p}^\prime ( l_{p},  \cfgnt{f} )$ and $\cfgnt{R}_{s} ( l_{q},  \cfgnt{f} ) = \cfgnt{R}_{q}^\prime ( l_{q},  \cfgnt{f})$.  Let $l_\alpha$ be the location such that ${(\phi_a\ l_\alpha)} = \cfgnt{L}_{p}(\cfgnt{R}_{p} (\cfgnt{l}_{p},f )) $. In this case, let $h^\prime(l_\alpha) = h(l_\alpha)$. Since $p \rightharpoonup_h {q}$, we may surmise that: 
$$(\phi_a\ \cfgnt{l}_\alpha) \in \cfgnt{L}_{p}^\prime(\cfgnt{R}_{p}^\prime (\cfgnt{l}_{p},f )) \Rightarrow (\phi_b\ h^\prime (\cfgnt{l}_\alpha))\in \cfgnt{L}_{s^\prime}(\cfgnt{R}_{q}^\prime (\cfgnt{l}_{q},f ))$$

Class 2: $\cfgnt{R}_{p} ( l_{p},  \cfgnt{f}) = \cfgnt{R}_{p}^\prime ( l_{p},  \cfgnt{f})$ and $\cfgnt{R}_{s} ( l_{q},  \cfgnt{f}) \neq \cfgnt{R}_{q}^\prime ( l_{q},  \cfgnt{f})$. Since $\cfgnt{R}_{s} ( l_{q},  \cfgnt{f}) \neq \cfgnt{R}_{q}^\prime ( l_{q},  \cfgnt{f})$, the Summarize rule must have altered this reference. A reference created by the Summarize rule has a value set $\theta_{all}$ with four subsets: $\theta_{null}$, $\theta_{new}$, $\theta_{alias}$, and $\theta_\mathit{orig}$. Because $\cfgnt{R}_{p} ( l_{p},  \cfgnt{f}) = \cfgnt{R}_{p}^\prime ( l_{p},  \cfgnt{f})$, we know that the location we want to map to lies in $\theta_\mathit{orig}$. Let $l_\alpha$ be the location such that ${(\phi_a\ l_\alpha)} = \cfgnt{L}_{p}(\cfgnt{R}_{p} (\cfgnt{l}_{p},f )) $, and let $l_{orig} = h(l_\alpha)$. In this case, we let $h^\prime(l_\alpha) = h(l_\alpha)$. Since $(\phi_a\ l_\alpha) \in \cfgnt{L}_{p}^\prime(\cfgnt{R}_{p}^\prime (\cfgnt{l}_{p},f ))$. Let $l_{orig} = h(l_\alpha)$. We can see that by the Summarize rule $ (\phi_b\ l_\mathit{orig})\in \cfgnt{L}_{s^\prime}(\cfgnt{R}_{q}^\prime (\cfgnt{l}_{q},f ))$, so therefore:
$$(\phi_a\ \cfgnt{l}_\alpha) \in \cfgnt{L}_{p}^\prime(\cfgnt{R}_{p}^\prime (\cfgnt{l}_{p},f )) \Rightarrow (\phi_b\ h^\prime (\cfgnt{l}_\alpha))\in \cfgnt{L}_{s^\prime}(\cfgnt{R}_{q}^\prime (\cfgnt{l}_{q},f ))$$

Class 3: $\cfgnt{R}_{p} ( l_{p},  \cfgnt{f}) \neq \cfgnt{R}_{p}^\prime ( l_{p},  \cfgnt{f})$ and $\cfgnt{R}_{s} ( l_{q},  \cfgnt{f}) \neq \cfgnt{R}_{q}^\prime ( l_{q},  \cfgnt{f})$. In this case, there are two possibilities: either the new reference $\cfgnt{R}_{p}^\prime ( l_{p},  \cfgnt{f})$ points to some location we've seen before $l_\alpha$, or it points to a previously unobserved location $l_\beta$. By establishing which of these possibilities has happened, we can build $h^\prime$. To construct $h^\prime$, let $l_\alpha$ be any location such that $(\phi_a\ \cfgnt{l}_\alpha) \in \cfgnt{L}_{p}^\prime(\cfgnt{R}_{p}^\prime (\cfgnt{l}_{p},f )) $. If there exists $\phi_\alpha$ such that $\lp \phi_\alpha\ l_\alpha \rp \in \cfgnt{L}_{p}^\rightarrow $, let $h^\prime(l_\alpha) = h(l_\alpha)$. Otherwise, let $l_\beta$ be the location such that $(\phi_b\ \cfgnt{l}_\beta) \in \cfgnt{L}_{s^\prime}(\cfgnt{R}_{q}^\prime (\cfgnt{l}_{q},f )) $ and $(\phi_b\ \cfgnt{l}_\beta) \notin \cfgnt{L}_{s}(\cfgnt{R}_{s} (\cfgnt{l}_{q},f )) $. Now, let $h^\prime(l_\alpha) = l_\beta$. Observe that either way,
$$(\phi_a\ \cfgnt{l}_\alpha) \in \cfgnt{L}_{p}^\prime(\cfgnt{R}_{p}^\prime (\cfgnt{l}_{p},f )) \Rightarrow (\phi_b\ h^\prime (\cfgnt{l}_\alpha))\in \cfgnt{L}_{s^\prime}(\cfgnt{R}_{q}^\prime (\cfgnt{l}_{q},f ))$$

Furthemore, since $l_\alpha$ and $l_\beta$ are new locations with uninitialized fields, we know that for any field $f^\prime$, $\{(\phi_p\ \bot)\} = \cfgnt{L}_{p}^\prime(\cfgnt{R}_{p}^\prime (\cfgnt{l}_\alpha,f^\prime ))$ and $ \{(\phi_p\ \bot)\} = \cfgnt{L}_{s^\prime}(\cfgnt{R}_{q}^\prime (\cfgnt{l}_\beta,f^\prime ))$ therefore, we know that:
$$(\phi_p\ \cfgnt{l}_x) \in \cfgnt{L}_{p}^\prime(\cfgnt{R}_{p}^\prime (\cfgnt{l}_\alpha,f^\prime )) \Rightarrow (\phi_q\ h^\prime (\cfgnt{l}_x))\in \cfgnt{L}_{s^\prime}(\cfgnt{R}_{q}^\prime ( h^\prime(\cfgnt{l}_\alpha),f ))$$

We have now shown that there exists a mapping $h^\prime: \mathcal{L} \mapsto \mathcal{L}$ for all $\cfgnt{l}_{\gse^\prime} \in \cfgnt{L}_{p}^\prime^\rightarrow$ such that:
$$ (\phi_a\ \cfgnt{l}_\alpha) \in \cfgnt{L}_{p}^\prime(\cfgnt{R}_{p}^\prime (\cfgnt{l}_{\gse^\prime},f )) \Rightarrow (\phi_b\ h^\prime (\cfgnt{l}_\alpha))\in \cfgnt{L}_{s^\prime}(\cfgnt{R}_{s^\prime} (\cfgnt{l}_{\gse^\prime},f )) $$
By Definition~\ref{def:homomorphism} we know that $p^+ \rightharpoonup_{h^\prime} q^+$. 

It remains to show that $\mathbb{S}( \phi_\sym^\prime \wedge \mathbb{HC}(p^+ \rightharpoonup_{h} q^+) )$. For locations in Class 1, no new conjuncts are added to $\mathbb{HC}(p^+ \rightharpoonup_{h} q^+)$, and therefore the satisfiability cannot be changed. For locations in Class 2 or Class 3, the new constraints take either the form $\phi_x \wedge \phi_\mathit{orig}$, or $\phi_x \wedge (r_f\ op\ r_{a}) \wedge (r_f\ op\ r_{b}) \wedge ...$. Constraints of the form  $\phi_x \wedge \phi_\mathit{orig}$ contain terms $\phi_x$ and $\phi_\mathit{orig}$ which were already conjoined to prior heap constraint, so satisfiability is not affected. In constraints of the form $\phi_x \wedge (r_f\ op\ r_{a}) \wedge (r_f\ op\ r_{b}) \wedge ...$, the term $\phi_x$ is conjoined to the prior heap constraint, and all the other terms involve the new variable $r_f$, so satisfiability is not affected. Since the previous heap constraint is satisfiable, and none of the new terms can impact the satisfiability, we know that the new heap constraint must also be satisfiable.

Since the heap constraint is satisfiable, we know that $p^+ \sqsubset q^+ $. We have therefore shown that for some symbolic state ${q}$ and an arbitrary GSE state $p$ such that $p \sqsubset {q}$ :
\begin{equation} 
(p \rinit^* p^+ \wedge {q} \rsum^* q^+) \Rightarrow p^+ \sqsubset q^+ 
\end{equation}

It follows immediately that $P^+ \subset P^q$. We now prove the reverse case, that $P^q \subset P^+$. Suppose that $q^+$ represents some infeasible state. This means that $q$ represents some GSE state that has some reference r which points somewhere that no place in the feasible set points to. Since the path condition is identical between $q$ and $q^+$, all references common to the two states point exactly to the same places under the same initial conditions. 

So, the problem must be with one of the references which is present in $q^+$, but not $q$. All of these references point to either a new location, the null location, the uninitialized location, or some alias. In the Summarize rule, the values and constraints for the new, null, uninitialized, and alias locations are contained in the sets $\theta_{new}$, $\theta_{null}$, $\theta_\mathit{orig}$, and $\theta_{alias}$. Since the null, and uninitialized locations are already accounted for by the homomorphism $p \rightharpoonup_h {q}$, and since a new location was created symmetrically for both $p^+$ and $q^+$, the problem must be with some alias location that is part of ${q}$ but not $p$. This means that there must be a feasible path to a target location that does not exist for any GSE heap. So, pick an arbitrary GSE heap containing the location and field in question. If said target location does not exist, then there is no reference in the GSE heap pointing to that location. In the symbolic heap, the path constraint on the path leading to the undesired target contains an aliasing condition that states that the source reference only points to this target location on condition that the parent reference points there. However, since we already know that no other reference in the GSE heap points there, this condition must be infeasible. Therefore, it is not part of the represented state. We have a contradiction. Therefore, there is no alias that points somewhere it's not supposed to.

We have now proven that 
$$ {p}^* \sqsubset {q}^*  \Rightarrow  {p}^* \in \{\forall {p}^\prime | \exists {p} ({p} \sqsubset {q} \wedge ({p} \rinit^* {p}^\prime) ) \}$$
From this result, it is certain that $P^q \subset P^+$. This fact, combined with our previous result, proves that $P^q \cong P^+$. This, with our previous assumption that $P^{q} = \{\forall {p}^\prime | {p}^\prime \sqsubset {q}^\prime \}$, leads us to our final conclusion:
$$ P^+ \sqsubset q^+$$.

\end{proof}

\begin{comment}
Choose any P and Q such that P is represented by Q on field access. Choose some P' that P transitions too.
Pick some p' in P'. By definition, it is related to a p in P. Use that p. Show how that p is related to p'. Then relate to q'.

Arbitrary P' such that P transitions to P'. Reason over each possible P' (how every many choices might exist--Only one for FA). Pick arbitrary p' in one of the successor. Do the forward proof. Show connection in q'. Connect to P' since it holds for arbitrary. 

Repeat process for each possible successor P'

For backward, for each q', there must exists a p'. All we need is one p and then that implies there exists a P. Must do for all Q'. 
\end{comment}

\begin{lemma}[\textrm{F{\footnotesize IELD}} \textrm{A{\footnotesize CCESS}} preserves $\sqsubset\ \subseteq F_\sim(\sqsubset)$]
If $P \in 2^{S_\mathit{FA}}$ and $q \in S$ are such that $P \sqsubset q$, then $(P\ q)$ is in the functional associated to bisumlation.
\label{lem:access}
$$
\forall P \in 2^{S_\mathit{FA}}\ (P \sqsubset q \Rightarrow (P\ q) \in F_\sim(\sqsubset))
$$
\end{lemma}

\begin{proof}
%Proof by contradiction: assume $P \sqsubset q \wedge (P\ q) \not\in F_\sim(\sqsubset)$.

By \lemref{lem:init}, we may assume without loss of generality that \cfgnt{r} points to no locations where field \cfgnt{f} is uninitialized. Furthermore, \lemref{lem:succ} ensures at least one successor of $P$ and $q$. Let $p^\prime \in P^\prime$ be any
state in any successor set and $p \rgse p^\prime$ that state's
predecessor. Let $q^\prime$ be similarly defined relative to $q$.  The states $p$ and $q$, by virtue of the
representation relation, are only differentiated by their heaps and
global constraints since the environment ($\eta$), expression ($\cfgnt{r}$),
and continuation ($\lp \cfgt{*}\ \cfgt{\$}\ \cfgnt{f} \rightarrow
\cfgnt{k}\rp$) are the same in all states 
(\defref{representation}). 

There are two relations defined for field access in both $\rsgse$ and
$\rsym$ for a null and non-null pointing reference (\figref{fig:lazy}
and \figref{fig:fHeap}). In each relation, the environment is
unchanged. For a null pointing reference, in both relations, the
expression is $\cfgt{error}$ and the continuation is $\cfgt{end}$. For
the non-null pointing reference, some care is needed. First each does
use the same continuation $\cfgnt{k}$ (it is the same for $p$ and $q$
by $\sqsubset$. As the environment and continuations are the same,
$\cfgnt{r}^\prime = \mathrm{stack}_\cfgnt{r}()$, returns the same next largest
number of current stake references from the partition. The first
condition for inclusion in $\sqsubset$, equality between environments,
expressions, and continuations, is met.

Now consider the heaps.  The functional in \defref{bisimulation} requires a forward
\eqref{eqn:BisimulationForwards} and backward
\eqref{eqn:BisimulationBackwards} relationship between successor
heaps. First the forward direction.

\noindent{\textrm{F{\footnotesize IELD}}
\textrm{A{\footnotesize CCESS (NULL)}}}: $p \rightarrow^\ell_\mathit{FA(N)}
p^\prime$ because $l = l_\mathit{null}$ (\figref{fig:lazy}). As $p$
is represented in $q$, $\exists q^\prime\ (q
\rightarrow^\sym_\mathit{FA(N)} q^\prime$ since $l_{\mathit{null}}$
must be feasible on the reference (\figref{fig:fHeap}). 

The successors $p^\prime$ and $q^\prime$ are only differentiated by
their heap and global constraint as mentioned previously. Since each heap in each successor is unchanged
from the predecessor state, the same homomorphism from $p \sqsubset q$
establishes that $p^\prime \sqsubset q^\prime$ holds in the forward direction.

\noindent\textrm{F{\footnotesize IELD}}
\textrm{A{\footnotesize CCESS}}: $p \rightarrow^\ell_\mathit{FA}
p^\prime $ because $\cfgnt{l} \neq \cfgnt{l}_\mathit{null}$. As $p$ is represented in $q$ there must be a feasible non-null location: $\exists q^\prime\ (q
\rightarrow^\sym_\mathit{FA} q^\prime)$. The non-null pointing field access relations initialize
uninitialized locations in the heap on reference $\cfgnt{r}$ for field
$\cfgnt{f}$. The heaps in $p$ and $q$ are still homomorphic after
initialization, and the heap constraint in the homomorphism is still
valid by \lemref{lem:init}. Let $(\cfgnt{L}^\prime_p\ \cfgnt{R}^\prime_p)$ and  
$(\cfgnt{L}^\prime_q\ \cfgnt{R}^\prime_q)$ be those new heaps and $\rightharpoonup_h$
the homomorphism after initialization on the heap in $p$ with
$\rightarrow_I^*$ (\figref{fig:lazyInit}) and the heap in $q$ with
$\rightarrow_S^*$ (\figref{fig:symInit}).

The final heap in the successor state $p^\prime$ is
$(\cfgnt{L}^\prime_p [\cfgnt{r}^\prime \mapsto
  \lp\phi^\prime\ l^\prime\rp]\ \cfgnt{R}^\prime_p)$, and for $q^\prime$
the final heap in the successor state is
$(\cfgnt{L}^\prime_q[\cfgnt{r}^\prime \mapsto \mathbb{VS}\lp
  \cfgnt{L}^\prime_q,\cfgnt{R}^\prime_q,\cfgnt{r},\cfgnt{f},\phi^\prime_g\rp
]\ \cfgnt{R}^\prime_q)$; each version adding in the
constraint-location pair or value set respectively on the same
reference $\cfgnt{r}^\prime$. As $\cfgnt{r}^\prime$ is a stack reference, and
identical in both $p^\prime$ and $q^\prime$, and no new locations are
added to either heap, the existing homomorphism after initialization
still holds: $p^\prime \rightharpoonup_h q^\prime$.


 To relate $p^\prime$ and $q^\prime$ on
$\sqsubset$, it is necessary to create a new homomorphism:
$\gsest^\prime \rightharpoonup_{h^\prime}
\symst^\prime$. To that end, construct the
function $h^\prime$ such that $h^\prime = h$.


Observe that since $p \rightharpoonup_{h} q$, and since $\cfgnt{R}_\gsest$ and $\cfgnt{R}_q$ are unchanged from states $p$ to $p^\prime$ and $q$ to $q^\prime$ respectively, we are guaranteed that $ \cfgnt{r} = \cfgnt{R}_\gsest(l,f) \Rightarrow \cfgnt{r} = \cfgnt{R}_q(h^\prime(l),f)$. Let $\{(\phi_\gsest^\prime\ l^\prime)\} =  \cfgnt{L}_\gsest(\cfgnt{R}_\gsest(l,f))$. Since $\mathbb{S}(\phi_g \wedge \mathbb{HC}(p \rightharpoonup_{h} q))$ is valid, we know that:
 $$(\phi_q \wedge \phi_q^\prime\ h(l^\prime)) \in \mathbb{VS}\lp \cfgnt{L}_{s},\cfgnt{R}_{s},\cfgnt{r},\cfgnt{f},\phi_g\rp$$ 
From this, we may deduce that:
$$ (\phi_\gsest\ l) \in \cfgnt{L}_\gsest^\prime(\cfgnt{r}^\prime) \Rightarrow (\phi_q \wedge \phi_q^\prime\ h^\prime(l))\in \cfgnt{L}_q^\prime(\cfgnt{r}^\prime)$$
Since $\cfgnt{r}^\prime$ is the only new addition to $L_\gsest^\prime$ and $L_q^\prime$, we now know that the assertion above holds for all $l \in \mathcal{L}$. Thus, we have shown that $p^\prime \rightharpoonup_{h} q^\prime$. Furthermore, since the constraints in $\mathbb{HC}(p^\prime \rightharpoonup_{h^\prime} q^\prime)$ are constructed using conjuncts already present in $ \mathbb{HC}(p \rightharpoonup_{h} q)$, we are guaranteed that $\mathbb{HC}(p^\prime \rightharpoonup_{h^\prime} q^\prime) \Leftrightarrow \mathbb{HC}(p \rightharpoonup_{h} q)$, and therefore $\mathbb{S}(\phi_g \wedge \mathbb{HC}(p^\prime \rightharpoonup_{h^\prime} q^\prime))$. This fact, and the fact that $\eta_\gsest = \eta_{s} ,\ \cfgnt{e}_\gsest = \cfgnt{e}_{s} ,\ \cfgnt{k}_\gsest = \cfgnt{k}_{s}$, means that by Definition~\ref{representation} we know $p^\prime \sqsubset q^\prime$. We have now shown that:
\begin{equation}
\forall \Gsest^\prime ( \Gsest \rsgse \Gsest^\prime \Rightarrow \exists q^\prime( (q \rsym q^\prime )\wedge (\Gsest^\prime\ \sqsubset\ q^\prime ))  )
\end{equation}

Now, to prove the reverse direction, choose any state $\gsest^\prime$ such that $\gsest^\prime \sqsubset q^\prime$. Since $\gsest^\prime \sqsubset q^\prime$, then by Definition~\ref{representation}, we know there exists a homomorphism $\gsest^\prime \rightharpoonup_{h^\prime} q^\prime$, and that $\mathbb{S}( \phi_\gsest^\prime \wedge \mathbb{HC}(\gsest^\prime \rightharpoonup_{h^\prime} q^\prime) )$. From state $\gsest^\prime$, construct state $\gsest$ such that 
\begin{align*}
\gsest &= \lp \cfgnt{L}_\gsest\ \cfgnt{R}_\gsest\ \phi_\gsest\ \eta\ \cfgnt{r}\ \lp \cfgt{*}\ \cfgt{\$}\ \cfgnt{f} \rightarrow \cfgnt{k}\rp \rp\\
L_\gsest &= L_{\gsest^\prime} \setminus \{\cfgnt{r}^\prime \}\\
R_\gsest &= R_{\gsest^\prime}\\
\phi_\gsest &= \phi_\gsest^\prime
\end{align*}
Observe that by virtue of the GSE Field Access rule, $\gsest \rgse \gsest^\prime$. Now, construct function $h_\gsest$ so that $h_\gsest = h^\prime$. Observe that by Definition~\ref{def:homomorphism} $\gsest \rightharpoonup_{h_\gsest} q$,  and that $\mathbb{S}( \phi_\gsest \wedge \mathbb{HC}(\gsest \rightharpoonup_{h_\gsest} q) )$, so $\gsest \sqsubset q$. Therefore:
\begin{equation}
\forall q^\prime ( q \rsym q^\prime\Rightarrow \exists \Gsest^\prime( (\Gsest \rsgse \Gsest^\prime )\wedge (\Gsest^\prime\ \sqsubset\ q^\prime ))  )
\end{equation}
By \defref{bisimulation}, we conclude:
$$(\Gsest \ \symst) \in F_\sim(\sqsubset)$$
\end{proof}

%proof for field write
%\begin{lemma}[Exactness of Field Write Rule]
%\label{lem:write}
%If there exists states $\gsest$ and $\symst$ such that $\symst \in \mathcal{FW}$ and $\gsest \sqsubset \symst$, then:
%\begin{equation}
%\forall \gsest^\prime ( \gsest \rgse \gsest^\prime \Rightarrow \exists \symst^\prime( (\symst \rsym \symst^\prime )\wedge (\gsest^\prime\ \sqsubset\ \symst^\prime ))  )
%\end{equation}
%and
%\begin{equation}
%\forall \symst^\prime ( \symst \rsym \symst^\prime\Rightarrow \exists \gsest^\prime( (\gsest \rgse \gsest^\prime )\wedge (\gsest^\prime\ \sqsubset\ \symst^\prime ))  )
%\end{equation}
%\end{lemma}

\begin{lemma}[\textrm{F{\footnotesize IELD}} \textrm{W{\footnotesize RITE}} preserves $\sqsubset\ \subseteq F_\sim(\sqsubset)$]
If $P \in 2^{S_\mathit{FW}}$ and $q \in S$ are such that $P \sqsubset q$, then $(P\ q)$ is in the functional associated to bisumlation.
\label{lem:write}
$$
\forall P \in 2^{S_\mathit{FW}}\ (P \sqsubset q \Rightarrow (P\ q) \in F_\sim(\sqsubset))
$$
\end{lemma}



\begin{proof}
Begin by assuming the conditions from Lemma~\ref{lem:write}: Take arbitrary state $q \in S$ and set of states $P \in 2^{S_\mathit{FW}}$ such that $P \sqsubset q$.

The first step is to prove that the conditions in \eqnref{eqn:BisimulationForwards} hold. Take state $\symst$ and compute state $\symst^\prime$ such that $\symst \rsym \symst^\prime$. Take any GSE state $\gsest$ such that $\gsest \sqsubset \symst$, and find state $\gsest^\prime$ such that $\gsest \rgse \gsest^\prime$. Let $l_\gsest$ be the location such that $\{(\phi_a\ l_\gsest )\} = \cfgnt{L}_\gsest(\cfgnt{r}_x) $ for some $\phi_a$. To show that $\gsest^\prime \sqsubset \symst^\prime$, we need to demonstrate that there exists a function $h^\prime$ such that $\gsest^\prime \rightharpoonup_{h^\prime} \symst^\prime$, and that $\mathbb{S}(\phi_{s^\prime} \wedge \mathbb{HC}(\gsest^\prime \rightharpoonup_{h^\prime} \symst^\prime) )$. Since $s_h \sqsubset \symst$, we know that there exists a function $h$ such that $\gsest \rightharpoonup_{h} \symst$. Let $h^\prime = h$. 

First, we consider how $\gsest^\prime \rightharpoonup_{h^\prime} \symst^\prime$. Let $l_\alpha$ and $l_\beta$ be arbitrary locations in $\cfgnt{L}_{\gsest^\prime}^\rightarrow$ such that $\{(\phi_a\ \cfgnt{l}_\alpha)\} = \cfgnt{L}_{\gsest^\prime}(\cfgnt{R}_{\gsest^\prime}(\cfgnt{l}_\beta,f ))$, let $\theta = \cfgnt{L}_\symst(\cfgnt{R}_\symst(h(l_\gsest),f))$, and let $\theta^\prime =   \cfgnt{L}_\symst^\prime(\cfgnt{R}_\symst^\prime(h(l_\gsest),f))$. 

Suppose $l_\beta \neq l_\gsest$. In this case either $\theta = \theta^\prime$ or $\theta \neq \theta^\prime$ . In the first case, we are guaranteed that the homomorphism works by default. Otherwise, if $\theta \neq \theta^\prime$.  We can see from the construction of the set $X$ in the symbolic Field Write rule that any feasible location in the set $\theta$ must also be in the set $\theta^\prime$. Since $\gsest \sqsubset \symst$, we know that $h(l_\alpha)$ is in $\theta$, and is likewise in $\theta^\prime$. We have now established that in either case where $l_\beta \neq l_\gsest$, $(\phi_b\ h(\cfgnt{l}_\alpha))\in \cfgnt{L}_{s^\prime}(\cfgnt{R}_{s^\prime} (h(\cfgnt{l}_\beta),f ))$.

On the other hand, suppose $l_\beta = l_\gsest$. In this case we know that $\{(\phi_a\ \cfgnt{l}_\alpha)\} = \cfgnt{L}_{\gsest^\prime}(\cfgnt{R}_{\gsest^\prime}(\cfgnt{l}_\gsest,f ))$. From the GSE field rule, we can surmise that $(\phi_a\ \cfgnt{l}_\alpha) \in \cfgnt{L}_{\gsest}(\cfgnt{r} )$, and since $\gsest \sqsubset \symst$, we know that $(\phi_b\ h(\cfgnt{l}_\alpha)) \in \cfgnt{L}_{s}(\cfgnt{r})$ for some constraint $\phi_b$. Using this fact, we can apply the symbolic Field Write rule to infer that $l_\alpha$ must be one of the locations in $\theta^\prime$, and therefore $(\phi_c\ h(\cfgnt{l}_\alpha)) \in \cfgnt{L}_{s^\prime}(\cfgnt{R}^\prime(l_\gsest,f))$

Thus, for arbitrary $l_\alpha$ and $l_\beta$:
$$(\phi_a\ \cfgnt{l}_\alpha) \in \cfgnt{L}_{\gsest^\prime}(\cfgnt{R}_{\gsest^\prime}(\cfgnt{l}_\beta,f )) \Rightarrow (\phi_b\ h(\cfgnt{l}_\alpha))\in \cfgnt{L}_{s^\prime}(\cfgnt{R}_{s^\prime} (h(\cfgnt{l}_\beta),f ))$$
Therefore, we have shown that $\gsest^\prime \rightharpoonup_{h^\prime} \symst^\prime$.

Since the path condition is unchanged, and since no new constraints are added to the represented heaps, we are guaranteed that $\mathbb{S}(\phi_{s^\prime} \wedge \mathbb{HC}(\gsest^\prime \rightharpoonup_{h^\prime} \symst^\prime) )$.

By proving the existence of a valid homomorphism, we have shown that for any state $\gsest^\prime$ such that $\gsest \rgse \gsest^\prime$, then the state $\symst^\prime$ such that $\symst \rsym \gsest^\prime$ represents $\gsest^\prime$. Therefore, $\forall \Gsest^\prime ( \Gsest \rsgse \Gsest^\prime \Rightarrow \exists \symst^\prime( (\symst \rsym \symst^\prime )\wedge (\Gsest^\prime\ \sqsubset\ \symst^\prime ))  )$.

To prove the reverse direction, we use the same argument as in the field read proof: that any state $\gsest^\prime$ represented by $\symst^\prime$ must have a counterpart state $\gsest$ such that $\gsest \rgse \gsest^\prime$ and $\gsest \sqsubset \symst$. Because $\gsest \sqsubset \symst$ , $\gsest$ must be a member of the set $\Gsest$. Therefore, $\forall \symst^\prime ( \symst \rsym \symst^\prime\Rightarrow \exists \Gsest^\prime( (\Gsest \rsgse \Gsest^\prime )\wedge (\Gsest^\prime\ \sqsubset\ \symst^\prime ))  )$.

By \defref{bisimulation}, we conclude:
$$(\Gsest \ \symst) \in F_\sim(\sqsubset)$$

\end{proof}


\begin{lemma}[$\rsum^*$ preserves heap determinism]
\label{lem:S-determ}
Given a deterministic heap, $(\cfgnt{L}_0\ \cfgnt{R}_0)$, from a
state with a reference $\cfgnt{r}$ and field $\cfgnt{f}$, the
new heap, $(\cfgnt{L}^\prime\ \cfgnt{R}^\prime)$, from the 
summary machine, $(\cfgnt{L}_0\ \cfgnt{R}_0\ \cfgnt{r}\ \cfgnt{f})
\rsum^*
(\cfgnt{L}^\prime\ \cfgnt{R}^\prime\ \cfgnt{r}\ \cfgnt{f})$, is also deterministic.
\end{lemma}
\begin{proof}
Induction over the number of steps in $\rsum^*$ in \figref{fig:symInit}.

\noindent\textbf{Base Case}. The relation makes one step: $(\cfgnt{L}_0\ \cfgnt{R}_0\ \cfgnt{r}\ \cfgnt{f})
\rsum
(\cfgnt{L}_1\ \cfgnt{R}_1\ \cfgnt{r}\ \cfgnt{f})$. Let $\Lambda = \mathbb{UN}(\cfgnt{L}_0, \cfgnt{R}_0, \cfgnt{r},
\cfgnt{f})$ be the set of uninitialized locations. If $\Lambda = \emptyset$, then the \textrm{S{\footnotesize UMMARIZE-END}}
rule is active and $(\cfgnt{L}_1\ \cfgnt{R}_1) = (\cfgnt{L}_0\ \cfgnt{R}_0)$, which is deterministic by the initial conditions in the lemma.

If $\Lambda \neq \emptyset$, then the \textrm{S{\footnotesize UMMARIZE}} rule is
active, and each new constraint location pair must be considered
individually. These pairs are partitioned into the sets
$\theta_\mathit{null}$, $\theta_\mathit{new}$,
$\theta_\mathit{alias}$, and $\theta_\mathit{orig}$ by the rule. 
\begin{itemize}
\item The original heap is
  deterministic by definition, so any constraint in any member of the
  set must have some term such that
  \[\begin{array}{l}
     \forall (\phi\ \cfgnt{l}),(\phi^\prime\ \cfgnt{l}^\prime) \in \theta_\mathit{orig}\ \\
     \ \ \ \ ((\cfgnt{l} \neq \cfgnt{l}^\prime \vee \phi \neq \phi^\prime) \Rightarrow (\phi \wedge \phi^\prime = \cfgt{false}))
     \end{array}
  \]
  Further, any member of $\theta_\mathrm{orig}$ has a constraint
  of the form $\phi = \neg \phi_x \wedge \ldots$ while any member of $\theta_\mathit{null}$, $\theta_\mathit{new}$, and
  $\theta_\mathit{alias}$ has a constraint of the form $\phi^\prime = \phi_x \wedge
  \ldots$; thus
  \[\begin{array}{l}
     \forall (\phi\ \cfgnt{l}) \in \theta_\mathit{orig}\ (\forall (\phi^\prime\ \cfgnt{l}^\prime) \in \theta_\mathit{null} \cup \theta_\mathit{new} \cup \theta_\mathit{alias}\ (\\
     \ \ \ \ \phi \wedge \phi^\prime = \cfgt{false}))
    \end{array}
  \]
\item The only member of $\theta_\mathit{null}= \{(\phi\ \cfgnt{l}_\mathit{null})\}$ has the form $\phi = \ldots \wedge
  \cfgnt{r}_f = \cfgnt{r}_\mathit{null}$ while any member of
  $\theta_\mathit{new}$ and $\theta_\mathit{alias}$ has the form
  $\phi^\prime = \ldots \wedge \cfgnt{r}_f \neq \cfgnt{r}_\mathit{null} \wedge
  \ldots$; thus
  \[\begin{array}{l}
     \forall (\phi^\prime\ \cfgnt{l}^\prime) \in \theta_\mathit{new} \cup \theta_\mathit{alias}\ (
     \phi \wedge \phi^\prime = \cfgt{false})
    \end{array}
  \]
\item The only member of $\theta_\mathit{new} =  \{(\phi\ \cfgnt{l}_f)\}$ has a constraint of the form
  $\phi = \ldots \wedge ( \wedge_{( \cfgnt{r}_a,\ \phi_a,\ l_a) \in \rho}
  \cfgnt{r}_f \ne \cfgnt{r}_a) )$ to assert it does not alias
  anything, while any member of $\theta_\mathit{alias}$ has the form
  $\phi^\prime = \ldots \wedge \cfgnt{r}_f = \cfgnt{r}_a \wedge \ldots$ to assert it aliases some
  $\cfgnt{r}_a$ with both partitions reasoning over the same set
  of aliases $\rho$; thus
  \[\begin{array}{l}
     \forall (\phi^\prime\ \cfgnt{l}^\prime) \in \theta_\mathit{alias}\ (
     \phi \wedge \phi^\prime = \cfgt{false})
    \end{array}
  \]
\item Any member of $\theta_\mathit{alias}$ has the form
\[\ldots \wedge \cfgnt{r}_f = \cfgnt{r}_a \wedge ( \wedge_{( \cfgnt{r}^{\prime}_a\ \phi^{\prime}_a\ l^{\prime}_a)  \in \rho\ ( \cfgnt{r}^\prime_a \neq \cfgnt{r}_a) } \cfgnt{r}_f \neq \cfgnt{r}^{\prime}_a)\]
And thus, 
  \[\begin{array}{l}
     \forall (\phi\ \cfgnt{l}),(\phi^\prime\ \cfgnt{l}^\prime) \in \theta_\mathit{alias}\\
     \ \ \ \ ((\cfgnt{l} \neq \cfgnt{l}^\prime \vee \phi \neq \phi^\prime) \Rightarrow (\phi \wedge \phi^\prime = \cfgt{false}))
     \end{array}
  \]
\end{itemize}
As $\theta$ is mapped to a single reference $r_f = \mathit{init}_r()$ in an already deterministic heap, the
resulting heap $(\cfgnt{L}_1\ \cfgnt{R}_1)$ is likewise deterministic.

\noindent\textbf{Inductive Step}. The machine takes $n$-steps:
\[
(\cfgnt{L}_0\ \cfgnt{R}_0\ \cfgnt{r}\ \cfgnt{f}) \rsum (\cfgnt{L}_1\ \cfgnt{R}_1\ \cfgnt{r}\ \cfgnt{f}) \rsum \ldots \rsum
(\cfgnt{L}_n\ \cfgnt{R}_n\ \cfgnt{r}\ \cfgnt{f})
\]
By the induction hypothesis, $(\cfgnt{L}_n\ \cfgnt{R}_n)$ is deterministic. This matches the base case, in that the heap on the
left side of $\rsum$ is deterministic, and by the
same argument as in the base case,
$(\cfgnt{L}_{n+1}\ \cfgnt{R}_{n+1})$ is thus deterministic.
\end{proof}

\begin{lemma}[$\rightarrow_\mathit{FA}$ preserves heap determinism]
\label{lem:FA-determ}
Given a state, $s$, with a deterministic heap,
$(\cfgnt{L}\ \cfgnt{R}) =
\mathrm{heap}(s)$, the new heap,
$(\cfgnt{L}^\prime\ \cfgnt{R}^\prime) =
\mathrm{heap}(s^\prime)$, in any state related by the field
access rule, $s \rightarrow_\mathit{FA} s^\prime$,
is also deterministic.
\end{lemma}
\begin{proof}
Proof by definition of $\rightarrow_\mathit{FA}$ in \figref{fig:fHeap}.

\lemref{lem:S-determ} establishes that the heap in the state on the right side of
$\rsum^*$ is deterministic if the heap in the state on the left side
is deterministic, so it is only needed to show that determinism is
preserved by the call to the value function, $\mathbb{VS}$, in the
rule. Let $(\cfgnt{L}\ \cfgnt{R})$ be the new deterministic heap related by $\rsum^*$.

Recall from \defref{def:VS} that each constraint in each member of the value set has the form
$(\phi\wedge \phi^\prime\ \cfgnt{l})$. Choose any two distinct members of the value set, $(\phi_\alpha \wedge
\phi_\alpha^\prime\ \cfgnt{l}_\alpha^\prime)$ and $(\phi_\beta \wedge
\phi_\beta^\prime\ \cfgnt{l}_\beta^\prime)$.
\begin{itemize}
\item If $\phi_\alpha = \phi_\beta$, then by \defref{def:VS} 
\[\begin{array}{l}
\exists (\phi_\alpha\ \cfgnt{l}) \in \cfgnt{L}(\cfgnt{r})\ (\exists \cfgnt{r}^\prime \in \cfgnt{R}(\cfgnt{l},\cfgnt{f})\ (\\
\ \ \ \ (\phi_\alpha^\prime\ l_\alpha^\prime) \in \cfgnt{L}(\cfgnt{r}^\prime) \wedge (\phi_\beta^\prime\ l_\beta^\prime) \in \cfgnt{L}(\cfgnt{r}^\prime)))
\end{array}
\]
As $(\phi_\alpha^\prime\ \cfgnt{l}_\alpha^\prime)$ and $(\phi_\beta^\prime\ \cfgnt{l}_\beta^\prime)$ are distinct and connected to the same reference $\cfgnt{r}^\prime$ in a deterministic heap, $\phi_\alpha^\prime \wedge  \phi_\beta^\prime = \cfgt{false}$ by definition.
\item If $\phi_\alpha \ne \phi_\beta$, then by \defref{def:VS} 
\[\begin{array}{l}
\exists \cfgnt{l}\ ((\phi_\alpha\ \cfgnt{l}) \in \cfgnt{L}(\cfgnt{r}) \wedge \exists \cfgnt{l}^\prime \neq \cfgnt{l} \ ((\phi_\beta\ \cfgnt{l}^\prime) \in \cfgnt{L}(\cfgnt{r})))
\end{array}
\]
As $(\phi_\alpha\ \cfgnt{l})$ and $(\phi_\beta\ \cfgnt{l}^\prime)$ are distinct and connected to the same reference $\cfgnt{r}$ in a deterministic heap, $\phi_\alpha \wedge  \phi_\beta = \cfgt{false}$ by definition.
\end{itemize}
The only change to the heap after the $S$-relation is the addition of
the new reference $\cfgnt{r}^\prime = \mathrm{stack}_\cfgnt{r}()$ to
point to the value set. As the value set meets the conditions for
determinism, the new heap with $r^\prime$ and the value set,
$(\cfgnt{L}^\prime\ \cfgnt{R}^\prime) =
\mathrm{heap}(s^\prime)$, is also deterministic.
\end{proof}

\begin{lemma}[$\rightarrow_\mathit{FW}$ preserves heap determinism]
\label{lem:FW-determ}
Given a state, $\symst$, with a deterministic heap,
$(\cfgnt{L}_\sym\ \cfgnt{R}_\sym) =
\mathrm{heap}(\symst)$, the new heap,
$(\cfgnt{L}_\sym^\prime\ \cfgnt{R}_\sym^\prime) =
\mathrm{heap}(\symst^\prime)$, in any state related by the field
write rule, $\symst \rightarrow_\mathit{FW} \symst^\prime$,
is also deterministic.
\end{lemma}
\begin{proof}
Proof by definition of $\rightarrow_\mathit{FW}$ in \figref{fig:fHeap}.

The $\rightarrow_\mathit{FW}$ rule relies on the $\mathbb{ST}$
function. Recall from \defref{def:ST} that each constraint in each
member of the strengthened set has the form $(\phi\wedge
\phi^\prime\ \cfgnt{l}^\prime)$ where every member,
$(\phi^\prime\ \cfgnt{l}^\prime)$, comes from
$\cfgnt{L}_\sym(\cfgnt{r})$ on the same reference in a
deterministic heap $(\cfgnt{L}_\sym\ \cfgnt{R}_\sym)$; thus,
that set of meets the criteria for determinism by definition. So any
application of $\mathbb{ST}$ preserves that criteria for determinism.

The $\rightarrow_\mathit{FW}$ makes two uses of the $\mathbb{ST}$
function, $\theta = \mathbb{ST}(\cfgnt{L},\cfgnt{r},\phi,\phi_g) \cup
\mathbb{ST}(\cfgnt{L},\cfgnt{r}_\mathit{cur},\neg\phi,\phi_g)$, to
build individual $\theta$ sets. Choose any two distinct members of $\theta$,
$(\phi_\alpha \wedge \phi_\alpha^\prime\ \cfgnt{l}_\alpha^\prime)$ and
$(\phi_\beta \wedge \phi_\beta^\prime\ \cfgnt{l}_\beta^\prime)$.
\begin{itemize}
\item If $\phi_\alpha = \phi_\beta$, then the constraints came from the
  same call to $\mathbb{ST}$ so $\phi_\alpha^\prime \wedge
  \phi_\beta^\prime = \cfgt{false}$ by definition.
\item If $\phi_\alpha \ne \phi_\beta$, then the constraints came from
  the different calls and are distinguished by the phase of $\phi$ in
  the call so $\phi_\alpha \wedge \phi_\beta = \cfgt{false}$.
\end{itemize}
Each $\theta$ is mapped to a new reference from
$\mathrm{fresh}_r()$. These are added to an already deterministic heap
$(\cfgnt{L}_\sym\ \cfgnt{R}_\sym)$ and meet the criteria so
that $(\cfgnt{L}_\sym^\prime\ \cfgnt{R}_\sym^\prime)$ is also
deterministic.
\end{proof}

\begin{lemma}[$\rcom$ preserves heap determinism]
\label{lem:J-determ}
Given a state, $\symst$, with a deterministic heap,
$(\cfgnt{L}_\sym\ \cfgnt{R}_\sym) =
\mathrm{heap}(\symst)$, the new heap,
$(\cfgnt{L}_\sym^\prime\ \cfgnt{R}_\sym^\prime) =
\mathrm{heap}(\symst^\prime)$, in any state related by the
Javalite relation, $\symst \rcom \symst^\prime$, is
also deterministic.
\end{lemma}
\begin{proof}
Proof by definition of $\rcom$ in \figref{fig:javalite-common}.

Every rule in $\rcom$ except \texttt{New} leaves the heap
unmodified. The rule for \texttt{New} adds a single new location
($\mathrm{fresh}_\cfgnt{l}(\cfgnt{C})$) to the heap on a single new
reference ($\mathrm{stack}_r()$). The rule also points every field in
the new location to $\cfgnt{r}_\mathit{null}$. As none of these
mutations alter the determinism of the heap, the new heap
$(\cfgnt{L}_\sym^\prime\ \cfgnt{R}_\sym^\prime)$ is also
deterministic.
\end{proof}

\begin{theorem}[$\rsym$ preserves heap well-formedness]
\label{thm:determ}
Given a state, $\symst$, with a deterministic heap,
$(\cfgnt{L}_\sym\ \cfgnt{R}_\sym) =
\mathrm{heap}(\symst)$, the new heap,
$(\cfgnt{L}_\sym^\prime\ \cfgnt{R}_\sym^\prime) =
\mathrm{heap}(\symst^\prime)$, in any state related by the
symbolic relation, $\symst \rsym \symst^\prime$, is
also deterministic.
\end{theorem}
\begin{proof}
Proof by \lemref{lem:FA-determ}, \lemref{lem:FW-determ}, and \lemref{lem:J-determ}
which represent all the rules that relate states in
$\rsym$.
\end{proof}

%\begin{lemma}[Exactness of Reference Compare Rule]
%\label{lem:compare}
%If there exists states $\gsest$ and $\symst$ such that $\symst \in \mathcal{RC}$ and $\gsest \sqsubset \symst$, then:
%\begin{equation}
%\forall \gsest^\prime ( \gsest \rgse \gsest^\prime \Rightarrow \exists \symst^\prime( (\symst \rsym \symst^\prime )\wedge (\gsest^\prime\ \sqsubset\ \symst^\prime ))  )
%\end{equation}
%and
%\begin{equation}
%\forall \symst^\prime ( \symst \rsym \symst^\prime\Rightarrow \exists \gsest^\prime( (\gsest \rgse \gsest^\prime )\wedge (\gsest^\prime\ \sqsubset\ \symst^\prime ))  )
%\end{equation}
%\end{lemma}


\begin{lemma}[\textrm{E{\footnotesize QUALS}} ] preserves $\sqsubset\ \subseteq F_\sim(\sqsubset)$]
If $P \in 2^{S_\mathit{E}}$ and $q \in S$ are such that $P \sqsubset q$, then $(P\ q)$ is in the functional associated to bisumlation.
\label{lem:compare}
$$
\forall P \in 2^{S_\mathit{E}}\ (P \sqsubset q \Rightarrow (P\ q) \in F_\sim(\sqsubset))
$$
\end{lemma}

There are two rules that apply to states in $2^{E}$, one for the $\cfgt{true}$ branch and one for the $\cfgt{false}$ branch. Since the proofs for both rules are nearly identical, for brevity we will only show the proofs for the case for the $\cfgt{true}$ branch. 
\begin{proof}
Assume there exists states $\gsest$ and $\symst$ such that $\symst \in \mathcal{RC}$ and $\gsest \sqsubset \symst$. Let $\symst^\prime$ be any state such that $\symst \rsym \symst$ and let $\zeta_T = \forall \gsest^\prime ( \gsest \rgse \gsest^\prime )$. Since $\gsest \sqsubset \symst$, we know that $\gsest \in \mathcal{RC}$, and that there exists a homomorphism $\gsest \rightharpoonup_{h} \symst$ such that $\mathbb{S}( \phi_\sym \wedge \mathbb{HC}(\gsest \rightharpoonup_{h} \symst) ) $. We partition $\zeta_T$ based on the values of $\cfgnt{L}_\gsest \lp \cfgnt{r}_0\rp$ and $\cfgnt{L}_\gsest \lp \cfgnt{r}_1 \rp$ as follows: Let
$$\zeta_t = \zeta_T \setminus \{ s_f | (s_f= \lp \cfgnt{L}_f\ \cfgnt{R}_f\ \phi_\gsest\ \eta\ \cfgnt{e}\ \cfgnt{k}\rp) \wedge (\cfgnt{L}_f \lp \cfgnt{r}_0\rp \neq \cfgnt{L}_f \lp \cfgnt{r}_1 \rp ) \}$$
and let
$$\zeta_f = \zeta_T \setminus \zeta_t$$ 

Furthermore, there are two possible configurations for $\symst^\prime$: $\lp \cfgnt{L}\ \cfgnt{R}\ \phi_g^\prime\ \eta\ \cfgt{true}\ \cfgnt{k}\rp $ and $\lp \cfgnt{L}\ \cfgnt{R}\ \phi_g^\prime\ \eta\ \cfgt{false}\ \cfgnt{k}\rp $. We now consider the partitions of $\zeta_T$ and configurations of $\symst^\prime$ in separate cases.

Case 1: Assume that $\cfgnt{L}_\gsest \lp \cfgnt{r}_0\rp = \cfgnt{L}_\gsest \lp \cfgnt{r}_1 \rp$. 
Compute state $\gsest^\prime$ such that $\gsest \rgse \gsest^\prime$. In this case, the GSE ``equals - references true" rule applied, therefore $\gsest^\prime$ is in $\zeta_t$. Observe that by applying Theorem~\ref{thm:determ}, $\phi_\sym^\prime \wedge \phi_0 \wedge \phi_1$ reduces to $\phi_\sym$. Therefore, $\mathbb{S}( \phi_\sym^\prime \wedge \mathbb{HC}(\gsest^\prime \rightharpoonup_{h} \symst^\prime) ) $ is true, and by extension, $\gsest^\prime \sqsubset \symst^\prime$. Since this relation holds for arbitrary $\gsest^\prime \in \zeta_t$, we now know that 
\begin{equation}
\label{eqn:RCForwardsTrue}
((\cfgnt{L}_\gsest \lp \cfgnt{r}_0\rp = \cfgnt{L}_\gsest \lp \cfgnt{r}_1 \rp) \wedge (\gsest^\prime \in \zeta_t)) \Rightarrow \gsest^\prime \sqsubset \symst^\prime
\end{equation}

Case 2:  Assume that $\symst^\prime$ has the form $\lp \cfgnt{L}\ \cfgnt{R}\ \phi_g^\prime\ \eta\ \cfgt{true}\ \cfgnt{k}\rp $, and define $\theta_\alpha$, $\theta_0$ and $\theta_1$ as in the ``equals (references-true) rule''. Since $\cfgnt{L}_\sym$ and $\cfgnt{R}_\sym$ are unchanged from $\symst$, and $\phi_\sym^\prime$ is only a strengthened version of $\phi_\sym$,  we know that
\begin{equation}
\label{eqn:RCSubset}
\{\gsest^\prime | \gsest^\prime \sqsubset \symst^\prime \} \subseteq \{\gsest^\prime | \exists \gsest \lp \gsest \sqsubset \symst \rp \wedge \gsest \rsym \gsest^\prime\}
\end{equation}
Suppose that there exists state $s_i^\prime$ such that $s_i^\prime \sqsubset \symst^\prime$ and $s_i^\prime \notin \zeta_t$. Because of Equation \ref{eqn:RCSubset}, we know that 
$$s_i^\prime \in \{\gsest^\prime | \exists \gsest \lp \gsest \sqsubset \symst\rp \wedge \gsest \rsym \gsest^\prime\}$$ 
Combining this with the assumption that $s_i^\prime \notin \zeta_t$, we must conclude that $\cfgnt{L}_\gsest \lp \cfgnt{r}_0\rp \neq  \cfgnt{L}_\gsest \lp \cfgnt{r}_1 \rp$. Because of this, and because of Theorem~\ref{thm:determ}, we know that either all constraints in the set
$$\{\phi_i \mid \exists \phi_\alpha (\phi_\alpha \in \theta_\alpha)\wedge \phi_i = \lp\phi_\alpha \wedge \phi_0 \wedge \phi_1\rp\}$$ are unsatisfiable, or that at least one constraint in the set
$$\{\phi_i \mid \exists \phi_\alpha (\phi_\alpha \in \lp \theta_0 \cup \theta_1 \rp)\wedge \lp \phi_i = \phi_\alpha \wedge \phi_0 \wedge \phi_1 \rp\}$$ 
is valid. Either way, $\mathbb{S}\lp\phi_i^\prime \wedge\phi_0\wedge \phi_1\rp$ is false and $\symst^\prime$ does not represent $s_i^\prime$. We have a contradiction. Therefore: 
\begin{equation}
\label{eqn:RCBackwardsTrue}
((\symst^\prime=\lp \cfgnt{L}\ \cfgnt{R}\ \phi_g^\prime\ \eta\ \cfgt{true}\ \cfgnt{k}\rp) \wedge (\gsest^\prime \sqsubset \symst^\prime)) \Rightarrow \gsest^\prime \in \zeta_t
\end{equation}

Case 3: Assume that $\cfgnt{L}_\gsest \lp \cfgnt{r}_0\rp \neq \cfgnt{L}_\gsest \lp \cfgnt{r}_1 \rp$.
This means that the GSE ``equals - references false" rule applies. The proof for the ``equals - references false" rule is highly similar to the proof for Case 1, so we omit it for the sake of brevity. The result for this case is:
\begin{equation}
\label{eqn:RCForwardsFalse}
((\cfgnt{L}_\gsest \lp \cfgnt{r}_0\rp = \cfgnt{L}_\gsest \lp \cfgnt{r}_1 \rp) \wedge (\gsest^\prime \in \zeta_t)) \Rightarrow \gsest^\prime \sqsubset \symst^\prime
\end{equation}

Case 4: Assume that $\symst^\prime$ has the form $\lp \cfgnt{L}\ \cfgnt{R}\ \phi_g^\prime\ \eta\ \cfgt{false}\ \cfgnt{k}\rp $.
The proof for this case is highly similar to the proof for Case 2, so we omit it for the sake of brevity. The result for this case is:
\begin{equation}
\label{eqn:RCBackwardsFalse}
\symst^\prime=\lp \cfgnt{L}\ \cfgnt{R}\ \phi_g^\prime\ \eta\ \cfgt{false}\ \cfgnt{k}\rp \wedge \gsest^\prime \sqsubset \symst^\prime \Rightarrow \gsest^\prime \in \zeta_f
\end{equation}

Since $\zeta_T = \zeta_t \cup \zeta_f$, we can combine Equation \ref{eqn:RCForwardsTrue} with \ref{eqn:RCForwardsFalse} to find that 
\begin{equation}
\forall \gsest^\prime ( \gsest \rgse \gsest^\prime \Rightarrow \exists \symst^\prime( (\symst \rsym \symst^\prime )\wedge (\gsest^\prime\ \sqsubset\ \symst^\prime ))  )
\end{equation}
Likewise, we can combine Equation \ref{eqn:RCBackwardsTrue} with Equation \ref{eqn:RCBackwardsFalse} to find that
\begin{equation}
\forall \symst^\prime ( \symst \rsym \symst^\prime\Rightarrow \exists \gsest^\prime( (\gsest \rgse \gsest^\prime )\wedge (\gsest^\prime\ \sqsubset\ \symst^\prime ))  )
\end{equation}
By \defref{bisimulation}, we conclude:
$$(\Gsest \ \symst) \in F_\sim(\sqsubset)$$
\end{proof}

%\begin{lemma}[Exactness of New Rule]
%\label{lem:new}
%If there exists states $\gsest$ and $\symst$ such that $\symst \in \mathcal{NW}$ and $\gsest \sqsubset \symst$, then:
%\begin{equation}
%\forall \gsest^\prime ( \gsest \rgse \gsest^\prime \Rightarrow \exists \symst^\prime( (\symst \rsym \symst^\prime )\wedge (\gsest^\prime\ \sqsubset\ \symst^\prime ))  )
%\end{equation}
%and
%\begin{equation}
%\forall \symst^\prime ( \symst \rsym \symst^\prime\Rightarrow \exists \gsest^\prime( (\gsest \rgse \gsest^\prime )\wedge (\gsest^\prime\ \sqsubset\ \symst^\prime ))  )
%\end{equation}

%\end{lemma}

%
%\begin{proof}
%The proof follows immediately from the rule.
%\end{proof}



\begin{theorem}
\label{th:bisim}
The representation relation $\sqsubset$ is a bisimulation.
\end{theorem}

\begin{proof}
Take any two states $\gsest$ and $\symst$ such that $\gsest \sqsubset \symst$. If $\symst \in \mathcal{FA} \cup \mathcal{FW} \cup \mathcal{RC}$, then by Lemmas \ref{lem:access}, \ref{lem:write}, and \ref{lem:compare} we know Equations \ref{eqn:BisimulationForwards} and \ref{eqn:BisimulationBackwards} hold. If $\symst$ has any other form, the heap is not modified for $\gsest^\prime$ or $\symst^\prime$, so then Equations \ref{eqn:BisimulationForwards} and \ref{eqn:BisimulationBackwards} hold by default. Thus, Equations \ref{eqn:BisimulationForwards} and \ref{eqn:BisimulationBackwards} hold for all  $\gsest$ and $\symst$ such that $\gsest \sqsubset \symst$. By Definition \ref{bisimulation}, $\sqsubset$ is a bisimulation.
\end{proof}

Note that in the above proof we omit consideration of states in $\mathcal{NW}$, as this case is trivial.

\begin{corollary}
For any given initial state, the set of possible control flow sequences under the GSE transition relation is exactly the set of possible control flow sequences under the symbolic transition relation.
\end{corollary}

\begin{corollary}
For any given initial state, the number of final symbolic states is exactly the number of possible control flow sequences.
\end{corollary}
