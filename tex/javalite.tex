\section{Javalite}
The algorithm for generalized symbolic execution with symbolic heaps
is presented using Javalite. Javalite is a syntactic machine defined
as rewrites on a string \cite{saints-MS}. The semantics are small-step
using a CESK model with a (C)ontrol string representing the expression
being evaluated, an (E)nvironment for local variables, a (S)tore for
the heap, and a (K)ontinuation for what is to be executed next. The
semantics of the original Javalite machine, and its machine syntax,
are changed to reflect generalized symbolic execution and then
generalized execution with symbolic heaps.

The surface syntax (input) for Javalite is in
\figref{fig:surface-syntax} with the machine syntax (state) in
\figref{fig:machine-syntax}. Terminals are in bold face while
non-terminals are italicized. Ellipses indicate zero or more
repetitions. Tuples omit the commas for compactness. The language
itself uses s-expressions for convenience.

\begin{figure}
\begin{center}
\cfgstart
\cfgrule{P}{\lp $\mu$ \lp \cfgnt{C} \cfgnt{m}\rp\rp}
\cfgrule{$\mu$}{(\cfgnt{CL} ...)}
\cfgrule{T}{\cfgt{bool} \cfgor \cfgnt{C}}
\cfgrule{CL}{\lp\cfgt{class} \cfgnt{C} \lp\lb\cfgnt{T} \cfgnt{f}\rb ...\rp \lp\cfgnt{M} ...\rp}
\cfgrule{M}{\lp\cfgnt{T} \cfgnt{m} \lb\cfgnt{T} \cfgnt{x}\rb\  e\rp}
\cfgrule{e}{\cfgnt{x}
\cfgor{\lp\cfgt{new} \cfgnt{C}\rp}
\cfgor{\lp\cfgnt{e} \cfgt{\$} \cfgnt{f}\rp}
\cfgor{\lp\cfgnt{x} \cfgt{\$} \cfgnt{f} \cfgt{:=} \cfgnt{e}\rp}
\cfgor{\lp\cfgnt{e} \cfgt{=} \cfgnt{e}\rp}}
\cfgorline{\lp\cfgt{if} \cfgnt{e} \cfgnt{e} \cfgt{else} \cfgnt{e}\rp 
\cfgor {\lp\cfgt{var} \cfgnt{T} \cfgnt{x} \cfgt{:=} \cfgnt{e} \cfgt{in} \cfgnt{e}\rp}
\cfgor {\lp\cfgnt{e} \cfgt{@} \cfgnt{m} \cfgnt{e} \rp}}
\cfgorline{\lp\cfgnt{x} \cfgt{:=} \cfgnt{e}\rp
\cfgor{\lp\cfgt{begin} \cfgnt{e} ...\rp}
\cfgor{\cfgnt{v}}}
\cfgrule{x}{\cfgt{this} \cfgor \cfgnt{id}}
\cfgrule{f,m,C}{\cfgnt{id}}
%\cfgrule{m}{\cfgnt{id}}
%\cfgrule{C}{\cfgnt{id}}
\cfgrule{v}{\cfgnt{r} \cfgor \cfgt{null} \cfgor \cfgt{true} \cfgor \cfgt{false} \cfgor \cfgt{error}}
\cfgrule{r}{\cfgt{number}}
\cfgrule{id}{\cfgt{variable-not-otherwise-mentioned}}
\cfgend
\end{center}
\caption{The Javalite surface syntax.}
\label{fig:surface-syntax}
\end{figure}

\begin{figure}
\begin{center}
\cfgstart
\cfgrule{e}{\lp ... \cfgor \lp \cfgnt{v} \cfgt{@} \cfgnt{m} \cfgnt{v} \rp\rp}
%\cfgrule{object}{ (\cfgnt{C} [ \cfgnt{f} \cfgnt{loc} ] ...) }
%\cfgrule{hv}{ (\cfgnt{v} \cfgnt{object})}
\cfgrule{$\phi$}{\cfgnt{constraint}}
\cfgrule{l}{\cfgt{number}}
%\cfgrule{h}{(\cfgnt{mt}\ (\cfgnt{h}\ [\cfgnt{loc} $\rightarrow$ \cfgnt{hv}]) )}
\cfgrule{$\eta$}{(\cfgnt{mt}\ ($\eta$ [\cfgnt{x} $\rightarrow$ \cfgnt{loc}]))}
\cfgrule{s}{\lp$\mu$ \cfgnt{L} \cfgnt{R} \cfgnt{g} $\eta$ \cfgnt{e} \cfgnt{k}\rp}
\cfgrule{k}{\cfgt{end}}
\cfgorline{\lp \cfgt{*} \cfgt{\$} \cfgnt{f} $\rightarrow$ \cfgnt{k}\rp}
\cfgorline{\lp \cfgt{*} \cfgt{@} \cfgnt{m} \lp \cfgnt{e} ... \rp $\rightarrow$ \cfgnt{k} \rp}
\cfgorline{\lp \cfgnt{v} \cfgt{@} \cfgnt{m} \cfgnt{v} \cfgt{*} \lp \cfgnt{e} ... \rp $\rightarrow$ \cfgnt{k} \rp}
\cfgorline{\lp \cfgt{*} \cfgt{=} \cfgnt{e} $\rightarrow$ \cfgnt{k}\rp}
\cfgorline{\lp \cfgt{v} \cfgt{=} \cfgnt{*} $\rightarrow$ \cfgnt{k}\rp}
\cfgorline{\lp \cfgt{x} \cfgt{:=} \cfgnt{*} $\rightarrow$ \cfgnt{k}\rp}
\cfgorline{\lp \cfgt{x} \cfgt{\$} \cfgnt{f}  \cfgt{:=} \cfgnt{*} $\rightarrow$ \cfgnt{k}\rp}
\cfgorline{\lp \cfgt{if} \cfgnt{*} \cfgnt{e} \cfgt{else} \cfgnt{e} $\rightarrow$ \cfgnt{k} \rp}
\cfgorline{\lp\cfgt{var} \cfgnt{T} \cfgnt{x} \cfgt{:=} \cfgnt{*} \cfgt{in} \cfgnt{e}  $\rightarrow$ \cfgnt{k} \rp}
\cfgorline{\lp\cfgt{begin}  \cfgnt{*} \lp \cfgnt{e} ...\rp $\rightarrow$ \cfgnt{k} \rp}
\cfgorline{\lp\cfgt{pop} $\eta$ \cfgnt{k}\rp}
\cfgend
\end{center}
\caption{The machine syntax for Javalite.}
\label{fig:machine-syntax}
\end{figure}


A Javalite program, $\cfgnt{P}$, is a registry of classes, $\mu$, with
a tuple indicating a class, $\cfgnt{C}$, and a method, $\cfgnt{m}$,
where execution starts. The only primitive type in Javalite is
Boolean. Classes have fields, $([\cfgnt{T}\ \cfgnt{f}]\ ...)$, and
methods, $(\cfgnt{M}\ ...)$. Methods only take a single
parameter. Expressions, $\cfgnt{e}$, are somewhat typical with
$\cfgt{:=}$ indicating assignment, $\cfgt{=}$ indicating comparison,
$\cfgt{\$}$ as the dot-operator for field access, and $\cfgt{@}$ as
the dot-operator for method invocation. There is no explicit return
statement in Javalite; rather, the value of the last expression is
used as a return value. A variable is always indicated by \cfgnt{x}
and a value by \cfgnt{v}. A value can be a reference in the heap,
$\cfgnt{r}$, or any of the special values. It is assumed that only
type correct programs are given as input to the machine.

The state of the Javalite machine (\figref{fig:machine-syntax}),
$\cfgnt{s}$, includes the program registry, a pair of functions
defining the heap $\lp\cfgnt{L}\ \cfgnt{R}\rp$, which is the store, a
constraint, $\phi$, defined over references in the heap, the
environment, $\eta$, for local variables, and the continuation
$\cfgnt{k}$. The more complex structures such as the environment are
defined as lists which start with empty, \cfgnt{mt}. The rewrite rules
that define the semantics treat these lists as partial
functions. As such, $\eta(\cfgnt{x}) = \cfgnt{r}$ is the
reference mapped to the variable $\cfgnt{x}$. The notation
$\eta^\prime = \eta[\cfgnt{x} \mapsto \cfgnt{v}]$ defines a new
partial function $\eta^\prime$ that is just like $\eta$ only the
variable $\cfgnt{x}$ now maps to $\cfgnt{v}$. 

The continuation $\cfgnt{k}$ helps accomplish the small-step semantics
by indicating where the hole, $\cfgt{*}$, is located, by storing
temporary computations, and by keeping track of the next
computation. For example, the continuation
$\lp\cfgt{*}\ \cfgt{\$}\ \cfgnt{f} \rightarrow \cfgnt{k}\rp$ indicates
that the machine is currently computing the expression for the object
reference on which the field $\cfgnt{f}$ is going to be accessed. Once
the field access is complete, the machine continues with $\cfgnt{k}$.

\subsection{Heaps as Bipartite Graphs}
An important aspect, and a novel contribution, of a symbolic heap is
its representation as a labeled bipartite graph consisting of
references, $\cfgnt{r}$, and constraint location pairs,
$\lp\phi\ \cfgnt{l}\rp$. The bipartite graph separates the constraint
problem from the morphology problem, that is to say, the shape of the
heap. Being bipartite also makes it so it is not needed to rewrite the
path condition every time the heap is updated. Without these two
properties, it would not be possible to lazily initialize heap
locations along a program path. The heap structure itself captures
that the actual location pointed to by a reference is conditional on
aliasing relationships within the heap. In this way, the graph
represents sets of heaps with the conditions under which those heaps
exist.

The graph is expressed in $\cfgnt{L}$, the location map, and
$\cfgnt{R}$, the reference map. As done with the environment,
$\cfgnt{L}$ and $\cfgnt{R}$ are treated as partial functions where
$\cfgnt{L}(r) = \{(\phi\ \cfgnt{l})\ ...\}$ is the set of
location-constraint pairs in the heap associated with the given
reference, and $\cfgnt{R}(\cfgnt{l},\cfgnt{f}) = \cfgnt{r}$ is the
reference associated with the given location-field pair in the
heap. Predicate calculus is used to describe the more complex updates
to the heap in the semantics.

There are several properties of the heap that are invariant and
important to the correctness proofs on the algorithm.
\begin{compactitem}
\item \textbf{Immutable references}: a reference does not change in
  the location map; it always points to the same set of constraint
  location pairs. The property ensures that constraints never need to
  be rewritten as the program evolves its state.
\item \textbf{Reference partitions}: new references are drawn from
  three distinct partitions $\mathrm{init}_r()$ for initialization,
  $\mathrm{fresh}_r()$ for the input heap, and $\mathrm{stack}_r()$
  for references in the expressions and continuations. The references
  strictly increase in value on each call, and modular arithmetic is
  used to determine the partition to which a reference belongs. The
  partitions are a critical piece to the existence proof of a
  bisimulation that relates states with a concrete heap to states with
  heap summaries.
\item \textbf{Determinism}: a reference in  $(\cfgnt{L}\ \cfgnt{R})$ cannot point to two locations at the same time.
\[
\begin{array}{l}
\forall \cfgnt{r} \in \cfgnt{L}^\leftarrow\ (\forall (\phi\ \cfgnt{l}),(\phi^\prime\ \cfgnt{l}^\prime) \in \cfgnt{L}(r)\ (\\
\ \ \ \ (\cfgnt{l} \neq \cfgnt{l}^\prime \vee \phi \neq \phi^\prime) \Rightarrow (\phi \wedge \phi^\prime = \cfgt{false}))
\end{array}
\]
$\cfgnt{L}^\leftarrow$ is the pre-image of the location map.
\item \textbf{Type consistency}: every location associated to a reference has the same type.
\[
\begin{array}{l}
\forall \cfgnt{r} \in \cfgnt{L}^\leftarrow\ (\forall (\phi\ \cfgnt{l}),(\phi^\prime\ \cfgnt{l}^\prime) \in \cfgnt{L}(r)\ (\\
\ \ \ \ (\mathrm{Type}\lp\cfgnt{l}\rp = \mathrm{Type}\lp\cfgnt{l}^\prime\rp)))
\end{array}
\]
\item \textbf{Null and Uninitialized}: every heap contains special
  locations for null ($\cfgnt{l}_\mathit{null}$) and uninitialized
  ($\cfgnt{l}_\mathit{un}$) with corresponding references
  $\cfgnt{r}_\mathit{null}$ and
  $\cfgnt{r}_\mathit{un}$. $\cfgnt{L}(\cfgnt{r}_\mathit{null}) =
  \{(\cfgt{true}\ \cfgnt{l}_\mathit{null})\}$ and
  $\cfgnt{L}(\cfgnt{r}_\mathit{un}) =
  \{(\cfgt{true}\ \cfgnt{l}_\mathit{un})\}$.
\end{compactitem}
The Javalite semantics are modified to operate on this new heap
structure so that a heap summary represents an equivalence class of
concrete heaps. All the concrete heaps that follow the same control
flow to a point of execution are in the equivalence class represented
by the summary. That equivalence class splits depending control flow
branches, partitioning the heaps depending on the produced outcome.

\subsection{State Transition Relation}

The initial state of a Javalite program is constructed from its
surface syntax on the $\cfgnt{P}$ production:
$\lp\mu\ \lp\cfgnt{C}\ \cfgnt{m}\rp\rp$. Assuming that
$\lp\cfgnt{L}\ \cfgnt{R}\rp$ is an empty heap with no references
or locations and $\eta$ is an empty environment with no
variables, the initial state is
$$
\begin{array}{l}
\lp\mu 
\ \cfgnt{L}[\cfgnt{r}_\mathit{null} \mapsto \{\lp\cfgt{true}\ \cfgnt{l}_\mathit{null})\}\rp] 
           [\cfgnt{r}_\mathit{un} \mapsto \{\lp\cfgt{true}\ \cfgnt{l}_\mathit{un}\rp\}] \\
\ \cfgnt{R}
\ \cfgt{true}\ \eta\  \lp\lp\cfgt{new}\ \cfgnt{C}\rp\ \cfgt{@}\ \cfgnt{m}\ \cfgt{true}\rp\rp\ \cfgt{end}\rp
\end{array}
$$
The first expression in the initial state is a call to a constructor
to create an instance of the class $\cfgnt{C}$. Once an instance of
$\cfgnt{C}$ is exists, the methods $\cfgnt{m}$ of the class is
invoked.

The common portion of Javalite for the new heap summary is defined
elsewhere, but in general, the semantics of Javalite are expressed as
rewrites on strings using pattern matching. Consider the rewrite rule
for the beginning of a field access instruction:
$$
\mprset{flushleft}
	\inferrule[Field Access(eval)]{}{
      \lp \cfgnt{L}\ \cfgnt{R}\ \phi\ \eta\ \lp \cfgnt{e}\ \cfgt{\$}\ \cfgnt{f}\rp \ \cfgnt{k}\rp  \rcom 
      \lp \cfgnt{L}\ \cfgnt{R}\ \phi\ \eta\ \cfgnt{e}\ \lp \cfgt{*}\ \cfgt{\$}\ \cfgnt{f} \rightarrow \cfgnt{k}\rp \rp 
	}
$$
If the string representing the state matches the left side, then it
creates the new string on the right. In this example, the new string
on the right is now evaluating the expression $\cfgnt{e}$ in the field
access, and it includes the continuation indicating that it still
needs to complete the actual field access once the expression is
evaluated.

Other more complex rules such as the one to create a new instance of a
class define constraints on the rewrites and more complex
transformations on the heap.
$$
\mprset{flushleft}
	\inferrule[New]{
      \cfgnt{r} = \mathrm{stack}_r\lp \rp\\
      l = \mathrm{fresh}_l\lp \cfgnt{C}\rp\\\\
      \cfgnt{R}^\prime = \cfgnt{R}[\forall \cfgnt{f} \in \mathit{fields}\lp \mathrm{C}\rp \ \lp \lp l\ \cfgnt{f}\rp  \mapsto \cfgnt{r}_\mathit{null} \rp ] \\\\
      \cfgnt{L}^\prime = \cfgnt{L}[\cfgnt{r} \mapsto \{\lp \cfgt{true}\ l\rp \}]
    }{
      \lp \cfgnt{L}\ \cfgnt{R}\ \phi\ \eta\ \lp \cfgt{new}\ \cfgnt{C}\rp \ \cfgnt{k}\rp  \rcom
      \lp \cfgnt{L}^\prime\ \cfgnt{R}^\prime\ \phi\ \eta\ \cfgnt{r}\ \cfgnt{k}\rp 
	}
$$
Here when the string matches the new-expression, it is written to use
a new heap where all the fields for the new object point to null and
the location map points a new stack reference to that new object.

The full operational semantics for generalized symbolic execution with
lazy initialization for Javalite are in the supplemental
document. From here forward, that machine is referred to by the
relation $\rgse$ or $p \rgse p^\prime$. The new algorithm for symbolic
execution with summary heaps is indicated by $\rsym$ or $q \rsym
q^\prime$, and it is defined in the next section.






