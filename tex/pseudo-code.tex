\section{Pseudo-code}
The presentation assumes a purely syntactic semantics for a CEKS
machine for a Java-like language \cite{saints-MS}. The machine state
for the semantics consists of a (C)ontrol string representing the
expression being evaluated, an (E)nvironment for local variables, a
(K)ontinuation for what is to be executed next, and a (S)tore for the
heap. This paper only defines salient features of the machine relevant
to understanding JPF's partial order reduction. 
\begin{itemize}
\item $n$ is a number, and the set of all positive natural numbers is $\mathbb{N}^+$.
\item $x$ is a variable, and the set of all variables is $\mathbb{X}$.
\item $f$ is a named field in an object.
\item $v$ is a value, and the set of all values is $\mathbb{V}$. A value can be a primitive, and object with named fields, or a reference to an object. 
\item $\eta : \mathbb{X} \rightarrow \mathbb{V}$ is an environment
\item $h : \mathbb{N}^+ \rightarrow \mathbb{V}$ is a heap
\item $k$ is a continuation to resolve evaluation order and return points from calls. Are particular interest are the continuations for thread completion (\texttt{ret}), thread creation 
\item $e$ is an expression. Expressions can be values $v$, sequencing ($(begin\ e\ \ldots)$),
  thread start ($(\mathtt{thread}\ e.m(e\ \ldots))$), field access ($(x.f)$ or $(x.f := e)$), and other constructs
  indicative of the Java language.
\item $t = (\eta\ e\ k)$ is a thread state.
\item $s = (\mu\ h\ (t\ \ldots))$ is a machine state.
\item $(h^\prime\ (\eta^\prime \ e^\prime\ k^\prime)) = \mathrm{execute}(\mu, h, \eta, e, k)$ is a function that returns a new heap and thread by evaluating the expression in the thread until the thread is either done, creates a new thread, or accesses a field in the heap that is shared:
  \begin{itemize}
    \item $(\eta^\prime\ v (* \$ f \rightarrow k))$
  \end{itemize}
\end{itemize}

% Algorithm2e environment
% http://en.wikibooks.org/wiki/LaTeX/Algorithms#Typesetting_using_the_algorithm2e_package

\begin{algorithm}
 \SetAlgoLined
 \KwData{this text}
 \KwResult{how to write algorithm with \LaTeX2e }
 initialization\;
 \While{not at end of this document $\wedge x < 2$}{
  read current\;
  \eIf{understand}{
   go to next section\;
   current section becomes this one\;
   }{
   go back to the beginning of current section\;
  }
 }
 \caption{How to write algorithms}
\end{algorithm}
