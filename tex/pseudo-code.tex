\section{Pseudo-code}
\figref{fig:surface-syntax} defines the surface syntax for the
Javalite language \cite{saints-MS}. The \figref{fig:machine-syntax} is
the machine syntax. The semantics of Javalite is syntax based and
defined as rewrites on a string. The semantics use a CEKS machine
model with a (C)ontrol string representing the expression being
evaluated, an (E)nvironment for local variables, a (K)ontinuation for
what is to be executed next, and a (S)tore for the heap. This paper
only defines salient features of the language and machine relevant to
understanding the new algorithm.

\begin{figure}
\begin{center}
\cfgstart
\cfgrule{P}{\lp $\mu$ \lp \cfgnt{C} \cfgnt{m}\rp\rp}
\cfgrule{$\mu$}{(\cfgnt{CL} ...)}
\cfgrule{T}{\cfgt{bool} \cfgor \cfgnt{C}}
\cfgrule{CL}{\lp\cfgt{class} \cfgnt{C} \lp\lb\cfgnt{T} \cfgnt{f}\rb ...\rp \lp\cfgnt{M} ...\rp}
\cfgrule{M}{\lp\cfgnt{T} \cfgnt{m} \lb\cfgnt{T} \cfgnt{x}\rb\  e\rp}
\cfgrule{e}{\cfgnt{x}
\cfgor{\lp\cfgt{new} \cfgnt{C}\rp}
\cfgor{\lp\cfgnt{e} \cfgt{\$} \cfgnt{f}\rp}
\cfgor{\lp\cfgnt{x} \cfgt{\$} \cfgnt{f} \cfgt{:=} \cfgnt{e}\rp}
\cfgor{\lp\cfgnt{e} \cfgt{=} \cfgnt{e}\rp}}
\cfgorline{\lp\cfgt{if} \cfgnt{e} \cfgnt{e} \cfgt{else} \cfgnt{e}\rp 
\cfgor {\lp\cfgt{var} \cfgnt{T} \cfgnt{x} \cfgt{:=} \cfgnt{e} \cfgt{in} \cfgnt{e}\rp}
\cfgor {\lp\cfgnt{e} \cfgt{@} \cfgnt{m} \cfgnt{e} \rp}}
\cfgorline{\lp\cfgnt{x} \cfgt{:=} \cfgnt{e}\rp
\cfgor{\lp\cfgt{begin} \cfgnt{e} ...\rp}
\cfgor{\cfgnt{v}}}
\cfgrule{x}{\cfgt{this} \cfgor \cfgnt{id}}
\cfgrule{f,m,C}{\cfgnt{id}}
%\cfgrule{m}{\cfgnt{id}}
%\cfgrule{C}{\cfgnt{id}}
\cfgrule{v}{\cfgnt{r} \cfgor \cfgt{null} \cfgor \cfgt{true} \cfgor \cfgt{false} \cfgor \cfgt{error}}
\cfgrule{r}{\cfgt{number}}
\cfgrule{id}{\cfgt{variable-not-otherwise-mentioned}}
\cfgend
\end{center}
\caption{The Javalite surface syntax.}
\label{fig:surface-syntax}
\end{figure}

\begin{figure}
\begin{center}
\cfgstart
\cfgrule{e}{\lp ... \cfgor \lp \cfgnt{v} \cfgt{@} \cfgnt{m} \cfgnt{v} \rp\rp}
%\cfgrule{object}{ (\cfgnt{C} [ \cfgnt{f} \cfgnt{loc} ] ...) }
%\cfgrule{hv}{ (\cfgnt{v} \cfgnt{object})}
\cfgrule{$\phi$}{\cfgnt{constraint}}
\cfgrule{l}{\cfgt{number}}
%\cfgrule{h}{(\cfgnt{mt}\ (\cfgnt{h}\ [\cfgnt{loc} $\rightarrow$ \cfgnt{hv}]) )}
\cfgrule{$\eta$}{(\cfgnt{mt}\ ($\eta$ [\cfgnt{x} $\rightarrow$ \cfgnt{loc}]))}
\cfgrule{s}{\lp$\mu$ \cfgnt{L} \cfgnt{R} \cfgnt{g} $\eta$ \cfgnt{e} \cfgnt{k}\rp}
\cfgrule{k}{\cfgt{end}}
\cfgorline{\lp \cfgt{*} \cfgt{\$} \cfgnt{f} $\rightarrow$ \cfgnt{k}\rp}
\cfgorline{\lp \cfgt{*} \cfgt{@} \cfgnt{m} \lp \cfgnt{e} ... \rp $\rightarrow$ \cfgnt{k} \rp}
\cfgorline{\lp \cfgnt{v} \cfgt{@} \cfgnt{m} \cfgnt{v} \cfgt{*} \lp \cfgnt{e} ... \rp $\rightarrow$ \cfgnt{k} \rp}
\cfgorline{\lp \cfgt{*} \cfgt{=} \cfgnt{e} $\rightarrow$ \cfgnt{k}\rp}
\cfgorline{\lp \cfgt{v} \cfgt{=} \cfgnt{*} $\rightarrow$ \cfgnt{k}\rp}
\cfgorline{\lp \cfgt{x} \cfgt{:=} \cfgnt{*} $\rightarrow$ \cfgnt{k}\rp}
\cfgorline{\lp \cfgt{x} \cfgt{\$} \cfgnt{f}  \cfgt{:=} \cfgnt{*} $\rightarrow$ \cfgnt{k}\rp}
\cfgorline{\lp \cfgt{if} \cfgnt{*} \cfgnt{e} \cfgt{else} \cfgnt{e} $\rightarrow$ \cfgnt{k} \rp}
\cfgorline{\lp\cfgt{var} \cfgnt{T} \cfgnt{x} \cfgt{:=} \cfgnt{*} \cfgt{in} \cfgnt{e}  $\rightarrow$ \cfgnt{k} \rp}
\cfgorline{\lp\cfgt{begin}  \cfgnt{*} \lp \cfgnt{e} ...\rp $\rightarrow$ \cfgnt{k} \rp}
\cfgorline{\lp\cfgt{pop} $\eta$ \cfgnt{k}\rp}
\cfgend
\end{center}
\caption{The machine syntax for Javalite.}
\label{fig:machine-syntax}
\end{figure}


%\item $n$ are numbers, and the set of all positive natural numbers is $\mathbb{N}^+$.
This paper uses a standard definition of constraints $\phi \in \Phi$
assuming all the usual relational operators and connectives. The heap
is a labeled bipartite graph consisting of references $R$ and
locations $L$ in the store. The functions $R$ and $L$ are defined for convenience
in manipulating the labeled bipartite graph.
\begin{itemize}
\item $R(l,f)$ maps location-field pairs from the store to a reference in $R$; and
\item $L(r)$ maps references to a set of location-constraint pairs in the store.
\end{itemize}
A reference is a node that gathers the possible store locations for an
object during symbolic execution. Each store location is guarded by a
constraint that determines the aliasing in the heap. Intuitively, the
reference is a level of indirection between a variable and the store,
and the reference is used to group a set of possible store locations
each predicated on the possible aliasing in the associated constraint.
For a variable (or field) to access any particular store location
associated with its reference, the corresponding constraint must be
satisfied.

\begin{comment}
 defined on value sets $L$ and references
$R$. $R$ and $L$ include the special symbol $\bot$ to indicate an
uninitialized reference and location respectively. Additionally, $L$
includes a special location $\mathtt{NULL}$ for null references.
\end{comment}

Locations are boxes in the graphical representation and indicated with
the letter $l$ in the math. References are circles in the graphical
representation and indicated with the letter $r$ in the math. Edges
from locations are labeled with field names $f \in F$. Edges from the
references are labeled with constraints $\phi \in \Phi$ (we assume
$\Phi$ is a power set over individual constraints and $\phi$ is a set
of constraints for the edge).

The function $\mathbb{VS}(L,R,r,f)$ constructs the value-set given a
heap, reference, and desired field such that
$(l^\prime\ \phi^\prime\wedge\phi) \in \mathbb{VS}(L,R,r,f)$ if and
only if
\[
\begin{array}{l}
  \exists (l\ \phi) \in L(r) ( \\
  \ \ \ \ \ \ \ \ \ \exists r^\prime \in R(l,f) ( \\
  \ \ \ \ \ \ \ \ \ \ \ \ \ \ \ \ \ \ \exists (l^\prime\ \phi^\prime) \in L(r^\prime) (\mathbb{S}(\phi\wedge\phi^\prime))))
\end{array}
\]
where $\mathbb{S}(\phi)$ returns true if $\phi$ is satisfiable.

The strengthen function $\mathbb{ST}(L,r,\phi^\prime)$ strengthens every
constraint from the reference $r$ and keeps only location-constraint
pairs that are satisfiable after strengthening. Formally,
$(l\ \phi\wedge\phi^\prime)\in\mathbb{ST}(L,r,\phi^\prime)$ if and
only if $\exists (l\ \phi)\in
L(r)\wedge\mathbb{S}(\phi\wedge\phi^\prime)$

The empty-reference function $\mathbb{ER}(L,\phi^\prime) = \{r\ |\ L(r) \neq
\emptyset \wedge \forall(l,\phi) \in L(r)(\neg \mathbb{S}(\phi \wedge
\phi^\prime))\}$ searches the heap for references that become
disconnected from all their locations after strengthening. These
references, if reachable, imply the heap is no longer valid on the
current search path. As such, the symbolic execution algorithm should
backtrack. This check is similar to a feasibility check in classic
symbolic execution with only primitive data types.

The consistency function is critical to the soundness of the algorithm
as it detects when a symbolic heap becomes invalid along a path,
similar to a feasibility check when doing classical symbolic execution
with just primitives. As the constraints in the heap are strengthened
with different aliasing requirements, it is possible to reach a point
where the heap is no longer connected. Meaning, a valid reference is
live, either in the local environment or the continuation, the reaches
another reference that is no longer connected to any locations due to
strengthening. The function relies on the empty-reference function to
identify disconnected references. The function in essence checks every
reference in the local environment and every reference found in the
continuation, as these are all considered live. This operation similar
to garbage collection where the local environment and stack are
inspected to find the roots of the heap for the scan.


The consistency function relies on two auxiliary functions which are
informally defined. The function $\mathrm{ref}(\eta,\cfgnt{k})$
inspects the local environment and continuation for all live
references, and it returns those references in a set. The function
$\mathrm{reach}(L, R, r, r^\prime)$ returns true if $r^\prime$ is
reachable from $r$ in the heap and false otherwise. The consistency function $\mathbb{C}(L,R,X,\eta,k)$ is now defined as
\[
 \left\{ \begin{array}{rl} 
        0 & \exists r \in \mathrm{ref}(\eta, \cfgnt{k})\ (\exists r^\prime \in X\ (\mathrm{reach}(L, R, r,r^\prime))) \\ 
        1 & \mbox{otherwise}\end{array}\right .
\]


\begin{figure*}[t]
\begin{center}
\mprset{flushleft}
\begin{mathpar}
	\inferrule[Variable lookup]{}{
      (L\ R\ g\ \eta\ \cfgnt{x}\ k) \rightarrow (L\ R\ g\ \eta\ \eta(\cfgnt{x})\ k)
	}
\and
	\inferrule[Field Access(eval)]{}{
      (L\ R\ g\ \eta\ (e\ \$\ \cfgnt{f})\ k) \rightarrow (L\ R\ g\ \eta\ e\ (\cfgt{init}\ \cfgnt{f}\ (*\ \$\ \cfgnt{f} \rightarrow k)))
	}
\and
%	\inferrule[Field Access (NULL)]{
 %     L(r) = \emptyset
 %   }{
 %     (L\ R\ g\ \eta\ r\ (*\ \$\ \cfgnt{f} \rightarrow k)) \rightarrow 
%      (L[r \mapsto \{(\bot,\phi_T)\}]\ R\ \eta\ r\ (*\ \$\ \cfgnt{f} \rightarrow k))
%	}
%\and
	\inferrule[Lazy Initialization]{
	  \Lambda = \{ l \mid \exists \phi\ ((l,\phi) \in L(r) \wedge  R(l,f) = \emptyset\}\\
      l_x = \mathrm{min}_l(\Lambda) \\\\
      \Lambda \neq \emptyset \\
      \mathrm{type}(\cfgnt{f}) = \mathrm{C}_r\\
      \mathrm{init}_r() = r_f\\ 
      \mathrm{init}_l(\mathrm{C}_r) = l_f \\\\
      \rho = \{ (r^\prime,\ \phi^\prime,\ l^\prime) \mid \mathrm{isInit}(r^\prime) \wedge (l^\prime, \phi^\prime) \in L(r^\prime) \wedge \mathrm{type}(l^\prime) = \mathrm{C}_r \} \\\\
      \theta_\mathit{alias} = \{ ( l ,\ \phi) | (r^\prime,\ \phi^\prime,\ \l) \in \rho \wedge \phi = (\phi^\prime \wedge r^\prime \neq r_\mathit{null} \wedge r^\prime = r_f \wedge (\wedge_{(r^{\prime\prime},\ \phi^{\prime\prime},\ l^{\prime\prime}) \in \rho\ (r^{\prime\prime} \neq r^\prime)} r^{\prime\prime} \neq r_f )) \} \\\\
      \theta_\mathit{new} = \{(l_f,\ r_f \neq r_\mathit{null} \wedge (\wedge_{(r^\prime,\ \phi^\prime,\ l^\prime) \in \rho} r_f \ne r^\prime))\}\\\\
      \theta_\mathit{null} = \{ (l_\mathit{null},\ r_f = r_\mathit{null}) \}\\\\
      \theta = \theta_\mathit{alias} \cup \theta_\mathit{new} \cup \theta_\mathit{null}
    }{
      (L\ R\ g\ \eta\ r\ (\cfgt{init}\ \cfgnt{f}\ k)) \rightarrow 
      (L[r_f \mapsto \theta]\ R[ (l_x,f) \mapsto r_f ]\ g\ \eta\ r\ (\cfgt{init}\ \cfgnt{f}\ k))
	}
\and
	\inferrule[Lazy Initialization (end)]{
	\forall (l,\phi) \in L(r)\ (R(l,f) \neq \emptyset)
    }{
      (L\ R\ g\ \eta\ r\ (\cfgt{init}\ \cfgnt{f}\ k)) \rightarrow 
      (L\ R\ g\ \eta\ r\ k)
	}
\and
	\inferrule[New]{
      \mathrm{fresh}_r(\cfgnt{C}) = r \\
      \mathrm{fresh}_l(\cfgnt{C}) = l \\\\
      R^\prime = R[\forall \cfgnt{f} \in \mathrm{C}\ ((l\ \cfgnt{f}) \mapsto \mathrm{fresh}_r(\mathrm{type}(\cfgnt{f})))] \\\\
      L^\prime = L[r \mapsto \{(l\ \phi_T)\}]
    }{
      (L\ R\ g\ \eta\ (\cfgt{new}\ \cfgnt{C})\ k) \rightarrow 
      (L^\prime\ R^\prime\ g\ \eta\ r\ k)
	}
\and
	\inferrule[Field Access]{
      \forall (l,\phi) \in L(r)\ (l = l_{\mathit{null}} \rightarrow \neg \mathbb{S}(\phi \wedge g)) \\
      \mathrm{fresh}_r() = r_f
    }{
      (L\ R\ g\ \eta\ r\ (*\ \$\ \cfgnt{f} \rightarrow k)) \rightarrow 
      (L[r_f \mapsto \mathbb{VS}(L,R,r,f,g)]\ R\ g\ \eta\ r_f\ k)
	}
\and
    \inferrule[Equals (l-operand eval)]{}{
      (L\ R\ g\ \eta\ (\cfgnt{e}_0 = \cfgnt{e}) \ k) \rightarrow 
      (L\ R\ g\ \eta\ \cfgnt{e}_0\ (\cfgt{*}\; \cfgt{=}\; \cfgnt{e} \rightarrow \cfgnt{k}))
    }
\and
    \inferrule[Equals (r-operand eval)]{}{
    (L\ R\ g\ \eta\ \cfgnt{v}\ (\cfgt{*}\; \cfgt{=}\; \cfgnt{e} \rightarrow \cfgnt{k})) \rightarrow
    (L\ R\ g\ \eta\ \cfgnt{e}\ (\cfgnt{v}\; \cfgt{=}\; \cfgt{*} \rightarrow \cfgnt{k}))
    }
\and
    \inferrule[Equals (bool)]{
    \cfgnt{v}_0 \in \{\cfgt{true}, \cfgt{false}\} \\
    \cfgnt{v}_1 \in \{\cfgt{true}, \cfgt{false}\} \\ 
    \mathrm{eq?}(v_0, v_1) = \cfgnt{v}_r}{
    (L\ R\ g\ \eta\ \cfgnt{v}_0\ (\cfgnt{v}_1\; \cfgt{=}\; \cfgt{*} \rightarrow \cfgnt{k})) \rightarrow
    (L\ R\ g\ \eta\ \cfgnt{v}_r\ \cfgnt{k})
    }
\and
    \inferrule[Equals (references-true)]{
    \cfgnt{v}_0 \not\in \{\cfgt{true}, \cfgt{false}\} \\
    \cfgnt{v}_1 \not\in \{\cfgt{true}, \cfgt{false}\}\\\\
    \theta_\alpha = \{\phi_0 \wedge \phi_1\mid\exists (l\ \phi_0) \in L(v_0)(\exists (l\ \phi_1) \in L(v_1) ( \mathbb{S}(\phi_0 \wedge \phi_1)))\} \\\\
    \theta_0 = \{\phi_0 \mid \exists (l_0\ \phi_0) \in L(v_0) \wedge \forall (l_1\ \phi_1) \in L(v_1)(l_0 \neq l_1)\} \\\\
    \theta_1 = \{\phi_1 \mid \exists (l_1\ \phi_1) \in L(v_1) \wedge \forall (l_0\ \phi_0) \in L(v_0)(l_0 \neq l_1)\} \\\\
    \phi_g = (\vee_{\phi_\alpha\in\theta_\alpha}\phi_\alpha)\wedge(\wedge_{\phi_0 \in \theta_0} \neg \phi_0)  \wedge(\wedge_{\phi_1
    \in \theta_1} \neg \phi_1)\\ g^\prime = g \wedge \phi_g}{
     % L^\prime = L[\forall r (\exists (\cfgnt{l}\ \phi )\in L(r)(r \mapsto \mathbb{ST}(L,r,\phi_g)))]\\\\
      %X = \mathbb{ER}(L, \phi_g) \\ \mathbb{C}(L^\prime, R, X, \eta, k) = 1}
    (L\ R\ g\ \eta\ \cfgnt{v}_0\ (\cfgnt{v}_1\; \cfgt{=}\; \cfgt{*} \rightarrow \cfgnt{k})) \rightarrow
    (L^\prime\ R\ g^\prime\ \eta\ \cfgt{true}\ \cfgnt{k})
    }
\and
    \inferrule[Equals (references-false)]{
    \cfgnt{v}_0 \not\in \{\cfgt{true}, \cfgt{false}\} \\
    \cfgnt{v}_1 \not\in \{\cfgt{true}, \cfgt{false}\}\\\\
    \theta_\alpha = \{\phi_0 \rightarrow \neg \phi_1\mid\exists (l\ \phi_0) \in L(v_0)(\exists (l\ \phi_1) \in L(v_1) ( \mathbb{S}(\phi_0 \wedge \phi_1)))\} \\\\
    \theta_0 = \{\phi_0 \mid \exists (l_0\ \phi_0) \in L(v_0) \wedge \forall (l_1\ \phi_1) \in L(v_1)(l_0 \neq l_1)\} \\\\
    \theta_1 = \{\phi_1 \mid \exists (l_1\ \phi_1) \in L(v_1) \wedge \forall (l_0\ \phi_0) \in L(v_0)(l_0 \neq l_1)\} \\\\
    \phi_g = (\wedge_{\phi_\alpha\in\theta_\alpha}\phi_\alpha)\vee((\vee_{\phi_0 \in \theta_0} \phi_0)  \vee(\vee_{\phi_1
    \in \theta_1} \phi_1)) \\ g^\prime = g \wedge \phi_g}{
    (L\ R\ g\ \eta\ \cfgnt{v}_0\ (\cfgnt{v}_1\; \cfgt{=}\; \cfgt{*} \rightarrow \cfgnt{k})) \rightarrow
    (L^\prime\ R\ g^\prime\ \eta\ \cfgt{true}\ \cfgnt{k})
    }
    
  %  \\\\
  %    L^\prime = L[\forall r (\exists (\cfgnt{l}\ \phi )\in L(r)(r \mapsto \mathbb{ST}(L,r,\phi_g)))]\\\\
  %    X = \mathbb{ER}(L, \phi_g) \\ \mathbb{C}(L^\prime, R, X, \eta, k) = 1}{
 %   (L\ R\ g\ \eta\ \cfgnt{v}_0\ (\cfgnt{v}_1\; \cfgt{=}\; \cfgt{*} \rightarrow \cfgnt{k})) \rightarrow
 %   (L^\prime\ R\ \eta\ \cfgt{false}\ \cfgnt{k})
 %   }
\and
   \inferrule[Method Invocation (object eval)]{}{
    (L\ R\ g\ \eta\ \lp\cfgnt{e}_0\ \cfgt{@}\ \cfgnt{m}\ \cfgnt{e}_1\rp\ \cfgnt{k}) \rightarrow
    (L\ R\ g\ \eta\ \cfgnt{e}_0\ (\cfgt{*}\ \cfgt{@}\ \cfgnt{m}\ \cfgnt{e}_1\ \rightarrow \cfgnt{k}))
   }
   
\and
   \inferrule[Method Invocation (arg eval)]{}{
    (L\ R\ g\ \eta\ \cfgnt{v}_0\ (\cfgt{*}\ \cfgt{@}\ \cfgnt{m}\ \cfgnt{e}_1\ \rightarrow \cfgnt{k})) \rightarrow
    (L\ R\ g\ \eta\ \cfgnt{e}_1\ (\cfgnt{v}_0\ \cfgt{@}\ \cfgnt{m}\ \cfgt{*}\ \rightarrow \cfgnt{k}))
   }
%\and
%   \inferrule[Method Invocation (arg0 eval)]{}{
%    (L\ R\ \eta\ \cfgnt{v}_0\ (\cfgt{*}\ \cfgt{@}\ \cfgnt{m} \lp\cfgnt{e}_1\ \cfgnt{e}_2\ ...\rp \rightarrow \cfgnt{k})) \rightarrow
%    (L\ R\ \eta\ \cfgnt{e}_1\ (\cfgnt{v}_0\ \cfgt{@}\ \cfgnt{m}\ ()\ \cfgt{*}\ \lp\cfgnt{e}_2\ ...\rp \rightarrow \cfgnt{k}))
%   }
%\and
%   \inferrule[Method Invocation (argi eval)]{}{
%    (L\ R\ \eta\ \cfgnt{v}_i\ (\cfgnt{v}_0\ \cfgt{@}\ \cfgnt{m}\ (\cfgnt{v}_1\ ...)\ \cfgt{*}\ \lp\cfgnt{e}_{i+1}\ \cfgnt{e}_{i+2}\ ...\rp \rightarrow \cfgnt{k})) \rightarrow
%    (L\ R\ \eta\ \cfgnt{e}_{i+1}\ (\cfgnt{v}_0\ \cfgt{@}\ \cfgnt{m}\ (\cfgnt{v}_1\ ...\ v_i)\ \cfgt{*}\ \lp\cfgnt{e}_{i+2}\ ...\rp \rightarrow \cfgnt{k}))
%   }
%\and
%   \inferrule[Method Invocation (args)]{}{
%    (L\ R\ \eta\ \cfgnt{v}_n\ (\cfgnt{v}_0\ \cfgt{@}\ \cfgnt{m}\ (\cfgnt{v}_1\ ...)\ \cfgt{*}\ \lp\rp \rightarrow \cfgnt{k})) \rightarrow
%    (L\ R\ \eta\ (\cfgt{raw}\ \cfgnt{v}_0\ \cfgt{@}\ \cfgnt{m}\ (\cfgnt{v}_1\ ...\ v_n))\ \cfgnt{k})
%   }
%\and
%   \inferrule[Method Invocation (no args)]{}{
%    (L\ R\ \eta\ \cfgnt{e}_0\ (\cfgt{*}\ \cfgt{@}\ \cfgnt{m}\ \lp\rp \rightarrow \cfgnt{k})) \rightarrow
%    (L\ R\ \eta\ (\cfgt{raw}\ \cfgnt{v}_0\ \cfgt{@}\ \cfgnt{m}\ ()\ \cfgnt{k}))
%   }
\and
   \inferrule[Method Invocation (raw)]{
    \mathrm{lookup}(\cfgnt{m}) = \lp\cfgnt{T}\ \cfgnt{m}\ \lb\cfgnt{T}\ \cfgnt{x}\rb\ \ e_m\rp \\
    \eta_m = \eta[\cfgt{this} \mapsto \cfgnt{v}_0][\cfgnt{x} \mapsto \cfgnt{v}_1]\ }{
    (L\ R\ g\ \eta\ \cfgnt{v}_1\ (\cfgnt{v}_0\ \cfgt{@}\ \cfgnt{m}\ \cfgt{*}\ \rightarrow \cfgnt{k})) \rightarrow
    (L\ R\ g\ \eta\ (\cfgt{raw}\ \cfgnt{v}_0\ \cfgt{@}\ \cfgnt{m}\ \cfgnt{v}_1\ \cfgnt{k})
   }
\and
   \inferrule[Method Invocation]{
    \mathrm{lookup}(\cfgnt{m}) = \lp\cfgnt{T}\ \cfgnt{m}\ \lb\cfgnt{T}\ \cfgnt{x}\rb\ \ e_m\rp \\
    \eta_m = \eta[\cfgt{this} \mapsto \cfgnt{v}_0][\cfgnt{x} \mapsto \cfgnt{v}_1]\ }{
    (L\ R\ g\ \eta\ (\cfgt{raw}\ \cfgnt{v}_0\ \cfgt{@}\ \cfgnt{m}\ \cfgnt{v}_1\ \cfgnt{k}) \rightarrow
    (L^\prime\ R^\prime\ g\ \eta_m\ \cfgnt{e}_m\ (\cfgt{pop}\ \eta\ \cfgnt{k}))
   }

\end{mathpar}
\end{center}
\caption{Uber-lazy state reductions}
\label{fig:expr:red}
\end{figure*}

\begin{figure*}[t]
\begin{center}
\mprset{flushleft}
\begin{mathpar}
	\inferrule[ITE (expr-eval)]{}{
      (L\ R\ g\ \eta\ (\cfgt{if}\ \cfgnt{e}_0\ \cfgnt{e}_1\ \cfgt{else}\ \cfgnt{e}_2)\ k) \rightarrow (L\ R\ g\ \eta\ \cfgnt{e}_0\ (\cfgt{if}\ \cfgnt{*}\ \cfgnt{e}_1\ \cfgt{else}\ \cfgnt{e}_2) \rightarrow k)
	}
\and
	\inferrule[ITE (true)]{}{
       (L\ R\ g\ \eta\ \cfgt{true}\ (\cfgt{if}\ \cfgnt{*}\ \cfgnt{e}_1\ \cfgt{else}\ \cfgnt{e}_2) \rightarrow k) \rightarrow (L\ R\ g\ \eta\ \cfgnt{e}_1\  k)
	}
\and
	\inferrule[ITE (false)]{}{
       (L\ R\ g\ \eta\ \cfgt{false}\ (\cfgt{if}\ \cfgnt{*}\ \cfgnt{e}_1\ \cfgt{else}\ \cfgnt{e}_2) \rightarrow k) \rightarrow (L\ R\ g\ \eta\ \cfgnt{e}_2\  k)
	}
\and
	\inferrule[Field Write (eval)]{}{
       (L\ R\ g\ \eta\ (\cfgnt{x}\ \cfgt{\$}\ \cfgnt{f}\ \cfgt{:=}\ \cfgnt{e})\ k) \rightarrow (L\ R\ g\ \eta\ \cfgnt{e}\ (\cfgt{init}\ \cfgnt{f}\ (\cfgnt{x}\ \cfgt{\$}\ \cfgnt{f}\ \cfgt{:=}\ \cfgnt{*}\ \rightarrow\ k)))
	}
\and
	\inferrule[Field Write]{
	r_x = \eta(\cfgnt{x})\\
      	\Lambda = \{ l_i | (l_i,*) \in L(r_x) \wedge R(l_i,f)= \emptyset\}\\\\
      \Lambda = \emptyset \\
      %(\bot,*) \notin L(r_x)  \\\\
     \{(l_{null},\phi)|(l_{null},\phi) \in L(r) \wedge \mathbb{S}(\phi)\}=\emptyset \\\\
      \Psi_x =\{(\phi,l,r^\prime)\ |\ \exists(\cfgnt{l},\phi) \in L(r_x) \wedge \exists r^\prime \in R(\cfgnt{l,f}) \}\\\\
      X = \{ (r^\prime, l^\prime, \theta)| (\phi^\prime,l^\prime,r^\prime)\in \Psi_x \wedge \theta = \mathbb{ST}(L,r,\phi^\prime) \cup \mathbb{ST}(L,r^\prime,\neg\phi^\prime) \}\\\\
      Y= \{l\ |\ (r^\prime, l^\prime, \theta) \in X\}\\ R^\prime = R[\forall l \in Y ((l,f) \mapsto \mathrm{fresh}_r(\mathrm{C}_f)) ]\\\\
      L^\prime = L[\forall l^\prime\ (\exists (r^\prime,l^\prime,\theta) \in X\ (\exists r^{\prime\prime} = R (l^\prime,f) ( r^{\prime\prime} \mapsto \theta) ) )]
    }{
      (L\ R\ g\ \eta\ \cfgnt{r}\ (\cfgnt{x}\ \cfgt{\$}\ \cfgnt{f}\ \cfgt{:=}\ \cfgnt{*}\ \rightarrow\ k)) \rightarrow 
      (L^\prime R^\prime\ g\ \eta\ k)
	}	
\and
   \inferrule[Begin (no args)]{}{
    (L\ R\ g\ \eta\ (\cfgt{begin})\ \cfgnt{k}) \rightarrow
    (L\ R\ g\ \eta\ \cfgnt{k})
   }
\and
   \inferrule[Begin (arg0 eval)]{}{
    (L\ R\ g\ \eta\ (\cfgt{begin}\ \lp \cfgnt{e}_0\ \cfgnt{e}_1\ ...\rp) \rightarrow \cfgnt{k}) \rightarrow
    (L\ R\ g\ \eta\ \cfgnt{e}_0\ (\cfgt{begin}\ \cfgt{*}\ \lp\cfgnt{e}_1\ ...\rp) \rightarrow \cfgnt{k})
   }
\and
   \inferrule[Begin (argI eval)]{}{
    (L\ R\ g\ \eta\ v\ (\cfgt{begin}\ \cfgt{*}\ \lp\cfgnt{e}_i\ \cfgnt{e}_{i+1}\ ...\rp) \rightarrow \cfgnt{k}) \rightarrow
    (L\ R\ g\ \eta\ \cfgnt{e}_i\ (\cfgt{begin}\ \cfgt{*}\ \lp\cfgnt{e}_{i+1}\ ...\rp) \rightarrow \cfgnt{k})
   }
\and
   \inferrule[Begin (argN eval)]{}{
    (L\ R\ g\ \eta\ v\ (\cfgt{begin}\ \lp\cfgnt{e}_{n}\rp) \rightarrow \cfgnt{k}) \rightarrow
    (L\ R\ g\ \eta\ \cfgnt{e}_n\ (\cfgt{begin}\ \cfgt{*}\ \lp\rp) \rightarrow \cfgnt{k})
   }
\and
   \inferrule[Begin]{}{
    (L\ R\ g\ \eta\ v\ (\cfgt{begin}\ \cfgt{*}\ \lp\rp \rightarrow \cfgnt{k})) \rightarrow
    (L\ R\ g\ \eta\ v\ \cfgnt{k})
   }	
\and
   \inferrule[Variable Declaration (eval)]{}{
    (L\ R\ g\ \eta\ \lp\cfgt{var}\ \cfgnt{T}\ \cfgnt{x}\ \cfgt{:=}\ \cfgnt{e}_0\ \cfgt{in}\ \cfgnt{e}_1\rp\ \cfgnt{k})) \rightarrow
    (L\ R\ g\ \eta\ \cfgnt{e}_0\ \lp\cfgt{var}\ \cfgnt{T}\ \cfgnt{x}\ \cfgt{*}\ \cfgt{:=}\ \cfgt{in}\ \cfgnt{e}_1\rp \rightarrow \cfgnt{k})
   }	
\and
   \inferrule[Variable Declaration]{
   \mathrm{fresh}_r(\cfgnt{T}) = r \\
   \mathrm{fresh}_l(\cfgnt{T}) = l \\
   \eta^\prime = \eta[x \mapsto  r]\\\\
   R^\prime = R[\forall \cfgnt{f} \in \mathrm{C}\ ((l\ \cfgnt{f}) \mapsto \mathrm{fresh}_r(\mathrm{type}(\cfgnt{f})))] \\\\
   L^\prime = L[r \mapsto \{(l\ \phi_T)\}]
   }{
    (L\ R\ g\ \eta\ v\ \lp\cfgt{var}\ \cfgnt{T}\ \cfgnt{x}\ \cfgt{*}\ \cfgt{:=}\ \cfgt{in}\ \cfgnt{e}_1\rp \rightarrow \cfgnt{k}) \rightarrow
    (L^\prime\ R^\prime\ g\ \eta^
    \prime\ \cfgnt{e}_1\ (\cfgt{pop}\ \eta\ \cfgnt{k}))
   }	
\and
   \inferrule[Pop]{}{
    (L\ R\ g\ \eta\ (\cfgt{pop}\ \eta_0\ \cfgnt{k})) \rightarrow
    (L\ R\ g\ \eta_0\ \cfgnt{k})
   }
\end{mathpar}
\end{center}
\caption{Uber-lazy state reductions}
\label{fig:expr:red}
\end{figure*}

%% Expression syntax




% Algorithm2e environment
% http://en.wikibooks.org/wiki/LaTeX/Algorithms#Typesetting_using_the_algorithm2e_package
\begin{comment}
\begin{algorithm}
 \SetAlgoLined
 \KwData{this text}
 \KwResult{how to write algorithm with \LaTeX2e }
 initialization\;
 \While{not at end of this document $\wedge x < 2$}{
  read current\;
  \eIf{understand}{
   go to next section\;
   current section becomes this one\;
   }{
   go back to the beginning of current section\;
  }
 }
 \caption{How to write algorithms}
\end{algorithm}
\end{comment}
