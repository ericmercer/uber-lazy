\section{Pseudo-code}


The presentation assumes a purely syntactic semantics for a CEKS
machine for a Java-like language \cite{saints-MS}. The machine state
for the semantics consists of a (C)ontrol string representing the
expression being evaluated, an (E)nvironment for local variables, a
(K)ontinuation for what is to be executed next, and a (S)tore for the
heap. This paper only defines salient features of the machine relevant to understanding the new algorithm.
\begin{itemize}
\item $n$ are numbers, and the set of all positive natural numbers is $\mathbb{N}^+$.
\item $x$ is a variable, and the set of all variables is $X$.
\item $f$ is a field name in an object, and the set of all field names is $F$.
\item $v$ is a value, and the set of all values is $V$. A value can only be a reference. 
\item $\eta : X \rightarrow V$ is an environment
\item $h$ is a heap represented as a bipartite graph (see below).
\item $k$ is a continuation to resolve evaluation order and return points from calls. Of particular interest are the continuations for completion (\texttt{ret}), ...
\item $e$ is an expression. Expressions can be values $v$, sequencing
  ($(begin\ e\ \ldots)$), field access ($(x.f)$ or $(x.f := e)$), and
  other constructs to be defined soon,...
\item $\mu$ is the text that defines classes and methods.
\item $P$ is a program $(\mu, (C m))$ where $C$ is a class defined in
  $\mu$ and $m$ is a method in that class.
\item $s = (\mu\ h\ \eta\ e\ k)$ is a machine state.
\item $(\mu\ h^\prime\ \eta^\prime\ e^\prime\ k^\prime) =
  \mathrm{execute}(\mu, h, \eta, e, k)$ is a function that returns a
  state by evaluating the expression.
\end{itemize}

The heap is a bipartite graph defined on locations $L$ and references
$R$. $R$ and $L$ include the special symbol $\bot$ to indicate an
uninitialized reference and location respectively. Additionally, $L$
includes a special location $\mathtt{NULL}$ for null references.

Locations are boxes in the graphical representation and indicated with
the letter $l$ in the math. References are circles in the graphical
representation and indicated with the letter $r$ in the math. Edges
from locations are labeled with field names $f \in F$. Edges from the
references are labeled with constraints $\phi \in \Phi$ (we assume
$\Phi$ is a power set over individual constraints and $phi$ is a set
of constraints for the edge).

The heap is now defined
as $h = (L\ R\ \mathit{ref}\ \mathit{loc})$ where
\begin{itemize}
\item $\mathit{ref} : L \times F \mapsto R$ is the
  connection between locations and references on a named field. The
  function can also be viewed as a set $\mathit{ref} \subseteq L
  \times F \times R$ in a set if that is preferred to a function.
\item $\mathit{loc} : R \mapsto 2^{L \times \Phi}$ is the connection
  between a reference and a location on a constraint. The function
  $\mathit{loc}$ can also be viewed as a set $\mathit{loc} \subseteq R
  \times \Phi \times L$.
\end{itemize}


\begin{figure*}[t]
\begin{center}
\mprset{flushleft}
\begin{mathpar}
	\inferrule[Variable lookup]{}{
      (L\ R\ \eta\ \mathrm{x}\ k) \rightarrow (L\ R\ \eta\ \eta(\mathrm{x})\ k)
	}
\and
	\inferrule[Field Access(eval)]{}{
      (L\ R\ \eta\ (e\ \$\ \mathrm{f})\ k) \rightarrow (L\ R\ \eta\ e\ (*\ \$\ \mathrm{f} \rightarrow k))
	}
\and
	\inferrule[Field Access (NULL)]{
      L(r) = \emptyset
    }{
      (L\ R\ \eta\ r\ (*\ \$\ \mathrm{f} \rightarrow k)) \rightarrow 
      (L[r \mapsto \{(\bot,\top)\}]\ R\ \eta\ r\ (*\ \$\ \mathrm{f} \rightarrow k))
	}
\and
	\inferrule[Field Access (non-NULL)]{
      L(r) = \emptyset \\
      \mathrm{type}(r) = \mathrm{C}_r\\
      \mathrm{fresh}_l(\mathrm{C}_r) = l \\\\
      R^\prime = R[\forall \mathrm{f} \in \mathrm{C}\ ((l\ \mathrm{f}) \mapsto \mathit{fresh_r}(\mathit{type}(\mathrm{f})))]
    }{
      (L\ R\ \eta\ r\ (*\ \$\ \mathrm{f} \rightarrow k)) \rightarrow 
      (L[r \mapsto \{(l,\top)\}]\ R^\prime\ \eta\ r\ (*\ \$\ \mathrm{f} \rightarrow k))
	}
\and
	\inferrule[Field Access]{
      h(r) \neq \emptyset \\
      \mathrm{type}(r) = C \\
      \mathrm{fresh}_r(C) = r_f \\\\
      Q = \{(r^\prime\ \phi)\ |\ (l\ \phi) \in L(r) \wedge r^\prime \in R(l\ \mathrm{f})\} \\\\
      L_f = L[r_f \mapsto \cup_{(r^\prime\ \phi)\in Q} \{(l\ \phi^\prime\phi)\ |\ (l\ \phi^\prime) \in L(r^\prime)\}]
    }{
      (L\ R\ \eta\ r\ (*\ \$\ \mathrm{f} \rightarrow k)) \rightarrow 
      (L_f\ R\ \eta\ r_f\ k)
	}
\end{mathpar}
\end{center}
\caption{Uber-lazy state reductions}
\label{fig:expr:red}
\end{figure*}


%% Expression syntax

\begin{comment}
\begin{figure}[ht]
\begin{center}
\cfgstart

\cfgrule{estate}{($\sigma$ $\eta$ \cfgnt{e} \cfgnt{k})}

\cfgrule{$\sigma$}{$\emptyset$ \cfgor \lp $\sigma$ \cfgq{[}
  \cfgnt{address} \cfgq{$\mapsto$} \cfgnt{v} \cfgq{]}\rp }

\cfgrule{$\eta$}{$\emptyset$ \cfgor \lp $\eta$ \cfgq{[} \cfgnt{id}
  \cfgq{$\mapsto$} \cfgnt{v} \cfgq{]}\rp }

\cfgrule{ae}{\cfgnt{v} \cfgor \cfgnt{id}}	
\cfgrule{e}{\cfgnt{ae} \cfgor \lp\cfgq{@} \cfgnt{e}\rp}
	\cfgorline{\lp\cfgnt{setop} \lp\cfgnt{pattern} \cfgq{in} \cfgnt{e}\rp ~\cfgnt{e}\rp}
	\cfgorline{\lp\cfgq{if} \cfgnt{e} \cfgnt{e} \cfgnt{e}\rp}
	\cfgorline{\lp\cfgq{let} \lp\cfgq{[}\cfgnt{id} \cfgnt{e}\cfgq{]}\rp ~\cfgnt{e}\rp}
	\cfgorline{\lp \cfgnt{op} \cfgnt{e} \cfgnt{e} ...\rp}
\cfgrule{v}{\cfgt{number}}
	\cfgorline{\cfgq{true} \cfgor \cfgq{false}}
	\cfgorline{\cfgq{error}}
	\cfgorline{\cfgt{string}}
	\cfgorline{\lp\cfgq{addr} \cfgnt{address}\rp}
	\cfgorline{\lp\cfgq{const-set} \cfgnt{v} ...\rp}
	\cfgorline{\lp\cfgq{const-tuple} \cfgnt{v} ...\rp}
\cfgrule{pattern}{\cfgnt{id} \cfgor \lp\cfgq{tuple} \cfgnt{id} ...\rp}
\cfgrule{op}{\cfgnt{binop} \cfgor \cfgnt{unaop} \cfgor \cfgq{set} \cfgor \cfgq{tuple}}
\cfgrule{setop}{\cfgq{setFilter} \cfgor \cfgq{setBuild}}

\cfgrule{k}{\cfgq{ret}}
	\cfgorline{\lp\cfgq{@} \cfgq{*} \cfgq{->} \cfgnt{k}\rp}
	\cfgorline{\lp\cfgnt{setop} \lp\cfgnt{pattern} \cfgq{in} \cfgq{*}\rp ~\cfgnt{e} \cfgq{->} \cfgnt{k}\rp}
	\cfgorline{\lp\cfgq{if} \cfgq{*} \cfgnt{e} \cfgnt{e} \cfgq{->} \cfgnt{k}\rp}
	\cfgorline{\lp\cfgq{pop} \cfgnt{$\eta$} \cfgnt{k}\rp}
	\cfgorline{\lp\cfgq{let} \lp\cfgq{[}\cfgnt{id} \cfgq{*}\cfgq{]}\rp ~\cfgnt{e} \cfgq{->} \cfgnt{k}\rp}
	\cfgorline{\lp \cfgnt{op} \lp\cfgnt{v} ...\rp \cfgq{*}
          \lp\cfgnt{e} ...\rp  \cfgq{->} \cfgnt{k}\rp}

\cfgend
\end{center}
\caption{Expression Machine Syntax}
\label{fig:expr:stx}
\end{figure}
\end{comment}





% Algorithm2e environment
% http://en.wikibooks.org/wiki/LaTeX/Algorithms#Typesetting_using_the_algorithm2e_package
\begin{comment}
\begin{algorithm}
 \SetAlgoLined
 \KwData{this text}
 \KwResult{how to write algorithm with \LaTeX2e }
 initialization\;
 \While{not at end of this document $\wedge x < 2$}{
  read current\;
  \eIf{understand}{
   go to next section\;
   current section becomes this one\;
   }{
   go back to the beginning of current section\;
  }
 }
 \caption{How to write algorithms}
\end{algorithm}
\end{comment}
