\section{Correctness}
\label{sec:bisim}
%A major drawback of previous methods for creating test inputs for
%symbolic heaps has been a lack of completeness. Incomplete heap
%representations may not create a set of inputs that represents
%the entire space of possible inputs. By missing some possible
%inputs, the subsequent analysis may fail to recognize 
%possible program behaviors. 

The theorems in this section establish the soundness and
completeness of the symbolic heap approach with respect to Generalized
Symbolic Execution (\gsetxt{}). Intuitively, the theorems imply that any properties proven
with~\gsetxt{} can also be proved using the symbolic heap
algorithm. What follows is a brief description of the requisite terminology,
followed by the theorem statements. The proofs for the theorems, as well as
the complete set of semantic rules for~\gsetxt{} and~\symtxt{} can
be found in ~\cite{Hillery:2015}. 

The theorems assert the existence of a
bisimulation between sets of states related by~\gsetxt{} ($p \rgse
p^\prime$) and states related by the symbolic heap and update rules in this paper ($q \rsym
q^\prime$). The relations are on the universe of \emph{well-formed}
states $S$, which have the properties in \secref{sec:sh} with the
constraint that the states are \emph{feasible}: successors exist
unless at \cfgt{end}. Let $p \rgse p^\prime$ be a union over relations
for~\gsetxt{} (see~\cite{Hillery:2015}): $\rgse = \rgse^\mathit{A} \cup
\rgse^\mathit{A^\prime} \cup \rgse^\mathit{W} \cup
\rgse^\mathit{W^\prime} \cup \rgse^\mathit{E} \cup
\rgse^\mathit{E^\prime} \cup \rcom$, where \emph{A} is a field access
after evaluating the expression for the base reference $\lp
\cfgt{*}\ \cfgt{\$}\ \cfgnt{f} \rightarrow \cfgnt{k}\rp$, \emph{W} is
a field write after evaluating the expression for the right operand
$\lp \cfgnt{x}\ \cfgt{\$}\ \cfgnt{f}\ \cfgt{:=}\ \cfgnt{*} \rightarrow
\cfgnt{k}\rp$, \emph{E} is a reference compare after evaluating the
left and right operands $\lp \cfgnt{v}\ \cfgt{=}\ \cfgnt{*}
\rightarrow \cfgnt{k}\rp$, the prime symbol indicates a null reference
in the operation or a false outcome, and \emph{\com} is everything
else in the language. Any state relation, say $\rightarrow_x$, is
trivially extended to sets of states as
$$
P \hookrightarrow_x P^\prime \Longleftrightarrow \forall p \in P\ (\forall p^\prime\ (p \rightarrow_x p^\prime \Leftrightarrow p^\prime \in P^\prime))
$$
Let $\hookrightarrow_\gse$ be the extension of $\rgse$ to sets of
states. From this extension, a new meta transition relation is defined
over sets of states as
$$
\rsgse = \hookrightarrow_\gse^\mathit{A}
\cup \hookrightarrow_\gse^\mathit{A^\prime} \cup \hookrightarrow_\gse^\mathit{W} \cup
\hookrightarrow_\gse^\mathit{W^\prime} \cup \hookrightarrow_\gse^\mathit{E} \cup \hookrightarrow_\gse^\mathit{E^\prime}
\cup \hookrightarrow_\com
$$
The relation captures the notion of splitting groups of heaps at
certain operations. For example, suppose a set $P$ contains a single state $p$
with all references uninitialized. If $p$ is a field access state, it has two
potential successors in~\gsetxt{}: a non-null reference and a null reference. 
Thus, the $\rsgse$ relation has two successors and divides $P$ into the two outcomes.

The functional equivalence between heaps in states $p$
and $q$ requires both a mapping to relate the two heaps and a
constraint on the feasibility of that mapping in the presence of a
path constraint from symbolic execution. Subscripts indicate state tuple
members as in $p = (
\cfgnt{L}_p\ \cfgnt{R}_p\ \phi_p\ \eta_p\ \cfgnt{e}_p\ \cfgnt{k}_p )$.
\begin{definition}
\label{def:homomorphism}
A \textbf{homomorphism}, given the universe of field indices $\mathcal{F}$ and the universe of locations $\mathcal{L}$, is 
$$
\begin{array}{l}
 s_p \rightharpoonup_{h} s_q \Leftrightarrow 
\exists h: \mathcal{L} \mapsto \mathcal{L}\ (\forall \cfgnt{l}_\alpha\ (\forall \cfgnt{l}_\beta\ (\forall f \in \mathcal{F}\ ( \forall \phi_\alpha\ (\\ 
\ \ \ \ \ \ \ \ (\phi_\alpha\ \cfgnt{l}_\alpha) \in \cfgnt{L}_p(\cfgnt{R}_p (\cfgnt{l}_\beta,f )) \Rightarrow \\
\ \ \ \ \ \ \ \ \exists \phi_\beta\ ( (\phi_\beta\ h(\cfgnt{l}_\alpha))\in \cfgnt{L}_q(\cfgnt{R}_q (h(\cfgnt{l}_\beta),f ))\ 
 )) ) ) ) )
\end{array}
$$
\end{definition}
\begin{definition}
\label{def:hc}
The \textbf{homomorphism constraint} is
%\begin{align*}
\[
\mathbb{HC}(p \rightharpoonup_{h} q) = 
\bigwedge \{ \phi_b\ | \exists (\phi_a\ l) \in \cfgnt{L}_p^\rightarrow ( (\phi_b\ h(l)) \in \cfgnt{L}_q^\rightarrow)\}
\]
%\end{align*}
\end{definition}
Functional equivalence asserts a common structure in the two heaps under certain
conditions. It is used to relate states in $p \rgse
p^\prime$ to states in $q \rsym q^\prime$.
\begin{definition}
\label{representation}
States $(p\ q)$ are in the \textbf{representation relation}, $p \sqsubset q$, if and only if, $\eta_p = \eta_q ,\ \cfgnt{e}_p =
\cfgnt{e}_q ,\ \cfgnt{k}_p = \cfgnt{k}_q$, and there exists a
homomorphism $p \rightharpoonup_{h} q$
such that $\mathbb{S}( \phi_q \wedge \mathbb{HC}(s_p \rightharpoonup_{h} s_q) )$.
The represented relation is extended to sets of states $P$ and a single state $q$ as
$P \sqsubset q \Longleftrightarrow \forall p\ (p \sqsubset q \Leftrightarrow p \in P)$.
\end{definition}
The statement $p \sqsubset q$ ensures that a functionally equivalent
heap to the one in $p$ is present, by the homomorphism, and valid, by
the heap constraint and path constraint, in $q$. As the states in $P$
are only differentiated by heaps and those states are only
differentiated from $q$ by both the heap and path constraint in $q$,
$P \sqsubset q$ implies that $q$ is representative of all the states
in $P$ up to the given point of execution expressed in $\phi_q$.
\begin{definition}
\label{bisimulation}
The \textbf{functional associated to bisimulation} applied to $\sqsubset$, denoted as $F_\sim(\sqsubset)$, is the set of all pairs
$(P\ q)$ such that
\begin{equation}
\label{eqn:BisimulationForwards}
\forall P^\prime\ ( P \rsgse P^\prime \Rightarrow \exists q^\prime\ ( q \rsym q^\prime \wedge P^\prime\ \sqsubset\ q^\prime))
\end{equation}
\begin{equation}
\label{eqn:BisimulationBackwards}
\forall q^\prime\ ( q \rsym q^\prime\Rightarrow \exists P^\prime\ ( P \rsgse P^\prime \wedge P^\prime\ \sqsubset\ q^\prime))
\end{equation}
If $\sqsubset$ is a bisimulation, then the greatest fixed point of $F_\sim(\sqsubset)$ is the bisimilarity relation denoted by $\sim$.
\end{definition}
Although the functional reasons over a forward and backward
simulation as typical~ \cite{Sangiorgi:2011}, its use of a meta-relation, $\rsgse$, is unique. 

In the following lemma, $S$ is the universe of well-formed states, and $S_{FA} \subseteq S$ is the set of states at a field access continuation having computed the base reference.
\begin{lemma}[\textrm{F{\footnotesize IELD}} \textrm{A{\footnotesize CCESS}} preserves $\sqsubset\ \subseteq F_\sim(\sqsubset)$]
If $P \in 2^{S_\mathit{FA}}$ and $q \in S$ are such that $P \sqsubset q$, then $(P\ q)$ is in the functional associated to bisimulation.
\label{lem:access}
$$
\forall P \in 2^{S_\mathit{FA}}\ (P \sqsubset q \Rightarrow (P\ q) \in F_\sim(\sqsubset))
$$
\end{lemma}

%\begin{proof}
%Proof by contradiction: assume $P \sqsubset q \wedge (P\ q) \not\in F_\sim(\sqsubset)$.
%
%\noindent\textbf{Sketch}: Choose any $P$ and $q$ at the field access continuation such
%that $P \sqsubset q$. In the forward simulation
%\eqref{eqn:BisimulationForwards}, for each $P^\prime$ such that $P
%\rsgse P^\prime$, pick some $p^\prime \in P^\prime$. By definition $p \rgse
%p^\prime$, $p \in P$, and $p \sqsubset q$. The proof shows the
%existence of $q^\prime$ such that $q \rsym q^\prime$ and $p^\prime
%\sqsubset q^\prime$ using the existing homomorphism in $p \sqsubset q$.
%
%In the backward simulation \eqref{eqn:BisimulationBackwards}, for each
%$q^\prime$ such that $q \rsym q^\prime$, the proof posits the existence of some 
%$p^\prime$ such that $p^\prime \sqsubset q^\prime$. It then uses the
%homomorphism in $p^\prime \sqsubset q^\prime$ to derive a $p$ such
%that $p \rgse p^\prime$ and $p \in P$. The existence of $P^\prime$ is established by the existence of $p^\prime$, and $P \sqsubset q$ makes $P^\prime$ the actual successor of $P$.
%
%As the forward and backward simulations hold for any $p \in P$, $P
%\sqsubset q$ must be in the functional, $F_\sim(\sqsubset)$, which is
%a contradiction.
%\end{proof}

Similar lemmas are proved for field write and equals reference. These
require additional lemmas on the summarize machine
$\rightarrow_S$: that it preserves determinism, the
homomorphism, and the satisfiability of the homomorphism constraint

\begin{theorem}
\label{th:bisim}
The relation $\sqsubset$ is a bisimulation: $\sqsubset\ \subseteq\ \sim$
\end{theorem}
%\begin{proof}
%By definition of bisimilarity. All of the common rules in $\rcom$
%make no changes to the heap, so $P \sqsubset q$ is included in the
%functional. The other rules that affect the heap are supported by lemmas
%such as \lemref{lem:access}.
%\end{proof}

%The notation $p \stackrel{n}{\rgse} p^\prime$ denotes that $p^\prime$
%is the $n$-step successor of $p$, when it exists, and $q
%\stackrel{n}{\rsym} q^\prime$ is similarly defined. Completeness and
%soundness fall out of \thref{th:bisim} by inducting
%over $n$. 

\begin{corollary}[$\rsym$ is complete]
If $P \in 2^{S_\mathit{A}}$ and $q \in S$ are such that $P \sqsubset q$ then for any $p \in P$
$\forall p^\prime\ (p \stackrel{n}{\rgse} p^\prime \Rightarrow \exists q^\prime\ (q \stackrel{n}{\rsym} q^\prime \wedge p^\prime \sqsubset q^\prime))$
\end{corollary}

\begin{corollary}[$\rsym$ is sound]
If $P \in 2^{S_\mathit{A}}$ and $q \in S$ are such that $P \sqsubset q$ then
$\forall q^\prime\ (q \stackrel{n}{\rsym} q^\prime \Rightarrow \exists p \in P\ (\exists p^\prime\ (p \stackrel{n}{\rgse} p^\prime \wedge p^\prime \sqsubset q^\prime)))$
\end{corollary}

%The relation, $P \sqsubset q$, is readily established in the initial
%state of a given program as there is single initial state, $P_o =
%\{p_o\}$, in any valid program. The initial state for $\rsym$ is
%then defined as $q_o = p_o$ so $P_o \sqsubset q_o$ trivially holds.
