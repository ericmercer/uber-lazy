\subsection{Heap Properties}


\begin{definition}
A heap, $(\cfgnt{L}\ \cfgnt{R})$, is \textbf{deterministic} if and only if 
\[
\begin{array}{l}
\forall \cfgnt{r} \in \cfgnt{L}^\leftarrow\ (\forall (\phi\ \cfgnt{l}),(\phi^\prime\ \cfgnt{l}^\prime) \in \cfgnt{L}(r)\ (\\
\ \ \ \ (\cfgnt{l} \neq \cfgnt{l}^\prime \vee \phi \neq \phi^\prime) \Rightarrow (\phi \wedge \phi^\prime = \cfgt{false}))
\end{array}
\]
\end{definition}

\begin{definition}
A state is \textbf{well formed} if its heap is deterministic, type consistent across locations on a common reference, every reference points to a location, and only references from the stack partition are in the environment, expression, or continuation.
\end{definition}

\begin{lemma}[Successors of well-formed states]
\label{lem:succ}
Any well-formed state has at least one successor unless the expression is $\cfgnt{v}$ and the continuation is $\cfgt{end}$.
\end{lemma}
\begin{proof}
Please write me!
\end{proof}

At times, it is useful to classify states in terms of patterns that the state strings match. In concrete terms, this is similar to asking what will be the next instruction to execute. For example, we know that left-hand states matching the pattern $ \lp \cfgnt{L}\ \cfgnt{R}\ \phi_g\ \eta\ \cfgnt{r}\ \lp \cfgt{*}\ \cfgt{\$}\ \cfgnt{f} \rightarrow \cfgnt{k}\rp \rp$ only appear in the Field Access rule.

\begin{definition}
The universe of reachable states is $S$. The universe is further partitioned into states that activate different relations.
\begin{itemize}
\item Field Access, $S_\mathit{FA} = \{\lp \cfgnt{L}\ \cfgnt{R}\ \phi\ \eta\ \cfgnt{e}\ \cfgnt{k} \rp \in S \mid \exists \cfgnt{r}\ (\cfgnt{e} = \cfgnt{r}) \wedge \exists \cfgnt{f},\cfgnt{k}_0\ (\cfgnt{k} = \lp \cfgt{*}\ \cfgt{\$}\ \cfgnt{f} \rightarrow \cfgnt{k}_0\rp) \}$
\item Field Write,  $S_\mathit{FW} = \{\lp \cfgnt{L}\ \cfgnt{R}\ \phi\ \eta\ \cfgnt{e}\ \cfgnt{k} \rp \in S \mid \exists \cfgnt{r}\ (\cfgnt{e} = \cfgnt{r}) \wedge \exists \cfgnt{x},\cfgnt{f},\cfgnt{k}_0\ (\cfgnt{k} = \lp \cfgnt{x}\ \cfgt{\$}\ \cfgnt{f}\ \cfgt{:=}\ \cfgt{*}\ \rightarrow\ \cfgnt{k}_0\rp) \}$
\item Equals, $S_\mathit{EQ} = \{\lp \cfgnt{L}\ \cfgnt{R}\ \phi\ \eta\ \cfgnt{e}\ \cfgnt{k} \rp \in S \mid \exists \cfgnt{r}_0 (\cfgnt{e} = \cfgnt{r}_0) \wedge \exists r_1, k_1(\cfgnt{k} = \lp \cfgnt{r}_1\; \cfgt{=}\; \cfgt{*} \rightarrow \cfgnt{k}_1\rp \}$
\item New, $S_\mathit{N} = \{\lp \cfgnt{L}\ \cfgnt{R}\ \phi\ \eta\ \cfgnt{e}\ \cfgnt{k} \rp \in S \mid \exists \cfgnt{C}\ (\cfgnt{e} = \lp \cfgt{new}\ \cfgnt{C}\rp) \}$
\end{itemize} 
\end{definition}


\begin{definition}
Given a sequence of states $$\Pi_n = s_0,s_1,...,s_n$$ where $$s_i = (\cfgnt{L}_i\ \cfgnt{R}_i\ \phi_i\ \eta_i\ \cfgnt{e}_i\ \cfgnt{k}_i )$$ the \textbf{control flow sequence} of $\Pi_n$ is the defined as the sequence of tuples $$ \pi_n = \mathbb{CF}(\Pi_n) = (\eta_0\ \cfgnt{e}_0\ \cfgnt{k}_0),(\eta_1\ \cfgnt{e}_1\ \cfgnt{k}_1),...,(\eta_n\ \cfgnt{e}_n\ \cfgnt{k}_n)$$
\end{definition}

We will later be concerned with establishing whether one state
represents another state. We want to say that one state represents
another state if equivalent paths lead out from each state. This
path-centric notion of equivalence is known as functional
equivalence. In establishing functional equivalence between states, it
is important to determine whether the heaps within the states are
themselves functionally equivalent. Two heaps are functionally
equivalent if the same sequence of field accesses in each heap
produces equivalent results. We define heap functional equivalence
using a co-inductive definition of homomorphism over the access paths
in the heaps.

\begin{definition}
\label{def:homomorphism}
A \textbf{homomorphism} $s_p \rightharpoonup_{h} s_q$, from state $s_p = ( \cfgnt{L}_p\ \cfgnt{R}_p\ \phi_p\ \eta_p\ \cfgnt{e}_p\ \cfgnt{k}_p )$ to state $s_q = ( \cfgnt{L}_q\ \cfgnt{R}_q\ \phi_q\ \eta_q\ \cfgnt{e}_q\ \cfgnt{k}_q )$, is defined as follows: 
$$
\begin{array}{l}
 s_p \rightharpoonup_{h} s_q \Leftrightarrow \\
\ \ \ \ \exists h: \mathcal{L} \mapsto \mathcal{L}\ (\forall \cfgnt{l}_\alpha\ (\forall \cfgnt{l}_\beta\ (\forall f \in \mathcal{F}( \exists \phi_\alpha\ (\exists \phi_\beta\ ( \\ 
\ \ \ \ \ \ \ \ \ \ \ \ (\phi_\alpha\ \cfgnt{l}_\alpha) \in \cfgnt{L}_p(\cfgnt{R}_p (\cfgnt{l}_\beta,f )) \Rightarrow (\phi_\beta\ h(\cfgnt{l}_\alpha))\in \cfgnt{L}_q(\cfgnt{R}_q (h(\cfgnt{l}_\beta),f ))\ \\
\ \ \ \ \ \ \ \ \ \ \ \ \ \ \ \  )) ) ) ) )
\end{array}
$$
\begin{comment}
$$
\begin{array}{l}
s_p \rightharpoonup_{h} s_q \Leftrightarrow \\
\ \ \ \ \exists h: \mathcal{L} \mapsto \mathcal{L}\ (\forall \cfgnt{l}_\beta\ (\forall f \in \mathit{fields}(\mathrm{type}(\cfgnt{l}_\beta))\ ( \\
\ \ \ \ \ \ \ \ \ \ \ \ \forall (\phi_\alpha\ \cfgnt{l}_\alpha) \in \cfgnt{L}_p(\cfgnt{R}_p (\cfgnt{l}_\beta,f ))\ ( \\
\ \ \ \ \ \ \ \ \ \ \ \ \ \ \ \ \exists \phi_\beta\ ((\phi_\beta\ h(\cfgnt{l}_\alpha))\in \cfgnt{L}_q(\cfgnt{R}_q (h(\cfgnt{l}_\beta),f )))))))\\
\end{array}
$$
\end{comment}
\end{definition}

Since the access paths in any given heap are bound by certain constraints, to preserve control flow equivalence we must establish whether the collection of any constraints in a given heap are collectively feasible. The homomorphism constraint is the conjunction of all constraints in the image of the represented heap in the representer heap.

\begin{definition}
\label{def:hc}
Given the homomorphism $s_p \rightharpoonup_{h} s_q$, the \textbf{homomorphism constraint} $\mathbb{HC}(s_p \rightharpoonup_{h} s_q)$ is defined as:
\begin{align*}
\mathbb{HC}(s_p \rightharpoonup_{h} s_q) &= \\
 \bigwedge \{ \phi_b\ | \exists (\phi_a\ l) \in \cfgnt{L}_p^\rightarrow ( (\phi_b\ h(l)) \in \cfgnt{L}_q^\rightarrow)\} 
\end{align*}
Where $\cfgnt{L}_p$ and $\cfgnt{L}_q$ are the locations maps from the heaps in $s_p$ and $s_q$.
\end{definition}

The representation relation combines the previously established
notions of heap homomorphism and feasibility with the added constraint
that the variables, expressions, and continuation strings must match
between the pairs of states.

\begin{definition}
\label{representation}
The \textbf{representation relation} $\sqsubset$ is defined as
follows: given state $s_p = (
\cfgnt{L}_p\ \cfgnt{R}_p\ \phi_p\ \eta_p\ \cfgnt{e}_p\ \cfgnt{k}_p )$
and state $s_q = (
\cfgnt{L}_q\ \cfgnt{R}_q\ \phi_q\ \eta_q\ \cfgnt{e}_q\ \cfgnt{k}_q )$,
$s_p \sqsubset s_q $ if and only if $\eta_p = \eta_q ,\ \cfgnt{e}_p =
\cfgnt{e}_q ,\ \cfgnt{k}_p = \cfgnt{k}_q$, the heaps are well-formed, and there exists a
homomorphism $(\cfgnt{L}_p\ \cfgnt{R}_p) \rightharpoonup_{h}
(\cfgnt{L}_q\ \cfgnt{R}_q)$ such that
\begin{equation}
\label{eqn:valid}
 \mathbb{S}( \phi_q \wedge \mathbb{HC}(s_p \rightharpoonup_{h} s_q) ) 
\end{equation}
The represented relation is extended to sets of states $P$ and $Q$ as
$$
P \sqsubset Q \Longleftrightarrow \forall q \in Q\ (\forall p\ (p \sqsubset q \Rightarrow p \in P))
$$
\end{definition}

\begin{lemma}[$Q$ is a singleton for $P \sqsubset Q$]
\label{lem:unique}
$$
P \sqsubset Q \Rightarrow \forall q,q^\prime \in Q\ (\forall p\ (p \sqsubset q \Rightarrow p \sqsubset q^\prime)
$$
The notation $P \sqsubset q$ is used to indicate that single member of
the set in the represented relation.
\end{lemma}
\begin{proof}
Please write me?
\end{proof}
\begin{definition}
\label{def:meta}
A state relation $p \rightarrow_x p^\prime$ is extended to sets of states $P \hookrightarrow_x P^\prime$ as
$$
P \hookrightarrow_x P^\prime \Longleftrightarrow \forall p \in P\ (\forall p^\prime\ (p \rightarrow_x p^\prime \Rightarrow p^\prime \in P^\prime))
$$
The $\rgse$ relation over sets of states is $\rsgse = \hookrightarrow_\mathit{FA} \cup \hookrightarrow_\mathit{FA(N)} \cup \hookrightarrow_\mathit{FW} \cup \hookrightarrow_\mathit{FW(N)} \cup \hookrightarrow_\mathit{EQ(T)} \cup \hookrightarrow_\mathit{EQ(F)} \cup \hookrightarrow_\com$.%, and 
%the $\rsym$ relation extended to sets of states is 
%$$
%\rssym = \hookrightarrow_\mathit{FA} \cup \hookrightarrow_\mathit{FW} \cup \hookrightarrow_\mathit{EQ}^T \cup \hookrightarrow_\mathit{EQ}^F \cup \hookrightarrow_\com
%$$
\end{definition}

\begin{lemma}[$P$ and $Q$ have same number of successors when $P \sqsubset q$]
\label{lem:succCount}
Let $X = \{P^\prime | P \leadsto P^\prime\}$ and $Y = \{q^\prime | q
\rsym q^\prime\}$: $P \sqsubset q \Rightarrow |X| = |Y|$.
\end{lemma}
\begin{proof}
Please write me?
\end{proof}

\begin{definition}
\label{bisimulation}
The functional associated to bisimulation,  $F_\sim$, is a function between binary relations. The functional applied to $\sqsubset$, denoted as $F_\sim(\sqsubset)$, is the set of all pairs
$(P\ q)$ such that
\begin{equation}
\label{eqn:BisimulationForwards}
\forall P^\prime\ ( P \rsgse P^\prime \Rightarrow \exists q^\prime\ ( (q \rsym q^\prime )\wedge (P^\prime\ \sqsubset\ q^\prime )))
\end{equation}
\begin{equation}
\label{eqn:BisimulationBackwards}
\forall q^\prime\ ( q \rsym q^\prime\Rightarrow \exists P^\prime\ ( (P \rsgse P^\prime )\wedge (P^\prime\ \sqsubset\ q^\prime )))
\end{equation}
The bisimularity relation is the greatest fixed point of the functional.
\end{definition}

Note that in the literature it is customary to define bisimulation in
terms of a single labeled transition system, whereas for the purposes
of this paper the definition of bisimulation refers to a pair of
transition relations $\rightarrow_x$ and $\rightarrow_y$ defined by
reduction rules. Since it is possible to create a union of the two
rule systems $\rightarrow_x \cup \rightarrow_y$, and since none of the
transitions in the reduction rules in this paper are labeled, this
definition is sufficient for all of the customary properties of
bisimulation to apply. For a more detailed treatment on the
application of bisimulation to reduction rule systems see
\cite{GSE:barbedbisimulation}.

