
\documentclass[pldi]{sigplanconf-pldi15}

\usepackage{amsmath}
\usepackage{comment}
\usepackage{mathpartir}
\usepackage{amssymb}
\usepackage{amsfonts}

\hyphenation{op-tical net-works semi-conduc-tor}

\input{cfg-commands}
\newcommand{\figref}[1]{Figure~\ref{#1}}
\newcommand{\secref}[1]{Section~\ref{#1}}
\newcommand{\thref}[1]{Theorem~\ref{#1}}
\newcommand{\lemref}[1]{Lemma~\ref{#1}}
\newcommand{\defref}[1]{Definition~\ref{#1}}

\newcommand{\sym}{\ensuremath{\varsigma}}
\newcommand{\gse}{\ensuremath{g}}
\newcommand{\rsym}{\ensuremath{\rightarrow_\sym}}
\newcommand{\rssym}{\ensuremath{\leadsto_\sym}}
\newcommand{\rgse}{\ensuremath{\rightarrow_\gse}}
\newcommand{\rsgse}{\ensuremath{\leadsto_\gse}}
\newcommand{\rsum}{\ensuremath{\rightarrow_S}}
\newcommand{\rinit}{\ensuremath{\rightarrow_I}}
\newcommand{\com}{\ensuremath{J}}
\newcommand{\rcom}{\ensuremath{\rightarrow_\com}}

\usepackage{url}
\usepackage{listings}
\usepackage{color}
\usepackage[T1]{fontenc}
%\usepackage{SIunits}            % typset units correctly
\usepackage{courier}            % standard fixed width font
\usepackage[scaled]{helvet} % see www.ctan.org/get/macros/latex/required/psnfss/psnfss2e.pdf

\newcommand{\doi}[1]{doi:~\href{http://dx.doi.org/#1}{\Hurl{#1}}}   % print a hyperlinked DOI

\definecolor{dkgreen}{rgb}{0,0.6,0}
\definecolor{gray}{rgb}{0.5,0.5,0.5}
\definecolor{mauve}{rgb}{0.58,0,0.82}

\lstset{frame=tb,
  language=Java,
  aboveskip=3mm,
  belowskip=3mm,
  showstringspaces=false,
  columns=flexible,
  basicstyle={\small\ttfamily},
  numbers=none,
  numberstyle=\tiny\color{gray},
  keywordstyle=\color{blue},
  commentstyle=\color{dkgreen},
  stringstyle=\color{mauve},
  breaklines=true,
  breakatwhitespace=true
  tabsize=3
}


\usepackage{graphicx}
\usepackage{graphics}
\usepackage{epsfig}
\usepackage{comment}

% used for inline lists
\usepackage{paralist}

%% This enables the xfig overlays to use the same font family as the document
%% (i.e., font family and size is the same in the figure as it is
%% in the text).

\gdef\SetFigFont#1#2#3#4#5{}

\usepackage{multirow}
\usepackage{algorithm} 
\RequirePackage[noend]{algorithmic}
\renewenvironment{algorithm}[1][\textwidth]%  
{\begin{minipage}[t][\totalheight][c]{#1}\begin{algorithmic}[1]}  %%% change [1] to [0] to turn off line numbers
{\end{algorithmic}\end{minipage}}

%% Use for each in ``FOR'' constructs

\renewcommand{\algorithmicfor}{\textbf{for each}}

%% All comments are in italics

\renewcommand{\algorithmiccomment}[1]{\textit{${/\ast}$~#1~${\ast/}$}}

%% Use ``procedure'' instead of ``Algorithm'' for off set

\renewcommand{\algorithmicensure}{\textbf{procedure}}

\newcommand{\algoname}[1]{\ENSURE #1}
\newcommand*{\algobox}[1]{\framebox{#1}}

\newcommand{\bo}[1]{\textbf{#1}}
\newcommand{\cc}[1]{\cellcolor[gray]{.6}{#1}}
\newcommand{\negspace}{\hspace{-.40cm}}

%define theorem environments
\usepackage{mathtools}
\usepackage{amsthm}
\newtheorem{theorem}{Theorem}
\newtheorem{lemma}[theorem]{Lemma}
\newtheorem{proposition}[theorem]{Proposition}
\newtheorem{corollary}[theorem]{Corollary}

\newtheorem{definition}{Definition}
%\newenvironment{definition}[1][Definition]{\begin{trivlist}
%\item[\hskip \labelsep {\bfseries #1}]}{\end{trivlist}}

%% This enables the xfig overlays to use the same font family as the document
%% (i.e., font family and size is the same in the figure as it is
%% in the text).

\gdef\SetFigFont#1#2#3#4#5{}




\begin{document}



%\title{Contraints-based Reasoning of Heaps in Symbolic Execution}

%\title{Using Constraints to Characterize Heaps in Symbolic Execution}

%\title{Symbolic Execution with precise heap constraints}

%\title{Precise Heap Summaries from Symbolic Execution}

\title{Supplemental Document}


\maketitle

\section{Proofs}

\subsection{Definitions}

\begin{definition}
A feasible symbolic heap is defined as...
\end{definition}

\begin{definition}
A subheap is defined as...
\end{definition}

\begin{definition}
A symbolic system state represents a set of lazy system states. To denote the representation relationship, we use the expression $\mathcal{L}\sqsubset \mathcal{S} $ to say that lazy system state $\mathcal{L}:\lp \mu_{\mathcal{L}}\ \cfgnt{L}_{\mathcal{L}}\ \cfgnt{R}_{\mathcal{L}}\ \phi_{\mathcal{L}}\ \eta_{\mathcal{L}}\ \cfgnt{e}_{\mathcal{L}}\ \cfgnt{k}_{\mathcal{L}}\rp$ is represented by symbolic state $\mathcal{S}:\lp \mu_{\mathcal{S}}\ \cfgnt{L}_{\mathcal{S}}\ \cfgnt{R}_{\mathcal{S}}\ \phi_{\mathcal{S}}\ \eta_{\mathcal{S}}\ \cfgnt{e}_{\mathcal{S}}\ \cfgnt{k}_{\mathcal{S}}\rp$ . The \textbf{representation relation} is defined as follows: $\mathcal{L}\sqsubset \mathcal{S} $ if and only if 
$$\eta_{\mathcal{L}} = \eta_{\mathcal{S}} ,\ \cfgnt{e}_{\mathcal{L}} = \cfgnt{e}_{\mathcal{S}} ,\ \cfgnt{k}_{\mathcal{L}} = \cfgnt{k}_{\mathcal{S}}$$
and there exists functions $g:\cfgnt{r} \mapsto \cfgnt{r}$ and $h:\cfgnt{l} \mapsto \cfgnt{l}$ such that for any reference $r \in \mathcal{L}$, location $l \in \mathcal{L}$, and field $f$, $$ l = L_{\mathcal{L}}(r) \leftrightarrow h(l)\in L_{\mathcal{S}}(g(r))$$ and $$ r = R_{\mathcal{L}}(\cfgnt{l},f) \leftrightarrow g(r) = R_{\mathcal{S}}(h(l),f)$$ and the subheap of state $\mathcal{S}$ formed by selecting the image of $g$ and $h$ is feasible.

\end{definition}

\begin{definition}
A symbolic state $\mathcal{S}$ is \textbf{exact} if and only if it represents the set of all feasible lazy states on the current execution path, and represents no infeasible state.
\end{definition}

\subsection{Theorems}

\begin{lemma}
If symbolic state $\mathcal{S}$ is exact prior to executing the Field Access rule, then the state $\mathcal{S}^\prime $ is also exact.
\end{lemma}

\begin{proof}
Suppose we have some symbolic state  $\mathcal{S} =  \lp \cfgnt{L}_{\mathcal{S}}\ \cfgnt{R}_{\mathcal{S}}\ \phi_g\ \eta\ \cfgnt{r}\ \lp \cfgt{*}\ \cfgt{\$}\ \cfgnt{f} \rightarrow \cfgnt{k}\rp \rp $ , and that all relevant fields are initialized. Take an arbitrary lazy state $\mathcal{L} \sqsubset \mathcal{S}$. Since $\mathcal{S}$ is exact,  $\mathcal{L} = \lp \cfgnt{L}_{\mathcal{L}}\ \cfgnt{R}_{\mathcal{L}}\ \phi_L\ \eta\ \cfgnt{r}\ \lp \cfgt{*}\ \cfgt{\$}\ \cfgnt{f} \rightarrow \cfgnt{k} \rp \rp$. If we apply the lazy field access rule to $\mathcal{L}$, we achieve state $\mathcal{L}^\prime = \lp \cfgnt{L}_{\mathcal{L}} [\cfgnt{r}^\prime \mapsto \lp\phi^\prime\ l^\prime\rp]\ \cfgnt{R}_{\mathcal{L}}\ \phi_L\ \eta\ \cfgnt{r}^\prime\ \cfgnt{k}\rp $. Likewise, by applying the symbolic field access rule to $\mathcal{S}$, we obtain state $\mathcal{S}^\prime = \lp \cfgnt{L}_{\mathcal{S}}[\cfgnt{r}^\prime \mapsto \mathbb{VS}\lp \cfgnt{L}_{\mathcal{S}},\cfgnt{R}_{\mathcal{S}},\cfgnt{r},\cfgnt{f},\phi_g\rp ]\ \cfgnt{R}_{\mathcal{S}}\ \phi_g\ \eta\ \cfgnt{r}^\prime\ \cfgnt{k}\rp $.

We now show that state $\mathcal{S}^\prime$ represents state $\mathcal{L}^\prime $. Since $\eta$, $e$, and $k$ are identical between $\mathcal{S}^\prime$ and $\mathcal{L}^\prime $, the first condition is met by default. Constructing the reference and location mappings is only slightly less trivial: $$g^\prime = g$$ $$h^\prime = h[ r^\prime \mapsto r^\prime]$$ 

It only remains to show that the image of $\mathcal{L}^\prime$ in $\mathcal{S}^\prime$ is feasible, WHICH WILL BE EASY ONCE WE DEFINE WHAT THAT MEANS.

Now, we show that no infeasible lazy states are allowed by $\mathcal{S}^\prime$. NEED TO FILL IN THIS BIT, TOO
\end{proof}

\section{Machine Syntax}
\begin{figure}
\begin{center}
\cfgstart
\cfgrule{e}{\lp .... \cfgor \lp\cfgt{raw} \cfgnt{v} \cfgt{@} \cfgnt{m} \cfgnt{v} \rp\rp}
%    ;; eval syntax
%  (object (C [f loc] ...))
%  (hv v
%      object)
\cfgrule{$\phi$}{\cfgnt{constraint}}
\cfgrule{l}{\cfgt{number}}
%  (h mt
%     (h [loc -> hv]))
%  (η mt
%     (η [x -> loc]))
\cfgrule{s}{\lp$\mu$ \cfgnt{L} \cfgnt{R} $\eta$ \cfgnt{e} \cfgnt{k}\rp}
\cfgrule{k}{\cfgt{end}}
\cfgorline{\lp \cfgt{*} \cfgt{\$} \cfgnt{f} $\rightarrow$ \cfgnt{k}\rp}
%     (* @ m (e ...) -> k)
%     (v @ m (v ...) * (e ...) -> k)
%     (* == e -> k)
%     (v == * -> k)
%     (x := * -> k)
%     (x $ f := * -> k)
%     (if * e else e -> k)
%     (var T x := * in e -> k)
%     (begin * (e ...) -> k)
%     (pop η k))
\cfgend
\end{center}
\caption{The machine syntax for Javalite.}
\label{fig:machine-syntax}
\end{figure}


\section{Common Rules for Javalite}

\begin{figure*}[t]
\begin{center}
\mprset{flushleft}
\begin{mathpar}
	\inferrule[Variable lookup]{}{
      \lp \cfgnt{L}\ \cfgnt{R}\ \phi\ \eta\ \cfgnt{x}\ \cfgnt{k}\rp  \rightarrow_J \\\\
      \lp \cfgnt{L}\ \cfgnt{R}\ \phi\ \eta\ \eta\lp \cfgnt{x}\rp \ \cfgnt{k}\rp 
	}
\and
	\inferrule[New]{
      \cfgnt{r} = \mathrm{stack}_r\lp \rp\\
      l = \mathrm{fresh}_l\lp \cfgnt{C}\rp\\\\
      \cfgnt{R}^\prime = \cfgnt{R}[\forall \cfgnt{f} \in \mathit{fields}\lp \mathrm{C}\rp \ \lp \lp l\ \cfgnt{f}\rp  \mapsto \cfgnt{r}_\mathit{null} \rp ] \\\\
      \cfgnt{L}^\prime = \cfgnt{L}[\cfgnt{r} \mapsto \{\lp \cfgt{true}\ l\rp \}]
    }{
      \lp \cfgnt{L}\ \cfgnt{R}\ \phi\ \eta\ \lp \cfgt{new}\ \cfgnt{C}\rp \ \cfgnt{k}\rp  \rightarrow_J
      \lp \cfgnt{L}^\prime\ \cfgnt{R}^\prime\ \phi\ \eta\ \cfgnt{r}\ \cfgnt{k}\rp 
	}
\and
	\inferrule[Field Access(eval)]{}{
      \lp \cfgnt{L}\ \cfgnt{R}\ \phi\ \eta\ \lp \cfgnt{e}\ \cfgt{\$}\ \cfgnt{f}\rp \ \cfgnt{k}\rp  \rightarrow_J \\\\
      \lp \cfgnt{L}\ \cfgnt{R}\ \phi\ \eta\ \cfgnt{e}\ \lp \cfgt{*}\ \cfgt{\$}\ \cfgnt{f} \rightarrow \cfgnt{k}\rp \rp 
	}
\and
	\inferrule[Field Write (eval)]{}{
       \lp \cfgnt{L}\ \cfgnt{R}\ \phi\ \eta\ \lp \cfgnt{x}\ \cfgt{\$}\ \cfgnt{f}\ \cfgt{:=}\ \cfgnt{e}\rp \ \cfgnt{k}\rp  \rightarrow_J \\\\
       \lp \cfgnt{L}\ \cfgnt{R}\ \phi\ \eta\ \cfgnt{e}\ \lp \cfgnt{x}\ \cfgt{\$}\ \cfgnt{f}\ \cfgt{:=}\ \cfgt{*}\ \rightarrow\ \cfgnt{k}\rp \rp 
	}
\and
    \inferrule[Equals (l-operand eval)]{}{
      \lp \cfgnt{L}\ \cfgnt{R}\ \phi\ \eta\ \lp \cfgnt{e}_0\ \cfgt{=}\ \cfgnt{e}\rp  \ \cfgnt{k}\rp  \rightarrow_J \\\\
      \lp \cfgnt{L}\ \cfgnt{R}\ \phi\ \eta\ \cfgnt{e}_0\ \lp \cfgt{*}\ \cfgt{=}\; \cfgnt{e} \rightarrow \cfgnt{k}\rp \rp 
    }
\and
    \inferrule[Equals (r-operand eval)]{}{
    \lp \cfgnt{L}\ \cfgnt{R}\ \phi\ \eta\ \cfgnt{v}\ \lp \cfgt{*}\; \cfgt{=}\; \cfgnt{e} \rightarrow \cfgnt{k}\rp \rp  \rightarrow_J \\\\
    \lp \cfgnt{L}\ \cfgnt{R}\ \phi\ \eta\ \cfgnt{e}\ \lp \cfgnt{v}\; \cfgt{=}\; \cfgt{*} \rightarrow \cfgnt{k}\rp \rp 
    }
\and
    \inferrule[Equals (bool)]{
    \cfgnt{v}_0 \in \{\cfgt{true}, \cfgt{false}\} \\
    \cfgnt{v}_1 \in \{\cfgt{true}, \cfgt{false}\} \\\\
    \cfgnt{v}_r = \mathrm{eq?}\lp \cfgnt{v}_0, \cfgnt{v}_1\rp}{
    \lp \cfgnt{L}\ \cfgnt{R}\ \phi\ \eta\ \cfgnt{v}_0\ \lp \cfgnt{v}_1\; \cfgt{=}\; \cfgt{*} \rightarrow \cfgnt{k}\rp \rp  \rightarrow_J \\\\
    \lp \cfgnt{L}\ \cfgnt{R}\ \phi\ \eta\ \cfgnt{v}_r\ \cfgnt{k}\rp 
    }
\and
    \inferrule[If-then-else (eval)]{}{
      \lp \cfgnt{L}\ \cfgnt{R}\ \phi\ \eta\ \lp \cfgt{if}\ \cfgnt{e}_0\ \cfgnt{e}_1\ \cfgt{else}\ \cfgnt{e}_2\rp \ \cfgnt{k}\rp  \rightarrow_J \\\\
      \lp \cfgnt{L}\ \cfgnt{R}\ \phi\ \eta\ \cfgnt{e}_0\ \lp \cfgt{if}\ \cfgt{*}\ \cfgnt{e}_1\ \cfgt{else}\ \cfgnt{e}_2\rp  \rightarrow \cfgnt{k}\rp 
	}
\and
	\inferrule[If-then-else (true) ]{}{
       \lp \cfgnt{L}\ \cfgnt{R}\ \phi\ \eta\ \cfgt{true}\ \lp \cfgt{if}\ \cfgt{*}\ \cfgnt{e}_1\ \cfgt{else}\ \cfgnt{e}_2\rp  \rightarrow_J \cfgnt{k}\rp  \rightarrow \\\\
       \lp \cfgnt{L}\ \cfgnt{R}\ \phi\ \eta\ \cfgnt{e}_1\  \cfgnt{k}\rp 
	}
\and
	\inferrule[If-then-else (false)]{}{
       \lp \cfgnt{L}\ \cfgnt{R}\ \phi\ \eta\ \cfgt{false}\ \lp \cfgt{if}\ \cfgt{*}\ \cfgnt{e}_1\ \cfgt{else}\ \cfgnt{e}_2\rp  \rightarrow_J \cfgnt{k}\rp  \rightarrow \\\\
       \lp \cfgnt{L}\ \cfgnt{R}\ \phi\ \eta\ \cfgnt{e}_2\  \cfgnt{k}\rp 
	}
\and
   \inferrule[Variable Declaration (eval)]{}{
    \lp \cfgnt{L}\ \cfgnt{R}\ \phi\ \eta\ \lp\cfgt{var}\ \cfgnt{T}\ \cfgnt{x}\ \cfgt{:=}\ \cfgnt{e}_0\ \cfgt{in}\ \cfgnt{e}_1\rp\ \cfgnt{k}\rp  \rightarrow_J \\\\
    \lp \cfgnt{L}\ \cfgnt{R}\ \phi\ \eta\ \cfgnt{e}_0\ \lp\cfgt{var}\ \cfgnt{T}\ \cfgnt{x}\ \cfgt{:=}\ \cfgt{*}\ \cfgt{in}\ \cfgnt{e}_1 \rightarrow \cfgnt{k}\rp\rp 
   }	
\and
   \inferrule[Variable Declaration]{}{
    \lp \cfgnt{L}\ \cfgnt{R}\ \phi\ \eta\ \cfgnt{v}\ \lp\cfgt{var}\ \cfgnt{T}\ \cfgnt{x}\ \cfgt{*}\ \cfgt{:=}\ \cfgt{in}\ \cfgnt{e}_1 \rightarrow \cfgnt{k}\rp\rp  \rightarrow_J \\\\
    \lp \cfgnt{L}\ \cfgnt{R}\ \phi\ \eta[x \mapsto \cfgnt{v}]\ \cfgnt{e}_1\ \lp \cfgt{pop}\ \eta\ \cfgnt{k}\rp \rp 
   }	
\and
   \inferrule[Method Invocation (object eval)]{}{
    \lp \cfgnt{L}\ \cfgnt{R}\ \phi\ \eta\ \lp\cfgnt{e}_0\ \cfgt{@}\ \cfgnt{m}\ \cfgnt{e}_1\rp\ \cfgnt{k}\rp  \rightarrow_J \\\\
    \lp \cfgnt{L}\ \cfgnt{R}\ \phi\ \eta\ \cfgnt{e}_0\ \lp \cfgt{*}\ \cfgt{@}\ \cfgnt{m}\ \cfgnt{e}_1\ \rightarrow \cfgnt{k}\rp \rp 
   }
\and
   \inferrule[Method Invocation (arg eval)]{}{
    \lp \cfgnt{L}\ \cfgnt{R}\ \phi\ \eta\ \cfgnt{v}_0\ \lp \cfgt{*}\ \cfgt{@}\ \cfgnt{m}\ \cfgnt{e}_1\ \rightarrow \cfgnt{k}\rp \rp  \rightarrow_J \\\\
    \lp \cfgnt{L}\ \cfgnt{R}\ \phi\ \eta\ \cfgnt{e}_1\ \lp \cfgnt{v}_0\ \cfgt{@}\ \cfgnt{m}\ \cfgt{*}\ \rightarrow \cfgnt{k}\rp \rp 
   }
\and
   \inferrule[Method Invocation]{
    \lp\cfgnt{T}\ \cfgnt{m}\ \lb\cfgnt{T}\ \cfgnt{x}\rb\ \ \cfgnt{e}_m\rp = \mathrm{lookup}\lp \cfgnt{m}\rp\\\\
    \eta_m = \eta[\cfgt{this} \mapsto \cfgnt{v}_0][\cfgnt{x} \mapsto \cfgnt{v}_1]}{
    \lp \cfgnt{L}\ \cfgnt{R}\ \phi\ \eta\ \cfgnt{v}_1\ \lp \cfgnt{v}_0\ \cfgt{@}\ \cfgnt{m}\ \cfgt{*}\ \rightarrow \cfgnt{k}\rp \rp  \rightarrow_J \\\\
    \lp \cfgnt{L}\ \cfgnt{R}\ \phi\ \eta_m\ \cfgnt{e}_m\ \lp \cfgt{pop}\ \eta\ \cfgnt{k}\rp \rp 
   }
\and
   \inferrule[Variable Assignment (eval)]{}{
    \lp \cfgnt{L}\ \cfgnt{R}\ \phi\ \eta\ \lp \cfgnt{x}\ \cfgt{:=}\ \cfgnt{e}\rp \ \cfgnt{k}\rp  \rightarrow_J \\\\
    \lp \cfgnt{L}\ \cfgnt{R}\ \phi\ \eta\ \cfgnt{e}\ \lp \cfgnt{x}\ \cfgt{:=}\ \cfgt{*} \rightarrow\ \cfgnt{k}\rp \rp 
   }	
\and
   \inferrule[Variable Assignment]{}{
    \lp \cfgnt{L}\ \cfgnt{R}\ \phi\ \eta\ \cfgnt{v}\ \lp \cfgnt{x}\ \cfgt{:=}\ \cfgt{*} \rightarrow\ \cfgnt{k}\rp \rp  \rightarrow_J \\\\
    \lp \cfgnt{L}\ \cfgnt{R}\ \phi\ \eta[\cfgnt{x} \mapsto \cfgnt{v}]\ \cfgnt{v}\ \cfgnt{k}\rp 
   }	
\and
   \inferrule[Begin (no args)]{}{
    \lp \cfgnt{L}\ \cfgnt{R}\ \phi\ \eta\ \lp \cfgt{begin}\rp \ \cfgnt{k}\rp  \rightarrow \\\\
    \lp \cfgnt{L}\ \cfgnt{R}\ \phi\ \eta\ \cfgnt{k}\rp 
   }
\and
   \inferrule[Begin (arg0 eval)]{}{
    \lp \cfgnt{L}\ \cfgnt{R}\ \phi\ \eta\ \lp \cfgt{begin}\ \cfgnt{e}_0\ \cfgnt{e}_1\ ...\rp \ \cfgnt{k}\rp  \rightarrow_J \\\\
    \lp \cfgnt{L}\ \cfgnt{R}\ \phi\ \eta\ \cfgnt{e}_0\ \lp \cfgt{begin}\ \cfgt{*}\ \lp\cfgnt{e}_1\ ...\rp \rightarrow \cfgnt{k}\rp \rp 
   }
\and
   \inferrule[Begin (argi eval)]{}{
    \lp \cfgnt{L}\ \cfgnt{R}\ \phi\ \eta\ \cfgnt{v}\ \lp \cfgt{begin}\ \cfgt{*}\ \lp\cfgnt{e}_i\ \cfgnt{e}_{i+1}\ ...\rp \rightarrow \cfgnt{k}\rp \rp  \rightarrow \\\\
    \lp \cfgnt{L}\ \cfgnt{R}\ \phi\ \eta\ \cfgnt{e}_i\ \lp \cfgt{begin}\ \cfgt{*}\ \lp\cfgnt{e}_{i+1}\ ...\rp \rightarrow \cfgnt{k}\rp \rp 
   }
\and
   \inferrule[Begin (argN eval)]{}{
    \lp \cfgnt{L}\ \cfgnt{R}\ \phi\ \eta\ \cfgnt{v}\ \lp \cfgt{begin}\ \cfgt{*}\ \lp\cfgnt{e}_{n}\rp \rightarrow \cfgnt{k}\rp \rp  \rightarrow \\\\
    \lp \cfgnt{L}\ \cfgnt{R}\ \phi\ \eta\ \cfgnt{e}_n\ \lp \cfgt{begin}\ \cfgt{*}\ \lp\rp \rightarrow \cfgnt{k}\rp \rp 
   }
\and
   \inferrule[Begin]{}{
    \lp \cfgnt{L}\ \cfgnt{R}\ \phi\ \eta\ \cfgnt{v}\ \lp \cfgt{begin}\ \cfgt{*}\ \lp\rp \rightarrow \cfgnt{k}\rp \rp  \rightarrow \\\\
    \lp \cfgnt{L}\ \cfgnt{R}\ \phi\ \eta\ \cfgnt{v}\ \cfgnt{k}\rp 
   }	
\and
	\inferrule[NULL]{}{
      \lp \cfgnt{L}\ \cfgnt{R}\ \phi\ \eta\ \cfgt{null}\ \cfgnt{k}\rp \rightarrow \\\\ 
      \lp \cfgnt{L}\ \cfgnt{R}\ \phi\ \eta\ \cfgnt{r}_\mathit{null}\ \cfgnt{k}\rp 
	}
\and
   \inferrule[Pop]{}{
    \lp \cfgnt{L}\ \cfgnt{R}\ \phi\ \eta\ \cfgnt{v}\ \lp \cfgt{pop}\ \eta_0\ \cfgnt{k}\rp \rp  \rightarrow \\\\
    \lp \cfgnt{L}\ \cfgnt{R}\ \phi\ \eta_0\  \cfgnt{v}\ \cfgnt{k}\rp 
   }
\end{mathpar}
\end{center}
\caption{Javalite rewrite rules that are common to generalized symbolic execution and precise heap summaries.}
\label{fig:javalite-common}
\end{figure*}


\section{Initialization of Symbolic References}

In this section we present the Javalite rewrite rules for the concrete
as well as summary initialization of symbolic references. The
initialization rules are defined on the bi-partite graph consisting of
references and locations. The lazy initialization of symbolic
references consists of three key points of non-determinism where each
symbolic reference can be initialized non-deterministically to null, a
new instance of the symbolic reference, or aliases to symbolic
references of the same type previously initialized. The initialization
in GSE consists of creating branches in the execution tree for all the
non-deterministic choices. On the other hand, the heap summarization
approach creates a single branch that contains the summarization for
all the initialization in a single bi-partitate graph.

\begin{figure*}[t]
\begin{center}
\mprset{flushleft}
\begin{mathpar}
	\inferrule[Initialize (null)]{
	  \Lambda = \mathbb{UN}\lp \cfgnt{L}, \cfgnt{R}, \cfgnt{r}, \cfgnt{f}\rp \\
      \Lambda \neq \emptyset\\\\
      \cfgnt{r}^\prime = \mathrm{fresh}_r\lp \rp\\ 
      \theta_\mathit{null} = \{ \lp \cfgt{true}\ l_\mathit{null}\rp \} \\\\
      l_x = \mathrm{min}_l\lp \Lambda\rp \\\\
      \phi_g^\prime = \lp\phi_g \wedge \cfgnt{r}^\prime = \cfgnt{r}_\mathit{null}\rp
    }{
      \lp \cfgnt{L}\ \cfgnt{R}\ \phi_g\ \cfgnt{r}\ \cfgnt{f}\ \cfgnt{C}\rp  \rightarrow_I 
      \lp \cfgnt{L}[\cfgnt{r}^\prime \mapsto \theta_\mathit{null}]\ \cfgnt{R}[ \lp l_x,\cfgnt{f}\rp  \mapsto \cfgnt{r}^\prime]\ \phi_g^\prime\ \cfgnt{r}\ \cfgnt{f}\ \cfgnt{C}\rp 
	}
\and
	\inferrule[Initialize (new)]{
	  \Lambda = \mathbb{UN}\lp \cfgnt{L}, \cfgnt{R}, \cfgnt{r}, \cfgnt{f}\rp \\
      \Lambda \neq \emptyset\\
      \lp\phi_x\ \cfgnt{l}_x\rp = \mathrm{min}_l\lp \Lambda\rp\\\\
      \cfgnt{r}_f = \mathrm{init}_r\lp \rp\\
      l_f = \mathrm{fresh}_l\lp \cfgnt{C}\rp \\\\
      \rho = \{ \lp\cfgnt{r}_a\ l_a\rp \mid \mathrm{isInit}\lp \cfgnt{r}_a\rp  \wedge \cfgnt{r}_a = \mathrm{min}_r\lp \cfgnt{R}^{-1}[l_a]\rp \wedge \mathrm{type}\lp l_a\rp  = \cfgnt{C} \}\\\\
      \theta_\mathit{new} = \{\lp \cfgt{true}\ l_f\rp \} \\\\
      \cfgnt{R}^\prime = \cfgnt{R}[\forall \cfgnt{f} \in \mathit{fields}\lp \mathrm{C}\rp \ \lp \lp l_f\ \cfgnt{f}\rp  \mapsto \cfgnt{r}_\mathit{un} \rp ] \\\\
      \phi_g^\prime = \lp\phi_g \wedge \cfgnt{r}_f \neq \cfgnt{r}_\mathit{null} \wedge \lp \wedge_{\lp\cfgnt{r}_a\ l_a\rp \in \rho} \cfgnt{r}_f \ne \cfgnt{r}_a\rp\rp
    }{
      \lp \cfgnt{L}\ \cfgnt{R}\ \phi_g\ \cfgnt{r}\ \cfgnt{f}\ \cfgnt{C}\rp  \rightarrow_I 
      \lp \cfgnt{L}[\cfgnt{r}_f \mapsto \theta_\mathit{new}]\ \cfgnt{R}^\prime[ \lp l_x,\cfgnt{f}\rp  \mapsto \cfgnt{r}_f ]\ \phi_g^\prime\ \cfgnt{r}\ \cfgnt{f}\ \cfgnt{C}\rp 
	}
\and
	\inferrule[Initialize (alias)]{
  	  \Lambda = \mathbb{UN}\lp \cfgnt{L}, \cfgnt{R}, \cfgnt{r}, \cfgnt{f}\rp \\
      \Lambda \neq \emptyset\\
      \lp\phi_x\ \cfgnt{l}_x\rp = \mathrm{min}_l\lp \Lambda\rp\\\\
      \cfgnt{r}^\prime = \mathrm{fresh}_r\lp \rp\\\\
      \rho = \{ \lp\cfgnt{r}_a\ l_a\rp \mid \mathrm{isInit}\lp \cfgnt{r}_a\rp  \wedge \cfgnt{r}_a = \mathrm{min}_r\lp \cfgnt{R}^{-1}[l_a]\rp \wedge \mathrm{type}\lp l_a\rp  = \cfgnt{C} \}\\\\
      \lp\cfgnt{r}_a\ l_a\rp \in \rho \\
      \theta_\mathit{alias} = \{ \lp \cfgt{true}\ l_a\rp\}\\\\
      \phi^\prime_g = \lp\phi_g \wedge \cfgnt{r}^\prime \neq \cfgnt{r}_\mathit{null} \wedge \cfgnt{r}^\prime = \cfgnt{r}_a \wedge \lp \wedge_{\lp \cfgnt{r}^{\prime}_a\ l_a\rp  \in \rho\ \lp \cfgnt{r}^{\prime}_a \neq \cfgnt{r}_a\rp } \cfgnt{r}^\prime \neq \cfgnt{r}^{\prime}_a \rp\rp
    }{
      \lp \cfgnt{L}\ \cfgnt{R}\ \phi_g\ \cfgnt{r}\ \cfgnt{f}\ \cfgnt{C}\rp  \rightarrow_I 
      \lp \cfgnt{L}[\cfgnt{r}^\prime \mapsto \theta_\mathit{alias}]\ \cfgnt{R}[ \lp l_x,\cfgnt{f}\rp  \mapsto \cfgnt{r}^\prime ]\ \phi_g^\prime\ \cfgnt{r}\ \cfgnt{f}\ \cfgnt{C}\rp 
	}
\and
	\inferrule[Initialize (end)]{
	  \Lambda = \mathbb{UN}\lp \cfgnt{L}, \cfgnt{R}, \cfgnt{r}, \cfgnt{f}\rp \\
      \Lambda = \emptyset
    }{
      \lp \cfgnt{L}\ \cfgnt{R}\ \phi_g\ \cfgnt{r}\ \cfgnt{f}\ \cfgnt{C}\rp  \rightarrow_I 
      \lp \cfgnt{L}\ \cfgnt{R}\ \phi_g\ \cfgnt{r}\ \cfgnt{f}\ \cfgnt{C}\rp 
	}
\end{mathpar}
\end{center}
\caption{The initialization machine, $s ::= \lp\cfgnt{L}\ \cfgnt{R}\ \phi_g\ \cfgnt{r}\ \cfgnt{f}\rp$, with $s \rightarrow_I^* s^\prime$ indicating stepping the machine until the state does not change.}
\label{fig:lazyInit}
\end{figure*}

\begin{figure*}[t]
\begin{center}
\mprset{flushleft}
\begin{mathpar}
	\inferrule[Summarize]{
	\Lambda = \mathbb{UN}\lp \cfgnt{L}, \cfgnt{R}, \cfgnt{r}, \cfgnt{f}\rp \\
      \Lambda \neq \emptyset \\
      \lp\phi_x\ \cfgnt{l}_x\rp = \mathrm{min}_l\lp \Lambda\rp\\
      \cfgnt{r}_f = \mathrm{init}_r\lp \rp \\
      l_f  = \mathrm{fresh}_l\lp \mathrm{C}\rp\\\\
      \rho = \{ \lp \cfgnt{r}_a\ \phi_a\ l_a\rp  \mid \mathrm{isInit}\lp \cfgnt{r}_a\rp  \wedge\cfgnt{r}_a = \mathrm{min}_r\lp \cfgnt{R}^{-1}[l_a]\rp \wedge \lp \phi_a\ l_a\rp  \in \cfgnt{L}\lp \cfgnt{r}_a\rp \wedge \mathrm{type}\lp l_a\rp  = \mathrm{C} \} \\\\
      \theta_\mathit{null} = \{ \lp \phi\ l_\mathit{null}\rp  \mid \phi = \lp \phi_x \wedge \cfgnt{r}_f = \cfgnt{r}_\mathit{null} \rp  \} \\\\
      \theta_\mathit{new} = \{\lp \phi\ l_f\rp  \mid \phi = \lp \phi_x \wedge \cfgnt{r}_f \neq \cfgnt{r}_\mathit{null} \wedge \lp \wedge_{\lp \cfgnt{r}^\prime_a,\ \phi^\prime_a,\ l^\prime_a\rp  \in \rho} \cfgnt{r}_f \ne \cfgnt{r}^\prime_a\rp \rp \}\\\\
      \theta_\mathit{alias} = \{ \lp \phi\ l_a\rp  \mid \exists\cfgnt{r}_a\ \lp\exists \phi_a\ \lp\lp\cfgnt{r}_a\ \phi_a\ l_a\rp  \in \rho \wedge \phi = \lp \phi_x \wedge \phi_a \wedge \cfgnt{r}_f \neq \cfgnt{r}_\mathit{null} \wedge \cfgnt{r}_f = \cfgnt{r}_a \wedge \lp \wedge_{\lp \cfgnt{r}^{\prime}_a\ \phi^{\prime}_a\ l^{\prime}_a\rp  \in \rho\ \lp \cfgnt{r}^\prime_a \neq \cfgnt{r}_a\rp } \cfgnt{r}_f \neq \cfgnt{r}^{\prime}_a \rp \rp \rp \rp \} \\\\
      \theta_\mathit{orig} = \{\lp\phi\ \cfgnt{l}_\mathit{orig}\rp \mid \exists \phi_\mathit{orig} \lp \lp\phi_\mathit{orig}\ \cfgnt{l}_\mathit{orig}\rp \in \cfgnt{L}\lp\cfgnt{R}\lp\cfgnt{l}_x,\cfgnt{f}\rp\rp \wedge \phi = \lp\neg\phi_x \wedge \phi_\mathit{orig}\rp\}\\\\ 
      \theta = \theta_\mathit{alias} \cup \theta_\mathit{new} \cup \theta_\mathit{null} \cup \theta_\mathit{old} \\
\cfgnt{R}^\prime = \cfgnt{R}[\forall \cfgnt{f} \in \mathit{fields}\lp \mathrm{C}\rp \ \lp \lp l_f\ \cfgnt{f}\rp  \mapsto \cfgnt{r}_\mathit{un} \rp ]
    }{
      \lp \cfgnt{L}\ \cfgnt{R}\ \cfgnt{r}\ \cfgnt{f}\ \cfgnt{C}\rp \rightarrow_S 
      \lp \cfgnt{L}[\cfgnt{r}_f \mapsto \theta]\ \cfgnt{R}^{\prime}[ \lp l_x,\cfgnt{f}\rp  \mapsto \cfgnt{r}_f ]\ \cfgnt{r}\ \cfgnt{f}\ \cfgnt{C}\rp
	}
\and
	\inferrule[Summarize (end)]{
	  \Lambda = \{ l \mid \exists \phi\ \lp \lp \phi\ l\rp  \in \cfgnt{L}\lp \cfgnt{r}\rp  \wedge  \cfgnt{R}\lp l,\cfgnt{f}\rp  = \bot\ \rp\}\\
      \Lambda = \emptyset
    }{
      \lp \cfgnt{L}\ \cfgnt{R}\ \cfgnt{r}\ \cfgnt{f}\ \cfgnt{C}\rp  \rightarrow_S
      \lp \cfgnt{L}\ \cfgnt{R}\ \cfgnt{r}\ \cfgnt{f}\ \cfgnt{C}\rp 
	}
\end{mathpar}
\end{center}
\caption{The summary machine, $s ::= \lp\cfgnt{L}\ \cfgnt{R}\ \cfgnt{r}\ \cfgnt{f}\ \cfgnt{C}\rp$, with $s\rightarrow_S^*s^\prime$ indicating stepping the machine until the state does not change.}
\label{fig:symInit}
\end{figure*}



The initialization rules are invoked when an uninitialized field in a
symbolic reference is accessed. The function $\mathbb{UN}(\cfgnt{L},
\cfgnt{R}, \cfgnt{r}, \cfgnt{f}) = \{\cfgnt{l}\ ...\}$ returns
constraint-location pairs in which the field $\cfgnt{f}$ is
uninitialized:
\[
\begin{array}{rcl}
\mathbb{UN}(\cfgnt{L}, \cfgnt{R}, \cfgnt{r}, \cfgnt{f}) & = &\{ \lp\phi\ \cfgnt{l}\rp \mid \lp \phi\ \cfgnt{l}\rp  \in \cfgnt{L}\lp \cfgnt{r}\rp  \wedge \\
& & \ \ \ \ \exists \phi^\prime \lp \lp \phi^\prime\ \cfgnt{l}_\mathit{un}\rp  \in \cfgnt{L}\lp \cfgnt{R}\lp l,\cfgnt{f}\rp\rp \wedge \\
& & \ \ \ \ \ \ \ \ \mathbb{S}\lp \phi \wedge \phi^\prime \rp\rp\}\\
\end{array}
\]
where $\mathbb{S}(\phi)$ returns true if $\phi$ is
satisfiable. Intutively, for the reference, $\cfgnt{r}$, it constructs
the set, $\theta$, that contains all contraint-location pairs that
point to the field $\cfgnt{f}$ and $\cfgnt{f}$ points to
$\cfgnt{l}_\mathit{un}$. The cardinality of the set, $\theta$ is never
greater than one in GSE and the constraint is always satisfiable
because all constraints are constant. This property is relaxed in GSE
with heap summaries.

The rules in~\figref{fig:lazyInit} present the rewrite rules for the
concrete initialization of symbolic heap objects.  These rules are
invoked until a fix pointed is reached. 

The initialize (null) rewrite rule in~\figref{fig:lazyInit} first
checks that the field, $\cfgnt{r}$ is uninitialized. The fresh method
returns a new input heap reference from the partition 


\section{Accessing and Writing to Field References}

\section{Equality and InEquality of References}

\begin{figure*}[t]
\begin{center}
\mprset{flushleft}
\begin{mathpar}
	\inferrule[Field Access]{
      \{\lp\phi\ l\rp\} = \cfgnt{L}\lp\cfgnt{r}\rp\\
      l \neq \cfgnt{l}_\mathit{null}\\
      \cfgnt{C} = \mathrm{type}\lp\cfgnt{l},\cfgnt{f}\rp\\\\
      \lp \cfgnt{L}\ \cfgnt{R}\ \cfgnt{r}\ \cfgnt{f}\ \cfgnt{C}\rp \rinit^*
      \lp \cfgnt{L}^\prime\ \cfgnt{R}^\prime\ \cfgnt{r}\ \cfgnt{f}\  \cfgnt{C}\rp \\\\ 
      \{\lp\phi^\prime\ l^\prime\rp\} = \cfgnt{L}^\prime\lp\cfgnt{R}^\prime\lp l,\cfgnt{f}\rp\rp \\
      \cfgnt{r}^\prime = \mathrm{stack}_r\lp\rp \\
    }{
      \lp \cfgnt{L}\ \cfgnt{R}\ \phi_g\ \eta\ \cfgnt{r}\ \lp \cfgt{*}\ \cfgt{\$}\ \cfgnt{f} \rightarrow \cfgnt{k}\rp \rp  \rightarrow_\ell \\\\
      \lp \cfgnt{L}^\prime[\cfgnt{r}^\prime \mapsto \lp\phi^\prime\ l^\prime\rp]\ \cfgnt{R}^\prime\ \phi_g^\prime\ \eta\ \cfgnt{r}^\prime\ \cfgnt{k}\rp 
	}
\and
	\inferrule[Field Write]{
      \cfgnt{r}_x = \eta\lp \cfgnt{x}\rp\\ 
      \theta = \{\lp\phi\ l\rp\} = \cfgnt{L}\lp\cfgnt{r}_x\rp \\\\
      l \neq \cfgnt{l}_\mathit{null}\\
      \cfgnt{r}^\prime = \mathrm{fresh}_r\lp\rp\\
    }{
      \lp \cfgnt{L}\ \cfgnt{R}\ \phi_g\ \eta\ \cfgnt{r}\ \lp \cfgnt{x}\ \cfgt{\$}\ \cfgnt{f}\ \cfgt{:=}\ \cfgt{*}\ \rightarrow\ \cfgnt{k}\rp \rp  \rightarrow_\ell \\\\
      \lp \cfgnt{L}[\cfgnt{r}^\prime \mapsto \theta]\ \cfgnt{R}[\lp l\ \cfgnt{f}\rp  \mapsto \cfgnt{r}^\prime]\ \phi_g\ \eta\ \cfgnt{r}\ \cfgnt{k}\rp 
	}
\and
  \inferrule[Equals (reference-true)]{
    \cfgnt{L}\lp \cfgnt{r}_0\rp = \cfgnt{L}\lp \cfgnt{r}_1\rp\\
    \phi^\prime = \lp\phi \wedge r_0 = r_1\rp
    }{
    \lp \cfgnt{L}\ \cfgnt{R}\ \phi\ \eta\ \cfgnt{r}_0\ \lp \cfgnt{r}_1\ \cfgt{=}\ \cfgt{*} \rightarrow \cfgnt{k}\rp \rp  \rightarrow_\ell \\\\
    \lp \cfgnt{L}\ \cfgnt{R}\ \phi^\prime\ \eta\ \cfgt{true}\ \cfgnt{k}\rp 
    }
\and
    \inferrule[Equals (reference-false)]{
    \cfgnt{L}\lp \cfgnt{r}_0\rp \neq \cfgnt{L}\lp \cfgnt{r}_1\rp\\
    \phi^\prime = \lp\phi \wedge r_0 \neq r_1\rp
   }{
    \lp \cfgnt{L}\ \cfgnt{R}\ \phi\ \eta\ \cfgnt{r}_0\ \lp \cfgnt{r}_1\ \cfgt{=}\ \cfgt{*} \rightarrow \cfgnt{k}\rp \rp  \rightarrow_\ell \\\\
    \lp \cfgnt{L}\ \cfgnt{R}\ \phi^\prime\ \eta\ \cfgt{false}\ \cfgnt{k}\rp 
    }	
\end{mathpar}
\end{center}
\caption{GSE with lazy initialization indicated by $\rgse = \rightarrow_\ell \cup \rcom$.}
\label{fig:lazy}
\end{figure*}



	

\section{Summary Machine Rules}

\begin{definition}
\label{def:VS}
The $\mathbb{VS}$ function constructs the value set given a
heap, reference, and desired field where $\mathbb{S}(\phi)$ returns true if $\phi$ is satisfiable.
\[
\begin{array}{rcl}
  \mathbb{VS}(L,R,\phi_g,\cfgnt{r},\cfgnt{f}) & \defeq &
  \{(\phi\wedge\phi^\prime\ l^\prime) \mid \exists
  l\ ((\phi\ l) \in L(\cfgnt{r})\ \wedge \\ & &
  \ \ (\phi^\prime\ l^\prime) \in L(R(l,\cfgnt{f}))\ \wedge \\ & &
  \ \ \mathbb{S}(\phi\wedge\phi^\prime\wedge \phi_g))\}
\end{array}
\]
\end{definition}

\begin{definition}
\label{def:ST}
The function $\mathbb{ST}(\cfgnt{L},\cfgnt{r},\phi,\phi_g)$
strengthens every constraint in $\cfgnt{L}\lp\cfgnt{r}\rp$ with
$\phi$ and retains strengthened location-constraint pairs that are satisfiable
with the path condition $\phi_g$:
\[
\begin{array}{rcl} 
\mathbb{ST}(\cfgnt{L},\cfgnt{r},\phi,\phi_g) & = & \{ (\phi\wedge\phi^\prime\ \cfgnt{l}^\prime) \mid (\phi^\prime\ \cfgnt{l}^\prime)\in \cfgnt{L}(\cfgnt{r})\wedge \\ & & 
\ \ \mathbb{S}(\phi\wedge\phi^\prime\wedge\phi_g) \}
\end{array}
\]
\end{definition}



\begin{figure*}[t]
\begin{center}
\mprset{flushleft}
\begin{mathpar}
	\inferrule[Summarize]{
	\Lambda = \mathbb{UN}\lp \cfgnt{L}, \cfgnt{R}, \cfgnt{r}, \cfgnt{f}\rp \\
      \Lambda \neq \emptyset \\
      \lp\phi_x\ \cfgnt{l}_x\rp = \mathrm{min}_l\lp \Lambda\rp\\
      \cfgnt{r}_f = \mathrm{init}_r\lp \rp \\
      l_f  = \mathrm{fresh}_l\lp \mathrm{C}\rp\\\\
      \rho = \{ \lp \cfgnt{r}_a\ \phi_a\ l_a\rp  \mid \mathrm{isInit}\lp \cfgnt{r}_a\rp  \wedge\cfgnt{r}_a = \mathrm{min}_r\lp \cfgnt{R}^{-1}[l_a]\rp \wedge \lp \phi_a\ l_a\rp  \in \cfgnt{L}\lp \cfgnt{r}_a\rp \wedge \mathrm{type}\lp l_a\rp  = \mathrm{C} \} \\\\
      \theta_\mathit{null} = \{ \lp \phi\ l_\mathit{null}\rp  \mid \phi = \lp \phi_x \wedge \cfgnt{r}_f = \cfgnt{r}_\mathit{null} \rp  \} \\\\
      \theta_\mathit{new} = \{\lp \phi\ l_f\rp  \mid \phi = \lp \phi_x \wedge \cfgnt{r}_f \neq \cfgnt{r}_\mathit{null} \wedge \lp \wedge_{\lp \cfgnt{r}^\prime_a,\ \phi^\prime_a,\ l^\prime_a\rp  \in \rho} \cfgnt{r}_f \ne \cfgnt{r}^\prime_a\rp \rp \}\\\\
      \theta_\mathit{alias} = \{ \lp \phi\ l_a\rp  \mid \exists\cfgnt{r}_a\ \lp\exists \phi_a\ \lp\lp\cfgnt{r}_a\ \phi_a\ l_a\rp  \in \rho \wedge \phi = \lp \phi_x \wedge \phi_a \wedge \cfgnt{r}_f \neq \cfgnt{r}_\mathit{null} \wedge \cfgnt{r}_f = \cfgnt{r}_a \wedge \lp \wedge_{\lp \cfgnt{r}^{\prime}_a\ \phi^{\prime}_a\ l^{\prime}_a\rp  \in \rho\ \lp \cfgnt{r}^\prime_a \neq \cfgnt{r}_a\rp } \cfgnt{r}_f \neq \cfgnt{r}^{\prime}_a \rp \rp \rp \rp \} \\\\
      \theta_\mathit{orig} = \{\lp\phi\ \cfgnt{l}_\mathit{orig}\rp \mid \exists \phi_\mathit{orig} \lp \lp\phi_\mathit{orig}\ \cfgnt{l}_\mathit{orig}\rp \in \cfgnt{L}\lp\cfgnt{R}\lp\cfgnt{l}_x,\cfgnt{f}\rp\rp \wedge \phi = \lp\neg\phi_x \wedge \phi_\mathit{orig}\rp\}\\\\ 
      \theta = \theta_\mathit{alias} \cup \theta_\mathit{new} \cup \theta_\mathit{null} \cup \theta_\mathit{old} \\
\cfgnt{R}^\prime = \cfgnt{R}[\forall \cfgnt{f} \in \mathit{fields}\lp \mathrm{C}\rp \ \lp \lp l_f\ \cfgnt{f}\rp  \mapsto \cfgnt{r}_\mathit{un} \rp ]
    }{
      \lp \cfgnt{L}\ \cfgnt{R}\ \cfgnt{r}\ \cfgnt{f}\ \cfgnt{C}\rp \rightarrow_S 
      \lp \cfgnt{L}[\cfgnt{r}_f \mapsto \theta]\ \cfgnt{R}^{\prime}[ \lp l_x,\cfgnt{f}\rp  \mapsto \cfgnt{r}_f ]\ \cfgnt{r}\ \cfgnt{f}\ \cfgnt{C}\rp
	}
\and
	\inferrule[Summarize (end)]{
	  \Lambda = \{ l \mid \exists \phi\ \lp \lp \phi\ l\rp  \in \cfgnt{L}\lp \cfgnt{r}\rp  \wedge  \cfgnt{R}\lp l,\cfgnt{f}\rp  = \bot\ \rp\}\\
      \Lambda = \emptyset
    }{
      \lp \cfgnt{L}\ \cfgnt{R}\ \cfgnt{r}\ \cfgnt{f}\ \cfgnt{C}\rp  \rightarrow_S
      \lp \cfgnt{L}\ \cfgnt{R}\ \cfgnt{r}\ \cfgnt{f}\ \cfgnt{C}\rp 
	}
\end{mathpar}
\end{center}
\caption{The summary machine, $s ::= \lp\cfgnt{L}\ \cfgnt{R}\ \cfgnt{r}\ \cfgnt{f}\ \cfgnt{C}\rp$, with $s\rightarrow_S^*s^\prime$ indicating stepping the machine until the state does not change.}
\label{fig:symInit}
\end{figure*}

\newsavebox{\boxPFAFW}
\savebox{\boxPFAFW}{
%\begin{figure}[t]
%\begin{center}
\mprset{flushleft}
\begin{mathpar}
	\inferrule[Field Access]{
      \exists \lp \phi\ l\rp \in \cfgnt{L}\lp \cfgnt{r}\rp\ \lp l \neq l_{\mathit{null}} \wedge \mathbb{S}\lp \phi \wedge \phi_g\rp \rp \\\\
      \theta = \{ \phi \mid \lp \phi\ l_\mathit{null} \rp \wedge \mathbb{S}\lp \phi \wedge \phi_g\rp \} \\\\
      \phi_g^\prime = \phi_g \wedge (\wedge_{\phi \in \theta} \neg \phi) \\\\
      \{\cfgnt{C}\} = \{\cfgnt{C} \mid \exists \lp \phi\ l\rp  \in \cfgnt{L}\lp \cfgnt{r}\rp\ \lp\cfgnt{C} = \mathrm{type}\lp \cfgnt{l},\cfgnt{f}\rp\rp\} \\\\
      \lp \cfgnt{L}\ \cfgnt{R}\ \cfgnt{r}\ \cfgnt{f}\ \cfgnt{C}\rp \rsum^* \lp \cfgnt{L}^\prime\ \cfgnt{R}^\prime\ \cfgnt{r}\ \cfgnt{f}\ \cfgnt{C}\rp \\
      \cfgnt{r}^\prime = \mathrm{stack}_r\lp \rp
    }{
      \lp \cfgnt{L}\ \cfgnt{R}\ \phi_g\ \eta\ \cfgnt{r}\ \lp \cfgt{*}\ \cfgt{\$}\ \cfgnt{f} \rightarrow \cfgnt{k}\rp \rp  \rightarrow_\mathit{A}
      \lp \cfgnt{L}^\prime[\cfgnt{r}^\prime \mapsto \mathbb{VS}\lp \cfgnt{L}^\prime,\cfgnt{R}^\prime,\cfgnt{r},\cfgnt{f},\phi_g^\prime\rp ]\ \cfgnt{R}^\prime\ \phi_g^\prime\ \eta\ \cfgnt{r}^\prime\ \cfgnt{k}\rp 
	}
\and
	\inferrule[Field Access (NULL)]{
      \exists \lp \phi\ l\rp \in \cfgnt{L}\lp \cfgnt{r}\rp\ \lp l = l_{\mathit{null}} \wedge \mathbb{S}\lp \phi \wedge \phi_g\rp \rp
    }{
      \lp \cfgnt{L}\ \cfgnt{R}\ \phi_g\ \eta\ \cfgnt{r}\ \lp \cfgt{*}\ \cfgt{\$}\ \cfgnt{f} \rightarrow \cfgnt{k}\rp \rp  \rightarrow_\mathit{A}
      \lp \cfgnt{L}\ \cfgnt{R}\ \phi_g\ \eta\ \cfgt{error}\ \cfgt{end} \rp
	}
\and
	\inferrule[Field Write]{
      \cfgnt{r}_x = \eta\lp\cfgnt{x}\rp \\
      \exists \lp \phi\ l\rp \in \cfgnt{L}\lp \cfgnt{r}_x\rp\ \lp l \neq l_{\mathit{null}} \wedge \mathbb{S}\lp \phi \wedge \phi_g\rp \rp \\\\
      \theta = \{ \phi \mid \lp \phi\ l_\mathit{null} \rp \wedge \mathbb{S}\lp \phi \wedge \phi_g\rp \} \\\\
      \phi_g^\prime = \phi_g \wedge (\wedge_{\phi \in \theta} \neg \phi) \\\\
      \Psi_x =\{\lp \phi\ l\ \cfgnt{r}_\mathit{cur} \rp  \mid \lp \phi\ \cfgnt{l}\rp  \in \cfgnt{L}\lp \cfgnt{r}_x\rp  \wedge \cfgnt{r}_\mathit{cur} = \cfgnt{R}\lp l,\cfgnt{f}\rp  \}\\\\
      X = \{ \lp l\ \theta \rp  \mid \exists \phi\ \lp \lp \phi\ l\ \cfgnt{r}_\mathit{cur} \rp \in \Psi_x \wedge \theta = \mathbb{ST}\lp \cfgnt{L},\cfgnt{r},\phi,\phi_g^\prime\rp  \cup \mathbb{ST}\lp \cfgnt{L},\cfgnt{r}_\mathit{cur},\neg\phi,\phi_g^\prime\rp \rp  \}\\\\
      \cfgnt{R}^{\prime} = \cfgnt{R}[\forall \lp l\ \theta \rp  \in X\ \lp \lp l\ \cfgnt{f}\rp  \mapsto \mathrm{fresh}_r\lp \rp \rp ]\\\\
      \cfgnt{L}^{\prime} = \cfgnt{L}[\forall \lp l\ \theta \rp  \in X\ \lp \exists \cfgnt{r}_\mathit{targ}\ \lp \cfgnt{r}_\mathit{targ} = \cfgnt{R}^\prime\lp l,\cfgnt{f}\rp \wedge \lp\cfgnt{r}_\mathit{targ} \mapsto \theta\rp  \rp \rp ]
    }{
      \lp \cfgnt{L}\ \cfgnt{R}\ \phi_g\ \eta\ \cfgnt{r}\ \lp \cfgnt{x}\ \cfgt{\$}\ \cfgnt{f}\ \cfgt{:=}\ \cfgt{*}\ \rightarrow\ \cfgnt{k}\rp \rp  \rightarrow_\mathit{FW}
      \lp \cfgnt{L}^{\prime}\ \cfgnt{R}^{\prime}\ \phi_g^\prime\ \eta\ \cfgnt{r}\ \cfgnt{k}\rp 
	}	
\and
	\inferrule[Field Write (NULL)]{
      \cfgnt{r}_x = \eta\lp \cfgnt{x}\rp \\
      \exists \lp \phi\ l\rp \in \cfgnt{L}\lp \cfgnt{r}_x\rp\ \lp l \neq l_{\mathit{null}} \wedge \mathbb{S}\lp \phi \wedge \phi_g\rp \rp
    }{
      \lp \cfgnt{L}\ \cfgnt{R}\ \phi_g\ \eta\ \cfgnt{r}\ \lp \cfgnt{x}\ \cfgt{\$}\ \cfgnt{f}\ \cfgt{:=}\ \cfgt{*}\ \rightarrow\ \cfgnt{k}\rp \rp  \rightarrow_\mathit{FW}
      \lp \cfgnt{L}\ \cfgnt{R}\ \phi_g\ \eta\ \cfgt{error}\ \cfgt{end}\rp
	}	
\end{mathpar}}
%\end{center}
%\caption{Precise symbolic heap summaries from symbolic execution indicated by $\rsym = \rightarrow_\mathit{FA} \cup \rightarrow_\mathit{FW} \cup \rightarrow_\mathit{EQ} \cup \rcom$.}
%\label{fig:symfield}
%\end{figure}


\begin{figure*}[t]
\begin{center}
\setlength{\tabcolsep}{50pt}
\begin{tabular}[c]{c}
\usebox{\boxPFAFW} \\
\end{tabular}
\end{center}
\caption{FIXME: Write the caption}
\label{fig:fHeap}
\end{figure*}

\newsavebox{\boxPEQ}
\savebox{\boxPEQ}{
%\begin{figure}[t]
%\begin{center}
\mprset{flushleft}
\begin{tabular}[c]{c}
\begin{mathpar}
    \inferrule[Equals (references-true)]{
    \Phi_\alpha = \{\lp\phi_0 \wedge \phi_1\rp \mid \exists l\ \lp \lp \phi_0\ l\rp  \in \cfgnt{L}\lp \cfgnt{r}_0\rp  \wedge \lp \phi_1\ l\rp  \in \cfgnt{L}\lp \cfgnt{r}_1\rp \rp \} \\\\
    \Phi_0 = \{\phi_0 \mid \exists l_0\ \lp \lp \phi_0\ l_0\rp  \in \cfgnt{L}\lp \cfgnt{r}_0\rp  \wedge \forall \lp \phi_1\ l_1\rp  \in \cfgnt{L}\lp \cfgnt{r}_1\rp \ \lp l_0 \neq l_1\rp \rp \} \\\\
    \Phi_1 = \{\phi_1 \mid \exists l_1\ \lp \lp \phi_1\ l_1\rp  \in \cfgnt{L}\lp \cfgnt{r}_1\rp  \wedge \forall \lp \phi_0\ l_0\rp  \in \cfgnt{L}\lp \cfgnt{r}_0\rp \ \lp l_0 \neq l_1\rp \rp \} \\\\
    \phi^\prime =  \phi \wedge \lp \vee_{\phi_\alpha\in\Phi_\alpha}\phi_\alpha\rp \wedge\lp \wedge_{\phi_0 \in \Phi_0} \neg \phi_0\rp \wedge\lp \wedge_{\phi_1
    \in \Phi_1} \neg \phi_1\rp \\\\ 
    \mathbb{S}(\phi^\prime)}{
    \lp \cfgnt{L}\ \cfgnt{R}\ \phi\ \eta\ \cfgnt{r}_0\ \lp \cfgnt{r}_1\; \cfgt{=}\; \cfgt{*} \rightarrow \cfgnt{k}\rp \rp  \rsym^\mathit{E}
    \lp \cfgnt{L}\ \cfgnt{R}\ \phi^\prime\ \eta\ \cfgt{true}\ \cfgnt{k}\rp 
    }
%\and
%    \inferrule[Equals (references-false)]{
%    \Phi_\alpha = \{\lp\phi_0 \Rightarrow \neg \phi_1\rp \mid \exists l\ \lp \lp \phi_0\ l\rp  \in \cfgnt{L}\lp \cfgnt{r}_0\rp  \wedge \lp \phi_1\ l\rp  \in \cfgnt{L}\lp \cfgnt{r}_1\rp \rp \} \\\\
%    \Phi_0 = \{\phi_0 \mid \exists l_0\ \lp \lp \phi_0\ l_0\rp  \in \cfgnt{L}\lp \cfgnt{r}_0\rp  \wedge \forall \lp \phi_1\ l_1\rp  \in \cfgnt{L}\lp \cfgnt{r}_1\rp \ \lp l_0 \neq l_1\rp \rp \} \\\\
%    \Phi_1 = \{\phi_1 \mid \exists l_1\ \lp \lp \phi_1\ l_1\rp  \in \cfgnt{L}\lp \cfgnt{r}_1\rp  \wedge \forall \lp \phi_0\ l_0\rp  \in \cfgnt{L}\lp \cfgnt{r}_0\rp \ \lp l_0 \neq l_1\rp \rp \} \\\\
%    \phi^\prime = \phi \wedge \lp \wedge_{\phi_\alpha\in\theta_\alpha}\phi_\alpha\rp \vee\lp \lp \vee_{\phi_0 \in \theta_0} \phi_0\rp   \vee\lp \vee_{\phi_1
%    \in \theta_1} \phi_1\rp \rp  \\\\ 
%    \mathbb{S}(\phi^\prime)}{
%    \lp \cfgnt{L}\ \cfgnt{R}\ \phi\ \eta\ \cfgnt{r}_0\ \lp \cfgnt{r}_1\; \cfgt{%=}\; \cfgt{*} \rightarrow \cfgnt{k}\rp \rp  \rsym^\mathit{E^\prime}
%    \lp \cfgnt{L}\ \cfgnt{R}\ \phi^\prime\ \eta\ \cfgt{false}\ \cfgnt{k}\rp 
%    }
\end{mathpar}
\end{tabular}}
%\end{center}
%\caption{FIX THIS CAPTION AND MOVE $\rsym$ DEFINITION (MAY NOT BE NEEDED BECAUSE OF \defref{def:meta}: Precise symbolic heap summaries from symbolic execution indicated by $\rsym = \rightarrow_\mathit{FA} \cup \rightarrow_\mathit{FW} \cup \rightarrow_\mathit{EQ}^T \cup \rightarrow_\mathit{EQ}^F \cup \rcom$.}
%\label{fig:symeq}
%\end{figure}


\begin{figure*}
\begin{tabular}[c]{c}
\usebox{\boxPEQ} \\
\end{tabular}
\caption{equals true for this.x == this.y}
\label{fig:eqs}
\end{figure*}



\end{document}

