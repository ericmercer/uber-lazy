\section{Evaluation}
\label{sec:evaluation}
%\begin{figure*}[t]
\begin{table*} [t]
  \centering
  \caption{Comparing \symtxt{} with the~\gsetxt{} and Lazier\# algorithms.}
  \scalebox{0.88}{\begin{tabular}{| c | c | r | r | r | r | r | r | r | r | r |}
  \hline
   \multirow{2}{*}{Method }&\multirow{2}{*}{ $k$ }
   &\multicolumn{3}{c|}{Time (seconds)} &\multicolumn{3}{c|}{States} &\multicolumn{3}{c|}{ Paths }\\
								&	&\gsetxt{} & L\#		&SL		&\gsetxt{}	& L\# & SL&\gsetxt{}	& L\# 	& SL\\
   \hline
    \multirow{3}{*}{LinkedList }			&3	& 0.91	& 1.21	& 0.69	& 2465	& 2844	& 99		&1656	& 1269	& 25\\
   		 						& 4	& 2.92	& 3.35	& 0.91	& 25774	& 29977	& 155	&17485	& 13550	& 39\\
   								& 5	& 20.78	& 19.47	& 1.59	& 341164	& 400296	& 223	&232743	& 181849	& 56\\
								& 6	& 280.56	& 299.19	& 2.36	&5447980	&6437201	& 303	&3731094	&2933027	& 76\\
								& 7	& -		& -		& 5.07	& -		&-		& 395	&-		&-		& 99\\
								& 8	& -		& -		& 17.49	& -		&-		& 499	&-		&-		& 125\\
								& 9	& -		& -		& 63.96	& -		&-		& 615	&-		&-		& 154\\
								& 10	& -		& -		& 206.93	& -		&-		& 743	&-		&-		& 186\\
    \hline
    \multirow{3}{*}{BinarySearchTree }	& 1	& 0.26	& 0.28	& 0.36	& 19		& 23		& 29		& 6		& 6		& 6\\
   		 						& 2	& 0.83	& 1.28	& 0.93	& 143	& 143	& 145	& 43		& 42		& 33\\
   								& 3	& 20.63	& 25.55	& 4.03	& 1953	& 1703	& 1485	& 515	& 515	& 328\\
								& 4	& -		& -		& 410.89	& -		&-		& 73635	&-		&-		& 15563\\
    \hline
      \multirow{3}{*}{TreeMap}			& 1	& 0.47	& 0.52	& 0.77	& 65		& 70		& 215	& 11		& 11		& 11\\
   		 						& 2	& 8.99	& 9.73	& 4.72	& 1009	& 942	& 3219	& 127	& 122	& 73\\
   								& 3	& -		& -		& 145.56	& -		& -		& 78695	& -		& -		& 887\\
						
    \hline
  \end{tabular}}

  \label{tab:results}
\end{table*}

%\end{figure*}

%\subsection{Analysis}
The \symtxt{} algorithm is implemented as an extension to the Symbolic
PathFinder (SPF)~\cite{DBLP:journals/ase/PasareanuVBGMR13}
framework. In addition to the operations presented in this paper, the
implementation contains support for operations over integers,
calculating per-path preconditions and postconditions, as well as
generating test input heaps that exercise all feasible control flow
paths. We are currently working on adding support for floating point
operations, arrays, and bit-operations. The \symtxt{} implementation
uses the \texttt{jConstraints}
library~\cite{ase2014-ghilrr,jpf2014-dghirr} to encode the heap
constraints and the z3 solver~\cite{deMouraBjorner08Z3} as the
underlying constraint solver. We leverage the incremental solving
feature in z3 in our experiments. We also implement other
optimizations that cache and re-use solutions to previous
constraints. These optimizations are an important part of our
implementation because only small portions of the heap constraint
change during execution.

SPF includes an implementation of~\gsetxt{} with lazy initialization.
In recent work, we implemented the Lazier and Lazier\# algorithms in
SPF~\cite{Hillery:2014}; these constitute the state of art approaches
case-splitting based lazy initialization techniques. The goal of our
experiment is to evaluate the efficacy of our approach with respect to
other state-of-the-art techniques for symbolic execution of programs
with unbounded complex data input. We empirically evaluate the
\symtxt{} algorithm by considering the following research question:
How does the cost of the \symtxt{} algorithm (SL) compare with that of
\gsetxt{} (GSE), Lazier, Lazier\#(L\#) algorithms?

The \emph{independent} variable in our study is the $k$-bound;
$k$-bounding bounds the length of a reference chain from the root of
the heap~\cite{Kiasan06}.  We select three dependent variables and
measures for our study: (i) time, (ii) states explored, and (iii)
paths generated. The \emph{time} is the total time taken by each
algorithm to explore the symbolic execution tree as well as total
constraint solving time. The \emph{states explored} represents the
number of nodes in the symbolic execution tree, and the \emph{paths
  generated} represents the number of unique paths in the symbolic
execution tree.

The data structures evaluated are a standard set commonly used in
analyses involving heap-manipulating
programs~\cite{Deng:2006,Deng:2007,boyapati2002korat,Ferrara:2014,Rosner:2015},
including a linked list, binary search tree, and red/black tree. We
use a repOk() method (a class invariant) for data structures in
object-oriented code~\cite{boyapati2002korat}. The repOk() methods
provide a mechanism to generate valid inputs for the methods under
test; symbolic execution has been used to generate test inputs using
repOk() methods in~\cite{visser2004test}. In this work, we define and
use repOk() methods to generate valid heaps with the appropriate
$k$-bounds before executing the methods under test. Note that this
allows us to have precision in checking properties of the heap that is
not possible in static analysis based techniques~\cite{Dillig:2011}.

The results of the experiments are presented in
Table~\ref{tab:results}.  Each row reports the results for the
specified $k$-bound for each artifact evaluated. The columns show the
total time in seconds, states explored, and paths generated for each
algorithm. The headings \gsetxt{}, L\#, and SL correspond to the GSE,
Lazier\#, and \symtxt{} algorithms, respectively. The results for the
Lazier algorithm are omitted because they are nearly identical to the
Lazier\# results for these artifacts. A table entry of '-' indicates
the analysis exceeded the allotted time bound of 1 hour.

The number of possible non-isomorphic heap configurations grows
exponentially for case-splitting techniques with a monotonic increase
in $k$, resulting a corresponding exponential increase in analysis
times . This is event in all the examples in
Table~\ref{tab:results}. The GSE and L\# algorithms are unable to
finish exploration in a time bound of one hour for $k\geq 7$ for the
LinkedList, $k\geq 4$ for the BinarySearchTree, and $k\geq 3$ for the
TreeMap examples. The improvement in analysis time for the \symtxt{}
over the state of the art case splitting techniques range from $4.8x$
for BinarySearchTree at $k$=3, to $118x$ for LinkedList at $k$=6. In
fact, for some $k$-bounds a number of experiments complete exploration
using the \symtxt{} algorithm in a few seconds whereas \gsetxt{} or
Lazier\# are unable to finish, e.g., BinarySearchTree for $k$=4 and
LinkedList for $k$=8. 

The number of path explored by \symtxt{} algorithm are strictly less
than or equal to the number of paths explored by \gsetxt{} for all the
evaluated artifacts. We can use this result to do more efficient test
case generation. From a generated path and corresponding symbolic heap
we can use the solutions provided by the constraint solver to
instantiate a set of concrete heaps to provide as test input. These
provide a smaller test suite to achieve control-flow path coverage as
compared to GSE and Lazier\#. Note that the ability to perform test
input generation is another advantage of \symtxt{} technique over
static analysis approaches.

The number of states varies between algorithms, for example, \gsetxt{}
has additional points of nondeterminism during field reads, but in
contrast reference compares are completely deterministic. Thus, in
example programs with large numbers of reference compares, such as
TreeMap, state counts for Lazier\# and the \symtxt{} algorithm may
exceed those for \gsetxt{}. We observed that the additional states
generated by the summary heap algorithm are unsatisfiable at the point
of reference compares; this is why they do not contribute any
additional branches in the final symbolic execution tree. Furthermore,
the larger state count and the corresponding satisfiability checks on
the constraint solver does not increase the overall runtime of the
technique.

In summary, the benefits of avoiding case-splitting based
non-determinism outweigh the increased complexity in the constraints
over heaps due to the advances made in SMT solvers in these
examples. In fact, the \symtxt{} algorithm can analyze certain types
of programs with orders of magnitude greater efficiency than with
\gsetxt{} or Lazier\#, while covering exactly the same feasible
control flow paths in the program.



