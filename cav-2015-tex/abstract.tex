A fundamental challenge of using symbolic execution for software analysis 
has been the treatment of dynamically allocated data. The main problems 
center on dereferencing symbolic data. When dereferencing symbolic values 
from an unconstrained symbolic input heap, existing methods resort to either 
approximation or case splitting. Since both options negate significant 
advantages of symbolic execution techniques, finding a definitive symbolic 
heap representation has been an open problem. This paper introduces a 
new technique for representing and manipulating symbolic heaps that (i) 
reasons over arbitrary, unbounded, and otherwise unconstrained symbolic 
input values, and (ii) produces a path condition that exactly represents all 
possible program values for a given execution path. The symbolic heap 
representation is the first to produce path conditions that are provably sound 
and complete with respect to all program properties verifiable through 
symbolic execution, without any restrictions or simplifying assumptions. An 
evaluation of a proof-of-concept implementation in the Java Pathfinder 
framework is presented, demonstrating the computational feasibility of the 
approach.

%A fundamental challenge of using symbolic execution for software
%analysis has been the treatment of dynamically allocated
%data. Generalized symbolic execution (GSE) techniques have addressed
%this challenge by materializing a concrete heap lazily during field
%accesses.  These techniques however, exacerbate the path explosion
%problem.  In this work, we present a summary heap approach which takes
%inspiration from the static analysis domain and defines a bipartite
%graph to encode multiple heaps in single representation, which, unlike
%state-of-the-art static analysis approaches, (i) supports aliasing,
%(ii) does not require rewrite of constraints on heap updates, and
%(iii) can initialize the heap-on-the-fly. We also define a summary
%heap algorithm to perform heap updates during symbolic execution on
%this representation in order to mitigate the path explosion problem in
%GSE techniques. Our approach is sound and complete with respect to the
%properties provable by GSE. A preliminary evaluation of our
%implementation in Java Pathfinder framework shows the summary heap
%enables the initialization of larger, more complex heaps than
%previously possible.
