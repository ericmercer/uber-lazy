\section{System State}
The program state is represented using a path condition, a program location, a stack, a symbolic heap, and a symbolic store. The path condition is a collection of predicates over the program inputs that indicates constraints on the values of those inputs at the present location. The program location indicates the instruction to be executed presently. The symbolic heap is a mapping from heap locations symbolic objects. 
\paragraph{Heap Symbols}
A heap symbol $\pi$ is a symbol which is created in the process of performing heap operations. The dynamic nature of heap symbols distinguishes them from statically-created symbols like those found in the path condition. Heap symbols may be symbolic primitives, symbolic references, symbolic locations, or symbolic types.
\paragraph{Symbolic References}
Symbolic references are symbolic points-to relations. The symbolic reference has a state parameter that indicates whether the reference is uninitialized, non-null, or initialized. The uninitialized and non-null states indicate that the location pointed to by the reference has yet to be resolved. The non-null state reflects the additional constraint that the reference does not point to the null location. References in the initialized state are associated with constraints reflecting which location the reference points to and under what condition it points to that location. References point to one and only one location at a time, so the conditions must be mutually exclusive, yet collectively exhaustive.
\paragraph{Symbolic Locations}
A symbolic location $h$ represents the index of a slot in the heap structure that holds a symbolic object. Each symbolic location represents a unique slot on the symbolic heap. There are two special symbolic locations null and non-null, which are never in the symbolic heap.
\paragraph{Symbolic Objects}
A symbolic object represents the contents of a heap slot. A symbolic object is characterized by a symbolic type, and contains a mapping from fields to heap symbols.
\paragraph{Constrained Symbol}
A constrained symbol $\psi$ is a symbol that assumes a value $\pi$ contingent upon the satisfiability of a constraint condition $\phi$.
\begin{equation}
\psi\models \phi\uparrow \Rightarrow\pi
\end{equation}
\paragraph{Symbolic Value Set}
A symbolic value set $\theta$ is a set of constrained symbols $\theta\colon \{\psi _1,\psi _2,...,\psi _n\}$, for which the constraints are mutually exclusive and collectively exhaustive.
\paragraph{Symbolic Store}
The symbolic store is a mapping from heap symbols to symbolic value sets $\mathbb{S}\colon \pi \mapsto \theta$. The abstract store represents a set of heaps common to the current program execution path. 
\paragraph{Heap}
The symbolic heap $\mathbb{H}$ is an indexed set of symbolic objects. Like the path condition, the symbolic heap contains heap state which is common to all heaps on the current execution path. 
\paragraph{Stack}
The Java Virtual Machine uses a system stack.