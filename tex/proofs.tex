\section{Proofs}

\subsection{Definitions}

\begin{definition}
A \textbf{state transition function} $\rightarrow_{\Phi}$ is a mapping $\rightarrow_{\Phi} : s \mapsto s$ , which takes one machine state and transforms it into another machine state. The state transition function with superscript represents composition of the state transition function: $$ s_a \rightarrow_{\Phi} s_b \rightarrow_{\Phi} s_c \implies s_a \rightarrow_{\Phi}^2 s_c $$
\end{definition}

\begin{definition}
A \textbf{feasible state sequence} is defined as a sequence of states resulting from repeated application of the state transition relation to some initial state $s_0$: $$\Pi_n = s_0,s_1,...,s_n$$ where the relation $s_i \rightarrow_{\Phi} s_{i+1}$ holds for all $i \in \{ i | 0 \leq i < n \}$
\end{definition}

\begin{definition}
Given a sequence of states $$\Pi_n = s_0,s_1,...,s_n$$ where $$s_i = ( \mu_i\ \cfgnt{L}_i\ \cfgnt{R}_i\ \phi_i\ \eta_i\ \cfgnt{e}_i\ \cfgnt{k}_i )$$ the \textbf{control flow sequence} of $\Pi_n$ is the defined as the sequence of tuples $$ \pi_n = \mathbb{CF}(\Pi_n) = (\eta_0\ \cfgnt{e}_0\ \cfgnt{k}_0),(\eta_1\ \cfgnt{e}_1\ \cfgnt{k}_1),...,(\eta_n\ \cfgnt{e}_n\ \cfgnt{k}_n)$$
\end{definition}

\begin{definition}
Given a state transition function $\rightarrow_{\Phi}$, an initial state $s_0$ and a control flow sequence $\pi_n$, the \textbf{feasible state set} $\zeta = \mathbb{FS}(\rightarrow_{\Phi},s_0,\pi_n)$ is defined as
 $$\zeta = \{ \forall s | \pi_n = \mathbb{CF}(\Pi_n) \wedge s = max_s(\Pi) \wedge s_0 \rightarrow_{\Phi}^{n-1} s\} $$
\end{definition}

\begin{definition}
A \textbf{heap homomorphism} $(g\ h)$ between two states $s_x$ and $s_y$ is defined as a pair of functions $g:\cfgnt{r} \mapsto \cfgnt{r}$ and $h:\cfgnt{l} \mapsto \cfgnt{l}$ such that for any reference $r \in s_x$, location $l \in s_x$, and field $f$, $$ l \in \cfgnt{L}_x(r) \Leftrightarrow h(l)\in \cfgnt{L}_y(g(r))$$ and $$ r = \cfgnt{R}_x(l,f) \Leftrightarrow g(r) = \cfgnt{R}_y(h(l),f)$$ If a such a pair of functions exists from one state $s_x$ to another state $s_y$ we say that state $s_x$ is \textbf{heap homomorphic} to state $s_y$, indicated by the notation $(g\ f):\ s_x \rightarrow s_y$. 

Suppose we take the set of constraints from the image of $s_x$ in $s_y$ under $(g\ h)$ :
$$\chi = \{ \phi\ | \exists (r \in \cfgnt{R}_x,\  l \in \cfgnt{L}_x) ( (\phi\ h(l)) \in \cfgnt{L}_y (g(r))  \}$$ and we conjoin those constraints with the global invariant from $s_y$ :
$$\phi_G = \phi_y \bigwedge_{\phi_i \in \chi} \phi_i $$
 If the expression $\phi_G$ is satisfiable, we say that the heap homomorphism $(g\ h):\ s_x \rightarrow s_y $ is \textbf{valid}.
\end{definition}
\begin{definition}
The \textbf{representation relation} is defined as follows: given two states $s_\ell$ and $s_s$, $s_\ell \sqsubset s_s $ if and only if $\eta_{\ell} = \eta_{s} ,\ \cfgnt{e}_{\ell} = \cfgnt{e}_{s} ,\ \cfgnt{k}_{\ell} = \cfgnt{k}_{s}$, and there exists a valid heap homomorphism $(g\ h):\ s_\ell \rightarrow s_s $. The expression $s_\ell \sqsubset s_s $ can be read as "state $s_\ell$ is represented by state $s_s$ . 
\end{definition}

\begin{definition}
A state $s_s$ is \textbf{congruent} to a set of states $\mathcal{S}$ if and only if $s$ represents every state in $\mathcal{S}$ and represents no other state: 
$$ s_s \equiv \mathcal{S} : s_i \in \mathcal{S} \Leftrightarrow s_i \sqsubset s_s $$
\end{definition}

\begin{definition}
A symbolic state $s_s$ is \textbf{exact} with respect to an initial state $s_0$ and control flow sequence $\pi$ if and only if it is congruent the set of feasible lazy states on the same control flow path:
$$ s_s \equiv \mathbb{FS}(\rightarrow_{\ell},s_0,\pi)$$
\end{definition}

\subsection{Theorems}

\begin{lemma}
If symbolic state $s_s = \lp \cfgnt{L}_{\mathcal{S}}\ \cfgnt{R}_{\mathcal{S}}\ \phi_g\ \eta\ \cfgnt{r}\ \lp \cfgt{*}\ \cfgt{\$}\ \cfgnt{f} \rightarrow \cfgnt{k}\rp \rp$ is exact with respect to some initial state $s_0$ and control flow path $\pi_n$, then the state $s_s^\prime : s_s \rightarrow_s s_s^\prime$ is exact with respect to $s_0$ and $\pi_{n+1}.
\end{lemma}

\begin{proof}
We will consider two cases for this proof. In the first case, we assume that all of the fields involved in the read are initialized. In the second case we consider the case of uninitialized fields. 

Case 1: suppose all of the pertinent fields in $s_s$ are initialized. Take an arbitrary lazy state $s_\ell \sqsubset s_s$. Since $s_s$ is exact,  $s_\ell = \lp \cfgnt{L}_{\ell}\ \cfgnt{R}_{\ell}\ \phi_L\ \eta\ \cfgnt{r}\ \lp \cfgt{*}\ \cfgt{\$}\ \cfgnt{f} \rightarrow \cfgnt{k} \rp \rp$. If we apply the state transition functions to achieve states $s_\ell^\prime : s_\ell \rightarrow_\ell s_\ell^\prime$ and $s_s^\prime : s_s \rightarrow_s s_s^\prime$, we find that:
$$s_\ell^\prime = \lp \cfgnt{L}_{\ell} [\cfgnt{r}^\prime \mapsto \lp\phi^\prime\ l^\prime\rp]\ \cfgnt{R}_{\ell}\ \phi_L\ \eta\ \cfgnt{r}^\prime\ \cfgnt{k}\rp $$
 and 
 $$ s_s^\prime = \lp \cfgnt{L}_{s}[\cfgnt{r}^\prime \mapsto \mathbb{VS}\lp \cfgnt{L}_{\mathcal{S}},\cfgnt{R}_{s},\cfgnt{r},\cfgnt{f},\phi_g\rp ]\ \cfgnt{R}_{\mathcal{S}}\ \phi_g\ \eta\ \cfgnt{r}^\prime\ \cfgnt{k}\rp $$

We now show that $s_\ell^\prime \sqsubset s_s^\prime$. Since $\eta$, $e$, and $k$ are identical between $s_s^\prime$ and $s_\ell^\prime $, the first condition is met by default. To find a valid heap homomorphism, first construct functions $g^\prime : g^\prime = g[ r^\prime \mapsto r^\prime]$ and $h^\prime : h^\prime = h$. 

%we now need to show that we have a heap homomorphism here, and that the homomorphism is valid. once we've done that , we've proved the soundness case

Now, we show that no infeasible lazy states are allowed by $\mathcal{S}^\prime$. NEED TO FILL IN THIS BIT, TOO

$\mathbb{S}(\bigwedge_{(\phi l) \in L_s^i} \phi)) \wedge \forall l \in L_L (\exists l \in L_s^i)$

Since $s_\ell \subseteq s_s$
\end{proof}


