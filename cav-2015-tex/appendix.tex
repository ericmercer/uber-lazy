\section*{Appendix: Common Rules for Javalite}
\begin{figure*}[t]
\begin{center}
\mprset{flushleft}
\begin{mathpar}
	\inferrule[Variable lookup]{}{
      \lp \cfgnt{L}\ \cfgnt{R}\ \phi\ \eta\ \cfgnt{x}\ \cfgnt{k}\rp  \rcom \\\\
      \lp \cfgnt{L}\ \cfgnt{R}\ \phi\ \eta\ \eta\lp \cfgnt{x}\rp \ \cfgnt{k}\rp 
	}
\and
	\inferrule[New]{
      \cfgnt{r} = \mathrm{stack}_r\lp \rp\\
      l = \mathrm{fresh}_l\lp \cfgnt{C}\rp\\\\
      \cfgnt{R}^\prime = \cfgnt{R}[\forall \cfgnt{f} \in \mathit{fields}\lp \mathrm{C}\rp \ \lp \lp l\ \cfgnt{f}\rp  \mapsto \cfgnt{r}_\mathit{null} \rp ] \\\\
      \cfgnt{L}^\prime = \cfgnt{L}[\cfgnt{r} \mapsto \{\lp \cfgt{true}\ l\rp \}]
    }{
      \lp \cfgnt{L}\ \cfgnt{R}\ \phi\ \eta\ \lp \cfgt{new}\ \cfgnt{C}\rp \ \cfgnt{k}\rp  \rcom
      \lp \cfgnt{L}^\prime\ \cfgnt{R}^\prime\ \phi\ \eta\ \cfgnt{r}\ \cfgnt{k}\rp 
	}
\and
	\inferrule[Field Access(eval)]{}{
      \lp \cfgnt{L}\ \cfgnt{R}\ \phi\ \eta\ \lp \cfgnt{e}\ \cfgt{\$}\ \cfgnt{f}\rp \ \cfgnt{k}\rp  \rcom \\\\
      \lp \cfgnt{L}\ \cfgnt{R}\ \phi\ \eta\ \cfgnt{e}\ \lp \cfgt{*}\ \cfgt{\$}\ \cfgnt{f} \rightarrow \cfgnt{k}\rp \rp 
	}
\and
	\inferrule[Field Write (eval)]{}{
       \lp \cfgnt{L}\ \cfgnt{R}\ \phi\ \eta\ \lp \cfgnt{x}\ \cfgt{\$}\ \cfgnt{f}\ \cfgt{:=}\ \cfgnt{e}\rp \ \cfgnt{k}\rp  \rcom \\\\
       \lp \cfgnt{L}\ \cfgnt{R}\ \phi\ \eta\ \cfgnt{e}\ \lp \cfgnt{x}\ \cfgt{\$}\ \cfgnt{f}\ \cfgt{:=}\ \cfgt{*}\ \rightarrow\ \cfgnt{k}\rp \rp 
	}
\and
    \inferrule[Equals (l-operand eval)]{}{
      \lp \cfgnt{L}\ \cfgnt{R}\ \phi\ \eta\ \lp \cfgnt{e}_0\ \cfgt{=}\ \cfgnt{e}\rp  \ \cfgnt{k}\rp  \rcom \\\\
      \lp \cfgnt{L}\ \cfgnt{R}\ \phi\ \eta\ \cfgnt{e}_0\ \lp \cfgt{*}\ \cfgt{=}\; \cfgnt{e} \rightarrow \cfgnt{k}\rp \rp 
    }
\and
    \inferrule[Equals (r-operand eval)]{}{
    \lp \cfgnt{L}\ \cfgnt{R}\ \phi\ \eta\ \cfgnt{v}\ \lp \cfgt{*}\; \cfgt{=}\; \cfgnt{e} \rightarrow \cfgnt{k}\rp \rp  \rcom \\\\
    \lp \cfgnt{L}\ \cfgnt{R}\ \phi\ \eta\ \cfgnt{e}\ \lp \cfgnt{v}\; \cfgt{=}\; \cfgt{*} \rightarrow \cfgnt{k}\rp \rp 
    }
\and
    \inferrule[Equals (bool)]{
    \cfgnt{v}_0 \in \{\cfgt{true}, \cfgt{false}\} \\
    \cfgnt{v}_1 \in \{\cfgt{true}, \cfgt{false}\} \\\\
    \cfgnt{v}_r = \mathrm{eq?}\lp \cfgnt{v}_0, \cfgnt{v}_1\rp}{
    \lp \cfgnt{L}\ \cfgnt{R}\ \phi\ \eta\ \cfgnt{v}_0\ \lp \cfgnt{v}_1\; \cfgt{=}\; \cfgt{*} \rightarrow \cfgnt{k}\rp \rp  \rcom \\\\
    \lp \cfgnt{L}\ \cfgnt{R}\ \phi\ \eta\ \cfgnt{v}_r\ \cfgnt{k}\rp 
    }
\and
    \inferrule[If-then-else (eval)]{}{
      \lp \cfgnt{L}\ \cfgnt{R}\ \phi\ \eta\ \lp \cfgt{if}\ \cfgnt{e}_0\ \cfgnt{e}_1\ \cfgt{else}\ \cfgnt{e}_2\rp \ \cfgnt{k}\rp  \rcom \\\\
      \lp \cfgnt{L}\ \cfgnt{R}\ \phi\ \eta\ \cfgnt{e}_0\ \lp \cfgt{if}\ \cfgt{*}\ \cfgnt{e}_1\ \cfgt{else}\ \cfgnt{e}_2\rp  \rightarrow \cfgnt{k}\rp 
	}
\and
	\inferrule[If-then-else (true) ]{}{
       \lp \cfgnt{L}\ \cfgnt{R}\ \phi\ \eta\ \cfgt{true}\ \lp \cfgt{if}\ \cfgt{*}\ \cfgnt{e}_1\ \cfgt{else}\ \cfgnt{e}_2\rp  \rcom \cfgnt{k}\rp  \rightarrow \\\\
       \lp \cfgnt{L}\ \cfgnt{R}\ \phi\ \eta\ \cfgnt{e}_1\  \cfgnt{k}\rp 
	}
\and
	\inferrule[If-then-else (false)]{}{
       \lp \cfgnt{L}\ \cfgnt{R}\ \phi\ \eta\ \cfgt{false}\ \lp \cfgt{if}\ \cfgt{*}\ \cfgnt{e}_1\ \cfgt{else}\ \cfgnt{e}_2\rp  \rcom \cfgnt{k}\rp  \rightarrow \\\\
       \lp \cfgnt{L}\ \cfgnt{R}\ \phi\ \eta\ \cfgnt{e}_2\  \cfgnt{k}\rp 
	}
\and
   \inferrule[Variable Declaration (eval)]{}{
    \lp \cfgnt{L}\ \cfgnt{R}\ \phi\ \eta\ \lp\cfgt{var}\ \cfgnt{T}\ \cfgnt{x}\ \cfgt{:=}\ \cfgnt{e}_0\ \cfgt{in}\ \cfgnt{e}_1\rp\ \cfgnt{k}\rp  \rcom \\\\
    \lp \cfgnt{L}\ \cfgnt{R}\ \phi\ \eta\ \cfgnt{e}_0\ \lp\cfgt{var}\ \cfgnt{T}\ \cfgnt{x}\ \cfgt{:=}\ \cfgt{*}\ \cfgt{in}\ \cfgnt{e}_1 \rightarrow \cfgnt{k}\rp\rp 
   }	
\and
   \inferrule[Variable Declaration]{}{
    \lp \cfgnt{L}\ \cfgnt{R}\ \phi\ \eta\ \cfgnt{v}\ \lp\cfgt{var}\ \cfgnt{T}\ \cfgnt{x}\ \cfgt{*}\ \cfgt{:=}\ \cfgt{in}\ \cfgnt{e}_1 \rightarrow \cfgnt{k}\rp\rp  \rcom \\\\
    \lp \cfgnt{L}\ \cfgnt{R}\ \phi\ \eta[x \mapsto \cfgnt{v}]\ \cfgnt{e}_1\ \lp \cfgt{pop}\ \eta\ \cfgnt{k}\rp \rp 
   }	
\and
   \inferrule[Method Invocation (object eval)]{}{
    \lp \cfgnt{L}\ \cfgnt{R}\ \phi\ \eta\ \lp\cfgnt{e}_0\ \cfgt{@}\ \cfgnt{m}\ \cfgnt{e}_1\rp\ \cfgnt{k}\rp  \rcom \\\\
    \lp \cfgnt{L}\ \cfgnt{R}\ \phi\ \eta\ \cfgnt{e}_0\ \lp \cfgt{*}\ \cfgt{@}\ \cfgnt{m}\ \cfgnt{e}_1\ \rightarrow \cfgnt{k}\rp \rp 
   }
\and
   \inferrule[Method Invocation (arg eval)]{}{
    \lp \cfgnt{L}\ \cfgnt{R}\ \phi\ \eta\ \cfgnt{v}_0\ \lp \cfgt{*}\ \cfgt{@}\ \cfgnt{m}\ \cfgnt{e}_1\ \rightarrow \cfgnt{k}\rp \rp  \rcom \\\\
    \lp \cfgnt{L}\ \cfgnt{R}\ \phi\ \eta\ \cfgnt{e}_1\ \lp \cfgnt{v}_0\ \cfgt{@}\ \cfgnt{m}\ \cfgt{*}\ \rightarrow \cfgnt{k}\rp \rp 
   }
\and
   \inferrule[Method Invocation]{
    \cfgnt{C} = \mathrm{type}\lp\cfgnt{r}\rp \\\\
    \lp\cfgnt{T}_o\ \cfgnt{m}\ \lb\cfgnt{T}_1\ \cfgnt{x}\rb\ \ \cfgnt{e}_m\rp = \mathrm{lookup}\lp \cfgnt{C},\cfgnt{m}\rp\\\\
    \eta_m = \eta[\cfgt{this} \mapsto \cfgnt{r}][\cfgnt{x} \mapsto \cfgnt{v}]}{
    \lp \cfgnt{L}\ \cfgnt{R}\ \phi\ \eta\ \cfgnt{v}\ \lp \cfgnt{r}\ \cfgt{@}\ \cfgnt{m}\ \cfgt{*}\ \rightarrow \cfgnt{k}\rp \rp  \rcom \\\\
    \lp \cfgnt{L}\ \cfgnt{R}\ \phi\ \eta_m\ \cfgnt{e}_m\ \lp \cfgt{pop}\ \eta\ \cfgnt{k}\rp \rp 
   }
\and
   \inferrule[Variable Assignment (eval)]{}{
    \lp \cfgnt{L}\ \cfgnt{R}\ \phi\ \eta\ \lp \cfgnt{x}\ \cfgt{:=}\ \cfgnt{e}\rp \ \cfgnt{k}\rp  \rcom \\\\
    \lp \cfgnt{L}\ \cfgnt{R}\ \phi\ \eta\ \cfgnt{e}\ \lp \cfgnt{x}\ \cfgt{:=}\ \cfgt{*} \rightarrow\ \cfgnt{k}\rp \rp 
   }	
\and
   \inferrule[Variable Assignment]{}{
    \lp \cfgnt{L}\ \cfgnt{R}\ \phi\ \eta\ \cfgnt{v}\ \lp \cfgnt{x}\ \cfgt{:=}\ \cfgt{*} \rightarrow\ \cfgnt{k}\rp \rp  \rcom \\\\
    \lp \cfgnt{L}\ \cfgnt{R}\ \phi\ \eta[\cfgnt{x} \mapsto \cfgnt{v}]\ \cfgnt{v}\ \cfgnt{k}\rp 
   }	
\and
   \inferrule[Begin (no args)]{}{
    \lp \cfgnt{L}\ \cfgnt{R}\ \phi\ \eta\ \lp \cfgt{begin}\rp \ \cfgnt{k}\rp  \rightarrow \\\\
    \lp \cfgnt{L}\ \cfgnt{R}\ \phi\ \eta\ \cfgnt{k}\rp 
   }
\and
   \inferrule[Begin (arg0 eval)]{}{
    \lp \cfgnt{L}\ \cfgnt{R}\ \phi\ \eta\ \lp \cfgt{begin}\ \cfgnt{e}_0\ \cfgnt{e}_1\ ...\rp \ \cfgnt{k}\rp  \rcom \\\\
    \lp \cfgnt{L}\ \cfgnt{R}\ \phi\ \eta\ \cfgnt{e}_0\ \lp \cfgt{begin}\ \cfgt{*}\ \lp\cfgnt{e}_1\ ...\rp \rightarrow \cfgnt{k}\rp \rp 
   }
\and
   \inferrule[Begin (argi eval)]{}{
    \lp \cfgnt{L}\ \cfgnt{R}\ \phi\ \eta\ \cfgnt{v}\ \lp \cfgt{begin}\ \cfgt{*}\ \lp\cfgnt{e}_i\ \cfgnt{e}_{i+1}\ ...\rp \rightarrow \cfgnt{k}\rp \rp  \rcom \\\\
    \lp \cfgnt{L}\ \cfgnt{R}\ \phi\ \eta\ \cfgnt{e}_i\ \lp \cfgt{begin}\ \cfgt{*}\ \lp\cfgnt{e}_{i+1}\ ...\rp \rightarrow \cfgnt{k}\rp \rp 
   }
\and
   \inferrule[Begin (argN eval)]{}{
    \lp \cfgnt{L}\ \cfgnt{R}\ \phi\ \eta\ \cfgnt{v}\ \lp \cfgt{begin}\ \cfgt{*}\ \lp\cfgnt{e}_{n}\rp \rightarrow \cfgnt{k}\rp \rp  \rcom \\\\
    \lp \cfgnt{L}\ \cfgnt{R}\ \phi\ \eta\ \cfgnt{e}_n\ \cfgnt{k}\rp 
   }
\and
	\inferrule[NULL]{}{
      \lp \cfgnt{L}\ \cfgnt{R}\ \phi\ \eta\ \cfgt{null}\ \cfgnt{k}\rp \rcom \\\\ 
      \lp \cfgnt{L}\ \cfgnt{R}\ \phi\ \eta\ \cfgnt{r}_\mathit{null}\ \cfgnt{k}\rp 
	}
\and
   \inferrule[Pop]{}{
    \lp \cfgnt{L}\ \cfgnt{R}\ \phi\ \eta\ \cfgnt{v}\ \lp \cfgt{pop}\ \eta_0\ \cfgnt{k}\rp \rp  \rcom \\\\
    \lp \cfgnt{L}\ \cfgnt{R}\ \phi\ \eta_0\  \cfgnt{v}\ \cfgnt{k}\rp 
   }
\end{mathpar}
\end{center}
\caption{Javalite rewrite rules, indicated by $\rcom$, that are common to generalized symbolic execution and precise heap summaries.}
\label{fig:javalite-common}
\end{figure*}


The surface syntax and the machine syntax for the Javalite syntactic
machine are in the main paper. \figref{fig:javalite-common} defines
the rewrite relations for the portion of the Javalite language
semantics that are common to both generalized symbolic execution and
the summary heap algorithm. The relation $\rcom$ includes all of these
rules. Excepting \textrm{N{\footnotesize EW}}, the rules do not update
the heap, and are largely concerned with argument evaluation in an
expected way. It is assumed that only type safe programs are input to
the machine so there is no type checking or error conditions. The
machine simply halts if no rewrite is enabled.

%% Includes its section definition
\section{Generating Heap Summaries}

\subsection{Initialization of Symbolic References}

In this section we present the Javalite rewrite rules for the concrete
as well as summary initialization of symbolic references. The
initialization rules are defined on the bi-partite graph consisting of
references and locations. The lazy initialization of symbolic
references consists of three key points of non-determinism where each
symbolic reference can be initialized non-deterministically to null, a
new instance of the symbolic reference, or aliases to symbolic
references of the same type previously initialized. The initialization
in GSE consists of creating branches in the execution tree for all the
non-deterministic choices. On the other hand, the heap summarization
approach creates a single branch that contains the summarization for
all the initialization in a single bi-partitate graph.

\begin{figure*}[t]
\begin{center}
\mprset{flushleft}
\begin{mathpar}
	\inferrule[Initialize (null)]{
	  \Lambda = \{ l \mid \exists \phi\ \lp \lp \phi\ l\rp  \in \cfgnt{L}\lp \cfgnt{r}\rp  \wedge  \cfgnt{R}\lp l,\cfgnt{f}\rp  = \bot\ \rp\}\\
      \Lambda \neq \emptyset\\\\
      \cfgnt{r}^\prime = \mathrm{fresh}_r\lp \rp\\ 
      \theta_\mathit{null} = \{ \lp \phi_T\ l_\mathit{null}\rp \} \\
      l_x = \mathrm{min}_l\lp \Lambda\rp \\\\
      \phi_g^\prime = \lp\phi_g \wedge \cfgnt{r}^\prime = \cfgnt{r}_\mathit{null}\rp
    }{
      \lp \cfgnt{L}\ \cfgnt{R}\ \phi_g\ \cfgnt{r}\ \cfgnt{f}\rp  \rightarrow_I 
      \lp \cfgnt{L}[\cfgnt{r}^\prime \mapsto \theta_\mathit{null}]\ \cfgnt{R}[ \lp l_x,\cfgnt{f}\rp  \mapsto \cfgnt{r}^\prime]\ \phi_g^\prime\ \cfgnt{r}\ \cfgnt{f}\rp 
	}
\and
	\inferrule[Initialize (new)]{
	  \Lambda  = \{ l \mid \exists \phi\ \lp \lp \phi\ l\rp  \in \cfgnt{L}\lp \cfgnt{r}\rp  \wedge  \cfgnt{R}\lp l,\cfgnt{f}\rp  = \bot\rp\}\\
      \Lambda \neq \emptyset\\\\
      \mathrm{C} = \mathrm{type}\lp \cfgnt{f}\rp\\
      \cfgnt{r}_f = \mathrm{init}_r\lp \rp\\
      l_f = \mathrm{init}_l\lp \mathrm{C}\rp \\\\
      \cfgnt{R}^\prime = \cfgnt{R}[\forall \cfgnt{f} \in \mathit{fields}\lp \mathrm{C}\rp \ \lp \lp l_f\ \cfgnt{f}\rp  \mapsto \bot \rp ] \\\\
      \rho = \{ \lp\cfgnt{r}_a\ l_a\rp \mid \mathrm{isInit}\lp \cfgnt{r}_a\rp  \wedge \mathrm{type}\lp l_a\rp  = \mathrm{C} \wedge \exists \phi_a\ \lp \lp \phi_a\ l_a\rp  \in \cfgnt{L}\lp \cfgnt{r}_a\rp\rp \}\\\\
      \theta_\mathit{new} = \{\lp \phi_T\ l_f\rp \} \\
      l_x = \mathrm{min}_l\lp \Lambda\rp \\\\
      \phi_g^\prime = \lp\phi_g \wedge \cfgnt{r}_f \neq \cfgnt{r}_\mathit{null} \wedge \lp \wedge_{\lp\cfgnt{r}_a\ l_a\rp \in \rho} \cfgnt{r}_f \ne \cfgnt{r}_a\rp\rp
    }{
      \lp \cfgnt{L}\ \cfgnt{R}\ \phi_g\ \cfgnt{r}\ \cfgnt{f}\rp  \rightarrow_I 
      \lp \cfgnt{L}[\cfgnt{r}_f \mapsto \theta_\mathit{new}]\ \cfgnt{R}^\prime[ \lp l_x,\cfgnt{f}\rp  \mapsto \cfgnt{r}_f ]\ \phi_g^\prime\ \cfgnt{r}\ \cfgnt{f}\rp 
	}
\and
	\inferrule[Initialize (alias)]{
	  \Lambda = \{ l \mid \exists \phi\ \lp \lp \phi\ l\rp  \in \cfgnt{L}\lp \cfgnt{r}\rp  \wedge  \cfgnt{R}\lp l,\cfgnt{f}\rp  = \bot\ \rp\}\\
      \Lambda \neq \emptyset\\\\
      \mathrm{C} = \mathrm{type}\lp \cfgnt{f}\rp\\
      \cfgnt{r}^\prime = \mathrm{fresh}_r\lp \rp\\\\
      \rho = \{ \lp\cfgnt{r}_a\ l_a\rp \mid \mathrm{isInit}\lp \cfgnt{r}_a\rp  \wedge \mathrm{type}\lp l_a\rp  = \mathrm{C} \wedge \exists \phi_a\ \lp \lp \phi_a\ l_a\rp  \in \cfgnt{L}\lp \cfgnt{r}_a\rp\rp \}\\\\
      \lp\cfgnt{r}_a\ l_a\rp \in \rho \\
      \theta_\mathit{alias} = \{ \lp \phi_T\ l_a\rp\}\\
      l_x = \mathrm{min}_l\lp \Lambda\rp\\\\
      \phi^\prime_g = \lp\phi_g \wedge \cfgnt{r}^\prime \neq \cfgnt{r}_\mathit{null} \wedge \cfgnt{r}^\prime = \cfgnt{r}_a \wedge \lp \wedge_{\lp \cfgnt{r}^{\prime}_a\ l_a\rp  \in \rho\ \lp \cfgnt{r}^{\prime}_a \neq \cfgnt{r}_a\rp } \cfgnt{r}^\prime \neq \cfgnt{r}^{\prime}_a \rp\rp
    }{
      \lp \cfgnt{L}\ \cfgnt{R}\ \phi_g\ \cfgnt{r}\ \cfgnt{f}\rp  \rightarrow_I 
      \lp \cfgnt{L}[\cfgnt{r}^\prime \mapsto \theta_\mathit{alias}]\ \cfgnt{R}[ \lp l_x,\cfgnt{f}\rp  \mapsto \cfgnt{r}^\prime ]\ \phi_g^\prime\ \cfgnt{r}\ \cfgnt{f}\rp 
	}
\and
	\inferrule[Initialize (end)]{
	  \Lambda = \{ l \mid \exists \phi\ \lp \lp \phi\ l\rp  \in \cfgnt{L}\lp \cfgnt{r}\rp  \wedge  \cfgnt{R}\lp l,\cfgnt{f}\rp  = \bot\ \rp\}\\
      \Lambda = \emptyset
    }{
      \lp \cfgnt{L}\ \cfgnt{R}\ \phi_g\ \cfgnt{r}\ \cfgnt{f}\rp  \rightarrow_I 
      \lp \cfgnt{L}\ \cfgnt{R}\ \phi_g\ \cfgnt{r}\ \cfgnt{f}\rp 
	}
\end{mathpar}
\end{center}
\caption{The initialization machine, $s ::= \lp\cfgnt{L}\ \cfgnt{R}\ \phi_g\ \cfgnt{r}\ \cfgnt{f}\rp$, with $s \rightarrow_I^* s^\prime$ indicating stepping the machine until the state does not change.}
\label{fig:lazyInit}
\end{figure*}

\newsavebox{\boxPi}
\savebox{\boxPi}{
%\begin{center}
\mprset{flushleft}
\begin{mathpar}
	\inferrule[Summarize]{
	\Lambda = \mathbb{UN}\lp \cfgnt{L}, \cfgnt{R}, \cfgnt{r}, \cfgnt{f}\rp \\
      \Lambda \neq \emptyset \\
      \lp\phi_x\ \cfgnt{l}_x\rp = \mathrm{min}_l\lp \Lambda\rp\\
      \cfgnt{r}_f = \mathrm{init}_r\lp \rp \\
      l_f  = \mathrm{fresh}_l\lp \mathrm{C}\rp\\\\
      \rho = \{ \lp \cfgnt{r}_a\ l_a\rp  \mid \mathrm{isInit}\lp \cfgnt{r}_a\rp  \wedge\cfgnt{r}_a = \mathrm{min}_r\lp \cfgnt{R}^{\leftarrow}[l_a]\rp \wedge \mathrm{type}\lp l_a\rp  = \mathrm{C} \} \\\\
      \theta_\mathit{null} = \{ \lp \phi\ l_\mathit{null}\rp  \mid \phi = \lp \phi_x \wedge \cfgnt{r}_f = \cfgnt{r}_\mathit{null} \rp  \} \\\\
      \theta_\mathit{new} = \{\lp \phi\ l_f\rp  \mid \phi = \lp \phi_x \wedge \cfgnt{r}_f \neq \cfgnt{r}_\mathit{null} \wedge \lp \wedge_{\lp \cfgnt{r}^\prime_a\ l^\prime_a\rp  \in \rho} \cfgnt{r}_f \ne \cfgnt{r}^\prime_a\rp \rp \}\\\\
      \theta_\mathit{alias} = \{ \lp \phi\ l_a\rp  \mid \exists\cfgnt{r}_a\ \lp\lp\cfgnt{r}_a\ l_a\rp  \in \rho \wedge \phi = \lp \phi_x \wedge \cfgnt{r}_f \neq \cfgnt{r}_\mathit{null} \wedge \cfgnt{r}_f = \cfgnt{r}_a \wedge \lp \wedge_{\lp \cfgnt{r}^{\prime}_a\ l^{\prime}_a\rp  \in \rho\ \lp \cfgnt{r}^\prime_a < \cfgnt{r}_a\rp } \cfgnt{r}_f \neq \cfgnt{r}^{\prime}_a \rp \rp \rp \} \\\\
      \theta_\mathit{orig} = \{\lp\phi\ \cfgnt{l}_\mathit{orig}\rp \mid \exists \phi_\mathit{orig} \lp \lp\phi_\mathit{orig}\ \cfgnt{l}_\mathit{orig}\rp \in \cfgnt{L}\lp\cfgnt{R}\lp\cfgnt{l}_x,\cfgnt{f}\rp\rp \wedge \phi = \lp\neg\phi_x \wedge \phi_\mathit{orig}\rp\}\\\\ 
      \theta = \theta_\mathit{null} \cup \theta_\mathit{new} \cup \theta_\mathit{alias} \cup \theta_\mathit{orig} \\
\cfgnt{R}^\prime = \cfgnt{R}[\forall \cfgnt{f} \in \mathit{fields}\lp \mathrm{C}\rp \ \lp \lp l_f\ \cfgnt{f}\rp  \mapsto \cfgnt{r}_\mathit{un} \rp ]
    }{
      \lp \cfgnt{L}\ \cfgnt{R}\ \cfgnt{r}\ \cfgnt{f}\ \cfgnt{C}\rp \rsum 
      \lp \cfgnt{L}[\cfgnt{r}_f \mapsto \theta]\ \cfgnt{R}^{\prime}[ \lp l_x,\cfgnt{f}\rp  \mapsto \cfgnt{r}_f ]\ \cfgnt{r}\ \cfgnt{f}\ \cfgnt{C}\rp
	}
\and
	\inferrule[Summarize-end]{
	  \Lambda = \mathbb{UN}\lp \cfgnt{L}, \cfgnt{R}, \cfgnt{r}, \cfgnt{f}\rp \\
      \Lambda = \emptyset
    }{
      \lp \cfgnt{L}\ \cfgnt{R}\ \cfgnt{r}\ \cfgnt{f}\ \cfgnt{C}\rp  \rsum
      \lp \cfgnt{L}\ \cfgnt{R}\ \cfgnt{r}\ \cfgnt{f}\ \cfgnt{C}\rp 
	}
\end{mathpar}}
%\end{center}
%\caption{The summary machine, $s ::= \lp\cfgnt{L}\ \cfgnt{R}\ \cfgnt{r}\ \cfgnt{f}\ \cfgnt{C}\rp$, with $s\rsum^*s^\prime =  s \rsum \cdots \rsum s^\prime \rsum s^\prime$.}
%\label{fig:symInit}
%\end{figure*}



The initialization rules are invoked when an uninitialized field in a
symbolic reference is accessed. The function $\mathbb{UN}(\cfgnt{L},
\cfgnt{R}, \cfgnt{r}, \cfgnt{f}) = \{\cfgnt{l}\ ...\}$ returns
constraint-location pairs in which the field $\cfgnt{f}$ is
uninitialized:
\[
\begin{array}{rcl}
\mathbb{UN}(\cfgnt{L}, \cfgnt{R}, \cfgnt{r}, \cfgnt{f}) & = &\{ \lp\phi\ \cfgnt{l}\rp \mid \lp \phi\ \cfgnt{l}\rp  \in \cfgnt{L}\lp \cfgnt{r}\rp  \wedge \\
& & \ \ \ \ \exists \phi^\prime \lp \lp \phi^\prime\ \cfgnt{l}_\mathit{un}\rp  \in \cfgnt{L}\lp \cfgnt{R}\lp l,\cfgnt{f}\rp\rp \wedge \\
& & \ \ \ \ \ \ \ \ \mathbb{S}\lp \phi \wedge \phi^\prime \rp\rp\}\\
\end{array}
\]
where $\mathbb{S}(\phi)$ returns true if $\phi$ is
satisfiable. Intutively, for the reference, $\cfgnt{r}$, it constructs
the set, $\theta$, that contains all contraint-location pairs that
point to the field $\cfgnt{f}$ and $\cfgnt{f}$ points to
$\cfgnt{l}_\mathit{un}$. The cardinality of the set, $\theta$ is never
greater than one in GSE and the constraint is always satisfiable
because all constraints are constant. This property is relaxed in GSE
with heap summaries.

The rules in~\figref{fig:lazyInit} present the rewrite rules for the
concrete initialization of symbolic heap objects.  These rules are
invoked until a fix pointed is reached. 

The initialize (null) rewrite rule in~\figref{fig:lazyInit} first
checks that the field, $\cfgnt{r}$ is uninitialized. The fresh method
returns a new input heap reference from the partition 



	

\newsavebox{\boxPFAFW}
\savebox{\boxPFAFW}{
%\begin{figure}[t]
%\begin{center}
\mprset{flushleft}
\begin{mathpar}
	\inferrule[Field Access]{
      \exists \lp \phi\ l\rp \in \cfgnt{L}\lp \cfgnt{r}\rp\ \lp l \neq l_{\mathit{null}} \wedge \mathbb{S}\lp \phi \wedge \phi_g\rp \rp \\\\
      \theta = \{ \phi \mid \lp \phi\ l_\mathit{null} \rp \wedge \mathbb{S}\lp \phi \wedge \phi_g\rp \} \\\\
      \phi_g^\prime = \phi_g \wedge (\wedge_{\phi \in \theta} \neg \phi) \\\\
      \{\cfgnt{C}\} = \{\cfgnt{C} \mid \exists \lp \phi\ l\rp  \in \cfgnt{L}\lp \cfgnt{r}\rp\ \lp\cfgnt{C} = \mathrm{type}\lp \cfgnt{l},\cfgnt{f}\rp\rp\} \\\\
      \lp \cfgnt{L}\ \cfgnt{R}\ \cfgnt{r}\ \cfgnt{f}\ \cfgnt{C}\rp \rsum^* \lp \cfgnt{L}^\prime\ \cfgnt{R}^\prime\ \cfgnt{r}\ \cfgnt{f}\ \cfgnt{C}\rp \\
      \cfgnt{r}^\prime = \mathrm{stack}_r\lp \rp
    }{
      \lp \cfgnt{L}\ \cfgnt{R}\ \phi_g\ \eta\ \cfgnt{r}\ \lp \cfgt{*}\ \cfgt{\$}\ \cfgnt{f} \rightarrow \cfgnt{k}\rp \rp  \rightarrow_\mathit{FA}
      \lp \cfgnt{L}^\prime[\cfgnt{r}^\prime \mapsto \mathbb{VS}\lp \cfgnt{L}^\prime,\cfgnt{R}^\prime,\cfgnt{r},\cfgnt{f},\phi_g^\prime\rp ]\ \cfgnt{R}^\prime\ \phi_g^\prime\ \eta\ \cfgnt{r}^\prime\ \cfgnt{k}\rp 
	}
\and
	\inferrule[Field Access (NULL)]{
      \exists \lp \phi\ l\rp \in \cfgnt{L}\lp \cfgnt{r}\rp\ \lp l = l_{\mathit{null}} \wedge \mathbb{S}\lp \phi \wedge \phi_g\rp \rp
    }{
      \lp \cfgnt{L}\ \cfgnt{R}\ \phi_g\ \eta\ \cfgnt{r}\ \lp \cfgt{*}\ \cfgt{\$}\ \cfgnt{f} \rightarrow \cfgnt{k}\rp \rp  \rightarrow_\mathit{FA}
      \lp \cfgnt{L}\ \cfgnt{R}\ \phi_g\ \eta\ \cfgt{error}\ \lp \cfgt{*}\ \cfgt{\$}\ \cfgnt{f} \rightarrow \cfgnt{k}\rp \rp
	}
\and
	\inferrule[Field Write]{
      \exists \lp \phi\ l\rp \in \cfgnt{L}\lp \cfgnt{r}\rp\ \lp l \neq l_{\mathit{null}} \wedge \mathbb{S}\lp \phi \wedge \phi_g\rp \rp \\\\
      \theta = \{ \phi \mid \lp \phi\ l_\mathit{null} \rp \wedge \mathbb{S}\lp \phi \wedge \phi_g\rp \} \\\\
      \phi_g^\prime = \phi_g \wedge (\wedge_{\phi \in \theta} \neg \phi) \\\\
      \cfgnt{r}_x = \eta\lp \cfgnt{x}\rp \\
      \Psi_x =\{\lp \phi\ l\ \cfgnt{r}_\mathit{cur} \rp  \mid \lp \phi\ \cfgnt{l}\rp  \in \cfgnt{L}\lp \cfgnt{r}_x\rp  \wedge \cfgnt{r}_\mathit{cur} = \cfgnt{R}\lp l,\cfgnt{f}\rp  \}\\\\
      X = \{ \lp l\ \theta \rp  \mid \exists \phi\ \lp \lp \phi\ l\ \cfgnt{r}_\mathit{cur} \rp \in \Psi_x \wedge \theta = \mathbb{ST}\lp \cfgnt{L},\cfgnt{r},\phi,\phi_g^\prime\rp  \cup \mathbb{ST}\lp \cfgnt{L},\cfgnt{r}_\mathit{cur},\neg\phi,\phi_g^\prime\rp \rp  \}\\\\
      \cfgnt{R}^{\prime} = \cfgnt{R}[\forall \lp l\ \theta \rp  \in X\ \lp \lp l\ \cfgnt{f}\rp  \mapsto \mathrm{fresh}_r\lp \rp \rp ]\\\\
      \cfgnt{L}^{\prime} = \cfgnt{L}[\forall \lp l\ \theta \rp  \in X\ \lp \exists \cfgnt{r}_\mathit{targ}\ \lp \cfgnt{r}_\mathit{targ} = \cfgnt{R}^\prime\lp l,\cfgnt{f}\rp \wedge \lp\cfgnt{r}_\mathit{targ} \mapsto \theta\rp  \rp \rp ]
    }{
      \lp \cfgnt{L}\ \cfgnt{R}\ \phi_g\ \eta\ \cfgnt{r}\ \lp \cfgnt{x}\ \cfgt{\$}\ \cfgnt{f}\ \cfgt{:=}\ \cfgt{*}\ \rightarrow\ \cfgnt{k}\rp \rp  \rightarrow_\mathit{FW}
      \lp \cfgnt{L}^{\prime}\ \cfgnt{R}^{\prime}\ \phi_g^\prime\ \eta\ \cfgnt{r}\ \cfgnt{k}\rp 
	}	
\and
	\inferrule[Field Write (NULL)]{
      \exists \lp \phi\ l\rp \in \cfgnt{L}\lp \cfgnt{r}\rp\ \lp l \neq l_{\mathit{null}} \wedge \mathbb{S}\lp \phi \wedge \phi_g\rp \rp
    }{
      \lp \cfgnt{L}\ \cfgnt{R}\ \phi_g\ \eta\ \cfgnt{r}\ \lp \cfgnt{x}\ \cfgt{\$}\ \cfgnt{f}\ \cfgt{:=}\ \cfgt{*}\ \rightarrow\ \cfgnt{k}\rp \rp  \rightarrow_\mathit{FW}
      \lp \cfgnt{L}\ \cfgnt{R}\ \phi_g\ \eta\ \cfgt{error}\ \lp \cfgnt{x}\ \cfgt{\$}\ \cfgnt{f}\ \cfgt{:=}\ \cfgt{*}\ \rightarrow\ \cfgnt{k}\rp \rp
	}	
\end{mathpar}}
%\end{center}
%\caption{Precise symbolic heap summaries from symbolic execution indicated by $\rsym = \rightarrow_\mathit{FA} \cup \rightarrow_\mathit{FW} \cup \rightarrow_\mathit{EQ} \cup \rcom$.}
%\label{fig:symfield}
%\end{figure}



\newsavebox{\boxPi}
\savebox{\boxPi}{
%\begin{center}
\mprset{flushleft}
\begin{mathpar}
	\inferrule[Summarize]{
	\Lambda = \mathbb{UN}\lp \cfgnt{L}, \cfgnt{R}, \cfgnt{r}, \cfgnt{f}\rp \\
      \Lambda \neq \emptyset \\
      \lp\phi_x\ \cfgnt{l}_x\rp = \mathrm{min}_l\lp \Lambda\rp\\
      \cfgnt{r}_f = \mathrm{init}_r\lp \rp \\
      l_f  = \mathrm{fresh}_l\lp \mathrm{C}\rp\\\\
      \rho = \{ \lp \cfgnt{r}_a\ l_a\rp  \mid \mathrm{isInit}\lp \cfgnt{r}_a\rp  \wedge\cfgnt{r}_a = \mathrm{min}_r\lp \cfgnt{R}^{\leftarrow}[l_a]\rp \wedge \mathrm{type}\lp l_a\rp  = \mathrm{C} \} \\\\
      \theta_\mathit{null} = \{ \lp \phi\ l_\mathit{null}\rp  \mid \phi = \lp \phi_x \wedge \cfgnt{r}_f = \cfgnt{r}_\mathit{null} \rp  \} \\\\
      \theta_\mathit{new} = \{\lp \phi\ l_f\rp  \mid \phi = \lp \phi_x \wedge \cfgnt{r}_f \neq \cfgnt{r}_\mathit{null} \wedge \lp \wedge_{\lp \cfgnt{r}^\prime_a\ l^\prime_a\rp  \in \rho} \cfgnt{r}_f \ne \cfgnt{r}^\prime_a\rp \rp \}\\\\
      \theta_\mathit{alias} = \{ \lp \phi\ l_a\rp  \mid \exists\cfgnt{r}_a\ \lp\lp\cfgnt{r}_a\ l_a\rp  \in \rho \wedge \phi = \lp \phi_x \wedge \cfgnt{r}_f \neq \cfgnt{r}_\mathit{null} \wedge \cfgnt{r}_f = \cfgnt{r}_a \wedge \lp \wedge_{\lp \cfgnt{r}^{\prime}_a\ l^{\prime}_a\rp  \in \rho\ \lp \cfgnt{r}^\prime_a < \cfgnt{r}_a\rp } \cfgnt{r}_f \neq \cfgnt{r}^{\prime}_a \rp \rp \rp \} \\\\
      \theta_\mathit{orig} = \{\lp\phi\ \cfgnt{l}_\mathit{orig}\rp \mid \exists \phi_\mathit{orig} \lp \lp\phi_\mathit{orig}\ \cfgnt{l}_\mathit{orig}\rp \in \cfgnt{L}\lp\cfgnt{R}\lp\cfgnt{l}_x,\cfgnt{f}\rp\rp \wedge \phi = \lp\neg\phi_x \wedge \phi_\mathit{orig}\rp\}\\\\ 
      \theta = \theta_\mathit{null} \cup \theta_\mathit{new} \cup \theta_\mathit{alias} \cup \theta_\mathit{orig} \\
\cfgnt{R}^\prime = \cfgnt{R}[\forall \cfgnt{f} \in \mathit{fields}\lp \mathrm{C}\rp \ \lp \lp l_f\ \cfgnt{f}\rp  \mapsto \cfgnt{r}_\mathit{un} \rp ]
    }{
      \lp \cfgnt{L}\ \cfgnt{R}\ \cfgnt{r}\ \cfgnt{f}\ \cfgnt{C}\rp \rsum 
      \lp \cfgnt{L}[\cfgnt{r}_f \mapsto \theta]\ \cfgnt{R}^{\prime}[ \lp l_x,\cfgnt{f}\rp  \mapsto \cfgnt{r}_f ]\ \cfgnt{r}\ \cfgnt{f}\ \cfgnt{C}\rp
	}
\and
	\inferrule[Summarize-end]{
	  \Lambda = \mathbb{UN}\lp \cfgnt{L}, \cfgnt{R}, \cfgnt{r}, \cfgnt{f}\rp \\
      \Lambda = \emptyset
    }{
      \lp \cfgnt{L}\ \cfgnt{R}\ \cfgnt{r}\ \cfgnt{f}\ \cfgnt{C}\rp  \rsum
      \lp \cfgnt{L}\ \cfgnt{R}\ \cfgnt{r}\ \cfgnt{f}\ \cfgnt{C}\rp 
	}
\end{mathpar}}
%\end{center}
%\caption{The summary machine, $s ::= \lp\cfgnt{L}\ \cfgnt{R}\ \cfgnt{r}\ \cfgnt{f}\ \cfgnt{C}\rp$, with $s\rsum^*s^\prime =  s \rsum \cdots \rsum s^\prime \rsum s^\prime$.}
%\label{fig:symInit}
%\end{figure*}


\section{Summary Heap Algorithm}


%\begin{figure*}[t]
%\begin{center}
%\setlength{\tabcolsep}{60pt}
%\hspace*{-35pt}
%\begin{tabular}[c]{cc}
%\scalebox{1.0}{\usebox{\boxPFAFW}} & 
%%\scalebox{0.91}{\input{faYHeap.pdf_t}} &
%\scalebox{0.91}{\input{fwXHeap.pdf_t}} \\ \\
%(a) & (b)
%\end{tabular}
%\end{center}
%\caption{Field read and write relations with an example heap. (a) Field-access, $\rsym^\mathit{A}$, and field-write, $\rsym^\mathit{W}$, rewrite rules for the $\rsym$ relation. (b) The final heap after $\lp\cfgt{this}\  \cfgt{\$}\ \cfgnt{x}\ \cfgt{:=}\ \lp\cfgt{this}\  \cfgt{\$}\ \cfgnt{y}\rp\rp$.}
%\label{fig:fHeap}
%\end{figure*}

%\begin{figure*}
%\begin{tabular}[c]{l}
%\scalebox{1.0}{\usebox{\boxPI}} \\
%\end{tabular}
%\caption{The summary machine, $s ::= \lp\cfgnt{L}\ \cfgnt{R}\ \cfgnt{r}\ \cfgnt{f}\ \cfgnt{C}\rp$, with $s\rsum^*s^\prime =  s \rsum \cdots \rsum s^\prime \rsum s^\prime$.}
%\label{fig:symInit}
%\end{figure*}

The rewrite rules for field access or field write when a null reference is possible leads to an error state.
$$
\begin{array}{c}
	\inferrule[Field Access (NULL)]{
      \exists \lp \phi\ l\rp \in \cfgnt{L}\lp \cfgnt{r}\rp\ \lp l = l_{\mathit{null}} \wedge \mathbb{S}\lp \phi \wedge \phi_g\rp \rp
    }{
      \lp \cfgnt{L}\ \cfgnt{R}\ \phi_g\ \eta\ \cfgnt{r}\ \lp \cfgt{*}\ \cfgt{\$}\ \cfgnt{f} \rightarrow \cfgnt{k}\rp \rp  \rsym^\mathit{A^\prime}
      \lp \cfgnt{L}\ \cfgnt{R}\ \phi_g\ \eta\ \cfgt{error}\ \cfgt{end} \rp
	} \\
\\
	\inferrule[Field Write (NULL)]{
      \cfgnt{r}_x = \eta\lp \cfgnt{x}\rp \\
      \exists \lp \phi\ l\rp \in \cfgnt{L}\lp \cfgnt{r}_x\rp\ \lp l \neq l_{\mathit{null}} \wedge \mathbb{S}\lp \phi \wedge \phi_g\rp \rp
    }{
      \lp \cfgnt{L}\ \cfgnt{R}\ \phi_g\ \eta\ \cfgnt{r}\ \lp \cfgnt{x}\ \cfgt{\$}\ \cfgnt{f}\ \cfgt{:=}\ \cfgt{*}\ \rightarrow\ \cfgnt{k}\rp \rp  \rsym^\mathit{F^\prime}
      \lp \cfgnt{L}\ \cfgnt{R}\ \phi_g\ \eta\ \cfgt{error}\ \cfgt{end}\rp
	}	
\end{array}
$$
The rules reflect a branch in the control flow for the summary heap
algorithm: one branch being the feasible null outcome (shown here), and the
other branch being the potential non-null
outcome (shown in the main paper). In the non-null case, the path constraint is updated to
restrict out all possible null instances.

%\subsection{Definitions}

\begin{definition}
The set of \textbf{states} $\mathcal{S}$ is defined as
\end{definition}

\begin{definition}
The set of \textbf{initial states} $\mathcal{S}_0$ is defined as
\end{definition}

\begin{definition}
For a given function $f:A \mapsto B$, the \textbf{image} $f^\rightarrow$ and \textbf{preimage} $f^\leftarrow$ are defined as
\begin{align}
 f^\rightarrow &= \{ f(a) \mid a \in A\}\\
 f^\leftarrow &= \{ a \mid f(a) \in B \}�
 \end{align}
\end{definition}

\begin{definition}
A \textbf{state transition function} $\rightarrow_{\Phi}$ is a mapping $\rightarrow_{\Phi} : s \mapsto s$, which takes one machine state and transforms it into another machine state. Two important state transition functions are the \textbf{lazy state transition function} $\rightarrow_\ell$ and the \textbf{summary state transition function} $\rightarrow_s$.
\end{definition}

\begin{definition}
A \textbf{state sequence} is as a sequence of states denoted as $\Pi_n = s_0,s_1,...,s_n$. A \textbf{feasible state sequence}, $\Pi_n^\phi = s_0,s_1,...,s_n$ is consistent with the transition: $\forall i\ (0 \leq i < n \Rightarrow s_i \rightarrow_{\Phi} s_{i+1})$, where $s_0\in \mathcal{S}_0$.
\end{definition}

\begin{comment}
\begin{definition}
A \textbf{feasible state sequence} is defined as a sequence of states resulting from repeated application of the state transition relation to some initial state $s_0\in \mathcal{S}_0$: $$\Pi_n = s_0,s_1,...,s_n$$ where the relation $s_i \rightarrow_{\Phi} s_{i+1}$ holds for all $i \in \{ i | 0 \leq i < n \}$
\end{definition}
\end{comment}

\begin{definition}
The set of \textbf{lazy states} $\mathcal{S}_\ell$ is defined as
\begin{align}
\mathcal{S}_\ell = \{s_\ell \mid \exists \Pi_n^\ell\ (\Pi_n^\ell = s_0, \ldots, s_\ell)\}
\end{align}
\end{definition}

\begin{definition}
The set of \textbf{summary states} $\mathcal{S}_s$ is defined as
\begin{align}
\mathcal{S}_s = \{s_s \mid \exists \Pi_n^s\ (\Pi_n^s = s_0, \ldots, s_s)\}
\end{align}
\end{definition}

\begin{definition}
Given a sequence of states $$\Pi_n = s_0,s_1,...,s_n$$ where $$s_i = ( \mu_i\ \cfgnt{L}_i\ \cfgnt{R}_i\ \phi_i\ \eta_i\ \cfgnt{e}_i\ \cfgnt{k}_i )$$ the \textbf{control flow sequence} of $\Pi_n$ is the defined as the sequence of tuples $$ \pi_n = \mathbb{CF}(\Pi_n) = (\eta_0\ \cfgnt{e}_0\ \cfgnt{k}_0),(\eta_1\ \cfgnt{e}_1\ \cfgnt{k}_1),...,(\eta_n\ \cfgnt{e}_n\ \cfgnt{k}_n)$$
\end{definition}

\begin{definition}
Given a state transition function $\rightarrow_{\Phi}$, an initial state $s_0$ and a control flow sequence $\pi_n$, the \textbf{feasible state set}, $\mathbb{FS}(\rightarrow_{\Phi},s_0,\pi_n)$, is defined as
 $$
\begin{array}{l}
\mathbb{FS}(\rightarrow_{\Phi},s_0,\pi_n) = \\
\ \ \ \ \ \ \{s \mid \exists \Pi_n^\phi\ (\pi_n = \mathbb{CF}(\Pi_n^\Phi) \wedge s = \mathit{last}(\Pi_n))\} 
\end{array}
$$
where $\mathit{last}(\Pi_n)$ returns the last state on the feasible sequence.
\end{definition}

\begin{definition}
A \textbf{field access descriptor} $\gamma_i$ is a tuple 
$$ \gamma_i = (\cfgnt{r}_i\ \phi_i\ \cfgnt{l}_i\ \cfgnt{f}_i)$$
\end{definition}

\begin{definition}
An \textbf{access path} $\Gamma_n$ is a sequence of field access descriptors
$ \Gamma_n = \gamma_0,\gamma_1,...,\gamma_n $.
\end{definition}

\begin{definition}
For a given access path $\Gamma_n$ the \textbf{access path constraint} $\mathbb{PC}(\Gamma_n)$ is defined as
$$\mathbb{PC}(\Gamma_n) =  \bigwedge \{\phi \mid \exists \gamma \in \Gamma_n \ (\gamma = (\cfgnt{r}\ \phi\ \cfgnt{e}\ \cfgnt{f}))\}$$  
\end{definition}

\begin{definition}
For a given state $s= ( \cfgnt{L}_s\ \cfgnt{R}_s\ \phi_s\ \eta_s\ \cfgnt{e}_s\ \cfgnt{k}_s )$, a \textbf{valid access path} $\Gamma_n^s = \gamma_0,\gamma_1,...,\gamma_n$ satisfies the properties
\begin{align*}
\cfgnt{r}_0 &\in \mathit{refs}(\eta_s) \\
\mathbb{S}(&\phi_s \wedge \mathbb{PC}(\Gamma_n^s)) \\
\forall i & \in \mathbb{N}\ ( 0 \leq i < n \Leftrightarrow \gamma_i \in \Gamma_n^s\  \wedge\\
&\ \ \ \  \gamma_{i+1} \in \Gamma_n^s\  \wedge\\
&\ \ \ \  (\phi_i\ \cfgnt{l}_i)\in \cfgnt{L}_s(\cfgnt{r}_i)\ \wedge \\
&\ \ \ \  \cfgnt{r}_{i+1} = \cfgnt{R}_s(\cfgnt{l}_i,\cfgnt{f}_i)\ \wedge\\
&\ \ \ \ (\phi_{i+1}\ \cfgnt{l}_{i+1}) = \cfgnt{L}_s(\cfgnt{r}_{i+1}) )
\end{align*}
where $\gamma_i = (\cfgnt{r}_i\ \phi_i\ \cfgnt{l}_i\ \cfgnt{f}_i)$
\end{definition}

\begin{definition}
A \textbf{homomorphism} $s_x \rightharpoonup_{h} s_y$, from state $s_x = ( \cfgnt{L}_x\ \cfgnt{R}_x\ \phi_x\ \eta_x\ \cfgnt{e}_x\ \cfgnt{k}_x )$ to state $s_y = ( \cfgnt{L}_y\ \cfgnt{R}_y\ \phi_y\ \eta_y\ \cfgnt{e}_y\ \cfgnt{k}_y )$, is defined as follows: 
$$
\begin{array}{l}
 s_x \rightharpoonup_{h} s_y \Leftrightarrow \\
\ \ \ \ \ \ \ \ \exists h: \mathcal{L} \mapsto \mathcal{L}\ (\forall \cfgnt{l}_\alpha \in \mathcal{L}\ (\forall \cfgnt{l}_\beta \in \mathcal{L}\ ( \forall f \in \mathcal{F}( \\ 
\ \ \ \ \ \ \ \ \ \ \ \ (\phi_a\ \cfgnt{l}_\alpha) \in \cfgnt{L}_x(\cfgnt{R}_x (\cfgnt{l}_\beta,f )) \Rightarrow (\phi_b\ h(\cfgnt{l}_\alpha))\in \cfgnt{L}_y(\cfgnt{R}_y (h(\cfgnt{l}_\beta),f ))\ \\
\ \ \ \ \ \ \ \ \ \ \ \ \ \ \ \  )) ) )
\end{array}
$$
\end{definition}

\begin{definition}
\label{def:hc}
Given homomorphism $s_x \rightharpoonup_{h} s_y$, the \textbf{homomorphism constraint} $\mathbb{HC}(s_x \rightharpoonup_{h} s_y)$ is defined as:
\begin{align*}
\mathbb{HC}(s_x \rightharpoonup_{h} s_y) &= \\
 \bigwedge \{ \phi_b\ | \exists r \in \cfgnt{L}_x^\leftarrow\ &(\exists (\phi_a\ l) \in \cfgnt{L}_x^\rightarrow ( (\phi_b\ h(l)) \in \cfgnt{L}_y (g(r)))) 
\end{align*}
\end{definition}

\begin{definition}
\label{representation}
The \textbf{representation relation} is defined as follows: given lazy state $s_\ell = ( \cfgnt{L}_\ell\ \cfgnt{R}_\ell\ \phi_\ell\ \eta_\ell\ \cfgnt{e}_\ell\ \cfgnt{k}_\ell )$ and summary state $s_s = ( \cfgnt{L}_s\ \cfgnt{R}_s\ \phi_s\ \eta_s\ \cfgnt{e}_s\ \cfgnt{k}_s )$, $s_\ell \sqsubset s_s $ if and only if $\eta_{\ell} = \eta_{s} ,\ \cfgnt{e}_{\ell} = \cfgnt{e}_{s} ,\ \cfgnt{k}_{\ell} = \cfgnt{k}_{s}$, and there exists a homomorphism $\ s_\ell \rightharpoonup_{h} s_s $ such that 
\begin{equation}
\label{eqn:valid}
 \mathbb{S}( \phi_s \wedge \mathbb{HC}(s_\ell \rightharpoonup_{h} s_s) ) 
\end{equation}
\end{definition}

\begin{definition}
\label{equivalent}
A summary state $s_s$ is \textbf{equivalent} to a set of lazy states $P$ if and only if $s_s$ represents every state in $P$ and represents no other state: 
$$s_s \cong P \Leftrightarrow \forall s_i \in \mathcal{S}\ (s_i \in P \Leftrightarrow s_i \sqsubset s_s) $$
\end{definition}

\begin{definition}
\label{sound}
A state $s$ is \textbf{sound} with respect to a transition relation, $\rightarrow_\phi$, initial state, $s_0$, and control flow path, $\pi_n$, if and only if 
$$ \forall s_\ell \in \mathcal{S}_\ell\ (s_\ell \sqsubset s_s \Rightarrow s_\ell \in \mathbb{FS}(\rightarrow_{\phi},s_0,\pi_n) ) $$
\end{definition}

\begin{definition}
\label{complete}
A state $s$ is \textbf{complete} with respect to a transition relation, $\rightarrow_\phi$, initial state, $s_0$, and control flow path, $\pi_n$, if and only if 
$$ \forall s_\ell \in \mathcal{S}_\ell\ ( s_\ell \in \mathbb{FS}(\rightarrow_{\phi},s_0,\pi_n)\Rightarrow s_\ell \sqsubset s_s ) $$
\end{definition}

\begin{definition}
\label{exact}
A state $s$ is \textbf{exact} with respect to a transition relation, $\rightarrow_\phi$, initial state, $s_0$, and control flow path, $\pi_n$, if and only if it is both sound and complete:
$$ s \cong \mathbb{FS}(\rightarrow_{\phi},s_0,\pi_n)$$
\end{definition}




