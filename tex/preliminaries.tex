

\subsection{Transition System}

In Javalite, states are strings that match a certain pattern. 

\begin{figure}
\begin{center}
\cfgstart
\cfgrule{e}{\lp ... \cfgor \lp \cfgnt{v} \cfgt{@} \cfgnt{m} \cfgnt{v} \rp\rp}
%\cfgrule{object}{ (\cfgnt{C} [ \cfgnt{f} \cfgnt{loc} ] ...) }
%\cfgrule{hv}{ (\cfgnt{v} \cfgnt{object})}
\cfgrule{$\phi$}{\cfgnt{constraint}}
\cfgrule{l}{\cfgt{number}}
%\cfgrule{h}{(\cfgnt{mt}\ (\cfgnt{h}\ [\cfgnt{loc} $\rightarrow$ \cfgnt{hv}]) )}
\cfgrule{$\eta$}{(\cfgnt{mt}\ ($\eta$ [\cfgnt{x} $\rightarrow$ \cfgnt{loc}]))}
\cfgrule{s}{\lp$\mu$ \cfgnt{L} \cfgnt{R} \cfgnt{g} $\eta$ \cfgnt{e} \cfgnt{k}\rp}
\cfgrule{k}{\cfgt{end}}
\cfgorline{\lp \cfgt{*} \cfgt{\$} \cfgnt{f} $\rightarrow$ \cfgnt{k}\rp}
\cfgorline{\lp \cfgt{*} \cfgt{@} \cfgnt{m} \lp \cfgnt{e} ... \rp $\rightarrow$ \cfgnt{k} \rp}
\cfgorline{\lp \cfgnt{v} \cfgt{@} \cfgnt{m} \cfgnt{v} \cfgt{*} \lp \cfgnt{e} ... \rp $\rightarrow$ \cfgnt{k} \rp}
\cfgorline{\lp \cfgt{*} \cfgt{=} \cfgnt{e} $\rightarrow$ \cfgnt{k}\rp}
\cfgorline{\lp \cfgt{v} \cfgt{=} \cfgnt{*} $\rightarrow$ \cfgnt{k}\rp}
\cfgorline{\lp \cfgt{x} \cfgt{:=} \cfgnt{*} $\rightarrow$ \cfgnt{k}\rp}
\cfgorline{\lp \cfgt{x} \cfgt{\$} \cfgnt{f}  \cfgt{:=} \cfgnt{*} $\rightarrow$ \cfgnt{k}\rp}
\cfgorline{\lp \cfgt{if} \cfgnt{*} \cfgnt{e} \cfgt{else} \cfgnt{e} $\rightarrow$ \cfgnt{k} \rp}
\cfgorline{\lp\cfgt{var} \cfgnt{T} \cfgnt{x} \cfgt{:=} \cfgnt{*} \cfgt{in} \cfgnt{e}  $\rightarrow$ \cfgnt{k} \rp}
\cfgorline{\lp\cfgt{begin}  \cfgnt{*} \lp \cfgnt{e} ...\rp $\rightarrow$ \cfgnt{k} \rp}
\cfgorline{\lp\cfgt{pop} $\eta$ \cfgnt{k}\rp}
\cfgend
\end{center}
\caption{The machine syntax for Javalite.}
\label{fig:machine-syntax}
\end{figure}


\begin{definition}
\label{def:state}
The set of \textbf{states} $\mathcal{S}$ is defined as the set of strings matching the pattern $s$ in \ref{fig:machine-syntax}.
\end{definition}

\begin{definition}
\label{def:initstate}
$\mathcal{S}_0$ is defined as the set of \textbf{initial states}. An initial state is a state meeting the following conditions:  The range of L has exactly three locations: $l_{null}$, $l_{un}$, and $l_0$, the function R is defined only for location $l_0$, and for any field $f$, $R(l_0,f)$ returns $r_{un}$. 
\end{definition}

The rewrite rules that define the Javalite semantics are presented in
the supplemental document accompanying the paper.


\begin{definition}
A \textbf{state transition relation} $\rightarrow_{\Phi}$ is a binary relation $\rightarrow_{\Phi}\ \subseteq\ \mathcal{S} \times \mathcal{S} $, which relates machine states with successor states. Two important state transition relations are \textbf{GSE} $\rgse$ and \textbf{symbolic} $\rsym$. Each of these use a separate relation for initialization: $\rinit$ for GSE and $\rsum$ for symbolic. All of these transition relations are defined in \figref{fig:lazy}, \figref{fig:lazyInit}, \figref{fig:sym}, and \figref{fig:symInit}.
\end{definition}

