\section{Proofs}

\subsection{Definitions}

\begin{definition}
The set of \textbf{states} $\mathcal{S}$ is defined as
\end{definition}

\begin{definition}
The set of \textbf{initial states} $\mathcal{S}_0$ is defined as
\end{definition}

\begin{definition}
For a given function $f:A \mapsto B$, the \textbf{image} $f^\rightarrow$ and \textbf{preimage} $f^\leftarrow$ are defined as
\begin{align}
 f^\rightarrow &= \{ f(a) \mid a \in A\}\\
 f^\leftarrow &= \{ a \mid f(a) \in B \}�
 \end{align}
\end{definition}

\begin{definition}
A \textbf{state transition function} $\rightarrow_{\Phi}$ is a mapping $\rightarrow_{\Phi} : s \mapsto s$, which takes one machine state and transforms it into another machine state. Two important state transition functions are the \textbf{lazy state transition function} $\rightarrow_\ell$ and the \textbf{summary state transition function} $\rightarrow_s$.
\end{definition}

\begin{definition}
A \textbf{state sequence} is as a sequence of states denoted as $\Pi_n = s_0,s_1,...,s_n$. A \textbf{feasible state sequence}, $\Pi_n^\phi = s_0,s_1,...,s_n$ is consistent with the transition: $\forall i\ (0 \leq i < n \Rightarrow s_i \rightarrow_{\Phi} s_{i+1})$, where $s_0\in \mathcal{S}_0$.
\end{definition}

\begin{comment}
\begin{definition}
A \textbf{feasible state sequence} is defined as a sequence of states resulting from repeated application of the state transition relation to some initial state $s_0\in \mathcal{S}_0$: $$\Pi_n = s_0,s_1,...,s_n$$ where the relation $s_i \rightarrow_{\Phi} s_{i+1}$ holds for all $i \in \{ i | 0 \leq i < n \}$
\end{definition}
\end{comment}

\begin{definition}
The set of \textbf{lazy states} $\mathcal{S}_\ell$ is defined as
\begin{align}
\mathcal{S}_\ell = \{s_\ell \mid \exists \Pi_n^\ell\ (\Pi_n^\ell = s_0, \ldots, s_\ell)\}
\end{align}
\end{definition}

\begin{definition}
The set of \textbf{summary states} $\mathcal{S}_s$ is defined as
\begin{align}
\mathcal{S}_s = \{s_s \mid \exists \Pi_n^s\ (\Pi_n^s = s_0, \ldots, s_s)\}
\end{align}
\end{definition}

\begin{definition}
Given a sequence of states $$\Pi_n = s_0,s_1,...,s_n$$ where $$s_i = ( \mu_i\ \cfgnt{L}_i\ \cfgnt{R}_i\ \phi_i\ \eta_i\ \cfgnt{e}_i\ \cfgnt{k}_i )$$ the \textbf{control flow sequence} of $\Pi_n$ is the defined as the sequence of tuples $$ \pi_n = \mathbb{CF}(\Pi_n) = (\eta_0\ \cfgnt{e}_0\ \cfgnt{k}_0),(\eta_1\ \cfgnt{e}_1\ \cfgnt{k}_1),...,(\eta_n\ \cfgnt{e}_n\ \cfgnt{k}_n)$$
\end{definition}

\begin{definition}
Given a state transition function $\rightarrow_{\Phi}$, an initial state $s_0$ and a control flow sequence $\pi_n$, the \textbf{feasible state set}, $\mathbb{FS}(\rightarrow_{\Phi},s_0,\pi_n)$, is defined as
 $$
\begin{array}{l}
\mathbb{FS}(\rightarrow_{\Phi},s_0,\pi_n) = \\
\ \ \ \ \ \ \{s \mid \exists \Pi_n^\phi\ (\pi_n = \mathbb{CF}(\Pi_n^\Phi) \wedge s = \mathit{last}(\Pi_n))\} 
\end{array}
$$
where $\mathit{last}(\Pi_n)$ returns the last state on the feasible sequence.
\end{definition}

\begin{definition}
A \textbf{field access descriptor} $\gamma_i$ is a tuple 
$$ \gamma_i = (\cfgnt{r}_i\ \phi_i\ \cfgnt{l}_i\ \cfgnt{f}_i)$$
\end{definition}

\begin{definition}
An \textbf{access path} $\Gamma_n$ is a sequence of field access descriptors
$ \Gamma_n = \gamma_0,\gamma_1,...,\gamma_n $.
\end{definition}

\begin{definition}
For a given access path $\Gamma_n$ the \textbf{access path constraint} $\mathbb{PC}(\Gamma_n)$ is defined as
$$\mathbb{PC}(\Gamma_n) =  \bigwedge \{\phi \mid \exists \gamma \in \Gamma_n \ (\gamma = (\cfgnt{r}\ \phi\ \cfgnt{e}\ \cfgnt{f}))\}$$  
\end{definition}

\begin{definition}
For a given state $s= ( \cfgnt{L}_s\ \cfgnt{R}_s\ \phi_s\ \eta_s\ \cfgnt{e}_s\ \cfgnt{k}_s )$, a \textbf{valid access path} $\Gamma_n^s = \gamma_0,\gamma_1,...,\gamma_n$ satisfies the properties
\begin{align*}
\cfgnt{r}_0 &\in \mathit{refs}(\eta_s) \\
\mathbb{S}(&\phi_s \wedge \mathbb{PC}(\Gamma_n^s)) \\
\forall i & \in \mathbb{N}\ ( 0 \leq i < n \Leftrightarrow \gamma_i \in \Gamma_n^s\  \wedge\\
&\ \ \ \  \gamma_{i+1} \in \Gamma_n^s\  \wedge\\
&\ \ \ \  (\phi_i\ \cfgnt{l}_i)\in \cfgnt{L}_s(\cfgnt{r}_i)\ \wedge \\
&\ \ \ \  \cfgnt{r}_{i+1} = \cfgnt{R}_s(\cfgnt{l}_i,\cfgnt{f}_i)\ \wedge\\
&\ \ \ \ (\phi_{i+1}\ \cfgnt{l}_{i+1}) = \cfgnt{L}_s(\cfgnt{r}_{i+1}) )
\end{align*}
where $\gamma_i = (\cfgnt{r}_i\ \phi_i\ \cfgnt{l}_i\ \cfgnt{f}_i)$
\end{definition}

\begin{definition}
A \textbf{homomorphism} $s_x \rightharpoonup_{h} s_y$, from state $s_x = ( \cfgnt{L}_x\ \cfgnt{R}_x\ \phi_x\ \eta_x\ \cfgnt{e}_x\ \cfgnt{k}_x )$ to state $s_y = ( \cfgnt{L}_y\ \cfgnt{R}_y\ \phi_y\ \eta_y\ \cfgnt{e}_y\ \cfgnt{k}_y )$, is defined as follows: 
$$
\begin{array}{l}
 s_x \rightharpoonup_{h} s_y \Leftrightarrow \\
\ \ \ \ \ \ \ \ \exists h: \mathcal{L} \mapsto \mathcal{L}\ (\forall \cfgnt{l}_\alpha \in \mathcal{L}\ (\forall \cfgnt{l}_\beta \in \mathcal{L}\ ( \forall f \in \mathcal{F}( \\ 
\ \ \ \ \ \ \ \ \ \ \ \ (\phi_a\ \cfgnt{l}_\alpha) \in \cfgnt{L}_x(\cfgnt{R}_x (\cfgnt{l}_\beta,f )) \Rightarrow (\phi_b\ h(\cfgnt{l}_\alpha))\in \cfgnt{L}_y(\cfgnt{R}_y (h(\cfgnt{l}_\beta),f ))\ \\
\ \ \ \ \ \ \ \ \ \ \ \ \ \ \ \  )) ) )
\end{array}
$$
\end{definition}

\begin{definition}
\label{def:hc}
Given homomorphism $s_x \rightharpoonup_{h} s_y$, the \textbf{homomorphism constraint} $\mathbb{HC}(s_x \rightharpoonup_{h} s_y)$ is defined as:
\begin{align*}
\mathbb{HC}(s_x \rightharpoonup_{h} s_y) &= \\
 \bigwedge \{ \phi_b\ | \exists r \in \cfgnt{L}_x^\leftarrow\ &(\exists (\phi_a\ l) \in \cfgnt{L}_x^\rightarrow ( (\phi_b\ h(l)) \in \cfgnt{L}_y (g(r)))) 
\end{align*}
\end{definition}

\begin{definition}
\label{representation}
The \textbf{representation relation} is defined as follows: given lazy state $s_\ell = ( \cfgnt{L}_\ell\ \cfgnt{R}_\ell\ \phi_\ell\ \eta_\ell\ \cfgnt{e}_\ell\ \cfgnt{k}_\ell )$ and summary state $s_s = ( \cfgnt{L}_s\ \cfgnt{R}_s\ \phi_s\ \eta_s\ \cfgnt{e}_s\ \cfgnt{k}_s )$, $s_\ell \sqsubset s_s $ if and only if $\eta_{\ell} = \eta_{s} ,\ \cfgnt{e}_{\ell} = \cfgnt{e}_{s} ,\ \cfgnt{k}_{\ell} = \cfgnt{k}_{s}$, and there exists a homomorphism $\ s_\ell \rightharpoonup_{h} s_s $ such that 
\begin{equation}
\label{eqn:valid}
 \mathbb{S}( \phi_s \wedge \mathbb{HC}(s_\ell \rightharpoonup_{h} s_s) ) 
\end{equation}
\end{definition}

\begin{definition}
\label{equivalent}
A summary state $s_s$ is \textbf{equivalent} to a set of lazy states $P$ if and only if $s_s$ represents every state in $P$ and represents no other state: 
$$s_s \cong P \Leftrightarrow \forall s_i \in \mathcal{S}\ (s_i \in P \Leftrightarrow s_i \sqsubset s_s) $$
\end{definition}

\begin{definition}
\label{sound}
A state $s$ is \textbf{sound} with respect to a transition relation, $\rightarrow_\phi$, initial state, $s_0$, and control flow path, $\pi_n$, if and only if 
$$ \forall s_\ell \in \mathcal{S}_\ell\ (s_\ell \sqsubset s_s \Rightarrow s_\ell \in \mathbb{FS}(\rightarrow_{\phi},s_0,\pi_n) ) $$
\end{definition}

\begin{definition}
\label{complete}
A state $s$ is \textbf{complete} with respect to a transition relation, $\rightarrow_\phi$, initial state, $s_0$, and control flow path, $\pi_n$, if and only if 
$$ \forall s_\ell \in \mathcal{S}_\ell\ ( s_\ell \in \mathbb{FS}(\rightarrow_{\phi},s_0,\pi_n)\Rightarrow s_\ell \sqsubset s_s ) $$
\end{definition}

\begin{definition}
\label{exact}
A state $s$ is \textbf{exact} with respect to a transition relation, $\rightarrow_\phi$, initial state, $s_0$, and control flow path, $\pi_n$, if and only if it is both sound and complete:
$$ s \cong \mathbb{FS}(\rightarrow_{\phi},s_0,\pi_n)$$
\end{definition}




\subsection{Theorems}

The goal of this section is to prove that the representation relation,
$\sqsubset$, is a bisimulation.  A bisimulation is a relation over pairs of states such that whenever two states, $s_\gse$ and
$s_\sym$, are related in the bisimulation, $s_\gse \sqsubset
s_\sym$, every successor, from either state, $s_\gse
\rgse s_\gse^\prime$ or $s_\sym \rsym
s_\sym^\prime$, has a corresponding mutual successor in the other
state such that both of those successors are also related in the
bisimulation: $s_\gse^\prime \sqsubset s_\sym^\prime$.

If $\rsym$ is considered to be a model of the
$\rgse$ (i.e., a representation of that machine), then the
representation relation, as a bisimulation, ensures that the
$\rsym$ is complete in that any property that can be
shown in states related on $\rgse$ can also be shown in the
$\rsym$; and further, the representation relation as a
bisimulation ensures that the $\rsym$ is also sound in
that any property that can be shown to hold in states related by $\rsym$
can also be shown to hold in the $\rgse$.

The proof of the representation relation as a bisimulation reasons
over individual rules in $\rsym$ to show that for
each rule, the representation relation, $\sqsubset$, exists. The heart
of the representation relation is the homomorphism that maps locations
in one heap to locations in the other heap. The proof reasons over of
each rule and as mentioned previously, and derives from the current
homomorphism in the representation relation for the current states, a
new homomorphism that is sufficient to use in the a new representation
relation that includes the successor states. The proof is constructive
in that it shows how given a valid homomorphism for the current
states, it is possible to derive a new homomorphism that includes
successor states. With such a new homomorphism, it is possible to
state that for $\rsym$, when restricted to a specific
rule (i.e., that is the only rule available in the relation), the
representation relation is a bisimulation for the restricted
$\rsym$. If such a bisimulation exists for all the
individual rules, then it exists for the rules collectively.

There are two slight complications in the proof: first, the field
access rule relies on $\rsum^*$ that operates on a different
state than $\rsym$; and second, constructing the new
homomorphism in the equals-reference rule relies on the incoming heap
being deterministic. The relation $\rsum^*$ effectively produces
an intermediary state that is the state where uninitialized references
are initialized before the field access actually takes place. In
essence, a state on the left of $\rsym$ that is a
field access, undergoes a transition in which its heap is changed to
initialize fields. Once the fields are initialized, then the actual
field access takes place. The state with the heap that has the
initialized fields is the intermediary state between the state on the
left of $\rsym$ and the state on the right of $\rsym$. The proof reasons separately about this
intermediary state to prove that it too is exact.

The equals-reference proof must show that any given reference in the
heap is not able to point to two distinct locations at the same
time. If the incoming heap is able to point to two valid locations at
a given reference at the same time, then the help is
non-deterministic, and it is not possible to construct a valid
homomorphism from the existing homomorphism: which location should be
used in the map? As such, the proof first establishes that
$\rsym$ preserves determinism when the incoming heap is
deterministic. Once that is shown, the exactness of the equals
reference rule is given.

The final statement that $\sqsubset$ is a bisimulation is in \thref{th:bisim}.

\begin{lemma}[$\sqsubset$ is a bisimulation for $\rinit$ and $\rsum$]
\label{lem:init}
If $s_\sym \cong \mathbb{FS}(\rightarrow_{\phi},s_0,\pi_n)$ for symbolic state $s_\sym = \lp \cfgnt{L}_\sym\ \cfgnt{R}_{s}\ \phi_\sym\ \eta\ \cfgnt{r}\ \lp \cfgt{*}\ \cfgt{\$}\ \cfgnt{f} \rightarrow \cfgnt{k}\rp \rp$, initial state $s_0$, and control flow path $\pi_n$, and if there exists some intermediate state state $s_\sym^\prime$ such that $(\cfgnt{L}_\sym\ \cfgnt{R}_{s}\ \cfgnt{r}\ \cfgnt{f} \ \cfgnt{C}) \rsum^* (\cfgnt{L}_{s^\prime}\ \cfgnt{R}_{s^\prime}\ \phi_{s^\prime}\ \cfgnt{r}\ \cfgnt{f} \ \cfgnt{C})$, then:
 $$s_\sym^\prime \cong \{\forall s_\gse^\prime | \exists s_\gse (s_\gse \sqsubset s_\sym \wedge (s_\gse \rinit^* s_\gse^\prime) ) \}$$
\end{lemma}

\begin{proof}
In order for a state to be equivalent to a set of GSE states, it must be both sound and complete with respect to the set. We will begin the proof by showing completeness, and then finish by demonstrating soundness.

To show completeness, we must show that any state in the set is represented by $s_\sym$. The definition of representation requires the both the existence of a homomorphism, and proof that the homomorphism constraint is satisfiable. To show that a homomorphism exits, take any GSE state $s_\gse$ such that $s_\gse \sqsubset s_\sym$. By Definition~\ref{representation}, we know $s_\gse = \lp \cfgnt{L}_\gse\ \cfgnt{R}_\gse\ \phi_\gse\ \eta\ \cfgnt{r}\ \lp \cfgt{*}\ \cfgt{\$}\ \cfgnt{f} \rightarrow \cfgnt{k}\rp \rp$. Take any state $s_\gse^\prime$ where $s_\gse \rinit^* s_\gse^\prime$, and state $s_\sym^\prime$ where $s_\sym \rsum^* s_\sym^\prime$. Note that  state $s_\gse^\prime$ has the form: $s_\gse^\prime = \lp \cfgnt{L}_{\gse^\prime}\ \cfgnt{R}_{\gse^\prime}\ \phi_{\gse^\prime}\ \eta\ \cfgnt{r}\ \lp \cfgt{*}\ \cfgt{\$}\ \cfgnt{f} \rightarrow \cfgnt{k}\rp \rp$.Take any location, field pair $\lp l_\gse\  \cfgnt{f}\rp$ such that $\lp l_\gse\  \cfgnt{f}\rp \in \cfgnt{R}_{\gse}^\leftarrow$, and let $l_\sym = h(l_\gse)$. We may classify $l_\gse$ into one of three ways, based on the values of the $\cfgnt{R}$ function in each of the states $s_\gse$, $s_\gse^\prime$, $s_\sym$, and $s_\sym^\prime$, and we may define a function $h^\prime: \mathcal{L} \mapsto \mathcal{L}$ based on that classification.

Class 1: $\cfgnt{R}_\gse ( l_\gse,  \cfgnt{f} ) = \cfgnt{R}_{\gse^\prime} ( l_\gse,  \cfgnt{f} )$ and $\cfgnt{R}_{s} ( l_\sym,  \cfgnt{f} ) = \cfgnt{R}_{s^\prime} ( l_\sym,  \cfgnt{f})$.  Let $l_\alpha$ be the location such that ${(\phi_a\ l_\alpha)} = \cfgnt{L}_{\gse}(\cfgnt{R}_{\gse} (\cfgnt{l}_\gse,f )) $. In this case, let $h^\prime(l_\alpha) = h(l_\alpha)$. Since $s_\gse \rightharpoonup_h s_\sym$, we may surmise that: 
$$(\phi_a\ \cfgnt{l}_\alpha) \in \cfgnt{L}_{\gse^\prime}(\cfgnt{R}_{\gse^\prime} (\cfgnt{l}_\gse,f )) \Rightarrow (\phi_b\ h^\prime (\cfgnt{l}_\alpha))\in \cfgnt{L}_{s^\prime}(\cfgnt{R}_{s^\prime} (\cfgnt{l}_\sym,f ))$$

Class 2: $\cfgnt{R}_\gse ( l_\gse,  \cfgnt{f}) = \cfgnt{R}_{\gse^\prime} ( l_\gse,  \cfgnt{f})$ and $\cfgnt{R}_{s} ( l_\sym,  \cfgnt{f}) \neq \cfgnt{R}_{s^\prime} ( l_\sym,  \cfgnt{f})$. Since $\cfgnt{R}_{s} ( l_\sym,  \cfgnt{f}) \neq \cfgnt{R}_{s^\prime} ( l_\sym,  \cfgnt{f})$, the Summarize rule must have altered this reference. A reference created by the Summarize rule has a value set $\theta_{all}$ with four subsets: $\theta_{null}$, $\theta_{new}$, $\theta_{alias}$, and $\theta_\mathit{orig}$. Because $\cfgnt{R}_\gse ( l_\gse,  \cfgnt{f}) = \cfgnt{R}_{\gse^\prime} ( l_\gse,  \cfgnt{f})$, we know that the location we want to map to lies in $\theta_\mathit{orig}$. Let $l_\alpha$ be the location such that ${(\phi_a\ l_\alpha)} = \cfgnt{L}_{\gse}(\cfgnt{R}_{\gse} (\cfgnt{l}_\gse,f )) $, and let $l_{orig} = h(l_\alpha)$. In this case, we let $h^\prime(l_\alpha) = h(l_\alpha)$. Since $(\phi_a\ l_\alpha) \in \cfgnt{L}_{\gse^\prime}(\cfgnt{R}_{\gse^\prime} (\cfgnt{l}_\gse,f ))$. Let $l_{orig} = h(l_\alpha)$. We can see that by the Summarize rule $ (\phi_b\ l_\mathit{orig})\in \cfgnt{L}_{s^\prime}(\cfgnt{R}_{s^\prime} (\cfgnt{l}_\sym,f ))$, so therefore:
$$(\phi_a\ \cfgnt{l}_\alpha) \in \cfgnt{L}_{\gse^\prime}(\cfgnt{R}_{\gse^\prime} (\cfgnt{l}_\gse,f )) \Rightarrow (\phi_b\ h^\prime (\cfgnt{l}_\alpha))\in \cfgnt{L}_{s^\prime}(\cfgnt{R}_{s^\prime} (\cfgnt{l}_\sym,f ))$$

Class 3: $\cfgnt{R}_\gse ( l_\gse,  \cfgnt{f}) \neq \cfgnt{R}_{\gse^\prime} ( l_\gse,  \cfgnt{f})$ and $\cfgnt{R}_{s} ( l_\sym,  \cfgnt{f}) \neq \cfgnt{R}_{s^\prime} ( l_\sym,  \cfgnt{f})$. In this case, there are two possibilities: either the new reference $\cfgnt{R}_{\gse^\prime} ( l_\gse,  \cfgnt{f})$ points to some location we've seen before $l_\alpha$, or it points to a previously unobserved location $l_\beta$. By establishing which of these possibilities has happened, we can build $h^\prime$. To construct $h^\prime$, let $l_\alpha$ be any location such that $(\phi_a\ \cfgnt{l}_\alpha) \in \cfgnt{L}_{\gse^\prime}(\cfgnt{R}_{\gse^\prime} (\cfgnt{l}_\gse,f )) $. If there exists $\phi_\alpha$ such that $\lp \phi_\alpha\ l_\alpha \rp \in \cfgnt{L}_{\gse}^\rightarrow $, let $h^\prime(l_\alpha) = h(l_\alpha)$. Otherwise, let $l_\beta$ be the location such that $(\phi_b\ \cfgnt{l}_\beta) \in \cfgnt{L}_{s^\prime}(\cfgnt{R}_{s^\prime} (\cfgnt{l}_\sym,f )) $ and $(\phi_b\ \cfgnt{l}_\beta) \notin \cfgnt{L}_{s}(\cfgnt{R}_{s} (\cfgnt{l}_\sym,f )) $. Now, let $h^\prime(l_\alpha) = l_\beta$. Observe that either way,
$$(\phi_a\ \cfgnt{l}_\alpha) \in \cfgnt{L}_{\gse^\prime}(\cfgnt{R}_{\gse^\prime} (\cfgnt{l}_\gse,f )) \Rightarrow (\phi_b\ h^\prime (\cfgnt{l}_\alpha))\in \cfgnt{L}_{s^\prime}(\cfgnt{R}_{s^\prime} (\cfgnt{l}_\sym,f ))$$

Furthemore, since $l_\alpha$ and $l_\beta$ are new locations with uninitialized fields, we know that for any field $f^\prime$, $\{(\phi_p\ \bot)\} = \cfgnt{L}_{\gse^\prime}(\cfgnt{R}_{\gse^\prime} (\cfgnt{l}_\alpha,f^\prime ))$ and $ \{(\phi_p\ \bot)\} = \cfgnt{L}_{s^\prime}(\cfgnt{R}_{s^\prime} (\cfgnt{l}_\beta,f^\prime ))$ therefore, we know that:
$$(\phi_p\ \cfgnt{l}_x) \in \cfgnt{L}_{\gse^\prime}(\cfgnt{R}_{\gse^\prime} (\cfgnt{l}_\alpha,f^\prime )) \Rightarrow (\phi_q\ h^\prime (\cfgnt{l}_x))\in \cfgnt{L}_{s^\prime}(\cfgnt{R}_{s^\prime} ( h^\prime(\cfgnt{l}_\alpha),f ))$$

We have now shown that there exists a mapping $h^\prime: \mathcal{L} \mapsto \mathcal{L}$ for all $\cfgnt{l}_{\gse^\prime} \in \cfgnt{L}_{\gse^\prime}^\rightarrow$ such that:
$$ (\phi_a\ \cfgnt{l}_\alpha) \in \cfgnt{L}_{\gse^\prime}(\cfgnt{R}_{\gse^\prime} (\cfgnt{l}_{\gse^\prime},f )) \Rightarrow (\phi_b\ h^\prime (\cfgnt{l}_\alpha))\in \cfgnt{L}_{s^\prime}(\cfgnt{R}_{s^\prime} (\cfgnt{l}_{\gse^\prime},f )) $$
By Definition~\ref{def:homomorphism} we know that $s_\gse^\prime \rightharpoonup_{h^\prime} s_\sym^\prime$. 

It remains to show that $\mathbb{S}( \phi_\sym^\prime \wedge \mathbb{HC}(s_\gse^\prime \rightharpoonup_{h} s_\sym^\prime) )$. For locations in Class 1, no new conjuncts are added to $\mathbb{HC}(s_\gse^\prime \rightharpoonup_{h} s_\sym^\prime)$, and therefore the satisfiability cannot be changed. For locations in Class 2 or Class 3, the new constraints take either the form $\phi_x \wedge \phi_\mathit{orig}$, or $\phi_x \wedge (r_f\ op\ r_{a}) \wedge (r_f\ op\ r_{b}) \wedge ...$. Constraints of the form  $\phi_x \wedge \phi_\mathit{orig}$ contain terms $\phi_x$ and $\phi_\mathit{orig}$ which were already conjoined to prior heap constraint, so satisfiability is not affected. In constraints of the form $\phi_x \wedge (r_f\ op\ r_{a}) \wedge (r_f\ op\ r_{b}) \wedge ...$, the term $\phi_x$ is conjoined to the prior heap constraint, and all the other terms involve the new variable $r_f$, so satisfiability is not affected. Since the previous heap constraint is satisfiable, and none of the new terms can impact the satisfiability, we know that the new heap constraint must also be satisfiable.

Since the heap constraint is satisfiable, we know that $s_\gse^\prime \sqsubset s_\sym^\prime $. We have therefore shown that for some symbolic state $s_\sym$ and an arbitrary GSE state $s_\gse$ such that $s_\gse \sqsubset s_\sym$ :
\begin{equation} 
(s_\gse \rinit^* s_\gse^\prime \wedge s_\sym \rsum^* s_\sym^\prime) \Rightarrow s_\gse^\prime \sqsubset s_\sym^\prime 
\end{equation}

We now prove the reverse case, that $s_\sym^*$ represents no infeasible states. Suppose that $s_\sym^\prime$ represents some infeasible state. This means that we represent some GSE state that has some reference r which points somewhere that no place in the feasible set points to. Since we don't change the path condition, all the old references still point exactly to the same places they used to. 

So, the problem must be with one of the new references. All of the new references point to either a new location, the null location, the uninitialized location, or some alias. In the Summarize rule, the values and constraints for the new, null, uninitialized, and alias locations are contained in the sets $\theta_{new}$, $\theta_{null}$, $\theta_\mathit{orig}$, and $\theta_{alias}$. Since the null, and uninitialized locations are already accounted for by the homomorphism $s_\gse \rightharpoonup_h s_\sym$, and since a new location was created symmetrically for both $s_\gse^\prime$ and $s_\sym^\prime$, the problem must be with some alias location that is part of $s_\sym$ but not $s_\gse$. This means that there must be a feasible path to a target location that does not exist for any GSE heap. So, pick an arbitrary GSE heap containing the location and field in question. If said target location does not exist, then there is no reference in the GSE heap pointing to that location. In the symbolic heap, the path constraint on the path leading to the undesired target contains an aliasing condition that states that the source reference only points to this target location on condition that the parent reference points there. However, since we already know that no other reference in the GSE heap points there, this condition must be infeasible. Therefore, it is not part of the represented state. We have a contradiction. Therefore, there is no alias that points somewhere it's not supposed to.

We have now proven that 
$$ s_\gse^* \sqsubset s_\sym^*  \Rightarrow  s_\gse^* \in \{\forall s_\gse^\prime | \exists s_\gse (s_\gse \sqsubset s_\sym \wedge (s_\gse \rinit^* s_\gse^\prime) ) \}$$
This fact, combined with our previous result, proves that
$$s_\sym^*  \cong \{\forall s_\gse^\prime | \exists s_\gse (s_\gse \sqsubset s_\sym \wedge (s_\gse \rinit^* s_\gse^\prime) ) \}$$

\end{proof}

%we need to define how r' is arrived at for this one to work...
\begin{lemma}[\textrm{F{\footnotesize IELD}} \textrm{A{\footnotesize CCESS}} preserves $\sqsubset\ \subseteq F_\sim(\sqsubset)$]
If two field access states are related in the represented by relation,
then they are also related in the functional associated to bisumlation.
\label{lem:access}
$$
\forall p \in S_\mathit{FA}\ (\forall q \in S_\mathit{FA}\ (p \sqsubset q \Rightarrow (p\ q) \in F_\sim(\sqsubset)))
$$
\end{lemma}

\begin{proof}
Proof by contradiction: assume $p \sqsubset q \wedge (p\ q) \not\in F_\sim(\sqsubset)$. 

The case where neither $p$ nor $q$ have successors is trivially satisfied by
\defref{bisimulation} since the conditions for inclusion in
$F_\sim(\sqsubset)$ are vacuously met. That is a contradiction.

Consider now the case where $p$ and $q$ have successor states. The
statement $p \sqsubset q$ (\defref{representation}) means that $p$ and
$q$ are only differentiated by their heaps and global constraints
since the environment ($\eta$), expression ($e$), and continuation
($k$) are the same in both states.

Ignoring the heaps, and given that $p$ and $q$ currently have
the same environment, expression, and continuation, all successors
of $p$ related by \textrm{F{\footnotesize IELD}}
\textrm{A{\footnotesize CCESS}} in $\rgse$ (\figref{fig:lazy}) have
the same unchanged environment from $p$, $\eta$, new expression
$r^\prime = \mathrm{stack}_\cfgnt{r}()$, and continuation $\cfgnt{k}$
from the old continuation in $p$ having completed the field
access. Similarly, the one successor of $q$ related by $\rsym$
(\figref{fig:symfield}), has that same $\eta$, $r^\prime =
\mathrm{stack}_\cfgnt{r}()$, and $\cfgnt{k}$. As such, every successor
of $p$/$q$ has a matching successor of $q$/$p$ that agrees on the
environment, expression, and continuation meeting the first condition
necessary to relate the successors in $\sqsubset$.

Turning now to the heaps, \textrm{F{\footnotesize IELD}}
\textrm{A{\footnotesize CCESS}}, in $\rgse$ and $\rsym$, initializes
uninitialized locations in the heap on reference $\cfgnt{r}$ for field
$\cfgnt{f}$. The heaps in $p$ and $q$ are still homomorphic after
initialization, and the heap constraint in the homomorphism is still
valid by \lemref{lem:init}. Let
$(\cfgnt{L}^\prime_p\ \cfgnt{R}^\prime_p) \rightharpoonup_h
(\cfgnt{L}^\prime_q\ \cfgnt{R}^\prime_q)$ be those new heaps and
the homomorphism after initialization on the heap in $p$ with
$\rightarrow_I^*$ (\figref{fig:lazyInit}) and the heap in $q$ with
$\rightarrow_S^*$ (\figref{fig:symInit}).

After initialization, \textrm{F{\footnotesize IELD}} \textrm{A{\footnotesize CCESS}} for $\rgse$ creates a new heap
$(\cfgnt{L}^\prime_p [\cfgnt{r}^\prime \mapsto \lp\phi^\prime\ l^\prime\rp]\ \cfgnt{R}^\prime_p$

and for $\rsym$ it relates the state 
$$ 
s_\sym^\prime = \lp \cfgnt{L}_{s}[\cfgnt{r}^\prime \mapsto \mathbb{VS}\lp \cfgnt{L}_{s},\cfgnt{R}_{s},\cfgnt{r},\cfgnt{f},\phi_g\rp ]\ \cfgnt{R}_{s}\ \phi_g\ \eta\ \cfgnt{r}^\prime\ \cfgnt{k}\rp 
$$

The functional in \defref{} requires and forward
\eqref{eqn:BisimulationForwards} and backward
\eqref{eqn:BisimulationBackwards} relationship between successor
heaps.



\eqref{eqn:BisimulationForwards}
\eqref{eqn:BisimulationBackwards}


We now show that $s_\gse^\prime \sqsubset s_\sym^\prime$. Since $\eta$, $e$, and $k$ are identical between $s_\sym^\prime$ and $s_\gse^\prime $, the first condition is met by default. Now we construct the function $h^\prime$ such that $h^\prime = h$. Observe that since $s_\gse \rightharpoonup_{h} s_\sym$, and since $\cfgnt{R}_\gse$ and $\cfgnt{R}_\sym$ are unchanged from states $s_\gse$ to $s_\gse^\prime$ and $s_\sym$ to $s_\sym^\prime$ respectively, we are guaranteed that $ \cfgnt{r} = \cfgnt{R}_\gse(l,f) \Rightarrow \cfgnt{r} = \cfgnt{R}_\sym(h^\prime(l),f)$. Let $\{(\phi_\gse^\prime\ l^\prime)\} =  \cfgnt{L}_\gse(\cfgnt{R}_\gse(l,f))$. Since $\mathbb{S}(\phi_g \wedge \mathbb{HC}(s_\gse \rightharpoonup_{h} s_\sym))$ is valid, we know that:
 $$(\phi_\sym \wedge \phi_\sym^\prime\ h(l^\prime)) \in \mathbb{VS}\lp \cfgnt{L}_{s},\cfgnt{R}_{s},\cfgnt{r},\cfgnt{f},\phi_g\rp$$ 
From this, we may deduce that:
$$ (\phi_\gse\ l) \in \cfgnt{L}_\gse^\prime(\cfgnt{r}^\prime) \Rightarrow (\phi_\sym \wedge \phi_\sym^\prime\ h^\prime(l))\in \cfgnt{L}_\sym^\prime(\cfgnt{r}^\prime)$$
Since $\cfgnt{r}^\prime$ is the only new addition to $L_\gse^\prime$ and $L_\sym^\prime$, we now know that the assertion above holds for all $l \in \mathcal{L}$. Thus, we have shown that $s_\gse^\prime \rightharpoonup_{h} s_\sym^\prime$. Furthermore, since the constraints in $\mathbb{HC}(s_\gse^\prime \rightharpoonup_{h^\prime} s_\sym^\prime)$ are constructed using conjuncts already present in $ \mathbb{HC}(s_\gse \rightharpoonup_{h} s_\sym)$, we are guaranteed that $\mathbb{HC}(s_\gse^\prime \rightharpoonup_{h^\prime} s_\sym^\prime) \Leftrightarrow \mathbb{HC}(s_\gse \rightharpoonup_{h} s_\sym)$, and therefore $\mathbb{S}(\phi_g \wedge \mathbb{HC}(s_\gse^\prime \rightharpoonup_{h^\prime} s_\sym^\prime))$. This fact, and the fact that $\eta_{\gse} = \eta_{s} ,\ \cfgnt{e}_{\gse} = \cfgnt{e}_{s} ,\ \cfgnt{k}_{\gse} = \cfgnt{k}_{s}$, means that by Definition~\ref{representation} we know $s_\gse^\prime \sqsubset s_\sym^\prime$. We have now shown that:
\begin{equation}
\forall s_\gse^\prime ( s_\gse \rgse s_\gse^\prime \Rightarrow \exists s_\sym^\prime( (s_\sym \rsym s_\sym^\prime )\wedge (s_\gse^\prime\ \sqsubset\ s_\sym^\prime ))  )
\end{equation}

Now, suppose that there exists a state $s_i^\prime$ such that $s_i^\prime \sqsubset s_\sym^\prime$. Since $s_i^\prime \sqsubset s_\sym^\prime$, then by Definition~\ref{representation}, we know there exists a homomorphism $s_i^\prime \rightharpoonup_{h^\prime} s_\sym^\prime$, and that $\mathbb{S}( \phi_i^\prime \wedge \mathbb{HC}(s_i^\prime \rightharpoonup_{h^\prime} s_\sym^\prime) )$. From state $s_i^\prime$, construct state $s_i$ such that 
\begin{align*}
s_i &= \lp \cfgnt{L}_{i}\ \cfgnt{R}_{i}\ \phi_i\ \eta\ \cfgnt{r}\ \lp \cfgt{*}\ \cfgt{\$}\ \cfgnt{f} \rightarrow \cfgnt{k}\rp \rp\\
L_i &= L_{i^\prime} \setminus \{\cfgnt{r}^\prime \}\\
R_i &= R_{i^\prime}\\
\phi_i &= \phi_i^\prime
\end{align*}
Observe that by virtue of the GSE Field Access rule, $s_i \rgse s_i^\prime$. Now, construct function $h_i$ so that $h_i = h^\prime$. Observe that by Definition~\ref{def:homomorphism} $s_i \rightharpoonup_{h_i} s_\sym$,  and that $\mathbb{S}( \phi_i \wedge \mathbb{HC}(s_i \rightharpoonup_{h_i} s_\sym) )$, so $s_i \sqsubset s_\sym$. Therefore:
\begin{equation}
\forall s_\sym^\prime ( s_\sym \rsym s_\sym^\prime\Rightarrow \exists s_\gse^\prime( (s_\gse \rgse s_\gse^\prime )\wedge (s_\gse^\prime\ \sqsubset\ s_\sym^\prime ))  )
\end{equation}
 This concludes the proof.
\end{proof}

%proof for field write
\begin{lemma}[Exactness of Field Write Rule]
\label{lem:write}
If there exists states $s_\gse$ and $s_\sym$ such that $s_\sym \in \mathcal{FW}$ and $s_\gse \sqsubset s_\sym$, then:
\begin{equation}
\forall s_\gse^\prime ( s_\gse \rgse s_\gse^\prime \Rightarrow \exists s_\sym^\prime( (s_\sym \rsym s_\sym^\prime )\wedge (s_\gse^\prime\ \sqsubset\ s_\sym^\prime ))  )
\end{equation}
and
\begin{equation}
\forall s_\sym^\prime ( s_\sym \rsym s_\sym^\prime\Rightarrow \exists s_\gse^\prime( (s_\gse \rgse s_\gse^\prime )\wedge (s_\gse^\prime\ \sqsubset\ s_\sym^\prime ))  )
\end{equation}
\end{lemma}
\begin{proof}
Begin by assuming the conditions from Lemma~\ref{lem:write}.

The first step is to show that there exists a state $s_\sym^\prime$ that is complete with respect to the feasible set. Take state $s_\sym$ and compute state $s_\sym^\prime$ such that $s_\sym \rsym s_\sym^\prime$. Take any GSE state $s_\gse$ such that $s_\gse \sqsubset s_\sym$, and find state $s_\gse^\prime$ such that $s_\gse \rgse s_\gse^\prime$. Let $l_\gse$ be the location such that $\{(\phi_a\ l_\gse )\} = \cfgnt{L}_\gse(\cfgnt{r}_x) $ for some $\phi_a$. To show that $s_\gse^\prime \sqsubset s_\sym^\prime$, we need to demonstrate that there exists a function $h^\prime$ such that $s_\gse^\prime \rightharpoonup_{h^\prime} s_\sym^\prime$, and that $\mathbb{S}(\phi_{s^\prime} \wedge \mathbb{HC}(s_\gse^\prime \rightharpoonup_{h^\prime} s_\sym^\prime) )$. Since $s_h \sqsubset s_\sym$, we know that there exists a function $h$ such that $s_\gse \rightharpoonup_{h} s_\sym$. Let $h^\prime = h$. 

First, we consider how $s_\gse^\prime \rightharpoonup_{h^\prime} s_\sym^\prime$. Let $l_\alpha$ and $l_\beta$ be arbitrary locations in $\cfgnt{L}_{\gse^\prime}^\rightarrow$ such that $\{(\phi_a\ \cfgnt{l}_\alpha)\} = \cfgnt{L}_{\gse^\prime}(\cfgnt{R}_{\gse^\prime}(\cfgnt{l}_\beta,f ))$, let $\theta = \cfgnt{L}_\sym(\cfgnt{R}_\sym(h(l_\gse),f))$, and let $\theta^\prime =   \cfgnt{L}_\sym^\prime(\cfgnt{R}_\sym^\prime(h(l_\gse),f))$. 

Suppose $l_\beta \neq l_\gse$. In this case either $\theta = \theta^\prime$ or $\theta \neq \theta^\prime$ . In the first case, we are guaranteed that the homomorphism works by default. Otherwise, if $\theta \neq \theta^\prime$.  We can see from the construction of the set $X$ in the symbolic Field Write rule that any feasible location in the set $\theta$ must also be in the set $\theta^\prime$. Since $s_\gse \sqsubset s_\sym$, we know that $h(l_\alpha)$ is in $\theta$, and is likewise in $\theta^\prime$. We have now established that in either case where $l_\beta \neq l_\gse$, $(\phi_b\ h(\cfgnt{l}_\alpha))\in \cfgnt{L}_{s^\prime}(\cfgnt{R}_{s^\prime} (h(\cfgnt{l}_\beta),f ))$.

On the other hand, suppose $l_\beta = l_\gse$. In this case we know that $\{(\phi_a\ \cfgnt{l}_\alpha)\} = \cfgnt{L}_{\gse^\prime}(\cfgnt{R}_{\gse^\prime}(\cfgnt{l}_\gse,f ))$. From the GSE field rule, we can surmise that $(\phi_a\ \cfgnt{l}_\alpha) \in \cfgnt{L}_{\gse}(\cfgnt{r} )$, and since $s_\gse \sqsubset s_\sym$, we know that $(\phi_b\ h(\cfgnt{l}_\alpha)) \in \cfgnt{L}_{s}(\cfgnt{r})$ for some constraint $\phi_b$. Using this fact, we can apply the symbolic Field Write rule to infer that $l_\alpha$ must be one of the locations in $\theta^\prime$, and therefore $(\phi_c\ h(\cfgnt{l}_\alpha)) \in \cfgnt{L}_{s^\prime}(\cfgnt{R}^\prime(l_\gse,f))$

Thus, for arbitrary $l_\alpha$ and $l_\beta$:
$$(\phi_a\ \cfgnt{l}_\alpha) \in \cfgnt{L}_{\gse^\prime}(\cfgnt{R}_{\gse^\prime}(\cfgnt{l}_\beta,f )) \Rightarrow (\phi_b\ h(\cfgnt{l}_\alpha))\in \cfgnt{L}_{s^\prime}(\cfgnt{R}_{s^\prime} (h(\cfgnt{l}_\beta),f ))$$
Therefore, we have shown that $s_\gse^\prime \rightharpoonup_{h^\prime} s_\sym^\prime$.

Establishing the fact that $\mathbb{S}(\phi_{s^\prime} \wedge \mathbb{HC}(s_\gse^\prime \rightharpoonup_{h^\prime} s_\sym^\prime) )$ is left as an exercise to the reader (waving hands in the air).

By proving the existence of a valid homomorphism, we have shown that for any state $s_\gse^\prime$ such that $s_\gse \rgse s_\gse^\prime$, then the state $s_\sym^\prime$ such that $s_\sym \rsym s_\gse^\prime$ represents $s_\gse^\prime$. Therefore, $\forall s_\gse^\prime ( s_\gse \rgse s_\gse^\prime \Rightarrow \exists s_\sym^\prime( (s_\sym \rsym s_\sym^\prime )\wedge (s_\gse^\prime\ \sqsubset\ s_\sym^\prime ))  )$. This concludes the proof of completeness.

To show that $s_\sym^\prime$ is sound with respect to the feasible set, we use the same argument as in the field read proof: that any state $s_\gse^\prime$ represented by $s_\sym^\prime$ must have a counterpart state $s_\gse$ such that $s_\gse \rgse s_\gse^\prime$ and $s_\gse \sqsubset s_\sym$. Because $s_\sym$ is exact, $s_\gse^\prime$ must be a part of the feasible set. Therefore, $\forall s_\sym^\prime ( s_\sym \rsym s_\sym^\prime\Rightarrow \exists s_\gse^\prime( (s_\gse \rgse s_\gse^\prime )\wedge (s_\gse^\prime\ \sqsubset\ s_\sym^\prime ))  )$.

\end{proof}


\begin{lemma}[$\rsum^*$ preserves heap determinism]
\label{lem:S-determ}
Given a deterministic heap, $(\cfgnt{L}_0\ \cfgnt{R}_0)$, from a
state with a reference $\cfgnt{r}$ and field $\cfgnt{f}$, the
new heap, $(\cfgnt{L}^\prime\ \cfgnt{R}^\prime)$, from the 
summary machine, $(\cfgnt{L}_0\ \cfgnt{R}_0\ \cfgnt{r}\ \cfgnt{f})
\rsum^*
(\cfgnt{L}^\prime\ \cfgnt{R}^\prime\ \cfgnt{r}\ \cfgnt{f})$, is also deterministic.
\end{lemma}
\begin{proof}
Induction over the number of steps in $\rsum^*$ in \figref{fig:symInit}.

\noindent\textbf{Base Case}. The relation makes one step: $(\cfgnt{L}_0\ \cfgnt{R}_0\ \cfgnt{r}\ \cfgnt{f})
\rsum
(\cfgnt{L}_1\ \cfgnt{R}_1\ \cfgnt{r}\ \cfgnt{f})$. Let $\Lambda = \mathbb{UN}(\cfgnt{L}_0, \cfgnt{R}_0, \cfgnt{r},
\cfgnt{f})$ be the set of uninitialized locations. If $\Lambda = \emptyset$, then the \textrm{S{\footnotesize UMMARIZE-END}}
rule is active and $(\cfgnt{L}_1\ \cfgnt{R}_1) = (\cfgnt{L}_0\ \cfgnt{R}_0)$, which is deterministic by the initial conditions in the lemma.

If $\Lambda \neq \emptyset$, then the \textrm{S{\footnotesize UMMARIZE}} rule is
active, and each new constraint location pair must be considered
individually. These pairs are partitioned into the sets
$\theta_\mathit{null}$, $\theta_\mathit{new}$,
$\theta_\mathit{alias}$, and $\theta_\mathit{orig}$ by the rule. 
\begin{itemize}
\item The original heap is
  deterministic by definition, so any constraint in any member of the
  set must have some term such that
  \[\begin{array}{l}
     \forall (\phi\ \cfgnt{l}),(\phi^\prime\ \cfgnt{l}^\prime) \in \theta_\mathit{orig}\ \\
     \ \ \ \ ((\cfgnt{l} \neq \cfgnt{l}^\prime \vee \phi \neq \phi^\prime) \Rightarrow (\phi \wedge \phi^\prime = \cfgt{false}))
     \end{array}
  \]
  Further, any member of $\theta_\mathrm{orig}$ has a constraint
  of the form $\phi = \neg \phi_x \wedge \ldots$ while any member of $\theta_\mathit{null}$, $\theta_\mathit{new}$, and
  $\theta_\mathit{alias}$ has a constraint of the form $\phi^\prime = \phi_x \wedge
  \ldots$; thus
  \[\begin{array}{l}
     \forall (\phi\ \cfgnt{l}) \in \theta_\mathit{orig}\ (\forall (\phi^\prime\ \cfgnt{l}^\prime) \in \theta_\mathit{null} \cup \theta_\mathit{new} \cup \theta_\mathit{alias}\ (\\
     \ \ \ \ \phi \wedge \phi^\prime = \cfgt{false}))
    \end{array}
  \]
\item The only member of $\theta_\mathit{null}= \{(\phi\ \cfgnt{l}_\mathit{null})\}$ has the form $\phi = \ldots \wedge
  \cfgnt{r}_f = \cfgnt{r}_\mathit{null}$ while any member of
  $\theta_\mathit{new}$ and $\theta_\mathit{alias}$ has the form
  $\phi^\prime = \ldots \wedge \cfgnt{r}_f \neq \cfgnt{r}_\mathit{null} \wedge
  \ldots$; thus
  \[\begin{array}{l}
     \forall (\phi^\prime\ \cfgnt{l}^\prime) \in \theta_\mathit{new} \cup \theta_\mathit{alias}\ (
     \phi \wedge \phi^\prime = \cfgt{false})
    \end{array}
  \]
\item The only member of $\theta_\mathit{new} =  \{(\phi\ \cfgnt{l}_f)\}$ has a constraint of the form
  $\phi = \ldots \wedge ( \wedge_{( \cfgnt{r}_a,\ \phi_a,\ l_a) \in \rho}
  \cfgnt{r}_f \ne \cfgnt{r}_a) )$ to assert it does not alias
  anything, while any member of $\theta_\mathit{alias}$ has the form
  $\phi^\prime = \ldots \wedge \cfgnt{r}_f = \cfgnt{r}_a \wedge \ldots$ to assert it aliases some
  $\cfgnt{r}_a$ with both partitions reasoning over the same set
  of aliases $\rho$; thus
  \[\begin{array}{l}
     \forall (\phi^\prime\ \cfgnt{l}^\prime) \in \theta_\mathit{alias}\ (
     \phi \wedge \phi^\prime = \cfgt{false})
    \end{array}
  \]
\item Any member of $\theta_\mathit{alias}$ has the form
\[\ldots \wedge \cfgnt{r}_f = \cfgnt{r}_a \wedge ( \wedge_{( \cfgnt{r}^{\prime}_a\ \phi^{\prime}_a\ l^{\prime}_a)  \in \rho\ ( \cfgnt{r}^\prime_a \neq \cfgnt{r}_a) } \cfgnt{r}_f \neq \cfgnt{r}^{\prime}_a)\]
And thus, 
  \[\begin{array}{l}
     \forall (\phi\ \cfgnt{l}),(\phi^\prime\ \cfgnt{l}^\prime) \in \theta_\mathit{alias}\\
     \ \ \ \ ((\cfgnt{l} \neq \cfgnt{l}^\prime \vee \phi \neq \phi^\prime) \Rightarrow (\phi \wedge \phi^\prime = \cfgt{false}))
     \end{array}
  \]
\end{itemize}
As $\theta$ is mapped to a single reference $r_f = \mathit{init}_r()$ in an already deterministic heap, the
resulting heap $(\cfgnt{L}_1\ \cfgnt{R}_1)$ is likewise deterministic.

\noindent\textbf{Inductive Step}. The machine takes $n$-steps:
\[
(\cfgnt{L}_0\ \cfgnt{R}_0\ \cfgnt{r}\ \cfgnt{f}) \rsum (\cfgnt{L}_1\ \cfgnt{R}_1\ \cfgnt{r}\ \cfgnt{f}) \rsum \ldots \rsum
(\cfgnt{L}_n\ \cfgnt{R}_n\ \cfgnt{r}\ \cfgnt{f})
\]
By the induction hypothesis, $(\cfgnt{L}_n\ \cfgnt{R}_n)$ is deterministic. This matches the base case, in that the heap on the
left side of $\rsum$ is deterministic, and by the
same argument as in the base case,
$(\cfgnt{L}_{n+1}\ \cfgnt{R}_{n+1})$ is thus deterministic.
\end{proof}

\begin{lemma}[$\rightarrow_\mathit{FA}$ preserves heap determinism]
\label{lem:FA-determ}
Given a state, $s$, with a deterministic heap,
$(\cfgnt{L}\ \cfgnt{R}) =
\mathrm{heap}(s)$, the new heap,
$(\cfgnt{L}^\prime\ \cfgnt{R}^\prime) =
\mathrm{heap}(s^\prime)$, in any state related by the field
access rule, $s \rightarrow_\mathit{FA} s^\prime$,
is also deterministic.
\end{lemma}
\begin{proof}
Proof by definition of $\rightarrow_\mathit{FA}$ in \figref{fig:sym}.

\lemref{lem:S-determ} establishes that the heap in the state on the right side of
$\rsum^*$ is deterministic if the heap in the state on the left side
is deterministic, so it is only needed to show that determinism is
preserved by the call to the value function, $\mathbb{VS}$, in the
rule. Let $(\cfgnt{L}\ \cfgnt{R})$ be the new deterministic heap related by $\rsum^*$.

Recall from \defref{def:VS} that each constraint in each member of the value set has the form
$(\phi\wedge \phi^\prime\ \cfgnt{l})$. Choose any two distinct members of the value set, $(\phi_\alpha \wedge
\phi_\alpha^\prime\ \cfgnt{l}_\alpha^\prime)$ and $(\phi_\beta \wedge
\phi_\beta^\prime\ \cfgnt{l}_\beta^\prime)$.
\begin{itemize}
\item If $\phi_\alpha = \phi_\beta$, then by \defref{def:VS} 
\[\begin{array}{l}
\exists (\phi_\alpha\ \cfgnt{l}) \in \cfgnt{L}(\cfgnt{r})\ (\exists \cfgnt{r}^\prime \in \cfgnt{R}(\cfgnt{l},\cfgnt{f})\ (\\
\ \ \ \ (\phi_\alpha^\prime\ l_\alpha^\prime) \in \cfgnt{L}(\cfgnt{r}^\prime) \wedge (\phi_\beta^\prime\ l_\beta^\prime) \in \cfgnt{L}(\cfgnt{r}^\prime)))
\end{array}
\]
As $(\phi_\alpha^\prime\ \cfgnt{l}_\alpha^\prime)$ and $(\phi_\beta^\prime\ \cfgnt{l}_\beta^\prime)$ are distinct and connected to the same reference $\cfgnt{r}^\prime$ in a deterministic heap, $\phi_\alpha^\prime \wedge  \phi_\beta^\prime = \cfgt{false}$ by definition.
\item If $\phi_\alpha \ne \phi_\beta$, then by \defref{def:VS} 
\[\begin{array}{l}
\exists \cfgnt{l}\ ((\phi_\alpha\ \cfgnt{l}) \in \cfgnt{L}(\cfgnt{r}) \wedge \exists \cfgnt{l}^\prime \neq \cfgnt{l} \ ((\phi_\beta\ \cfgnt{l}^\prime) \in \cfgnt{L}(\cfgnt{r})))
\end{array}
\]
As $(\phi_\alpha\ \cfgnt{l})$ and $(\phi_\beta\ \cfgnt{l}^\prime)$ are distinct and connected to the same reference $\cfgnt{r}$ in a deterministic heap, $\phi_\alpha \wedge  \phi_\beta = \cfgt{false}$ by definition.
\end{itemize}
The only change to the heap after the $S$-relation is the addition of
the new reference $\cfgnt{r}^\prime = \mathrm{stack}_\cfgnt{r}()$ to
point to the value set. As the value set meets the conditions for
determinism, the new heap with $r^\prime$ and the value set,
$(\cfgnt{L}^\prime\ \cfgnt{R}^\prime) =
\mathrm{heap}(s^\prime)$, is also deterministic.
\end{proof}

\begin{lemma}[$\rightarrow_\mathit{FW}$ preserves heap determinism]
\label{lem:FW-determ}
Given a state, $s_\sym$, with a deterministic heap,
$(\cfgnt{L}_\sym\ \cfgnt{R}_\sym) =
\mathrm{heap}(s_\sym)$, the new heap,
$(\cfgnt{L}_\sym^\prime\ \cfgnt{R}_\sym^\prime) =
\mathrm{heap}(s_\sym^\prime)$, in any state related by the field
write rule, $s_\sym \rightarrow_\mathit{FW} s_\sym^\prime$,
is also deterministic.
\end{lemma}
\begin{proof}
Proof by definition of $\rightarrow_\mathit{FW}$ in \figref{fig:sym}.

The $\rightarrow_\mathit{FW}$ rule relies on the $\mathbb{ST}$
function. Recall from \defref{def:ST} that each constraint in each
member of the strengthened set has the form $(\phi\wedge
\phi^\prime\ \cfgnt{l}^\prime)$ where every member,
$(\phi^\prime\ \cfgnt{l}^\prime)$, comes from
$\cfgnt{L}_\sym(\cfgnt{r})$ on the same reference in a
deterministic heap $(\cfgnt{L}_\sym\ \cfgnt{R}_\sym)$; thus,
that set of meets the criteria for determinism by definition. So any
application of $\mathbb{ST}$ preserves that criteria for determinism.

The $\rightarrow_\mathit{FW}$ makes two uses of the $\mathbb{ST}$
function, $\theta = \mathbb{ST}(\cfgnt{L},\cfgnt{r},\phi,\phi_g) \cup
\mathbb{ST}(\cfgnt{L},\cfgnt{r}_\mathit{cur},\neg\phi,\phi_g)$, to
build individual $\theta$ sets. Choose any two distinct members of $\theta$,
$(\phi_\alpha \wedge \phi_\alpha^\prime\ \cfgnt{l}_\alpha^\prime)$ and
$(\phi_\beta \wedge \phi_\beta^\prime\ \cfgnt{l}_\beta^\prime)$.
\begin{itemize}
\item If $\phi_\alpha = \phi_\beta$, then the constraints came from the
  same call to $\mathbb{ST}$ so $\phi_\alpha^\prime \wedge
  \phi_\beta^\prime = \cfgt{false}$ by definition.
\item If $\phi_\alpha \ne \phi_\beta$, then the constraints came from
  the different calls and are distinguished by the phase of $\phi$ in
  the call so $\phi_\alpha \wedge \phi_\beta = \cfgt{false}$.
\end{itemize}
Each $\theta$ is mapped to a new reference from
$\mathrm{fresh}_r()$. These are added to an already deterministic heap
$(\cfgnt{L}_\sym\ \cfgnt{R}_\sym)$ and meet the criteria so
that $(\cfgnt{L}_\sym^\prime\ \cfgnt{R}_\sym^\prime)$ is also
deterministic.
\end{proof}

\begin{lemma}[$\rcom$ preserves heap determinism]
\label{lem:J-determ}
Given a state, $s_\sym$, with a deterministic heap,
$(\cfgnt{L}_\sym\ \cfgnt{R}_\sym) =
\mathrm{heap}(s_\sym)$, the new heap,
$(\cfgnt{L}_\sym^\prime\ \cfgnt{R}_\sym^\prime) =
\mathrm{heap}(s_\sym^\prime)$, in any state related by the
Javalite relation, $s_\sym \rcom s_\sym^\prime$, is
also deterministic.
\end{lemma}
\begin{proof}
Proof by definition of $\rcom$ in \figref{fig:javalite-common}.

Every rule in $\rcom$ except \texttt{New} leaves the heap
unmodified. The rule for \texttt{New} adds a single new location
($\mathrm{fresh}_\cfgnt{l}(\cfgnt{C})$) to the heap on a single new
reference ($\mathrm{stack}_r()$). The rule also points every field in
the new location to $\cfgnt{r}_\mathit{null}$. As none of these
mutations alter the determinism of the heap, the new heap
$(\cfgnt{L}_\sym^\prime\ \cfgnt{R}_\sym^\prime)$ is also
deterministic.
\end{proof}

\begin{theorem}[$\rsym$ preserves heap determinism]
\label{thm:determ}
Given a state, $s_\sym$, with a deterministic heap,
$(\cfgnt{L}_\sym\ \cfgnt{R}_\sym) =
\mathrm{heap}(s_\sym)$, the new heap,
$(\cfgnt{L}_\sym^\prime\ \cfgnt{R}_\sym^\prime) =
\mathrm{heap}(s_\sym^\prime)$, in any state related by the
symbolic relation, $s_\sym \rsym s_\sym^\prime$, is
also deterministic.
\end{theorem}
\begin{proof}
Proof by \lemref{lem:FA-determ}, \lemref{lem:FW-determ}, and \lemref{lem:J-determ}
which represent all the rules that relate states in
$\rsym$.
\end{proof}

\begin{lemma}[Exactness of Reference Compare Rule]
\label{lem:compare}
If there exists states $s_\gse$ and $s_\sym$ such that $s_\sym \in \mathcal{RC}$ and $s_\gse \sqsubset s_\sym$, then:
\begin{equation}
\forall s_\gse^\prime ( s_\gse \rgse s_\gse^\prime \Rightarrow \exists s_\sym^\prime( (s_\sym \rsym s_\sym^\prime )\wedge (s_\gse^\prime\ \sqsubset\ s_\sym^\prime ))  )
\end{equation}
and
\begin{equation}
\forall s_\sym^\prime ( s_\sym \rsym s_\sym^\prime\Rightarrow \exists s_\gse^\prime( (s_\gse \rgse s_\gse^\prime )\wedge (s_\gse^\prime\ \sqsubset\ s_\sym^\prime ))  )
\end{equation}
\end{lemma}
There are two rules that apply to state $s_\sym$, one for the $\cfgt{true}$ branch and one for the $\cfgt{false}$ branch. Since the proofs for both rules are nearly identical, for brevity we will only show the proofs for the case for the $\cfgt{true}$ branch. 
\begin{proof}
Assume there exists states $s_\gse$ and $s_\sym$ such that $s_\sym \in \mathcal{RC}$ and $s_\gse \sqsubset s_\sym$. Let $s_\sym^\prime$ be any state such that $s_\sym \rsym s_\sym$ and let $\zeta_T = \forall s_\gse^\prime ( s_\gse \rgse s_\gse^\prime )$. Since $s_\gse \sqsubset s_\sym$, we know that $s_\gse \in \mathcal{RC}$, and that there exists a homomorphism $s_\gse \rightharpoonup_{h} s_\sym$ such that $\mathbb{S}( \phi_\sym \wedge \mathbb{HC}(s_\gse \rightharpoonup_{h} s_\sym) ) $. We partition $\zeta_T$ based on the values of $\cfgnt{L}_\gse \lp \cfgnt{r}_0\rp$ and $\cfgnt{L}_\gse \lp \cfgnt{r}_1 \rp$ as follows: Let
$$\zeta_t = \zeta_T \setminus \{ s_f | (s_f= \lp \cfgnt{L}_f\ \cfgnt{R}_f\ \phi_\gse\ \eta\ \cfgnt{e}\ \cfgnt{k}\rp) \wedge (\cfgnt{L}_f \lp \cfgnt{r}_0\rp \neq \cfgnt{L}_f \lp \cfgnt{r}_1 \rp ) \}$$
and let
$$\zeta_f = \zeta_T \setminus \zeta_t$$ 

Furthermore, there are two possible configurations for $s_\sym^\prime$: $\lp \cfgnt{L}\ \cfgnt{R}\ \phi_g^\prime\ \eta\ \cfgt{true}\ \cfgnt{k}\rp $ and $\lp \cfgnt{L}\ \cfgnt{R}\ \phi_g^\prime\ \eta\ \cfgt{false}\ \cfgnt{k}\rp $. We now consider the partitions of $\zeta_T$ and configurations of $s_\sym^\prime$ in separate cases.

Case 1: Assume that $\cfgnt{L}_\gse \lp \cfgnt{r}_0\rp = \cfgnt{L}_\gse \lp \cfgnt{r}_1 \rp$. 
Compute state $s_\gse^\prime$ such that $s_\gse \rgse s_\gse^\prime$. In this case, the GSE ``equals - references true" rule applied, therefore $s_\gse^\prime$ is in $\zeta_t$. Observe that by applying Theorem~\ref{thm:mutex}, $\phi_\sym^\prime \wedge \phi_0 \wedge \phi_1$ reduces to $\phi_\sym$. Therefore, $\mathbb{S}( \phi_\sym^\prime \wedge \mathbb{HC}(s_\gse^\prime \rightharpoonup_{h} s_\sym^\prime) ) $ is true, and by extension, $s_\gse^\prime \sqsubset s_\sym^\prime$. Since this relation holds for arbitrary $s_\gse^\prime \in \zeta_t$, we now know that 
\begin{equation}
\label{eqn:RCForwardsTrue}
((\cfgnt{L}_\gse \lp \cfgnt{r}_0\rp = \cfgnt{L}_\gse \lp \cfgnt{r}_1 \rp) \wedge (s_\gse^\prime \in \zeta_t)) \Rightarrow s_\gse^\prime \sqsubset s_\sym^\prime
\end{equation}

Case 2:  Assume that $s_\sym^\prime$ has the form $\lp \cfgnt{L}\ \cfgnt{R}\ \phi_g^\prime\ \eta\ \cfgt{true}\ \cfgnt{k}\rp $, and define $\theta_\alpha$, $\theta_0$ and $\theta_1$ as in the ``equals (references-true) rule''. Since $\cfgnt{L}_\sym$ and $\cfgnt{R}_\sym$ are unchanged from $s_\sym$, and $\phi_\sym^\prime$ is only a strengthened version of $\phi_\sym$,  we know that
\begin{equation}
\label{eqn:RCSubset}
\{s_\gse^\prime | s_\gse^\prime \sqsubset s_\sym^\prime \} \subseteq \{s_\gse^\prime | \exists s_\gse \lp s_\gse \sqsubset s_\sym \rp \wedge s_\gse \rsym s_\gse^\prime\}
\end{equation}
Suppose that there exists state $s_i^\prime$ such that $s_i^\prime \sqsubset s_\sym^\prime$ and $s_i^\prime \notin \zeta_t$. Because of Equation \ref{eqn:RCSubset}, we know that 
$$s_i^\prime \in \{s_\gse^\prime | \exists s_\gse \lp s_\gse \sqsubset s_\sym\rp \wedge s_\gse \rsym s_\gse^\prime\}$$ 
Combining this with the assumption that $s_i^\prime \notin \zeta_t$, we must conclude that $\cfgnt{L}_\gse \lp \cfgnt{r}_0\rp \neq  \cfgnt{L}_\gse \lp \cfgnt{r}_1 \rp$. Because of this, and because of Theorem~\ref{thm:mutex}, we know that either all constraints in the set
$$\{\phi_i \mid \exists \phi_\alpha (\phi_\alpha \in \theta_\alpha)\wedge \phi_i = \lp\phi_\alpha \wedge \phi_0 \wedge \phi_1\rp\}$$ are unsatisfiable, or that at least one constraint in the set
$$\{\phi_i \mid \exists \phi_\alpha (\phi_\alpha \in \lp \theta_0 \cup \theta_1 \rp)\wedge \lp \phi_i = \phi_\alpha \wedge \phi_0 \wedge \phi_1 \rp\}$$ 
is valid. Either way, $\mathbb{S}\lp\phi_i^\prime \wedge\phi_0\wedge \phi_1\rp$ is false and $s_\sym^\prime$ does not represent $s_i^\prime$. We have a contradiction. Therefore: 
\begin{equation}
\label{eqn:RCBackwardsTrue}
((s_\sym^\prime=\lp \cfgnt{L}\ \cfgnt{R}\ \phi_g^\prime\ \eta\ \cfgt{true}\ \cfgnt{k}\rp) \wedge (s_\gse^\prime \sqsubset s_\sym^\prime)) \Rightarrow s_\gse^\prime \in \zeta_t
\end{equation}

Case 3: Assume that $\cfgnt{L}_\gse \lp \cfgnt{r}_0\rp \neq \cfgnt{L}_\gse \lp \cfgnt{r}_1 \rp$.
This means that the GSE ``equals - references false" rule applies. The proof for the ``equals - references false" rule is highly similar to the proof for Case 1, so we omit it for the sake of brevity. The result for this case is:
\begin{equation}
\label{eqn:RCForwardsFalse}
((\cfgnt{L}_\gse \lp \cfgnt{r}_0\rp = \cfgnt{L}_\gse \lp \cfgnt{r}_1 \rp) \wedge (s_\gse^\prime \in \zeta_t)) \Rightarrow s_\gse^\prime \sqsubset s_\sym^\prime
\end{equation}

Case 4: Assume that $s_\sym^\prime$ has the form $\lp \cfgnt{L}\ \cfgnt{R}\ \phi_g^\prime\ \eta\ \cfgt{false}\ \cfgnt{k}\rp $.
The proof for this case is highly similar to the proof for Case 2, so we omit it for the sake of brevity. The result for this case is:
\begin{equation}
\label{eqn:RCBackwardsFalse}
s_\sym^\prime=\lp \cfgnt{L}\ \cfgnt{R}\ \phi_g^\prime\ \eta\ \cfgt{false}\ \cfgnt{k}\rp \wedge s_\gse^\prime \sqsubset s_\sym^\prime \Rightarrow s_\gse^\prime \in \zeta_f
\end{equation}

Since $\zeta_T = \zeta_t \cup \zeta_f$, we can combine Equation \ref{eqn:RCForwardsTrue} with \ref{eqn:RCForwardsFalse} to find that 
\begin{equation}
\forall s_\gse^\prime ( s_\gse \rgse s_\gse^\prime \Rightarrow \exists s_\sym^\prime( (s_\sym \rsym s_\sym^\prime )\wedge (s_\gse^\prime\ \sqsubset\ s_\sym^\prime ))  )
\end{equation}
Likewise, we can combine Equation \ref{eqn:RCBackwardsTrue} with Equation \ref{eqn:RCBackwardsFalse} to find that
\begin{equation}
\forall s_\sym^\prime ( s_\sym \rsym s_\sym^\prime\Rightarrow \exists s_\gse^\prime( (s_\gse \rgse s_\gse^\prime )\wedge (s_\gse^\prime\ \sqsubset\ s_\sym^\prime ))  )
\end{equation}
\end{proof}

\begin{lemma}[Exactness of New Rule]
\label{lem:new}
If there exists states $s_\gse$ and $s_\sym$ such that $s_\sym \in \mathcal{NW}$ and $s_\gse \sqsubset s_\sym$, then:
\begin{equation}
\forall s_\gse^\prime ( s_\gse \rgse s_\gse^\prime \Rightarrow \exists s_\sym^\prime( (s_\sym \rsym s_\sym^\prime )\wedge (s_\gse^\prime\ \sqsubset\ s_\sym^\prime ))  )
\end{equation}
and
\begin{equation}
\forall s_\sym^\prime ( s_\sym \rsym s_\sym^\prime\Rightarrow \exists s_\gse^\prime( (s_\gse \rgse s_\gse^\prime )\wedge (s_\gse^\prime\ \sqsubset\ s_\sym^\prime ))  )
\end{equation}

\begin{proof}
The proof is left as an exercise to the reader.
\end{proof}

\end{lemma}


\begin{theorem}
\label{th:bisim}
The representation relation $\sqsubset$ is a bisimulation.
\end{theorem}

\begin{proof}
Take any two states $s_\gse$ and $s_\sym$ such that $s_\gse \sqsubset s_\sym$. If $s_\sym \in \mathcal{FA} \cup \mathcal{FW} \cup \mathcal{RC} \cup \mathcal{NW}$, then by Lemmas \ref{lem:access}, \ref{lem:write}, \ref{lem:compare}, and \ref{lem:new} we know Equations \ref{eqn:BisimulationForwards} and \ref{eqn:BisimulationBackwards} hold. If $s_\sym$ has any other form, the heap is not modified for $s_\gse^\prime$ or $s_\sym^\prime$, so then Equations \ref{eqn:BisimulationForwards} and \ref{eqn:BisimulationBackwards} hold by default. Thus, Equations \ref{eqn:BisimulationForwards} and \ref{eqn:BisimulationBackwards} hold for all  $s_\gse$ and $s_\sym$ such that $s_\gse \sqsubset s_\sym$. By Definition \ref{bisimulation}, $\sqsubset$ is a bisimulation.

\end{proof}

\begin{corollary}
For any given initial state, the set of possible control flow sequences under the GSE transition relation is exactly the set of possible control flow sequences under the symbolic transition relation.
\end{corollary}

\begin{corollary}
For any given initial state, the number of final symbolic states is exactly the number of possible control flow sequences.
\end{corollary}
