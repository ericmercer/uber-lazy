%\begin{figure*}[t]
\begin{table*} [t]
  \centering
  \begin{tabular}{| c | c | r | r | r | r | r | r | r | r | r |}
  \hline
   \multirow{2}{*}{Method }&\multirow{2}{*}{ $k$ }
   &\multicolumn{3}{|c|}{Time} &\multicolumn{3}{|c|}{States} &\multicolumn{3}{|c|}{ Paths }\\
								&	&\gsetxt{} & L\#		&SL		&\gsetxt{}	& L\# & SL&\gsetxt{}	& L\# 	& SL\\
   \hline
    \multirow{3}{*}{LinkedList }			&3	& 0.91	& 1.21	& 0.69	& 2465	& 2844	& 99		&1656	& 1269	& 25\\
   		 						& 4	& 2.92	& 3.35	& 0.91	& 25774	& 29977	& 155	&17485	& 13550	& 39\\
   								& 5	& 20.78	& 19.47	& 1.59	& 341164	& 400296	& 223	&232743	& 181849	& 56\\
								& 6	& 280.56	& 299.19	& 2.36	&5447980	&6437201	& 303	&3731094	&2933027	& 76\\
    \hline
    \multirow{3}{*}{BinarySearchTree }	& 1	& 0.26	& 0.28	& 0.36	& 19		& 23		& 29		& 6		& 6		& 6\\
   		 						& 2	& 0.83	& 1.28	& 0.93	& 143	& 143	& 145	& 43		& 42		& 33\\
   								& 3	& 20.63	& 25.55	& 4.03	& 1953	& 1703	& 1485	& 515	& 515	& 328\\
    \hline
      \multirow{3}{*}{TreeMap}			& 1	& 0.47	& 0.52	& 0.77	& 65		& 70		& 215	& 11		& 11		& 11\\
   		 						& 2	& 8.99	& 9.73	& 4.72	& 1009	& 942	& 3219	& 127	& 122	& 73\\
   								& 3	& -		& -		& 145.56	& -		& -		& 78695	& -		& -		& 887\\
						
    \hline
  \end{tabular}
  \caption{Test results.}
  \label{tab:results}
\end{table*}
%\end{figure*}

\subsection{Analysis}
The results of the experiments are presented in Table \ref{tab:results}. The rows contain to individual $k$-bounded test runs for each of the evaluated programs. The columns show the run time, state count and path count for each of the three algorithms tested. The headings \gsetxt{}, L\#, and SL correspond to the Generalized Symblic Execution, lazier\#, and \symtxt{} algorithms, respectively. Table entries containing - correspond to test runs that exceeded the allotted time bounds. 

In most tests, run times strongly favor \symtxt{}, especially for nontrivial k-values. Performance improvement ranges from 4.8x for BinarySearchTree at $k$=3, to 118x for LinkedList at $k$=6. In fact, a number of test cases can be efficiently completed using \symtxt{} for k-bounds that are not practical to be attempted with \gsetxt{} or lazier\#, including the BinarySearchTree for $k$=4 and LinkedList for $k$=8. Some of this discrepancy may be accounted for by the fact that \symtxt{} is equipped for incremental solving, however, during informal testing with the incremental solver turned off, \symtxt{} usually performed better than \gsetxt and lazier\#, especially in test runs with high k-bounds. 

Path counts likewise substantially favor \symtxt{}. In agreement with the theoretical results, the number of paths explored by \symtxt{} is strictly less than or equal the number of paths explored by \gsetxt{} for all test cases. More surprisingly, \symtxt{} path growth appears to be sub-exponential in $k$ for the LinkedList program.

Curiously, the pattern established with run-times and path counts does not hold true for state counts. State counts can vary between algorithms, because states in JPF represent points of nondeterminism, and each algorithm differs in this regard. For example, \gsetxt{} has additional points of nondeterminism during field reads, but in contrast address compares are completely deterministic. Thus, in example programs with large numbers address compares, such as TreeMap, state counts for lazier\# and \symtxt{} may exceed those for \gsetxt{}. Despite this, in practice this is often mitigated by the fact that \symtxt{} can dispatch address compares without consulting the constraint solver.



