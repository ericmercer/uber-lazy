\section{Generating Heap Summaries}

\subsection{Initialization of Symbolic References}

In this section we present the Javalite rewrite rules for the summary
initialization of symbolic references. The initialization rules are
defined on the bi-partite graph consisting of references and
locations. Recall that in generalized symbolic execution (GSE) for
lazy initialization of symbolic references consists of three key
points of non-determinism where each symbolic reference can be
initialized non-deterministically to null, a new instance of the
symbolic reference, or aliases to symbolic references of the same type
previously initialized. The initialization in GSE consists of creating
branches in the execution tree for all the non-deterministic
choices. In contrast, the heap summarization approach creates a
single branch that contains the summarization for all the
initialization in a single bi-partitate graph.


\newsavebox{\boxPi}
\savebox{\boxPi}{
%\begin{center}
\mprset{flushleft}
\begin{mathpar}
	\inferrule[Summarize]{
	\Lambda = \mathbb{UN}\lp \cfgnt{L}, \cfgnt{R}, \cfgnt{r}, \cfgnt{f}\rp \\
      \Lambda \neq \emptyset \\
      \lp\phi_x\ \cfgnt{l}_x\rp = \mathrm{min}_l\lp \Lambda\rp\\
      \cfgnt{r}_f = \mathrm{init}_r\lp \rp \\
      l_f  = \mathrm{fresh}_l\lp \mathrm{C}\rp\\\\
      \rho = \{ \lp \cfgnt{r}_a\ l_a\rp  \mid \mathrm{isInit}\lp \cfgnt{r}_a\rp  \wedge\cfgnt{r}_a = \mathrm{min}_r\lp \cfgnt{R}^{\leftarrow}[l_a]\rp \wedge \mathrm{type}\lp l_a\rp  = \mathrm{C} \} \\\\
      \theta_\mathit{null} = \{ \lp \phi\ l_\mathit{null}\rp  \mid \phi = \lp \phi_x \wedge \cfgnt{r}_f = \cfgnt{r}_\mathit{null} \rp  \} \\\\
      \theta_\mathit{new} = \{\lp \phi\ l_f\rp  \mid \phi = \lp \phi_x \wedge \cfgnt{r}_f \neq \cfgnt{r}_\mathit{null} \wedge \lp \wedge_{\lp \cfgnt{r}^\prime_a\ l^\prime_a\rp  \in \rho} \cfgnt{r}_f \ne \cfgnt{r}^\prime_a\rp \rp \}\\\\
      \theta_\mathit{alias} = \{ \lp \phi\ l_a\rp  \mid \exists\cfgnt{r}_a\ \lp\lp\cfgnt{r}_a\ l_a\rp  \in \rho \wedge \phi = \lp \phi_x \wedge \cfgnt{r}_f \neq \cfgnt{r}_\mathit{null} \wedge \cfgnt{r}_f = \cfgnt{r}_a \wedge \lp \wedge_{\lp \cfgnt{r}^{\prime}_a\ l^{\prime}_a\rp  \in \rho\ \lp \cfgnt{r}^\prime_a < \cfgnt{r}_a\rp } \cfgnt{r}_f \neq \cfgnt{r}^{\prime}_a \rp \rp \rp \} \\\\
      \theta_\mathit{orig} = \{\lp\phi\ \cfgnt{l}_\mathit{orig}\rp \mid \exists \phi_\mathit{orig} \lp \lp\phi_\mathit{orig}\ \cfgnt{l}_\mathit{orig}\rp \in \cfgnt{L}\lp\cfgnt{R}\lp\cfgnt{l}_x,\cfgnt{f}\rp\rp \wedge \phi = \lp\neg\phi_x \wedge \phi_\mathit{orig}\rp\}\\\\ 
      \theta = \theta_\mathit{null} \cup \theta_\mathit{new} \cup \theta_\mathit{alias} \cup \theta_\mathit{orig} \\
\cfgnt{R}^\prime = \cfgnt{R}[\forall \cfgnt{f} \in \mathit{fields}\lp \mathrm{C}\rp \ \lp \lp l_f\ \cfgnt{f}\rp  \mapsto \cfgnt{r}_\mathit{un} \rp ]
    }{
      \lp \cfgnt{L}\ \cfgnt{R}\ \cfgnt{r}\ \cfgnt{f}\ \cfgnt{C}\rp \rsum 
      \lp \cfgnt{L}[\cfgnt{r}_f \mapsto \theta]\ \cfgnt{R}^{\prime}[ \lp l_x,\cfgnt{f}\rp  \mapsto \cfgnt{r}_f ]\ \cfgnt{r}\ \cfgnt{f}\ \cfgnt{C}\rp
	}
\and
	\inferrule[Summarize-end]{
	  \Lambda = \mathbb{UN}\lp \cfgnt{L}, \cfgnt{R}, \cfgnt{r}, \cfgnt{f}\rp \\
      \Lambda = \emptyset
    }{
      \lp \cfgnt{L}\ \cfgnt{R}\ \cfgnt{r}\ \cfgnt{f}\ \cfgnt{C}\rp  \rsum
      \lp \cfgnt{L}\ \cfgnt{R}\ \cfgnt{r}\ \cfgnt{f}\ \cfgnt{C}\rp 
	}
\end{mathpar}}
%\end{center}
%\caption{The summary machine, $s ::= \lp\cfgnt{L}\ \cfgnt{R}\ \cfgnt{r}\ \cfgnt{f}\ \cfgnt{C}\rp$, with $s\rsum^*s^\prime =  s \rsum \cdots \rsum s^\prime \rsum s^\prime$.}
%\label{fig:symInit}
%\end{figure*}


The initialization rules are invoked when an uninitialized field in a
symbolic reference is accessed. The function $\mathbb{UN}(\cfgnt{L},
\cfgnt{R}, \cfgnt{r}, \cfgnt{f}) = \{\cfgnt{l}\ ...\}$ returns
constraint-location pairs in which the field $\cfgnt{f}$ is
uninitialized:
\[
\begin{array}{rcl}
\mathbb{UN}(\cfgnt{L}, \cfgnt{R}, \cfgnt{r}, \cfgnt{f}) & = &\{ \lp\phi\ \cfgnt{l}\rp \mid \lp \phi\ \cfgnt{l}\rp  \in \cfgnt{L}\lp \cfgnt{r}\rp  \wedge \\
& & \ \ \ \ \exists \phi^\prime \lp \lp \phi^\prime\ \cfgnt{l}_\mathit{un}\rp  \in \cfgnt{L}\lp \cfgnt{R}\lp l,\cfgnt{f}\rp\rp \wedge \\
& & \ \ \ \ \ \ \ \ \mathbb{S}\lp \phi \wedge \phi^\prime \rp\rp\}\\
\end{array}
\]
where $\mathbb{S}(\phi)$ returns true if $\phi$ is
satisfiable. Intutively, for the reference, $\cfgnt{r}$, it constructs
the set, $\theta$, that contains all contraint-location pairs that
point to the field $\cfgnt{f}$ and $\cfgnt{f}$ points to
$\cfgnt{l}_\mathit{un}$.

% The cardinality of the set, $\theta$ is never
%greater than one in GSE and the constraint is always satisfiable
%because all constraints are constant. This property is relaxed in GSE
%with heap summaries.

The rewrite rule to initialize a heap in the symbolic summarization
technique is shown in~\figref{fig:initHeap}. The set $\mathbb{UN}$

 There are three
constraint-location pair sets $\theta_\mathit{null}$,
$\theta_\mathit{new}$, and $\theta_\mathit{alias}$ that correspond to
the non-deterministic choices in GSE. The $\theta_\mathit{null}$ creates
a constraint location pair where the constraint under which the field is read $\phi_x$ $l_\mathit{null}$ 

\begin{figure*}[t]
\begin{center}
\begin{tabular}[c]{c|c|c|c}
\begin{tabular}[c]{c}
\scalebox{0.81}{\input{origHeap.pdf_t}} \\
\end{tabular} &
\begin{tabular}[c]{c}
\scalebox{0.81}{\input{summarizeXHeap.pdf_t}} \\
\end{tabular} &
\begin{tabular}[c]{c}
\scalebox{0.81}{\input{summarizeYHeap.pdf_t}} \\
\end{tabular} &
\begin{tabular}[c]{l}
$\rho := \{ (r_1^i, r_1^i \neq r_\mathit{null}, l_1 \}$ \\
$\theta_\mathit{null} := \{ ( r_2^i = r_\mathit{null}, l_\mathit{null}) \}$\\
$\theta_\mathit{new} := \{ ( r_2^i \neq r_\mathit{null} \wedge r_2^i \neq r_1^i, l_2) \}$\\
$\theta_\mathit{alias} := \{ ( r_1^i \neq r_\mathit{null} \wedge r_2^i \neq r_\mathit{null} \wedge r_2^i = r_1^i, l_1) \}$\\
$\theta_\mathit{orig} := \{ \}$ \\
$\phi_{\mathit{1a}} := r_1^i = r_\mathit{null} $ \\
$\phi_{\mathit{1b}} := r_1^i \neq r_\mathit{null} $  \\
$\phi_{\mathit{2a}} := r_2^i = r_\mathit{null}$ \\
$\phi_{\mathit{2b}} := r_2^i \neq r_\mathit{null} \wedge r_2^i \neq r_1^i$ \\
$\phi_{\mathit{2c}} :=  r_2^i \neq r_\mathit{null} \wedge r_2^i = r_1^i $ \\
\end{tabular} \\
(a) & (b) & (c) & (d) \\
\end{tabular}
\end{center}
\caption{initialize this.x and this.y}
\label{fig:initHeap}
\end{figure*}

We visualize the initialization process in the symbolic summarization
technique in~\figref{fig:initHeap}. The heap
in~\figref{fig:initHeap}(a) represents the initial heap. The
references with the superscript $s$ indicates that it is a local
reference. In~\figref{fig:initHeap} $r_0^s$ represents the reference
for the $\mathit{this}$ instance which has two fields $x$ and $y$ of
the same type. The reference $r_0^s$ points to the location
$l_0$. Note that when no constraint is specified, then there is a
implicit $\mathit{true}$ constraint. In~\figref{fig:initHeap} $r_0^s$
points to $l_0$ on $\mathit{true}$. The fields $x$ and $y$ point to
the uninitialized reference $r_\mathit{un}$. The reference
$r_\mathit{un}$ points to the uninitialized location $l_\mathit{un}$
on the $\mathit{true}$ constraint.

The graph in~\figref{fig:initHeap}(b) represents the summary heap
after the initialization of the $\mathit{this}.x$ field while the
graph in~\figref{fig:initHeap}(c) represents the summary heap after
the initialization of the $\mathit{this}.y$ field following the
initialization of $\mathit{this}.x$. The list
in~\figref{fig:initHeap}(d) represents the various sets constructed in
the rewrite system when the initialization takes places. We also
define the constraints of references to their corresponding labels in
the graph.

The access on $l_0.x$ creates a new input reference $r_1^i$ in the
summary heap shown in~\figref{fig:initHeap}(b). The reference $r_1^1$
points to the $l_\mathit{null}$ location under the constraint that
$r_1^i$ is null: $\phi_{1a} := r_1^i = r_\mathit{null}$. The reference
$r_1^i$ points to the location $l_1$ under the constraint that $r_1^i$
is not null $\phi_{1b} :=r_1^i \neq r_\mathit{null}$. The location
$l_1$ represents a fresh location of type $C$ is created on the heap
such that $C$ is type of $\mathit{this}.x$.





