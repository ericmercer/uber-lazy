\section{Summary Heap Framework}
The summary heap algorithm is presented using Javalite, an
executable formalization of a Java-like language~\cite{saints-MS}. Javalite retains 
the essence of Java and other object-oriented languages but with formalizations 
in both PLT Redex and Coq so that the outcome of any program has a precise
and deterministic meaning.

Javalite is a syntactic machine defined as rewrites on a string.  The
semantics are small-step using the CESK architecture with a (C)ontrol string
representing the expression being evaluated, an (E)nvironment for
local variables, a (S)tore for the heap, and a (K)ontinuation for what
is to be executed next.  In this work we change the semantics of the original
 Javalite machine and its machine syntax to represent (i) GSE~\cite{GSE03}, and (ii) the summary heap algorithm
presented in this work.

The surface syntax (input) for Javalite is shown 
in~\figref{fig:surface-syntax}, and the machine syntax (state)
in~\figref{fig:machine-syntax}. Terminals are in bold face while
non-terminals are italicized. Ellipses indicate zero or more
repetitions. Tuples omit the commas for compactness. The language
itself uses s-expressions for convenience.

\begin{figure}
\begin{center}
\cfgstart
\cfgrule{P}{\lp $\mu$ \lp \cfgnt{C} \cfgnt{m}\rp\rp}
\cfgrule{$\mu$}{(\cfgnt{CL} ...)}
\cfgrule{T}{\cfgt{bool} \cfgor \cfgnt{C}}
\cfgrule{CL}{\lp\cfgt{class} \cfgnt{C} \lp\lb\cfgnt{T} \cfgnt{f}\rb ...\rp \lp\cfgnt{M} ...\rp}
\cfgrule{M}{\lp\cfgnt{T} \cfgnt{m} \lb\cfgnt{T} \cfgnt{x}\rb\  e\rp}
\cfgrule{e}{\cfgnt{x}
\cfgor{\lp\cfgt{new} \cfgnt{C}\rp}
\cfgor{\lp\cfgnt{e} \cfgt{\$} \cfgnt{f}\rp}
\cfgor{\lp\cfgnt{x} \cfgt{\$} \cfgnt{f} \cfgt{:=} \cfgnt{e}\rp}
\cfgor{\lp\cfgnt{e} \cfgt{=} \cfgnt{e}\rp}}
\cfgorline{\lp\cfgt{if} \cfgnt{e} \cfgnt{e} \cfgt{else} \cfgnt{e}\rp 
\cfgor {\lp\cfgt{var} \cfgnt{T} \cfgnt{x} \cfgt{:=} \cfgnt{e} \cfgt{in} \cfgnt{e}\rp}
\cfgor {\lp\cfgnt{e} \cfgt{@} \cfgnt{m} \cfgnt{e} \rp}}
\cfgorline{\lp\cfgnt{x} \cfgt{:=} \cfgnt{e}\rp
\cfgor{\lp\cfgt{begin} \cfgnt{e} ...\rp}
\cfgor{\cfgnt{v}}}
\cfgrule{x}{\cfgt{this} \cfgor \cfgnt{id}}
\cfgrule{f,m,C}{\cfgnt{id}}
%\cfgrule{m}{\cfgnt{id}}
%\cfgrule{C}{\cfgnt{id}}
\cfgrule{v}{\cfgnt{r} \cfgor \cfgt{null} \cfgor \cfgt{true} \cfgor \cfgt{false} \cfgor \cfgt{error}}
\cfgrule{r}{\cfgt{number}}
\cfgrule{id}{\cfgt{variable-not-otherwise-mentioned}}
\cfgend
\end{center}
\caption{The Javalite surface syntax.}
\label{fig:surface-syntax}
\end{figure}

\begin{figure}
\begin{center}
\cfgstart
\cfgrule{e}{\lp .... \cfgor \lp\cfgt{raw} \cfgnt{v} \cfgt{@} \cfgnt{m} \cfgnt{v} \rp\rp}
%    ;; eval syntax
%  (object (C [f loc] ...))
%  (hv v
%      object)
\cfgrule{$\phi$}{\cfgnt{constraint}}
\cfgrule{l}{\cfgt{number}}
%  (h mt
%     (h [loc -> hv]))
%  (η mt
%     (η [x -> loc]))
\cfgrule{s}{\lp$\mu$ \cfgnt{L} \cfgnt{R} $\eta$ \cfgnt{e} \cfgnt{k}\rp}
\cfgrule{k}{\cfgt{end}}
\cfgorline{\lp \cfgt{*} \cfgt{\$} \cfgnt{f} $\rightarrow$ \cfgnt{k}\rp}
%     (* @ m (e ...) -> k)
%     (v @ m (v ...) * (e ...) -> k)
%     (* == e -> k)
%     (v == * -> k)
%     (x := * -> k)
%     (x $ f := * -> k)
%     (if * e else e -> k)
%     (var T x := * in e -> k)
%     (begin * (e ...) -> k)
%     (pop η k))
\cfgend
\end{center}
\caption{The machine syntax for Javalite.}
\label{fig:machine-syntax}
\end{figure}


A Javalite program, $\cfgnt{P}$, is a registry of classes, $\mu$, with
a tuple indicating a class, $\cfgnt{C}$, and a method, $\cfgnt{m}$,
where execution starts. The only primitive type in Javalite is
Boolean. Classes have non-primitive fields, $([\cfgnt{C}\ \cfgnt{f}]\ ...)$, and
methods, $(\cfgnt{M}\ ...)$.\footnote{Summary heaps are able to support non-primitive fields (see \secref{sec:eval}). They are omitted to focus and simplify the presentation.} Methods are limited to a single
parameter in our Javalite machine. Expressions, $\cfgnt{e}$,  
include statements, and they
 use '$\cfgt{:=}$' to indicate assignment and '$\cfgt{=}$' to indicate comparison.
The dot-operator for field access is replaced by '$\cfgt{\$}$', and the dot-operator
for method invocation is replaced by '$\cfgt{@}$' since '.' is reserved in PLT Redex. There is no explicit return
statement in Javalite; rather, the value of the last expression is
used as the return value. A variable is always indicated by \cfgnt{x}
and a value by \cfgnt{v}. A value can be a reference in the heap, $\cfgnt{r}$, or any of the special values shown in~\figref{fig:surface-syntax}. 
We assume that only
type correct programs are given as input to the machine.

The state of the Javalite machine (\figref{fig:machine-syntax}),
$\cfgnt{s}$, includes the program registry, $\mu$, a pair of functions defining the heap
$\lp\cfgnt{L}\ \cfgnt{R}\rp$, a constraint,
$\phi$, defined over references in the heap, the environment, $\eta$,
for local variables, and the continuation $\cfgnt{k}$.\footnote{The registry $\mu$ never changes so it is omitted from the state tuple in the rest of the presentation for compactness.}  The more
complex structures, such as the environment and heap, are defined as lists that
start with empty, \cfgnt{mt}. The rewrite rules that define the
semantics treat these lists as partial functions. As such,
$\eta(\cfgnt{x}) = \cfgnt{r}$ is the reference $\cfgnt{r}$ associated with variable
$\cfgnt{x}$. The notation $\eta^\prime = \eta[\cfgnt{x} \mapsto
  \cfgnt{v}]$ defines a new partial function $\eta^\prime$ that is
the same as $\eta$ except that the variable $\cfgnt{x}$ now maps to
$\cfgnt{v}$.

The continuation $\cfgnt{k}$ helps accomplish the small-step semantics
by indicating where the hole, $\cfgt{*}$, is located, by storing
temporary computations, and by keeping track of the next
computation. For example, the continuation
$\lp\cfgt{*}\ \cfgt{\$}\ \cfgnt{f} \rightarrow \cfgnt{k}\rp$ indicates
that the machine is currently computing the expression for the object
reference on which the field $\cfgnt{f}$ is going to be accessed. Once
the field access is complete, the machine continues with $\cfgnt{k}$.

\subsection{Summary Heaps as Bipartite Graphs}
A novel contribution of summary heaps for symbolic execution is the
labeled bipartite graph representation of the heap which enables a
summary heap to group heaps together that follow the same control flow
to a given point of execution. The graph structure is able to capture
group membership and compute value sets using entirely non-recursive
local lookup. Additionally, updates on the heap do not require
rewriting the path condition.

The bipartite graph itself consists of references, $\cfgnt{r}$, and
constraint-location pairs, $\lp\phi\ \cfgnt{l}\rp$. Locations may be 
altered to perform updates, whereas references are immutable
once they are added to the graph. The graph is
expressed in the location map, $\cfgnt{L}$, and the reference map,
$\cfgnt{R}$. As done with the environment, $\cfgnt{L}$ and $\cfgnt{R}$
are treated as partial functions where $\cfgnt{L}(r) =
\{(\phi\ \cfgnt{l})\ ...\}$ is the set of constraint-location pairs in
the heap associated with the given reference, and
$\cfgnt{R}(\cfgnt{l},\cfgnt{f}) = \cfgnt{r}$ is the reference
associated with the given location-field pair in the heap.  

There are several properties of the summary heap enforced by the
rewrite rules that are invariant and important to the correctness
proofs in \secref{sec:bisim}: \emph{reference partitions},
\emph{immutability}, \emph{determinism}, \emph{type
  consistency}, and \emph{null and uninitialized}.

\emph{Reference partitions}: new references drawn from
  three distinct partitions (i) $\mathrm{init}_\cfgnt{r}()$ for references for the \emph{input heap}, (ii) $\mathrm{fresh}_\cfgnt{r}()$ for references for \emph{auxiliary literals}, and
  (iii) $\mathrm{stack}_\cfgnt{r}()$ for references for \emph{stack literals} in the expressions and
  continuations. The references strictly increase in value on each
  call, and modular arithmetic is used to determine the partition to
  which a reference belongs.
\begin{compactdesc}
\item[Input heap:] these references have two
  interpretations: one as a literal and another as a variable. As a literal it is a node in
  the summary heap. As a variable it is a term in any number of constraints that label
  edges in the summary heap. Only input heap references and the
  special reference $\cfgnt{r}_\mathit{null}$ appear in constraints
\item[Auxiliary literals:] these references are only used to define additional structure in
  the bipartite graph and are not part of any constraint and do not appear
  in any environment, expression, or continuation.
\item[Stack references:] these references are also used to define additional structure in the bipartite graph but may also appear in the environment, expression, or continuation. These are not part of any constraint.
\end{compactdesc}
Reference partitions are a critical piece to the existence proof of a
bisimilation between the summary heap algorithm and generalized
symbolic execution (see \secref{sec:bisim}).

\emph{Immutability} means that a reference does not change in the
location map. It always points to the same set of constraint location
pairs. This property ensures that constraints never need to be
rewritten as the program evolves its state.

\emph{Determinism} means that a reference in $(\cfgnt{L}\ \cfgnt{R})$ cannot point to multiple locations simultaneously. In the definition $\cfgnt{L}^\leftarrow$ is the pre-image of the location map.
$$
\begin{array}{l}
\forall \cfgnt{r} \in \cfgnt{L}^\leftarrow\ (\forall (\phi\ \cfgnt{l}),(\phi^\prime\ \cfgnt{l}^\prime) \in \cfgnt{L}(r)\ (\\
\ \ \ \ (\cfgnt{l} \neq \cfgnt{l}^\prime \vee \phi \neq \phi^\prime) \Rightarrow (\phi \wedge \phi^\prime = \cfgt{false}))
\end{array}
$$
Informally, any two constraint location pairs connected to a
common reference form a contradiction: only one can be true in any
satisfying assignment. The property ensures that any satisfying assignment
resolving aliasing relationships results in a deterministic heap: a
field access returns a single location under that assignment.

\emph{Type consistency} implies that all locations associated with a reference have the same type.
\[
\begin{array}{l}
\forall \cfgnt{r} \in \cfgnt{L}^\leftarrow\ (\forall (\phi\ \cfgnt{l}),(\phi^\prime\ \cfgnt{l}^\prime) \in \cfgnt{L}(r)\ (\\
\ \ \ \ (\mathrm{Type}\lp\cfgnt{l}\rp = \mathrm{Type}\lp\cfgnt{l}^\prime\rp)))
\end{array}
\]

\emph{Null and Uninitialized} means that every summary heap contains 
a special location for null ($\cfgnt{l}_\mathit{null}$), and a special
location for uninitialized
  ($\cfgnt{l}_\mathit{un}$), with corresponding references
  $\cfgnt{r}_\mathit{null}$ and
  $\cfgnt{r}_\mathit{un}$, respectively. $\cfgnt{L}(\cfgnt{r}_\mathit{null}) =
  \{(\cfgt{true}\ \cfgnt{l}_\mathit{null})\}$ and
  $\cfgnt{L}(\cfgnt{r}_\mathit{un}) =
  \{(\cfgt{true}\ \cfgnt{l}_\mathit{un})\}$. These special references can be used in constraints.

To summarize, the heap summary algorithm creates
many references that are auxiliary literals or stack literals to
define the structure of the bipartite graph.  It only uses input heap
references for constraints expressing potential aliasing. A solver
reasons over the aliasing relationships, and a satisfying assignment
uniquely identifies active edges in the graph. Those active edges
describe a concrete heap and its shape.

For example, if the solver assigns input heap variables $\cfgnt{r}$ and $\cfgnt{r}^\prime$
such that $\cfgnt{r} = \cfgnt{r}^\prime$ is true, then there exists
some location $\cfgnt{l}$ and constraints $\phi$ and $\phi^\prime$ such
that
\begin{compactitem}
\item[-] $\lp\phi\ \cfgnt{l}\rp \in \cfgnt{L}(\cfgnt{r})$; and
\item[-] $\lp\phi^\prime\ \cfgnt{l}\rp \in \cfgnt{L}(\cfgnt{r}^\prime)$; and
\item[-] $\phi$ and $\phi^\prime$ hold in the aliasing assignments from the solver.
\end{compactitem}
In this way, the graph groups heaps with the conditions under which
those heaps exist. This representation separates the constraint
problem from the morphology problem: it is possible to update the heap
without having to rewrite constraints. Without this separation, it
would not be possible to lazily initialize heap locations along a
program path.

\subsection{State Transition Relation}

The initial state of a Javalite program is constructed from its
surface syntax on the $\cfgnt{P}$ production:
$\lp\mu\ \lp\cfgnt{C}\ \cfgnt{m}\rp\rp$. Assuming that
$\lp\cfgnt{L}\ \cfgnt{R}\rp$ is an empty heap with no references
or locations and $\eta$ is an empty environment with no
variables, the initial state is
$$
\begin{array}{l}
\lp\mu 
\ \cfgnt{L}[\cfgnt{r}_\mathit{null} \mapsto \{\lp\cfgt{true}\ \cfgnt{l}_\mathit{null})\}\rp] 
           [\cfgnt{r}_\mathit{un} \mapsto \{\lp\cfgt{true}\ \cfgnt{l}_\mathit{un}\rp\}] \\
\ \cfgnt{R}
\ \cfgt{true}\ \eta\  \lp\lp\cfgt{new}\ \cfgnt{C}\rp\ \cfgt{@}\ \cfgnt{m}\ \cfgt{true}\rp\rp\ \cfgt{end}\rp
\end{array}
$$
The first expression in the initial state is to construct a new object
in the summary heap of type $\cfgnt{C}$. Note that the resulting object is not referenced
by the input heap since it comes from the new-expression.  Once an
instance of $\cfgnt{C}$ exists in the summary heap, the method
$\cfgnt{m}$ of the class is invoked.

In general, the semantics of Javalite are expressed as
rewrites on strings using pattern matching. Consider the rewrite rule
for the beginning of a field access instruction in the Javalite
machine:
$$
\mprset{flushleft}
	\inferrule[Field Access(eval)]{}{
      \lp \cfgnt{L}\ \cfgnt{R}\ \phi\ \eta\ \lp \cfgnt{e}\ \cfgt{\$}\ \cfgnt{f}\rp \ \cfgnt{k}\rp  \rcom 
      \lp \cfgnt{L}\ \cfgnt{R}\ \phi\ \eta\ \cfgnt{e}\ \lp \cfgt{*}\ \cfgt{\$}\ \cfgnt{f} \rightarrow \cfgnt{k}\rp \rp 
	}
$$
If the string representing the current state matches the left side, then it
creates the new string on the right. In this example, the new string
on the right is now evaluating the expression $\cfgnt{e}$ in the field
access, and it includes the continuation indicating that it still
needs to complete the actual field access once the expression is
evaluated.

Other more complex rules, such as one to create a new instance of a
class, define constraints on the rewrites and more complex
transformations on the heap.
$$
\mprset{flushleft}
	\inferrule[New]{
      \cfgnt{r} = \mathrm{stack}_r\lp \rp\\
      l = \mathrm{fresh}_l\lp \cfgnt{C}\rp\\\\
      \cfgnt{R}^\prime = \cfgnt{R}[\forall \cfgnt{f} \in \mathit{fields}\lp \mathrm{C}\rp \ \lp \lp l\ \cfgnt{f}\rp  \mapsto \cfgnt{r}_\mathit{null} \rp ] \\\\
      \cfgnt{L}^\prime = \cfgnt{L}[\cfgnt{r} \mapsto \{\lp \cfgt{true}\ l\rp \}]
    }{
      \lp \cfgnt{L}\ \cfgnt{R}\ \phi\ \eta\ \lp \cfgt{new}\ \cfgnt{C}\rp \ \cfgnt{k}\rp  \rcom
      \lp \cfgnt{L}^\prime\ \cfgnt{R}^\prime\ \phi\ \eta\ \cfgnt{r}\ \cfgnt{k}\rp 
	}
$$
In this rule, when the string matches the new-expression, it is rewritten to use
a new heap location where all of the fields for the new object point to
$\cfgnt{r}_\mathit{null}$
and the location map points a new stack reference to that new object.

To show the heap summary algorithm is sound and complete with respect
to properties proved by GSE, we define a transition relation that
relates states according GSE and refer to it as $\rgse$ or $p \rgse
p^\prime$. Similarly we define for the new summary heap algorithm
$\rsym$ or $q \rsym q^\prime$. For simplicity, both use lazy
initialization as opposed to lazier or lazier\#.

The full operational semantics for $\rgse$ are in the supplemental
document. In the next section, we present the rewrite rules specific
to $\rsym$.

\subsection{Generating Summary Heaps}
\label{sec:precise}
The summary
heap algorithm only
partitions the heap at points of divergence in the control flow graph
such as field accesses or writes (execeptional control flow due to null
pointers) or comparisons of references (program control flow). There
are three sets of rewrite rules specific to the summary heap
algorithm: (i) rules to initialize symbolic references, (ii) rules to
complete field access and field write, and (iii) rules to check
equality and inequality of references. The other rules are common to
both $\rsym$ and $\rgse$. The complete set of rules are shown in the
supplemental attached with the paper.

Initialization of references in GSE ($\rgse$ relation) is a
non-deterministic choice where the reference in the heap is
initialized to null, a new instance of the object, or to any type
compatible object in the input heap (e.g., previously initialized by
$\rgse$). In contrast, initialization in the $\rsym$ relation creates
a single summary heap containing all of the possible initialization
choices.

\input{summarize-precise-incomplete}

\begin{figure*}
\begin{tabular}[c]{l}
\scalebox{1.0}{\usebox{\boxPI}} \\
\end{tabular}
\caption{The summary machine, $s ::= \lp\cfgnt{L}\ \cfgnt{R}\ \cfgnt{r}\ \cfgnt{f}\ \cfgnt{C}\rp$, with $s\rsum^*s^\prime =  s \rsum \cdots \rsum s^\prime \rsum s^\prime$.}
\label{fig:symInit}
\end{figure*}

The initialization rules, \figref{fig:symInit}, are invoked whenever a
field pointing to an uninitialized reference is accessed. The
interaction with the solver in the definition of the rules is denoted
by $\mathbb{S}(\phi)$ which returns true if $\phi$ is satisfiable. The
summarize-end rule becomes active when there is nothing to
initialize. This condition is determined by the function
$\mathbb{UN}(\cfgnt{L}, \cfgnt{R}, \cfgnt{r}, \cfgnt{f}) =
\{\lp\phi\ \cfgnt{l}\rp\ ...\}$ which returns constraint-location
pairs where the field $\cfgnt{f}$ is uninitialized.
\[
\begin{array}{rcl}
\mathbb{UN}(\cfgnt{L}, \cfgnt{R}, \cfgnt{r}, \cfgnt{f}) & = &\{ \lp\phi\ \cfgnt{l}\rp \mid \lp \phi\ \cfgnt{l}\rp  \in \cfgnt{L}\lp \cfgnt{r}\rp  \wedge \\
& & \ \ \ \ \exists \phi^\prime \lp \lp \phi^\prime\ \cfgnt{l}_\mathit{un}\rp  \in \cfgnt{L}\lp \cfgnt{R}\lp l,\cfgnt{f}\rp\rp \wedge \\
& & \ \ \ \ \ \ \ \ \mathbb{S}\lp \phi \wedge \phi^\prime \rp\rp\}\\
\end{array}
\]
A pair in the set means the field in the location maps to a reference
that then points to $\cfgnt{l}_\mathit{un}$, and that the path to that
location is feasible on some aliasing assignment. 

% The cardinality of the set, $\theta$ is never
%greater than one in GSE and the constraint is always satisfiable
%because all constraints are constant. This property is relaxed in GSE
%with heap summaries.

In~\figref{fig:symInit}, given the uninitialized set $\Lambda$ for
field $f$, the $\mathit{min}_l$ function returns
$(\phi_x\ \cfgnt{l}_x)$ based on a lexicographical ordering of
uninitialized locations in $\Lambda$ to make initialization
deterministic (i.e., the same heap is always initialized in the same
way as order determines how many aliases are available in the input heap at any given point in initialization). 
%The $\mathrm{fresh}_\cfgnt{l}(\cfgnt{C})$ call returns a new location of type $C$. 
The set $\rho$ contains the set of
reference-location pairs from the input heap that are potential
aliases. Minimizing over the set of references in the pre-image ensures
that any pair in $\rho$ is the reference $\cfgnt{r}_a$ and the location $\cfgnt{l}_a$ materialized at the same time in some prior initialization (i.e., they were once
$\cfgnt{r}_f$ and $\cfgnt{l}_f$). Aliasing constraints
in the new summary heap reason over these original creators; otherwise aliasing and local value set
computation breaks down.  The reference partitions with their
monotonicity make this determination possible.

There are three sets corresponding to the non-deterministic choices in
GSE with a fourth that preserves the pre-initialized heap structure.
Note that the constraints always preserve the access path to the location which is $\phi_x$.
(i) $\theta_\mathit{null}$ asserts the condition under which
$\cfgnt{l}_\mathit{null}$ is possible (any time the solver assigns
values to reference variables in such a way that both $\phi_x$ and $\cfgnt{r}_f =
\cfgnt{r}_\mathit{null}$ hold). (ii)
$\theta_\mathit{new}$ is active when $\cfgnt{r}_f$ points to the
fresh location. This occurs when $\cfgnt{r}_f$ is not new or any other alias. (iii) each member of $\theta_\mathit{alias}$ restricts $\cfgnt{r}_f$ to
a particular alias in $\rho$, and at the same time, not alias any member of $\rho$
that was initialized earlier than the current choice (a later
initialization may indeed also alias the current choice in a valid
heap as order determines the number of aliasing choices available). And (iv) $\theta_\mathit{orig}$  implements conditional initialization to preserve the
original heap structure. This property is needed for the bisimulation. 
These sets are added into the heap on $\cfgnt{r}_f$ after the fields for $\cfgnt{l}_f$ are initialized to $\cfgnt{r}_\mathit{un}$.

\begin{figure*}[t]
\begin{center}
\begin{tabular}[c]{c|c|c|c}
\begin{tabular}[c]{c}
\scalebox{0.81}{\input{origHeap.pdf_t}} \\
\end{tabular} &
\begin{tabular}[c]{c}
\scalebox{0.81}{\input{summarizeXHeap.pdf_t}} \\
\end{tabular} &
\begin{tabular}[c]{c}
\scalebox{0.81}{\input{summarizeYHeap.pdf_t}} \\
\end{tabular} &
\begin{tabular}[c]{l}
$\rho := \{ (\cfgnt{r}_1^i\ \cfgnt{l}_1 \}$ \\
$\theta_\mathit{null} := \{ ( \cfgnt{r}_2^i = \cfgnt{r}_\mathit{null}\ \cfgnt{l}_\mathit{null}) \}$\\
$\theta_\mathit{new} := \{ ( \cfgnt{r}_2^i \neq \cfgnt{r}_\mathit{null} \wedge \cfgnt{r}_2^i \neq \cfgnt{r}_1^i\ \cfgnt{l}_2) \}$\\
$\theta_\mathit{alias} := \{ ( \cfgnt{r}_1^i \neq \cfgnt{r}_\mathit{null} \wedge \cfgnt{r}_2^i \neq \cfgnt{r}_\mathit{null} \wedge \cfgnt{r}_2^i = \cfgnt{r}_1^i\ \cfgnt{l}_1) \}$\\
$\theta_\mathit{orig} := \{ \}$ \\
$\phi_{\mathit{1a}} := \cfgnt{r}_1^i = \cfgnt{r}_\mathit{null} $ \\
$\phi_{\mathit{1b}} := \cfgnt{r}_1^i \neq \cfgnt{r}_\mathit{null} $  \\
$\phi_{\mathit{2a}} := \cfgnt{r}_2^i = \cfgnt{r}_\mathit{null}$ \\
$\phi_{\mathit{2b}} := \cfgnt{r}_2^i \neq \cfgnt{r}_\mathit{null} \wedge \cfgnt{r}_2^i \neq \cfgnt{r}_1^i$ \\
$\phi_{\mathit{2c}} :=  \cfgnt{r}_2^i \neq \cfgnt{r}_\mathit{null} \wedge \cfgnt{r}_2^i = \cfgnt{r}_1^i $ \\
\end{tabular} \\
(a) & (b) & (c) & (d) \\
\end{tabular}
\end{center}
\caption{An example that initializes $\lp\cfgt{this}\  \cfgt{\$}\ \cfgnt{x}\rp$ and $\lp\cfgt{this}\  \cfgt{\$}\ \cfgnt{y}\rp$. (a) Initial heap structure. (b) After $\lp\cfgt{this}\  \cfgt{\$}\ \cfgnt{x}\rp$ is initialized. (c) After $\lp\cfgt{this}\  \cfgt{\$}\ \cfgnt{y}\rp$ is initialized. (d) Sets in the summarize rule and constraints on the edges.}
\label{fig:initHeap}
\end{figure*}

We visualize the initialization process in~\figref{fig:initHeap}. The
graph in~\figref{fig:initHeap}(a) represents the initial heap. The
reference superscripts $s$ and $i$ indicate the partition containing the reference.
 In~\figref{fig:initHeap}, $\cfgnt{r}_0^s$ represents a stack
reference for the $\cfgt{this}$ variable which has two fields $x$
and $y$ of the same type. Note that when no constraint is specified for a location,
there is an implicit $\mathit{true}$ constraint. For example, $\cfgnt{r}_0^s$
points to $\cfgnt{l}_0$ on the constraint $\mathit{true}$. The fields $x$ and
$y$ point to the uninitialized reference $\cfgnt{r}_\mathit{un}$.

The graph in~\figref{fig:initHeap}(b) represents the summary heap
after the initialization of the $\lp\cfgt{this}\  \cfgt{\$}\ \cfgnt{x}\rp$ field while
the graph in~\figref{fig:initHeap}(c) represents the symbolic heap
summary after the initialization of the $\lp\cfgt{this}\  \cfgt{\$}\ \cfgnt{y}\rp$ field
following the initialization of $\lp\cfgt{this}\  \cfgt{\$}\ \cfgnt{x}\rp$. The list
in~\figref{fig:initHeap}(d) represents the various sets constructed in
the summarize rule during the initialization of $\lp\cfgt{this}\  \cfgt{\$}\ \cfgnt{y}\rp$. We
also define values of constraints used as labels in the graph.

Accessing field $x$ from $\cfgnt{l}_0$ creates a new input heap reference $\cfgnt{r}_1^i$ in the
summary heap shown in~\figref{fig:initHeap}(b). Reference $\cfgnt{r}_1^i$
points to the $\cfgnt{l}_\mathit{null}$ location under the constraint that
$\cfgnt{r}_1^i$ is null (constraint $\phi_{1a}$ in  \figref{fig:initHeap}(d)), reference
$\cfgnt{r}_1^i$ points to location $\cfgnt{l}_1$ under the constraint that $\cfgnt{r}_1^i$
is not $\mathit{null}$ (constraint $\phi_{1b}$), and fresh location $\cfgnt{l}_1$ of type $C$ 
(the type of field $\cfgnt{x}$), is added to the heap. There are no aliases because $\cfgnt{l}_1$ 
is the first location of type $C$ to be initialized.

Following $\lp\cfgt{this}\ \cfgt{\$}\ \cfgnt{x}\rp$,
$\lp\cfgt{this}\ \cfgt{\$}\ \cfgnt{y}\rp$ is initialized leading to
the summary heap in~\figref{fig:initHeap}(c). The set $\rho$
in~\figref{fig:initHeap}(d) contains the one potential aliases
$\cfgnt{l}_1$ from the previous initialization for
$\lp\cfgt{this}\ \cfgt{\$}\ \cfgnt{x}\rp$. It also shows the
$\theta$-sets for this initialization and the resulting edge
labels. Note that the ordering between aliases in
$\theta_\mathit{alias}$ is insignificant until a third field is
initialized and there are two potential choices.
