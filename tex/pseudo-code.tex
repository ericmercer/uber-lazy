\section{Generalized Symbolic Execution with Lazy Initialization}
Generalized symbolic execution with lazy initialization is a technique
to apply symbolic execution to non-primitive data types (i.e.,
objects) \cite{GSE03}. The approach initializes each symbolic object
to be either null, a new instance of the object, or an alias to an
object previously created as part of lazy initialization. Each of
these outcomes is a possible choice, and a model checker is able to
exhaustively enumerates these choices effectively creating all
possible concrete heaps under the choice sets. The approach is first formalized in this section and then extended to describe the precise heap summaries.


Many of these concrete
heaps explore redundant control flow paths in the program under test,
but since symbolic model checking is stateless, subject to a bound to
terminate the search, the model checker needlessly explores many
redundant state leading to severe state explosion.


\figref{fig:surface-syntax} defines the surface syntax for the
Javalite language \cite{saints-MS}. \figref{fig:machine-syntax} is
the machine syntax. Javalite is syntactic machine
defined as rewrites on a string. The semantics use a CEKS model with a (C)ontrol string representing the expression being
evaluated, an (E)nvironment for local variables, a (K)ontinuation for
what is to be executed next, and a (S)tore for the heap. 

\begin{figure}
\begin{center}
\cfgstart
\cfgrule{P}{\lp $\mu$ \lp \cfgnt{C} \cfgnt{m}\rp\rp}
\cfgrule{$\mu$}{(\cfgnt{CL} ...)}
\cfgrule{T}{\cfgt{bool} \cfgor \cfgnt{C}}
\cfgrule{CL}{\lp\cfgt{class} \cfgnt{C} \lp\lb\cfgnt{T} \cfgnt{f}\rb ...\rp \lp\cfgnt{M} ...\rp}
\cfgrule{M}{\lp\cfgnt{T} \cfgnt{m} \lb\cfgnt{T} \cfgnt{x}\rb\  e\rp}
\cfgrule{e}{\cfgnt{x}
\cfgor{\lp\cfgt{new} \cfgnt{C}\rp}
\cfgor{\lp\cfgnt{e} \cfgt{\$} \cfgnt{f}\rp}
\cfgor{\lp\cfgnt{x} \cfgt{\$} \cfgnt{f} \cfgt{:=} \cfgnt{e}\rp}
\cfgor{\lp\cfgnt{e} \cfgt{=} \cfgnt{e}\rp}}
\cfgorline{\lp\cfgt{if} \cfgnt{e} \cfgnt{e} \cfgt{else} \cfgnt{e}\rp 
\cfgor {\lp\cfgt{var} \cfgnt{T} \cfgnt{x} \cfgt{:=} \cfgnt{e} \cfgt{in} \cfgnt{e}\rp}
\cfgor {\lp\cfgnt{e} \cfgt{@} \cfgnt{m} \cfgnt{e} \rp}}
\cfgorline{\lp\cfgnt{x} \cfgt{:=} \cfgnt{e}\rp
\cfgor{\lp\cfgt{begin} \cfgnt{e} ...\rp}
\cfgor{\cfgnt{v}}}
\cfgrule{x}{\cfgt{this} \cfgor \cfgnt{id}}
\cfgrule{f,m,C}{\cfgnt{id}}
%\cfgrule{m}{\cfgnt{id}}
%\cfgrule{C}{\cfgnt{id}}
\cfgrule{v}{\cfgnt{r} \cfgor \cfgt{null} \cfgor \cfgt{true} \cfgor \cfgt{false} \cfgor \cfgt{error}}
\cfgrule{r}{\cfgt{number}}
\cfgrule{id}{\cfgt{variable-not-otherwise-mentioned}}
\cfgend
\end{center}
\caption{The Javalite surface syntax.}
\label{fig:surface-syntax}
\end{figure}

\begin{figure}
\begin{center}
\cfgstart
\cfgrule{e}{\lp .... \cfgor \lp\cfgt{raw} \cfgnt{v} \cfgt{@} \cfgnt{m} \cfgnt{v} \rp\rp}
%    ;; eval syntax
%  (object (C [f loc] ...))
%  (hv v
%      object)
\cfgrule{$\phi$}{\cfgnt{constraint}}
\cfgrule{l}{\cfgt{number}}
%  (h mt
%     (h [loc -> hv]))
%  (η mt
%     (η [x -> loc]))
\cfgrule{s}{\lp$\mu$ \cfgnt{L} \cfgnt{R} $\eta$ \cfgnt{e} \cfgnt{k}\rp}
\cfgrule{k}{\cfgt{end}}
\cfgorline{\lp \cfgt{*} \cfgt{\$} \cfgnt{f} $\rightarrow$ \cfgnt{k}\rp}
%     (* @ m (e ...) -> k)
%     (v @ m (v ...) * (e ...) -> k)
%     (* == e -> k)
%     (v == * -> k)
%     (x := * -> k)
%     (x $ f := * -> k)
%     (if * e else e -> k)
%     (var T x := * in e -> k)
%     (begin * (e ...) -> k)
%     (pop η k))
\cfgend
\end{center}
\caption{The machine syntax for Javalite.}
\label{fig:machine-syntax}
\end{figure}


%\item $n$ are numbers, and the set of all positive natural numbers is $\mathbb{N}^+$.
The heap is a labeled bipartite graph consisting of references $R$ and
locations $L$ in the store. The functions $R$ and $L$ are defined for convenience
in manipulating the labeled bipartite graph.
\begin{itemize}
\item $R(l,f)$ maps location-field pairs from the store to a reference in $R$; and
\item $L(r)$ maps references to a set of location-constraint pairs in the store.
\end{itemize}
A reference is a node that gathers the possible store locations for an
object during symbolic execution. Each store location is guarded by a
constraint that determines the aliasing in the heap. Intuitively, the
reference is a level of indirection between a variable and the store,
and the reference is used to group a set of possible store locations
each predicated on the possible aliasing in the associated constraint.
For a variable (or field) to access any particular store location
associated with its reference, the corresponding constraint must be
satisfied.

Locations are boxes in the graphical representation and indicated with
the letter $l$ in the math. References are circles in the graphical
representation and indicated with the letter $\cfgnt{r}$ in the
math. The special symbol $\bot$ is an uninitialized reference. Edges
from locations are labeled with field names $\cfgnt{f}$. Edges from
the references are labeled with constraints $\phi \in \Phi$ (it is
assumed that $\Phi$ is a power set over individual constraints and
$\phi$ is a set of constraints for the edge).

The function $\mathbb{VS}(L,R,\phi_g,r,f)$ constructs the value-set given a
heap, reference, and desired field:
\[
\begin{array}{rcl}
  \mathbb{VS}(L,R,\phi_g,r,f) & = & \{(l^\prime\ \phi\wedge\phi^\prime) \mid \\
  & & \ \ \ \ \exists l\ ((l\ \phi) \in L(r)\ \wedge \\
  & & \ \ \ \ \ \ \ \ \exists r^\prime \in R(l,f) ( \\
  & & \ \ \ \ \ \ \ \ \ \ \ \ (l^\prime\ \phi^\prime) \in L(r^\prime)\ \wedge\\
  & & \ \ \ \ \ \ \ \ \ \ \ \ \mathbb{S}(\phi\wedge\phi^\prime\wedge \phi_g)))\}
\end{array}
\]
where $\mathbb{S}(\phi)$ returns true if $\phi$ is satisfiable.

The strengthen function $\mathbb{ST}(L,r,\phi^\prime)$ strengthens every
constraint from the reference $r$ with $\phi^\prime$ and keeps only location-constraint
pairs that are satisfiable after this strengthening:
\[
\begin{array}{rcl} 
\mathbb{ST}(L,r,\phi^\prime) & = & \{ (l\ \phi\wedge\phi^\prime) \mid 
(l\ \phi)\in L(r)\wedge\mathbb{S}(\phi\wedge\phi^\prime) \}
\end{array}
\]

\begin{figure*}[t]
\begin{center}
\mprset{flushleft}
\begin{mathpar}
	\inferrule[Variable lookup]{}{
      \lp \cfgnt{L}\ \cfgnt{R}\ \phi\ \eta\ \cfgnt{x}\ \cfgnt{k}\rp  \rightarrow_J \\\\
      \lp \cfgnt{L}\ \cfgnt{R}\ \phi\ \eta\ \eta\lp \cfgnt{x}\rp \ \cfgnt{k}\rp 
	}
\and
	\inferrule[New]{
      \cfgnt{r} = \mathrm{stack}_r\lp \rp\\
      l = \mathrm{fresh}_l\lp \cfgnt{C}\rp\\\\
      \cfgnt{R}^\prime = \cfgnt{R}[\forall \cfgnt{f} \in \mathit{fields}\lp \mathrm{C}\rp \ \lp \lp l\ \cfgnt{f}\rp  \mapsto \cfgnt{r}_\mathit{null} \rp ] \\\\
      \cfgnt{L}^\prime = \cfgnt{L}[\cfgnt{r} \mapsto \{\lp \cfgt{true}\ l\rp \}]
    }{
      \lp \cfgnt{L}\ \cfgnt{R}\ \phi\ \eta\ \lp \cfgt{new}\ \cfgnt{C}\rp \ \cfgnt{k}\rp  \rightarrow_J
      \lp \cfgnt{L}^\prime\ \cfgnt{R}^\prime\ \phi\ \eta\ \cfgnt{r}\ \cfgnt{k}\rp 
	}
\and
	\inferrule[Field Access(eval)]{}{
      \lp \cfgnt{L}\ \cfgnt{R}\ \phi\ \eta\ \lp \cfgnt{e}\ \cfgt{\$}\ \cfgnt{f}\rp \ \cfgnt{k}\rp  \rightarrow_J \\\\
      \lp \cfgnt{L}\ \cfgnt{R}\ \phi\ \eta\ \cfgnt{e}\ \lp \cfgt{*}\ \cfgt{\$}\ \cfgnt{f} \rightarrow \cfgnt{k}\rp \rp 
	}
\and
	\inferrule[Field Write (eval)]{}{
       \lp \cfgnt{L}\ \cfgnt{R}\ \phi\ \eta\ \lp \cfgnt{x}\ \cfgt{\$}\ \cfgnt{f}\ \cfgt{:=}\ \cfgnt{e}\rp \ \cfgnt{k}\rp  \rightarrow_J \\\\
       \lp \cfgnt{L}\ \cfgnt{R}\ \phi\ \eta\ \cfgnt{e}\ \lp \cfgnt{x}\ \cfgt{\$}\ \cfgnt{f}\ \cfgt{:=}\ \cfgt{*}\ \rightarrow\ \cfgnt{k}\rp \rp 
	}
\and
    \inferrule[Equals (l-operand eval)]{}{
      \lp \cfgnt{L}\ \cfgnt{R}\ \phi\ \eta\ \lp \cfgnt{e}_0\ \cfgt{=}\ \cfgnt{e}\rp  \ \cfgnt{k}\rp  \rightarrow_J \\\\
      \lp \cfgnt{L}\ \cfgnt{R}\ \phi\ \eta\ \cfgnt{e}_0\ \lp \cfgt{*}\ \cfgt{=}\; \cfgnt{e} \rightarrow \cfgnt{k}\rp \rp 
    }
\and
    \inferrule[Equals (r-operand eval)]{}{
    \lp \cfgnt{L}\ \cfgnt{R}\ \phi\ \eta\ \cfgnt{v}\ \lp \cfgt{*}\; \cfgt{=}\; \cfgnt{e} \rightarrow \cfgnt{k}\rp \rp  \rightarrow_J \\\\
    \lp \cfgnt{L}\ \cfgnt{R}\ \phi\ \eta\ \cfgnt{e}\ \lp \cfgnt{v}\; \cfgt{=}\; \cfgt{*} \rightarrow \cfgnt{k}\rp \rp 
    }
\and
    \inferrule[Equals (bool)]{
    \cfgnt{v}_0 \in \{\cfgt{true}, \cfgt{false}\} \\
    \cfgnt{v}_1 \in \{\cfgt{true}, \cfgt{false}\} \\\\
    \cfgnt{v}_r = \mathrm{eq?}\lp \cfgnt{v}_0, \cfgnt{v}_1\rp}{
    \lp \cfgnt{L}\ \cfgnt{R}\ \phi\ \eta\ \cfgnt{v}_0\ \lp \cfgnt{v}_1\; \cfgt{=}\; \cfgt{*} \rightarrow \cfgnt{k}\rp \rp  \rightarrow_J \\\\
    \lp \cfgnt{L}\ \cfgnt{R}\ \phi\ \eta\ \cfgnt{v}_r\ \cfgnt{k}\rp 
    }
\and
    \inferrule[If-then-else (eval)]{}{
      \lp \cfgnt{L}\ \cfgnt{R}\ \phi\ \eta\ \lp \cfgt{if}\ \cfgnt{e}_0\ \cfgnt{e}_1\ \cfgt{else}\ \cfgnt{e}_2\rp \ \cfgnt{k}\rp  \rightarrow_J \\\\
      \lp \cfgnt{L}\ \cfgnt{R}\ \phi\ \eta\ \cfgnt{e}_0\ \lp \cfgt{if}\ \cfgt{*}\ \cfgnt{e}_1\ \cfgt{else}\ \cfgnt{e}_2\rp  \rightarrow \cfgnt{k}\rp 
	}
\and
	\inferrule[If-then-else (true) ]{}{
       \lp \cfgnt{L}\ \cfgnt{R}\ \phi\ \eta\ \cfgt{true}\ \lp \cfgt{if}\ \cfgt{*}\ \cfgnt{e}_1\ \cfgt{else}\ \cfgnt{e}_2\rp  \rightarrow_J \cfgnt{k}\rp  \rightarrow \\\\
       \lp \cfgnt{L}\ \cfgnt{R}\ \phi\ \eta\ \cfgnt{e}_1\  \cfgnt{k}\rp 
	}
\and
	\inferrule[If-then-else (false)]{}{
       \lp \cfgnt{L}\ \cfgnt{R}\ \phi\ \eta\ \cfgt{false}\ \lp \cfgt{if}\ \cfgt{*}\ \cfgnt{e}_1\ \cfgt{else}\ \cfgnt{e}_2\rp  \rightarrow_J \cfgnt{k}\rp  \rightarrow \\\\
       \lp \cfgnt{L}\ \cfgnt{R}\ \phi\ \eta\ \cfgnt{e}_2\  \cfgnt{k}\rp 
	}
\and
   \inferrule[Variable Declaration (eval)]{}{
    \lp \cfgnt{L}\ \cfgnt{R}\ \phi\ \eta\ \lp\cfgt{var}\ \cfgnt{T}\ \cfgnt{x}\ \cfgt{:=}\ \cfgnt{e}_0\ \cfgt{in}\ \cfgnt{e}_1\rp\ \cfgnt{k}\rp  \rightarrow_J \\\\
    \lp \cfgnt{L}\ \cfgnt{R}\ \phi\ \eta\ \cfgnt{e}_0\ \lp\cfgt{var}\ \cfgnt{T}\ \cfgnt{x}\ \cfgt{:=}\ \cfgt{*}\ \cfgt{in}\ \cfgnt{e}_1 \rightarrow \cfgnt{k}\rp\rp 
   }	
\and
   \inferrule[Variable Declaration]{}{
    \lp \cfgnt{L}\ \cfgnt{R}\ \phi\ \eta\ \cfgnt{v}\ \lp\cfgt{var}\ \cfgnt{T}\ \cfgnt{x}\ \cfgt{*}\ \cfgt{:=}\ \cfgt{in}\ \cfgnt{e}_1 \rightarrow \cfgnt{k}\rp\rp  \rightarrow_J \\\\
    \lp \cfgnt{L}\ \cfgnt{R}\ \phi\ \eta[x \mapsto \cfgnt{v}]\ \cfgnt{e}_1\ \lp \cfgt{pop}\ \eta\ \cfgnt{k}\rp \rp 
   }	
\and
   \inferrule[Method Invocation (object eval)]{}{
    \lp \cfgnt{L}\ \cfgnt{R}\ \phi\ \eta\ \lp\cfgnt{e}_0\ \cfgt{@}\ \cfgnt{m}\ \cfgnt{e}_1\rp\ \cfgnt{k}\rp  \rightarrow_J \\\\
    \lp \cfgnt{L}\ \cfgnt{R}\ \phi\ \eta\ \cfgnt{e}_0\ \lp \cfgt{*}\ \cfgt{@}\ \cfgnt{m}\ \cfgnt{e}_1\ \rightarrow \cfgnt{k}\rp \rp 
   }
\and
   \inferrule[Method Invocation (arg eval)]{}{
    \lp \cfgnt{L}\ \cfgnt{R}\ \phi\ \eta\ \cfgnt{v}_0\ \lp \cfgt{*}\ \cfgt{@}\ \cfgnt{m}\ \cfgnt{e}_1\ \rightarrow \cfgnt{k}\rp \rp  \rightarrow_J \\\\
    \lp \cfgnt{L}\ \cfgnt{R}\ \phi\ \eta\ \cfgnt{e}_1\ \lp \cfgnt{v}_0\ \cfgt{@}\ \cfgnt{m}\ \cfgt{*}\ \rightarrow \cfgnt{k}\rp \rp 
   }
\and
   \inferrule[Method Invocation]{
    \lp\cfgnt{T}\ \cfgnt{m}\ \lb\cfgnt{T}\ \cfgnt{x}\rb\ \ \cfgnt{e}_m\rp = \mathrm{lookup}\lp \cfgnt{m}\rp\\\\
    \eta_m = \eta[\cfgt{this} \mapsto \cfgnt{v}_0][\cfgnt{x} \mapsto \cfgnt{v}_1]}{
    \lp \cfgnt{L}\ \cfgnt{R}\ \phi\ \eta\ \cfgnt{v}_1\ \lp \cfgnt{v}_0\ \cfgt{@}\ \cfgnt{m}\ \cfgt{*}\ \rightarrow \cfgnt{k}\rp \rp  \rightarrow_J \\\\
    \lp \cfgnt{L}\ \cfgnt{R}\ \phi\ \eta_m\ \cfgnt{e}_m\ \lp \cfgt{pop}\ \eta\ \cfgnt{k}\rp \rp 
   }
\and
   \inferrule[Variable Assignment (eval)]{}{
    \lp \cfgnt{L}\ \cfgnt{R}\ \phi\ \eta\ \lp \cfgnt{x}\ \cfgt{:=}\ \cfgnt{e}\rp \ \cfgnt{k}\rp  \rightarrow_J \\\\
    \lp \cfgnt{L}\ \cfgnt{R}\ \phi\ \eta\ \cfgnt{e}\ \lp \cfgnt{x}\ \cfgt{:=}\ \cfgt{*} \rightarrow\ \cfgnt{k}\rp \rp 
   }	
\and
   \inferrule[Variable Assignment]{}{
    \lp \cfgnt{L}\ \cfgnt{R}\ \phi\ \eta\ \cfgnt{v}\ \lp \cfgnt{x}\ \cfgt{:=}\ \cfgt{*} \rightarrow\ \cfgnt{k}\rp \rp  \rightarrow_J \\\\
    \lp \cfgnt{L}\ \cfgnt{R}\ \phi\ \eta[\cfgnt{x} \mapsto \cfgnt{v}]\ \cfgnt{v}\ \cfgnt{k}\rp 
   }	
\and
   \inferrule[Begin (no args)]{}{
    \lp \cfgnt{L}\ \cfgnt{R}\ \phi\ \eta\ \lp \cfgt{begin}\rp \ \cfgnt{k}\rp  \rightarrow \\\\
    \lp \cfgnt{L}\ \cfgnt{R}\ \phi\ \eta\ \cfgnt{k}\rp 
   }
\and
   \inferrule[Begin (arg0 eval)]{}{
    \lp \cfgnt{L}\ \cfgnt{R}\ \phi\ \eta\ \lp \cfgt{begin}\ \cfgnt{e}_0\ \cfgnt{e}_1\ ...\rp \ \cfgnt{k}\rp  \rightarrow_J \\\\
    \lp \cfgnt{L}\ \cfgnt{R}\ \phi\ \eta\ \cfgnt{e}_0\ \lp \cfgt{begin}\ \cfgt{*}\ \lp\cfgnt{e}_1\ ...\rp \rightarrow \cfgnt{k}\rp \rp 
   }
\and
   \inferrule[Begin (argi eval)]{}{
    \lp \cfgnt{L}\ \cfgnt{R}\ \phi\ \eta\ \cfgnt{v}\ \lp \cfgt{begin}\ \cfgt{*}\ \lp\cfgnt{e}_i\ \cfgnt{e}_{i+1}\ ...\rp \rightarrow \cfgnt{k}\rp \rp  \rightarrow \\\\
    \lp \cfgnt{L}\ \cfgnt{R}\ \phi\ \eta\ \cfgnt{e}_i\ \lp \cfgt{begin}\ \cfgt{*}\ \lp\cfgnt{e}_{i+1}\ ...\rp \rightarrow \cfgnt{k}\rp \rp 
   }
\and
   \inferrule[Begin (argN eval)]{}{
    \lp \cfgnt{L}\ \cfgnt{R}\ \phi\ \eta\ \cfgnt{v}\ \lp \cfgt{begin}\ \cfgt{*}\ \lp\cfgnt{e}_{n}\rp \rightarrow \cfgnt{k}\rp \rp  \rightarrow \\\\
    \lp \cfgnt{L}\ \cfgnt{R}\ \phi\ \eta\ \cfgnt{e}_n\ \lp \cfgt{begin}\ \cfgt{*}\ \lp\rp \rightarrow \cfgnt{k}\rp \rp 
   }
\and
   \inferrule[Begin]{}{
    \lp \cfgnt{L}\ \cfgnt{R}\ \phi\ \eta\ \cfgnt{v}\ \lp \cfgt{begin}\ \cfgt{*}\ \lp\rp \rightarrow \cfgnt{k}\rp \rp  \rightarrow \\\\
    \lp \cfgnt{L}\ \cfgnt{R}\ \phi\ \eta\ \cfgnt{v}\ \cfgnt{k}\rp 
   }	
\and
	\inferrule[NULL]{}{
      \lp \cfgnt{L}\ \cfgnt{R}\ \phi\ \eta\ \cfgt{null}\ \cfgnt{k}\rp \rightarrow \\\\ 
      \lp \cfgnt{L}\ \cfgnt{R}\ \phi\ \eta\ \cfgnt{r}_\mathit{null}\ \cfgnt{k}\rp 
	}
\and
   \inferrule[Pop]{}{
    \lp \cfgnt{L}\ \cfgnt{R}\ \phi\ \eta\ \cfgnt{v}\ \lp \cfgt{pop}\ \eta_0\ \cfgnt{k}\rp \rp  \rightarrow \\\\
    \lp \cfgnt{L}\ \cfgnt{R}\ \phi\ \eta_0\  \cfgnt{v}\ \cfgnt{k}\rp 
   }
\end{mathpar}
\end{center}
\caption{Javalite rewrite rules that are common to generalized symbolic execution and precise heap summaries.}
\label{fig:javalite-common}
\end{figure*}

\begin{figure*}[t]
\begin{center}
\mprset{flushleft}
\begin{mathpar}
	\inferrule[Initialize (null)]{
	  \Lambda = \mathbb{UN}\lp \cfgnt{L}, \cfgnt{R}, \cfgnt{r}, \cfgnt{f}\rp \\
      \Lambda \neq \emptyset\\\\
      \cfgnt{r}^\prime = \mathrm{fresh}_r\lp \rp\\ 
      \theta_\mathit{null} = \{ \lp \cfgt{true}\ l_\mathit{null}\rp \} \\\\
      l_x = \mathrm{min}_l\lp \Lambda\rp \\\\
      \phi_g^\prime = \lp\phi_g \wedge \cfgnt{r}^\prime = \cfgnt{r}_\mathit{null}\rp
    }{
      \lp \cfgnt{L}\ \cfgnt{R}\ \phi_g\ \cfgnt{r}\ \cfgnt{f}\ \cfgnt{C}\rp  \rightarrow_I 
      \lp \cfgnt{L}[\cfgnt{r}^\prime \mapsto \theta_\mathit{null}]\ \cfgnt{R}[ \lp l_x,\cfgnt{f}\rp  \mapsto \cfgnt{r}^\prime]\ \phi_g^\prime\ \cfgnt{r}\ \cfgnt{f}\ \cfgnt{C}\rp 
	}
\and
	\inferrule[Initialize (new)]{
	  \Lambda = \mathbb{UN}\lp \cfgnt{L}, \cfgnt{R}, \cfgnt{r}, \cfgnt{f}\rp \\
      \Lambda \neq \emptyset\\
      \lp\phi_x\ \cfgnt{l}_x\rp = \mathrm{min}_l\lp \Lambda\rp\\\\
      \cfgnt{r}_f = \mathrm{init}_r\lp \rp\\
      l_f = \mathrm{fresh}_l\lp \cfgnt{C}\rp \\\\
      \rho = \{ \lp\cfgnt{r}_a\ l_a\rp \mid \mathrm{isInit}\lp \cfgnt{r}_a\rp  \wedge \cfgnt{r}_a = \mathrm{min}_r\lp \cfgnt{R}^{-1}[l_a]\rp \wedge \mathrm{type}\lp l_a\rp  = \cfgnt{C} \}\\\\
      \theta_\mathit{new} = \{\lp \cfgt{true}\ l_f\rp \} \\\\
      \cfgnt{R}^\prime = \cfgnt{R}[\forall \cfgnt{f} \in \mathit{fields}\lp \mathrm{C}\rp \ \lp \lp l_f\ \cfgnt{f}\rp  \mapsto \cfgnt{r}_\mathit{un} \rp ] \\\\
      \phi_g^\prime = \lp\phi_g \wedge \cfgnt{r}_f \neq \cfgnt{r}_\mathit{null} \wedge \lp \wedge_{\lp\cfgnt{r}_a\ l_a\rp \in \rho} \cfgnt{r}_f \ne \cfgnt{r}_a\rp\rp
    }{
      \lp \cfgnt{L}\ \cfgnt{R}\ \phi_g\ \cfgnt{r}\ \cfgnt{f}\ \cfgnt{C}\rp  \rightarrow_I 
      \lp \cfgnt{L}[\cfgnt{r}_f \mapsto \theta_\mathit{new}]\ \cfgnt{R}^\prime[ \lp l_x,\cfgnt{f}\rp  \mapsto \cfgnt{r}_f ]\ \phi_g^\prime\ \cfgnt{r}\ \cfgnt{f}\ \cfgnt{C}\rp 
	}
\and
	\inferrule[Initialize (alias)]{
  	  \Lambda = \mathbb{UN}\lp \cfgnt{L}, \cfgnt{R}, \cfgnt{r}, \cfgnt{f}\rp \\
      \Lambda \neq \emptyset\\
      \lp\phi_x\ \cfgnt{l}_x\rp = \mathrm{min}_l\lp \Lambda\rp\\\\
      \cfgnt{r}^\prime = \mathrm{fresh}_r\lp \rp\\\\
      \rho = \{ \lp\cfgnt{r}_a\ l_a\rp \mid \mathrm{isInit}\lp \cfgnt{r}_a\rp  \wedge \cfgnt{r}_a = \mathrm{min}_r\lp \cfgnt{R}^{-1}[l_a]\rp \wedge \mathrm{type}\lp l_a\rp  = \cfgnt{C} \}\\\\
      \lp\cfgnt{r}_a\ l_a\rp \in \rho \\
      \theta_\mathit{alias} = \{ \lp \cfgt{true}\ l_a\rp\}\\\\
      \phi^\prime_g = \lp\phi_g \wedge \cfgnt{r}^\prime \neq \cfgnt{r}_\mathit{null} \wedge \cfgnt{r}^\prime = \cfgnt{r}_a \wedge \lp \wedge_{\lp \cfgnt{r}^{\prime}_a\ l_a\rp  \in \rho\ \lp \cfgnt{r}^{\prime}_a \neq \cfgnt{r}_a\rp } \cfgnt{r}^\prime \neq \cfgnt{r}^{\prime}_a \rp\rp
    }{
      \lp \cfgnt{L}\ \cfgnt{R}\ \phi_g\ \cfgnt{r}\ \cfgnt{f}\ \cfgnt{C}\rp  \rightarrow_I 
      \lp \cfgnt{L}[\cfgnt{r}^\prime \mapsto \theta_\mathit{alias}]\ \cfgnt{R}[ \lp l_x,\cfgnt{f}\rp  \mapsto \cfgnt{r}^\prime ]\ \phi_g^\prime\ \cfgnt{r}\ \cfgnt{f}\ \cfgnt{C}\rp 
	}
\and
	\inferrule[Initialize (end)]{
	  \Lambda = \mathbb{UN}\lp \cfgnt{L}, \cfgnt{R}, \cfgnt{r}, \cfgnt{f}\rp \\
      \Lambda = \emptyset
    }{
      \lp \cfgnt{L}\ \cfgnt{R}\ \phi_g\ \cfgnt{r}\ \cfgnt{f}\ \cfgnt{C}\rp  \rightarrow_I 
      \lp \cfgnt{L}\ \cfgnt{R}\ \phi_g\ \cfgnt{r}\ \cfgnt{f}\ \cfgnt{C}\rp 
	}
\end{mathpar}
\end{center}
\caption{The initialization machine, $s ::= \lp\cfgnt{L}\ \cfgnt{R}\ \phi_g\ \cfgnt{r}\ \cfgnt{f}\rp$, with $s \rightarrow_I^* s^\prime$ indicating stepping the machine until the state does not change.}
\label{fig:lazyInit}
\end{figure*}

\section{Initialization of Symbolic References}

In this section we present the Javalite rewrite rules for the concrete
as well as summary initialization of symbolic references. The
initialization rules are defined on the bi-partite graph consisting of
references and locations. The lazy initialization of symbolic
references consists of three key points of non-determinism where each
symbolic reference can be initialized non-deterministically to null, a
new instance of the symbolic reference, or aliases to symbolic
references of the same type previously initialized. The initialization
in GSE consists of creating branches in the execution tree for all the
non-deterministic choices. On the other hand, the heap summarization
approach creates a single branch that contains the summarization for
all the initialization in a single bi-partitate graph.

\begin{figure*}[t]
\begin{center}
\mprset{flushleft}
\begin{mathpar}
	\inferrule[Initialize (null)]{
	  \Lambda = \mathbb{UN}\lp \cfgnt{L}, \cfgnt{R}, \cfgnt{r}, \cfgnt{f}\rp \\
      \Lambda \neq \emptyset\\\\
      \cfgnt{r}^\prime = \mathrm{fresh}_r\lp \rp\\ 
      \theta_\mathit{null} = \{ \lp \cfgt{true}\ l_\mathit{null}\rp \} \\\\
      l_x = \mathrm{min}_l\lp \Lambda\rp \\\\
      \phi_g^\prime = \lp\phi_g \wedge \cfgnt{r}^\prime = \cfgnt{r}_\mathit{null}\rp
    }{
      \lp \cfgnt{L}\ \cfgnt{R}\ \phi_g\ \cfgnt{r}\ \cfgnt{f}\ \cfgnt{C}\rp  \rightarrow_I 
      \lp \cfgnt{L}[\cfgnt{r}^\prime \mapsto \theta_\mathit{null}]\ \cfgnt{R}[ \lp l_x,\cfgnt{f}\rp  \mapsto \cfgnt{r}^\prime]\ \phi_g^\prime\ \cfgnt{r}\ \cfgnt{f}\ \cfgnt{C}\rp 
	}
\and
	\inferrule[Initialize (new)]{
	  \Lambda = \mathbb{UN}\lp \cfgnt{L}, \cfgnt{R}, \cfgnt{r}, \cfgnt{f}\rp \\
      \Lambda \neq \emptyset\\
      \lp\phi_x\ \cfgnt{l}_x\rp = \mathrm{min}_l\lp \Lambda\rp\\\\
      \cfgnt{r}_f = \mathrm{init}_r\lp \rp\\
      l_f = \mathrm{fresh}_l\lp \cfgnt{C}\rp \\\\
      \rho = \{ \lp\cfgnt{r}_a\ l_a\rp \mid \mathrm{isInit}\lp \cfgnt{r}_a\rp  \wedge \cfgnt{r}_a = \mathrm{min}_r\lp \cfgnt{R}^{-1}[l_a]\rp \wedge \mathrm{type}\lp l_a\rp  = \cfgnt{C} \}\\\\
      \theta_\mathit{new} = \{\lp \cfgt{true}\ l_f\rp \} \\\\
      \cfgnt{R}^\prime = \cfgnt{R}[\forall \cfgnt{f} \in \mathit{fields}\lp \mathrm{C}\rp \ \lp \lp l_f\ \cfgnt{f}\rp  \mapsto \cfgnt{r}_\mathit{un} \rp ] \\\\
      \phi_g^\prime = \lp\phi_g \wedge \cfgnt{r}_f \neq \cfgnt{r}_\mathit{null} \wedge \lp \wedge_{\lp\cfgnt{r}_a\ l_a\rp \in \rho} \cfgnt{r}_f \ne \cfgnt{r}_a\rp\rp
    }{
      \lp \cfgnt{L}\ \cfgnt{R}\ \phi_g\ \cfgnt{r}\ \cfgnt{f}\ \cfgnt{C}\rp  \rightarrow_I 
      \lp \cfgnt{L}[\cfgnt{r}_f \mapsto \theta_\mathit{new}]\ \cfgnt{R}^\prime[ \lp l_x,\cfgnt{f}\rp  \mapsto \cfgnt{r}_f ]\ \phi_g^\prime\ \cfgnt{r}\ \cfgnt{f}\ \cfgnt{C}\rp 
	}
\and
	\inferrule[Initialize (alias)]{
  	  \Lambda = \mathbb{UN}\lp \cfgnt{L}, \cfgnt{R}, \cfgnt{r}, \cfgnt{f}\rp \\
      \Lambda \neq \emptyset\\
      \lp\phi_x\ \cfgnt{l}_x\rp = \mathrm{min}_l\lp \Lambda\rp\\\\
      \cfgnt{r}^\prime = \mathrm{fresh}_r\lp \rp\\\\
      \rho = \{ \lp\cfgnt{r}_a\ l_a\rp \mid \mathrm{isInit}\lp \cfgnt{r}_a\rp  \wedge \cfgnt{r}_a = \mathrm{min}_r\lp \cfgnt{R}^{-1}[l_a]\rp \wedge \mathrm{type}\lp l_a\rp  = \cfgnt{C} \}\\\\
      \lp\cfgnt{r}_a\ l_a\rp \in \rho \\
      \theta_\mathit{alias} = \{ \lp \cfgt{true}\ l_a\rp\}\\\\
      \phi^\prime_g = \lp\phi_g \wedge \cfgnt{r}^\prime \neq \cfgnt{r}_\mathit{null} \wedge \cfgnt{r}^\prime = \cfgnt{r}_a \wedge \lp \wedge_{\lp \cfgnt{r}^{\prime}_a\ l_a\rp  \in \rho\ \lp \cfgnt{r}^{\prime}_a \neq \cfgnt{r}_a\rp } \cfgnt{r}^\prime \neq \cfgnt{r}^{\prime}_a \rp\rp
    }{
      \lp \cfgnt{L}\ \cfgnt{R}\ \phi_g\ \cfgnt{r}\ \cfgnt{f}\ \cfgnt{C}\rp  \rightarrow_I 
      \lp \cfgnt{L}[\cfgnt{r}^\prime \mapsto \theta_\mathit{alias}]\ \cfgnt{R}[ \lp l_x,\cfgnt{f}\rp  \mapsto \cfgnt{r}^\prime ]\ \phi_g^\prime\ \cfgnt{r}\ \cfgnt{f}\ \cfgnt{C}\rp 
	}
\and
	\inferrule[Initialize (end)]{
	  \Lambda = \mathbb{UN}\lp \cfgnt{L}, \cfgnt{R}, \cfgnt{r}, \cfgnt{f}\rp \\
      \Lambda = \emptyset
    }{
      \lp \cfgnt{L}\ \cfgnt{R}\ \phi_g\ \cfgnt{r}\ \cfgnt{f}\ \cfgnt{C}\rp  \rightarrow_I 
      \lp \cfgnt{L}\ \cfgnt{R}\ \phi_g\ \cfgnt{r}\ \cfgnt{f}\ \cfgnt{C}\rp 
	}
\end{mathpar}
\end{center}
\caption{The initialization machine, $s ::= \lp\cfgnt{L}\ \cfgnt{R}\ \phi_g\ \cfgnt{r}\ \cfgnt{f}\rp$, with $s \rightarrow_I^* s^\prime$ indicating stepping the machine until the state does not change.}
\label{fig:lazyInit}
\end{figure*}

\begin{figure*}[t]
\begin{center}
\mprset{flushleft}
\begin{mathpar}
	\inferrule[Summarize]{
	\Lambda = \mathbb{UN}\lp \cfgnt{L}, \cfgnt{R}, \cfgnt{r}, \cfgnt{f}\rp \\
      \Lambda \neq \emptyset \\
      \lp\phi_x\ \cfgnt{l}_x\rp = \mathrm{min}_l\lp \Lambda\rp\\
      \cfgnt{r}_f = \mathrm{init}_r\lp \rp \\
      l_f  = \mathrm{fresh}_l\lp \mathrm{C}\rp\\\\
      \rho = \{ \lp \cfgnt{r}_a\ \phi_a\ l_a\rp  \mid \mathrm{isInit}\lp \cfgnt{r}_a\rp  \wedge\cfgnt{r}_a = \mathrm{min}_r\lp \cfgnt{R}^{-1}[l_a]\rp \wedge \lp \phi_a\ l_a\rp  \in \cfgnt{L}\lp \cfgnt{r}_a\rp \wedge \mathrm{type}\lp l_a\rp  = \mathrm{C} \} \\\\
      \theta_\mathit{null} = \{ \lp \phi\ l_\mathit{null}\rp  \mid \phi = \lp \phi_x \wedge \cfgnt{r}_f = \cfgnt{r}_\mathit{null} \rp  \} \\\\
      \theta_\mathit{new} = \{\lp \phi\ l_f\rp  \mid \phi = \lp \phi_x \wedge \cfgnt{r}_f \neq \cfgnt{r}_\mathit{null} \wedge \lp \wedge_{\lp \cfgnt{r}^\prime_a,\ \phi^\prime_a,\ l^\prime_a\rp  \in \rho} \cfgnt{r}_f \ne \cfgnt{r}^\prime_a\rp \rp \}\\\\
      \theta_\mathit{alias} = \{ \lp \phi\ l_a\rp  \mid \exists\cfgnt{r}_a\ \lp\exists \phi_a\ \lp\lp\cfgnt{r}_a\ \phi_a\ l_a\rp  \in \rho \wedge \phi = \lp \phi_x \wedge \phi_a \wedge \cfgnt{r}_f \neq \cfgnt{r}_\mathit{null} \wedge \cfgnt{r}_f = \cfgnt{r}_a \wedge \lp \wedge_{\lp \cfgnt{r}^{\prime}_a\ \phi^{\prime}_a\ l^{\prime}_a\rp  \in \rho\ \lp \cfgnt{r}^\prime_a \neq \cfgnt{r}_a\rp } \cfgnt{r}_f \neq \cfgnt{r}^{\prime}_a \rp \rp \rp \rp \} \\\\
      \theta_\mathit{orig} = \{\lp\phi\ \cfgnt{l}_\mathit{orig}\rp \mid \exists \phi_\mathit{orig} \lp \lp\phi_\mathit{orig}\ \cfgnt{l}_\mathit{orig}\rp \in \cfgnt{L}\lp\cfgnt{R}\lp\cfgnt{l}_x,\cfgnt{f}\rp\rp \wedge \phi = \lp\neg\phi_x \wedge \phi_\mathit{orig}\rp\}\\\\ 
      \theta = \theta_\mathit{alias} \cup \theta_\mathit{new} \cup \theta_\mathit{null} \cup \theta_\mathit{old} \\
\cfgnt{R}^\prime = \cfgnt{R}[\forall \cfgnt{f} \in \mathit{fields}\lp \mathrm{C}\rp \ \lp \lp l_f\ \cfgnt{f}\rp  \mapsto \cfgnt{r}_\mathit{un} \rp ]
    }{
      \lp \cfgnt{L}\ \cfgnt{R}\ \cfgnt{r}\ \cfgnt{f}\ \cfgnt{C}\rp \rightarrow_S 
      \lp \cfgnt{L}[\cfgnt{r}_f \mapsto \theta]\ \cfgnt{R}^{\prime}[ \lp l_x,\cfgnt{f}\rp  \mapsto \cfgnt{r}_f ]\ \cfgnt{r}\ \cfgnt{f}\ \cfgnt{C}\rp
	}
\and
	\inferrule[Summarize (end)]{
	  \Lambda = \{ l \mid \exists \phi\ \lp \lp \phi\ l\rp  \in \cfgnt{L}\lp \cfgnt{r}\rp  \wedge  \cfgnt{R}\lp l,\cfgnt{f}\rp  = \bot\ \rp\}\\
      \Lambda = \emptyset
    }{
      \lp \cfgnt{L}\ \cfgnt{R}\ \cfgnt{r}\ \cfgnt{f}\ \cfgnt{C}\rp  \rightarrow_S
      \lp \cfgnt{L}\ \cfgnt{R}\ \cfgnt{r}\ \cfgnt{f}\ \cfgnt{C}\rp 
	}
\end{mathpar}
\end{center}
\caption{The summary machine, $s ::= \lp\cfgnt{L}\ \cfgnt{R}\ \cfgnt{r}\ \cfgnt{f}\ \cfgnt{C}\rp$, with $s\rightarrow_S^*s^\prime$ indicating stepping the machine until the state does not change.}
\label{fig:symInit}
\end{figure*}



The initialization rules are invoked when an uninitialized field in a
symbolic reference is accessed. The function $\mathbb{UN}(\cfgnt{L},
\cfgnt{R}, \cfgnt{r}, \cfgnt{f}) = \{\cfgnt{l}\ ...\}$ returns
constraint-location pairs in which the field $\cfgnt{f}$ is
uninitialized:
\[
\begin{array}{rcl}
\mathbb{UN}(\cfgnt{L}, \cfgnt{R}, \cfgnt{r}, \cfgnt{f}) & = &\{ \lp\phi\ \cfgnt{l}\rp \mid \lp \phi\ \cfgnt{l}\rp  \in \cfgnt{L}\lp \cfgnt{r}\rp  \wedge \\
& & \ \ \ \ \exists \phi^\prime \lp \lp \phi^\prime\ \cfgnt{l}_\mathit{un}\rp  \in \cfgnt{L}\lp \cfgnt{R}\lp l,\cfgnt{f}\rp\rp \wedge \\
& & \ \ \ \ \ \ \ \ \mathbb{S}\lp \phi \wedge \phi^\prime \rp\rp\}\\
\end{array}
\]
where $\mathbb{S}(\phi)$ returns true if $\phi$ is
satisfiable. Intutively, for the reference, $\cfgnt{r}$, it constructs
the set, $\theta$, that contains all contraint-location pairs that
point to the field $\cfgnt{f}$ and $\cfgnt{f}$ points to
$\cfgnt{l}_\mathit{un}$. The cardinality of the set, $\theta$ is never
greater than one in GSE and the constraint is always satisfiable
because all constraints are constant. This property is relaxed in GSE
with heap summaries.

The rules in~\figref{fig:lazyInit} present the rewrite rules for the
concrete initialization of symbolic heap objects.  These rules are
invoked until a fix pointed is reached. 

The initialize (null) rewrite rule in~\figref{fig:lazyInit} first
checks that the field, $\cfgnt{r}$ is uninitialized. The fresh method
returns a new input heap reference from the partition 


\section{Accessing and Writing to Field References}

\section{Equality and InEquality of References}

\begin{figure*}[t]
\begin{center}
\mprset{flushleft}
\begin{mathpar}
	\inferrule[Field Access]{
      \{\lp\phi\ l\rp\} = \cfgnt{L}\lp\cfgnt{r}\rp\\
      l \neq \cfgnt{l}_\mathit{null}\\
      \cfgnt{C} = \mathrm{type}\lp\cfgnt{l},\cfgnt{f}\rp\\\\
      \lp \cfgnt{L}\ \cfgnt{R}\ \cfgnt{r}\ \cfgnt{f}\ \cfgnt{C}\rp \rinit^*
      \lp \cfgnt{L}^\prime\ \cfgnt{R}^\prime\ \cfgnt{r}\ \cfgnt{f}\  \cfgnt{C}\rp \\\\ 
      \{\lp\phi^\prime\ l^\prime\rp\} = \cfgnt{L}^\prime\lp\cfgnt{R}^\prime\lp l,\cfgnt{f}\rp\rp \\
      \cfgnt{r}^\prime = \mathrm{stack}_r\lp\rp \\
    }{
      \lp \cfgnt{L}\ \cfgnt{R}\ \phi_g\ \eta\ \cfgnt{r}\ \lp \cfgt{*}\ \cfgt{\$}\ \cfgnt{f} \rightarrow \cfgnt{k}\rp \rp  \rightarrow_\ell \\\\
      \lp \cfgnt{L}^\prime[\cfgnt{r}^\prime \mapsto \lp\phi^\prime\ l^\prime\rp]\ \cfgnt{R}^\prime\ \phi_g^\prime\ \eta\ \cfgnt{r}^\prime\ \cfgnt{k}\rp 
	}
\and
	\inferrule[Field Write]{
      \cfgnt{r}_x = \eta\lp \cfgnt{x}\rp\\ 
      \theta = \{\lp\phi\ l\rp\} = \cfgnt{L}\lp\cfgnt{r}_x\rp \\\\
      l \neq \cfgnt{l}_\mathit{null}\\
      \cfgnt{r}^\prime = \mathrm{fresh}_r\lp\rp\\
    }{
      \lp \cfgnt{L}\ \cfgnt{R}\ \phi_g\ \eta\ \cfgnt{r}\ \lp \cfgnt{x}\ \cfgt{\$}\ \cfgnt{f}\ \cfgt{:=}\ \cfgt{*}\ \rightarrow\ \cfgnt{k}\rp \rp  \rightarrow_\ell \\\\
      \lp \cfgnt{L}[\cfgnt{r}^\prime \mapsto \theta]\ \cfgnt{R}[\lp l\ \cfgnt{f}\rp  \mapsto \cfgnt{r}^\prime]\ \phi_g\ \eta\ \cfgnt{r}\ \cfgnt{k}\rp 
	}
\and
  \inferrule[Equals (reference-true)]{
    \cfgnt{L}\lp \cfgnt{r}_0\rp = \cfgnt{L}\lp \cfgnt{r}_1\rp\\
    \phi^\prime = \lp\phi \wedge r_0 = r_1\rp
    }{
    \lp \cfgnt{L}\ \cfgnt{R}\ \phi\ \eta\ \cfgnt{r}_0\ \lp \cfgnt{r}_1\ \cfgt{=}\ \cfgt{*} \rightarrow \cfgnt{k}\rp \rp  \rightarrow_\ell \\\\
    \lp \cfgnt{L}\ \cfgnt{R}\ \phi^\prime\ \eta\ \cfgt{true}\ \cfgnt{k}\rp 
    }
\and
    \inferrule[Equals (reference-false)]{
    \cfgnt{L}\lp \cfgnt{r}_0\rp \neq \cfgnt{L}\lp \cfgnt{r}_1\rp\\
    \phi^\prime = \lp\phi \wedge r_0 \neq r_1\rp
   }{
    \lp \cfgnt{L}\ \cfgnt{R}\ \phi\ \eta\ \cfgnt{r}_0\ \lp \cfgnt{r}_1\ \cfgt{=}\ \cfgt{*} \rightarrow \cfgnt{k}\rp \rp  \rightarrow_\ell \\\\
    \lp \cfgnt{L}\ \cfgnt{R}\ \phi^\prime\ \eta\ \cfgt{false}\ \cfgnt{k}\rp 
    }	
\end{mathpar}
\end{center}
\caption{GSE with lazy initialization indicated by $\rgse = \rightarrow_\ell \cup \rcom$.}
\label{fig:lazy}
\end{figure*}




\begin{figure*}[t]
\begin{center}
\mprset{flushleft}
\begin{mathpar}
	\inferrule[Summarize]{
	\Lambda = \mathbb{UN}\lp \cfgnt{L}, \cfgnt{R}, \cfgnt{r}, \cfgnt{f}\rp \\
      \Lambda \neq \emptyset \\
      \lp\phi_x\ \cfgnt{l}_x\rp = \mathrm{min}_l\lp \Lambda\rp\\
      \cfgnt{r}_f = \mathrm{init}_r\lp \rp \\
      l_f  = \mathrm{fresh}_l\lp \mathrm{C}\rp\\\\
      \rho = \{ \lp \cfgnt{r}_a\ \phi_a\ l_a\rp  \mid \mathrm{isInit}\lp \cfgnt{r}_a\rp  \wedge\cfgnt{r}_a = \mathrm{min}_r\lp \cfgnt{R}^{-1}[l_a]\rp \wedge \lp \phi_a\ l_a\rp  \in \cfgnt{L}\lp \cfgnt{r}_a\rp \wedge \mathrm{type}\lp l_a\rp  = \mathrm{C} \} \\\\
      \theta_\mathit{null} = \{ \lp \phi\ l_\mathit{null}\rp  \mid \phi = \lp \phi_x \wedge \cfgnt{r}_f = \cfgnt{r}_\mathit{null} \rp  \} \\\\
      \theta_\mathit{new} = \{\lp \phi\ l_f\rp  \mid \phi = \lp \phi_x \wedge \cfgnt{r}_f \neq \cfgnt{r}_\mathit{null} \wedge \lp \wedge_{\lp \cfgnt{r}^\prime_a,\ \phi^\prime_a,\ l^\prime_a\rp  \in \rho} \cfgnt{r}_f \ne \cfgnt{r}^\prime_a\rp \rp \}\\\\
      \theta_\mathit{alias} = \{ \lp \phi\ l_a\rp  \mid \exists\cfgnt{r}_a\ \lp\exists \phi_a\ \lp\lp\cfgnt{r}_a\ \phi_a\ l_a\rp  \in \rho \wedge \phi = \lp \phi_x \wedge \phi_a \wedge \cfgnt{r}_f \neq \cfgnt{r}_\mathit{null} \wedge \cfgnt{r}_f = \cfgnt{r}_a \wedge \lp \wedge_{\lp \cfgnt{r}^{\prime}_a\ \phi^{\prime}_a\ l^{\prime}_a\rp  \in \rho\ \lp \cfgnt{r}^\prime_a \neq \cfgnt{r}_a\rp } \cfgnt{r}_f \neq \cfgnt{r}^{\prime}_a \rp \rp \rp \rp \} \\\\
      \theta_\mathit{orig} = \{\lp\phi\ \cfgnt{l}_\mathit{orig}\rp \mid \exists \phi_\mathit{orig} \lp \lp\phi_\mathit{orig}\ \cfgnt{l}_\mathit{orig}\rp \in \cfgnt{L}\lp\cfgnt{R}\lp\cfgnt{l}_x,\cfgnt{f}\rp\rp \wedge \phi = \lp\neg\phi_x \wedge \phi_\mathit{orig}\rp\}\\\\ 
      \theta = \theta_\mathit{alias} \cup \theta_\mathit{new} \cup \theta_\mathit{null} \cup \theta_\mathit{old} \\
\cfgnt{R}^\prime = \cfgnt{R}[\forall \cfgnt{f} \in \mathit{fields}\lp \mathrm{C}\rp \ \lp \lp l_f\ \cfgnt{f}\rp  \mapsto \cfgnt{r}_\mathit{un} \rp ]
    }{
      \lp \cfgnt{L}\ \cfgnt{R}\ \cfgnt{r}\ \cfgnt{f}\ \cfgnt{C}\rp \rightarrow_S 
      \lp \cfgnt{L}[\cfgnt{r}_f \mapsto \theta]\ \cfgnt{R}^{\prime}[ \lp l_x,\cfgnt{f}\rp  \mapsto \cfgnt{r}_f ]\ \cfgnt{r}\ \cfgnt{f}\ \cfgnt{C}\rp
	}
\and
	\inferrule[Summarize (end)]{
	  \Lambda = \{ l \mid \exists \phi\ \lp \lp \phi\ l\rp  \in \cfgnt{L}\lp \cfgnt{r}\rp  \wedge  \cfgnt{R}\lp l,\cfgnt{f}\rp  = \bot\ \rp\}\\
      \Lambda = \emptyset
    }{
      \lp \cfgnt{L}\ \cfgnt{R}\ \cfgnt{r}\ \cfgnt{f}\ \cfgnt{C}\rp  \rightarrow_S
      \lp \cfgnt{L}\ \cfgnt{R}\ \cfgnt{r}\ \cfgnt{f}\ \cfgnt{C}\rp 
	}
\end{mathpar}
\end{center}
\caption{The summary machine, $s ::= \lp\cfgnt{L}\ \cfgnt{R}\ \cfgnt{r}\ \cfgnt{f}\ \cfgnt{C}\rp$, with $s\rightarrow_S^*s^\prime$ indicating stepping the machine until the state does not change.}
\label{fig:symInit}
\end{figure*}

\subsection{Accessing a Field Reference}



\newsavebox{\boxPFAFW}
\savebox{\boxPFAFW}{
%\begin{figure}[t]
%\begin{center}
\mprset{flushleft}
\begin{mathpar}
	\inferrule[Field Access]{
      \exists \lp \phi\ l\rp \in \cfgnt{L}\lp \cfgnt{r}\rp\ \lp l \neq l_{\mathit{null}} \wedge \mathbb{S}\lp \phi \wedge \phi_g\rp \rp \\\\
      \theta = \{ \phi \mid \lp \phi\ l_\mathit{null} \rp \wedge \mathbb{S}\lp \phi \wedge \phi_g\rp \} \\\\
      \phi_g^\prime = \phi_g \wedge (\wedge_{\phi \in \theta} \neg \phi) \\\\
      \{\cfgnt{C}\} = \{\cfgnt{C} \mid \exists \lp \phi\ l\rp  \in \cfgnt{L}\lp \cfgnt{r}\rp\ \lp\cfgnt{C} = \mathrm{type}\lp \cfgnt{l},\cfgnt{f}\rp\rp\} \\\\
      \lp \cfgnt{L}\ \cfgnt{R}\ \cfgnt{r}\ \cfgnt{f}\ \cfgnt{C}\rp \rsum^* \lp \cfgnt{L}^\prime\ \cfgnt{R}^\prime\ \cfgnt{r}\ \cfgnt{f}\ \cfgnt{C}\rp \\
      \cfgnt{r}^\prime = \mathrm{stack}_r\lp \rp
    }{
      \lp \cfgnt{L}\ \cfgnt{R}\ \phi_g\ \eta\ \cfgnt{r}\ \lp \cfgt{*}\ \cfgt{\$}\ \cfgnt{f} \rightarrow \cfgnt{k}\rp \rp  \rightarrow_\mathit{A}
      \lp \cfgnt{L}^\prime[\cfgnt{r}^\prime \mapsto \mathbb{VS}\lp \cfgnt{L}^\prime,\cfgnt{R}^\prime,\cfgnt{r},\cfgnt{f},\phi_g^\prime\rp ]\ \cfgnt{R}^\prime\ \phi_g^\prime\ \eta\ \cfgnt{r}^\prime\ \cfgnt{k}\rp 
	}
\and
	\inferrule[Field Access (NULL)]{
      \exists \lp \phi\ l\rp \in \cfgnt{L}\lp \cfgnt{r}\rp\ \lp l = l_{\mathit{null}} \wedge \mathbb{S}\lp \phi \wedge \phi_g\rp \rp
    }{
      \lp \cfgnt{L}\ \cfgnt{R}\ \phi_g\ \eta\ \cfgnt{r}\ \lp \cfgt{*}\ \cfgt{\$}\ \cfgnt{f} \rightarrow \cfgnt{k}\rp \rp  \rightarrow_\mathit{A}
      \lp \cfgnt{L}\ \cfgnt{R}\ \phi_g\ \eta\ \cfgt{error}\ \cfgt{end} \rp
	}
\and
	\inferrule[Field Write]{
      \cfgnt{r}_x = \eta\lp\cfgnt{x}\rp \\
      \exists \lp \phi\ l\rp \in \cfgnt{L}\lp \cfgnt{r}_x\rp\ \lp l \neq l_{\mathit{null}} \wedge \mathbb{S}\lp \phi \wedge \phi_g\rp \rp \\\\
      \theta = \{ \phi \mid \lp \phi\ l_\mathit{null} \rp \wedge \mathbb{S}\lp \phi \wedge \phi_g\rp \} \\\\
      \phi_g^\prime = \phi_g \wedge (\wedge_{\phi \in \theta} \neg \phi) \\\\
      \Psi_x =\{\lp \phi\ l\ \cfgnt{r}_\mathit{cur} \rp  \mid \lp \phi\ \cfgnt{l}\rp  \in \cfgnt{L}\lp \cfgnt{r}_x\rp  \wedge \cfgnt{r}_\mathit{cur} = \cfgnt{R}\lp l,\cfgnt{f}\rp  \}\\\\
      X = \{ \lp l\ \theta \rp  \mid \exists \phi\ \lp \lp \phi\ l\ \cfgnt{r}_\mathit{cur} \rp \in \Psi_x \wedge \theta = \mathbb{ST}\lp \cfgnt{L},\cfgnt{r},\phi,\phi_g^\prime\rp  \cup \mathbb{ST}\lp \cfgnt{L},\cfgnt{r}_\mathit{cur},\neg\phi,\phi_g^\prime\rp \rp  \}\\\\
      \cfgnt{R}^{\prime} = \cfgnt{R}[\forall \lp l\ \theta \rp  \in X\ \lp \lp l\ \cfgnt{f}\rp  \mapsto \mathrm{fresh}_r\lp \rp \rp ]\\\\
      \cfgnt{L}^{\prime} = \cfgnt{L}[\forall \lp l\ \theta \rp  \in X\ \lp \exists \cfgnt{r}_\mathit{targ}\ \lp \cfgnt{r}_\mathit{targ} = \cfgnt{R}^\prime\lp l,\cfgnt{f}\rp \wedge \lp\cfgnt{r}_\mathit{targ} \mapsto \theta\rp  \rp \rp ]
    }{
      \lp \cfgnt{L}\ \cfgnt{R}\ \phi_g\ \eta\ \cfgnt{r}\ \lp \cfgnt{x}\ \cfgt{\$}\ \cfgnt{f}\ \cfgt{:=}\ \cfgt{*}\ \rightarrow\ \cfgnt{k}\rp \rp  \rightarrow_\mathit{FW}
      \lp \cfgnt{L}^{\prime}\ \cfgnt{R}^{\prime}\ \phi_g^\prime\ \eta\ \cfgnt{r}\ \cfgnt{k}\rp 
	}	
\and
	\inferrule[Field Write (NULL)]{
      \cfgnt{r}_x = \eta\lp \cfgnt{x}\rp \\
      \exists \lp \phi\ l\rp \in \cfgnt{L}\lp \cfgnt{r}_x\rp\ \lp l \neq l_{\mathit{null}} \wedge \mathbb{S}\lp \phi \wedge \phi_g\rp \rp
    }{
      \lp \cfgnt{L}\ \cfgnt{R}\ \phi_g\ \eta\ \cfgnt{r}\ \lp \cfgnt{x}\ \cfgt{\$}\ \cfgnt{f}\ \cfgt{:=}\ \cfgt{*}\ \rightarrow\ \cfgnt{k}\rp \rp  \rightarrow_\mathit{FW}
      \lp \cfgnt{L}\ \cfgnt{R}\ \phi_g\ \eta\ \cfgt{error}\ \cfgt{end}\rp
	}	
\end{mathpar}}
%\end{center}
%\caption{Precise symbolic heap summaries from symbolic execution indicated by $\rsym = \rightarrow_\mathit{FA} \cup \rightarrow_\mathit{FW} \cup \rightarrow_\mathit{EQ} \cup \rcom$.}
%\label{fig:symfield}
%\end{figure}




\begin{figure*}[t]
\begin{center}
\setlength{\tabcolsep}{50pt}
\begin{tabular}[c]{cc}
\usebox{\boxPFAFW} & 
%\scalebox{0.91}{\input{faYHeap.pdf_t}} &
\scalebox{0.91}{\input{fwXHeap.pdf_t}} \\ \\
(a) & (b)
\end{tabular}
\end{center}
\caption{field access for this.y and field write for this.x = this.y}
\label{fig:fHeap}
\end{figure*}

There are two rewrite rules in~\figref{fig:fHeap}(a), one for field
access and the other for field write. Let us consider the rewrite rule
for field access. The first check we perform is whether there exists a
constraint location pair for the $r$ being accessed such that the
location is not null and the constraint when conjuncted with the
global constraint is satisfiable. Next we extract all possible
constraints under which $r$ points to a null location such that the
constraint is satisfiable under the current global constraint,
$\phi_g$. The negation of these constraints are added to the global
constraint to create a new global constraint $\phi_g^\prime$. The
update to the global constraint ensures that access of the field $f$
happens only on non-null locations. The type $C$ of the field is
extracted and passed to rewrite rule that performs the initialization
of the symbolic heap. The initialization rules are described earlier
in this section. 

Recall that during the initialization the rewrite rules
in~\figref{fig:symHeap} add input references that map to fields being
initialized.  Once the initialization is complete, however, we create
a new local reference $r^\prime$. An important property of the
references in the bi-partiate graph is that they are
\emph{immutable}. Hence we de-reference the initialized input
reference, assign its value to the the new local reference, and return
the local reference. In order to de-reference a field $r.f$ we define
a helper function which is called the value set.

\begin{definition}
\label{def:VS}
The function $\mathbb{VS}(L,R,\phi_g,r,f)$ constructs the value-set given a
heap, reference, and desired field:
\[
\begin{array}{rcl}
  \mathbb{VS}(\cfgnt{L},\cfgnt{R},\phi_g,\cfgnt{r},\cfgnt{f}) & = & \{(\phi\wedge\phi^\prime\ \cfgnt{l}^\prime) \mid \\
  & & \ \ \ \ \exists \cfgnt{l}\ ((\phi\ l) \in L(r)\ \wedge \\
  & & \ \ \ \ \ \ \ \ \exists \cfgnt{r}^\prime ( \cfgnt{r}^\prime = R(\cfgnt{l},\cfgnt{f})\ \wedge \\
  & & \ \ \ \ \ \ \ \ \ \ \ \ (\phi^\prime\ l^\prime) \in \cfgnt{L}(\cfgnt{r}^\prime)\ \wedge\\
  & & \ \ \ \ \ \ \ \ \ \ \ \ \mathbb{S}(\phi\wedge\phi^\prime\wedge \phi_g)))\}
\end{array}
\]
where $\mathbb{S}(\phi)$ returns true if $\phi$ is satisfiable.
\end{definition}


In the post-condition of the rewrite rule we assign
the value set of input reference $r$ to the local reference $r^\prime$
and return the local reference $r^\prime$ in the next state.

Consider the graph shown in~\figref{fig:fHeap}(b) the reference
$r_2^i$ is created during the initialization of $\mathit{this}.y$. The
reference that is returned during the access of the field, however, is
$r_3^s$. The reference $r_3^s$ points to the value set of $r_2^i$
which are: $(\phi_{2a}, l_\mathit{null})$, $(\phi_{2b}, l_2)$, and
$(\phi_{2c}, l_1)$. The values of the constraints $\phi_{2a}$,
$\phi_{2b}$, and $\phi_{2c}$ are defined in~\figref{fig:initHeap}(d).

\subsection{Writing to a Field}

The reference $r_x$ is the base pointer whose field is being written
to while $r$ is the target reference. We look up the value of the base
reference in the environment $\eta(x)$. The set $\Psi_x$ is the set of
tuples of constraints, locations, and references which provide the
reference chains leading from $r_x$ to the reference of the field,
$r_\mathit{curr}$, being written to through $\phi$ and $l$.The goal is
to overwrite the $r_\mathit{curr}$ references with the target
references. Since the target of the write is $r$, we first check that the
location constraint pairs of $L(r)$ are satisfiable when accessed
through the $r_x$ chain. This is accomplished by the strengthing
function.

\begin{definition}
\label{def:ST}
The strengthen function $\mathbb{ST}(\cfgnt{L},\cfgnt{r},\phi,\phi_g)$ strengthens every
constraint from the reference $\cfgnt{r}$ with $\phi$ and keeps only location-constraint
pairs that are satisfiable after this strengthening with the inclusion of the global heap constraint $\phi_g$:
\[
\begin{array}{rcl} 
\mathbb{ST}(\cfgnt{L},\cfgnt{r},\phi,\phi_g) & = & \{ (\phi\wedge\phi^\prime\ \cfgnt{l}^\prime) \mid  \\
& & \ \ \ \ (\phi^\prime\ \cfgnt{l}^\prime)\in \cfgnt{L}(\cfgnt{r})\wedge\mathbb{S}(\phi\wedge\phi^\prime\wedge\phi_g) \}
\end{array}
\]
\end{definition}

Additionally, we also check for conditions where the write is not possible 

\begin{comment}
\begin{figure}[t]
\begin{center}
\begin{tabular}[c]{l}
$\Psi_x = \{ (true, l_0, r_1^i) \}$\\
$ST (L, r_3^s, \phi, \phi_g)$ \\
$\theta = \{ (\phi_{2a}\; l_\mathit{null} ) (\phi_{2b}\; l_2) (\phi_{2c}\; l_1) \}$\\
$ST(L, r_0, \phi, \phi_g)$\\
$\theta = \{ \}$\\
\end{tabular}
\end{center}
\caption{field write for this.x = this.y sets}
\label{fig:faHeapSets}
\end{figure}
\end{comment}

\subsection{Equality and InEquality of References}

\newsavebox{\boxPEQ}
\savebox{\boxPEQ}{
%\begin{figure}[t]
%\begin{center}
\mprset{flushleft}
\begin{tabular}[c]{c}
\begin{mathpar}
    \inferrule[Equals (references-true)]{
    \Phi_\alpha = \{\lp\phi_0 \wedge \phi_1\rp \mid \exists l\ \lp \lp \phi_0\ l\rp  \in \cfgnt{L}\lp \cfgnt{r}_0\rp  \wedge \lp \phi_1\ l\rp  \in \cfgnt{L}\lp \cfgnt{r}_1\rp \rp \} \\\\
    \Phi_0 = \{\phi_0 \mid \exists l_0\ \lp \lp \phi_0\ l_0\rp  \in \cfgnt{L}\lp \cfgnt{r}_0\rp  \wedge \forall \lp \phi_1\ l_1\rp  \in \cfgnt{L}\lp \cfgnt{r}_1\rp \ \lp l_0 \neq l_1\rp \rp \} \\\\
    \Phi_1 = \{\phi_1 \mid \exists l_1\ \lp \lp \phi_1\ l_1\rp  \in \cfgnt{L}\lp \cfgnt{r}_1\rp  \wedge \forall \lp \phi_0\ l_0\rp  \in \cfgnt{L}\lp \cfgnt{r}_0\rp \ \lp l_0 \neq l_1\rp \rp \} \\\\
    \phi^\prime =  \phi \wedge \lp \vee_{\phi_\alpha\in\Phi_\alpha}\phi_\alpha\rp \wedge\lp \wedge_{\phi_0 \in \Phi_0} \neg \phi_0\rp \wedge\lp \wedge_{\phi_1
    \in \Phi_1} \neg \phi_1\rp \\\\ 
    \mathbb{S}(\phi^\prime)}{
    \lp \cfgnt{L}\ \cfgnt{R}\ \phi\ \eta\ \cfgnt{r}_0\ \lp \cfgnt{r}_1\; \cfgt{=}\; \cfgt{*} \rightarrow \cfgnt{k}\rp \rp  \rsym^\mathit{E}
    \lp \cfgnt{L}\ \cfgnt{R}\ \phi^\prime\ \eta\ \cfgt{true}\ \cfgnt{k}\rp 
    }
%\and
%    \inferrule[Equals (references-false)]{
%    \Phi_\alpha = \{\lp\phi_0 \Rightarrow \neg \phi_1\rp \mid \exists l\ \lp \lp \phi_0\ l\rp  \in \cfgnt{L}\lp \cfgnt{r}_0\rp  \wedge \lp \phi_1\ l\rp  \in \cfgnt{L}\lp \cfgnt{r}_1\rp \rp \} \\\\
%    \Phi_0 = \{\phi_0 \mid \exists l_0\ \lp \lp \phi_0\ l_0\rp  \in \cfgnt{L}\lp \cfgnt{r}_0\rp  \wedge \forall \lp \phi_1\ l_1\rp  \in \cfgnt{L}\lp \cfgnt{r}_1\rp \ \lp l_0 \neq l_1\rp \rp \} \\\\
%    \Phi_1 = \{\phi_1 \mid \exists l_1\ \lp \lp \phi_1\ l_1\rp  \in \cfgnt{L}\lp \cfgnt{r}_1\rp  \wedge \forall \lp \phi_0\ l_0\rp  \in \cfgnt{L}\lp \cfgnt{r}_0\rp \ \lp l_0 \neq l_1\rp \rp \} \\\\
%    \phi^\prime = \phi \wedge \lp \wedge_{\phi_\alpha\in\theta_\alpha}\phi_\alpha\rp \vee\lp \lp \vee_{\phi_0 \in \theta_0} \phi_0\rp   \vee\lp \vee_{\phi_1
%    \in \theta_1} \phi_1\rp \rp  \\\\ 
%    \mathbb{S}(\phi^\prime)}{
%    \lp \cfgnt{L}\ \cfgnt{R}\ \phi\ \eta\ \cfgnt{r}_0\ \lp \cfgnt{r}_1\; \cfgt{%=}\; \cfgt{*} \rightarrow \cfgnt{k}\rp \rp  \rsym^\mathit{E^\prime}
%    \lp \cfgnt{L}\ \cfgnt{R}\ \phi^\prime\ \eta\ \cfgt{false}\ \cfgnt{k}\rp 
%    }
\end{mathpar}
\end{tabular}}
%\end{center}
%\caption{FIX THIS CAPTION AND MOVE $\rsym$ DEFINITION (MAY NOT BE NEEDED BECAUSE OF \defref{def:meta}: Precise symbolic heap summaries from symbolic execution indicated by $\rsym = \rightarrow_\mathit{FA} \cup \rightarrow_\mathit{FW} \cup \rightarrow_\mathit{EQ}^T \cup \rightarrow_\mathit{EQ}^F \cup \rcom$.}
%\label{fig:symeq}
%\end{figure}


\newsavebox{\boxPEX}
\savebox{\boxPEX}{
\begin{tabular}[c]{l}
$L(r_1^i) = \{ (\phi_{1a}\; l_\mathit{null})\; (\phi_{1b}\; l_1) \}$ \\
$L(r_2^i) = \{ (\phi_{2a}\; l_\mathit{null}),\; (\phi_{2b}\; l_2),\; (\phi_{2c}\; l_1) \} $\\
  $\theta_0 = \{ \} $\\
$\theta_1 = \{ \phi_{2b}\} $\\ \hline
Equals true \\
$\theta_\alpha = \{ (\phi_{1a}\; \wedge\; \phi_{2a} ) (\phi_{1b}\; \wedge\; \phi_{2c} ) \}$\\
$\phi^\prime = \mathit{true} \wedge [ (\phi_{1a}\; \wedge\; \phi_{2a} )\vee (\phi_{1b}\; \wedge\; \phi_{2c} ) ] \wedge \neg\phi_{2b} $\\ \hline
Equals false \\
$\theta_\alpha = \{ (\phi_{1a}\; \implies\; \neg\phi_{2a} ) (\phi_{1b}\; \implies\; \neg\phi_{2c} ) \}$\\
$\phi^\prime = \mathit{true} \wedge  (\phi_{1a}\; \implies\; \neg\phi_{2a} )\wedge (\phi_{1b}\; \implies\; \neg\phi_{2c} )  \wedge \phi_{2b} $\\ \hline
\end{tabular}}

\begin{figure*}
\begin{tabular}[c]{cl}
\usebox{\boxPEQ} & \usebox{\boxPEX} \\ \\
(a) & (b) \\
\end{tabular}
\caption{equals true for this.x == this.y}
\label{fig:eqs}
\end{figure*}


The rewrite rules for check the equals true and equals false when
comparing two references in the symbolic summary heap is shown
in~\figref{fig:eqs}(a). First the equals reference true rewrite rule
returns true if two references $r_0$ and $r_1$ are equal. In GSE this
is a simple comparision of object refrences. In the symbolic summary
heap, however, there we campare sets of constraint location pairs for
each reference. We construct three sets of constraints (i) In order to check whether 
there are locations, $l$ in the heap such that under some constraints $\phi_0$ references $r_0$ and $r_1$
$\Phi_\alpha$ constains combinations of constraints ($\phi_0 \wedge
\phi_1$) from $r_0$ and $r_1$

