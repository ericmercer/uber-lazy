\subsection{Generalized Symbolic Execution}
\label{app:gse}
The rules to generalized symbolic execution with
lazy initialization using the symbolic heap are in
\figref{fig:lazyInit} for the initialization and \figref{fig:lazy} for
how the initialization is used. Specifically, the field-access rule in
\figref{fig:lazy} uses initialization, $\rightarrow_I^*$, to ensure
the accessed field is instantiated. Initialization in generalized
symbolic execution never happens for more than one object on any use
of $\rightarrow_I^*$: $\Lambda$ is either empty or contains exactly
one location.

An uninitialized object may be lazily initialized
non-deterministically to null, a new instance of the object, or an to
an object of the same type also initialized lazily. The
non-deterministic initialization creates branches in the execution
tree during state space exploration. 

The $\rgse$ relation is a union over branching points on control flow
to support the final proof of the existence of a
bisimulation. Although it is apparent that there is non-determinism in
initialization, it is not the initialization where control flow
diverges in program execution. Rather, that occurs at branches and
exceptions, and that is reflected in the composition of $\rgse$.

\begin{figure*}[t]
\begin{center}
\mprset{flushleft}
\begin{mathpar}
	\inferrule[Initialize (null)]{
	  \Lambda = \{ l \mid \exists \phi\ \lp \lp \phi\ l\rp  \in \cfgnt{L}\lp \cfgnt{r}\rp  \wedge  \cfgnt{R}\lp l,\cfgnt{f}\rp  = \bot\ \rp\}\\
      \Lambda \neq \emptyset\\\\
      \cfgnt{r}^\prime = \mathrm{fresh}_r\lp \rp\\ 
      \theta_\mathit{null} = \{ \lp \phi_T\ l_\mathit{null}\rp \} \\
      l_x = \mathrm{min}_l\lp \Lambda\rp \\\\
      \phi_g^\prime = \lp\phi_g \wedge \cfgnt{r}^\prime = \cfgnt{r}_\mathit{null}\rp
    }{
      \lp \cfgnt{L}\ \cfgnt{R}\ \phi_g\ \cfgnt{r}\ \cfgnt{f}\rp  \rightarrow_I 
      \lp \cfgnt{L}[\cfgnt{r}^\prime \mapsto \theta_\mathit{null}]\ \cfgnt{R}[ \lp l_x,\cfgnt{f}\rp  \mapsto \cfgnt{r}^\prime]\ \phi_g^\prime\ \cfgnt{r}\ \cfgnt{f}\rp 
	}
\and
	\inferrule[Initialize (new)]{
	  \Lambda  = \{ l \mid \exists \phi\ \lp \lp \phi\ l\rp  \in \cfgnt{L}\lp \cfgnt{r}\rp  \wedge  \cfgnt{R}\lp l,\cfgnt{f}\rp  = \bot\rp\}\\
      \Lambda \neq \emptyset\\\\
      \mathrm{C} = \mathrm{type}\lp \cfgnt{f}\rp\\
      \cfgnt{r}_f = \mathrm{init}_r\lp \rp\\
      l_f = \mathrm{init}_l\lp \mathrm{C}\rp \\\\
      \cfgnt{R}^\prime = \cfgnt{R}[\forall \cfgnt{f} \in \mathit{fields}\lp \mathrm{C}\rp \ \lp \lp l_f\ \cfgnt{f}\rp  \mapsto \bot \rp ] \\\\
      \rho = \{ \lp\cfgnt{r}_a\ l_a\rp \mid \mathrm{isInit}\lp \cfgnt{r}_a\rp  \wedge \mathrm{type}\lp l_a\rp  = \mathrm{C} \wedge \exists \phi_a\ \lp \lp \phi_a\ l_a\rp  \in \cfgnt{L}\lp \cfgnt{r}_a\rp\rp \}\\\\
      \theta_\mathit{new} = \{\lp \phi_T\ l_f\rp \} \\
      l_x = \mathrm{min}_l\lp \Lambda\rp \\\\
      \phi_g^\prime = \lp\phi_g \wedge \cfgnt{r}_f \neq \cfgnt{r}_\mathit{null} \wedge \lp \wedge_{\lp\cfgnt{r}_a\ l_a\rp \in \rho} \cfgnt{r}_f \ne \cfgnt{r}_a\rp\rp
    }{
      \lp \cfgnt{L}\ \cfgnt{R}\ \phi_g\ \cfgnt{r}\ \cfgnt{f}\rp  \rightarrow_I 
      \lp \cfgnt{L}[\cfgnt{r}_f \mapsto \theta_\mathit{new}]\ \cfgnt{R}^\prime[ \lp l_x,\cfgnt{f}\rp  \mapsto \cfgnt{r}_f ]\ \phi_g^\prime\ \cfgnt{r}\ \cfgnt{f}\rp 
	}
\and
	\inferrule[Initialize (alias)]{
	  \Lambda = \{ l \mid \exists \phi\ \lp \lp \phi\ l\rp  \in \cfgnt{L}\lp \cfgnt{r}\rp  \wedge  \cfgnt{R}\lp l,\cfgnt{f}\rp  = \bot\ \rp\}\\
      \Lambda \neq \emptyset\\\\
      \mathrm{C} = \mathrm{type}\lp \cfgnt{f}\rp\\
      \cfgnt{r}^\prime = \mathrm{fresh}_r\lp \rp\\\\
      \rho = \{ \lp\cfgnt{r}_a\ l_a\rp \mid \mathrm{isInit}\lp \cfgnt{r}_a\rp  \wedge \mathrm{type}\lp l_a\rp  = \mathrm{C} \wedge \exists \phi_a\ \lp \lp \phi_a\ l_a\rp  \in \cfgnt{L}\lp \cfgnt{r}_a\rp\rp \}\\\\
      \lp\cfgnt{r}_a\ l_a\rp \in \rho \\
      \theta_\mathit{alias} = \{ \lp \phi_T\ l_a\rp\}\\
      l_x = \mathrm{min}_l\lp \Lambda\rp\\\\
      \phi^\prime_g = \lp\phi_g \wedge \cfgnt{r}^\prime \neq \cfgnt{r}_\mathit{null} \wedge \cfgnt{r}^\prime = \cfgnt{r}_a \wedge \lp \wedge_{\lp \cfgnt{r}^{\prime}_a\ l_a\rp  \in \rho\ \lp \cfgnt{r}^{\prime}_a \neq \cfgnt{r}_a\rp } \cfgnt{r}^\prime \neq \cfgnt{r}^{\prime}_a \rp\rp
    }{
      \lp \cfgnt{L}\ \cfgnt{R}\ \phi_g\ \cfgnt{r}\ \cfgnt{f}\rp  \rightarrow_I 
      \lp \cfgnt{L}[\cfgnt{r}^\prime \mapsto \theta_\mathit{alias}]\ \cfgnt{R}[ \lp l_x,\cfgnt{f}\rp  \mapsto \cfgnt{r}^\prime ]\ \phi_g^\prime\ \cfgnt{r}\ \cfgnt{f}\rp 
	}
\and
	\inferrule[Initialize (end)]{
	  \Lambda = \{ l \mid \exists \phi\ \lp \lp \phi\ l\rp  \in \cfgnt{L}\lp \cfgnt{r}\rp  \wedge  \cfgnt{R}\lp l,\cfgnt{f}\rp  = \bot\ \rp\}\\
      \Lambda = \emptyset
    }{
      \lp \cfgnt{L}\ \cfgnt{R}\ \phi_g\ \cfgnt{r}\ \cfgnt{f}\rp  \rightarrow_I 
      \lp \cfgnt{L}\ \cfgnt{R}\ \phi_g\ \cfgnt{r}\ \cfgnt{f}\rp 
	}
\end{mathpar}
\end{center}
\caption{The initialization machine, $s ::= \lp\cfgnt{L}\ \cfgnt{R}\ \phi_g\ \cfgnt{r}\ \cfgnt{f}\rp$, with $s \rightarrow_I^* s^\prime$ indicating stepping the machine until the state does not change.}
\label{fig:lazyInit}
\end{figure*}

\begin{figure*}[t]
\begin{center}
\mprset{flushleft}
\begin{mathpar}
	\inferrule[Field Access]{
      \{\lp\phi\ l\rp\} = \cfgnt{L}\lp\cfgnt{r}\rp\\
      l \neq \cfgnt{l}_\mathit{null}\\
      \cfgnt{C} = \mathrm{type}\lp\cfgnt{l},\cfgnt{f}\rp\\\\
      \lp \cfgnt{L}\ \cfgnt{R}\ \cfgnt{r}\ \cfgnt{f}\ \cfgnt{C}\rp \rinit^*
      \lp \cfgnt{L}^\prime\ \cfgnt{R}^\prime\ \cfgnt{r}\ \cfgnt{f}\  \cfgnt{C}\rp \\\\ 
      \{\lp\phi^\prime\ l^\prime\rp\} = \cfgnt{L}^\prime\lp\cfgnt{R}^\prime\lp l,\cfgnt{f}\rp\rp \\
      \cfgnt{r}^\prime = \mathrm{stack}_r\lp\rp \\
    }{
      \lp \cfgnt{L}\ \cfgnt{R}\ \phi_g\ \eta\ \cfgnt{r}\ \lp \cfgt{*}\ \cfgt{\$}\ \cfgnt{f} \rightarrow \cfgnt{k}\rp \rp  \rightarrow_\ell \\\\
      \lp \cfgnt{L}^\prime[\cfgnt{r}^\prime \mapsto \lp\phi^\prime\ l^\prime\rp]\ \cfgnt{R}^\prime\ \phi_g^\prime\ \eta\ \cfgnt{r}^\prime\ \cfgnt{k}\rp 
	}
\and
	\inferrule[Field Access (NULL)]{
      l = \cfgnt{l}_\mathit{null}
    }{
      \lp \cfgnt{L}\ \cfgnt{R}\ \phi_g\ \eta\ \cfgnt{r}\ \lp \cfgt{*}\ \cfgt{\$}\ \cfgnt{f} \rightarrow \cfgnt{k}\rp \rp  \rightarrow_\ell \\\\
      \lp \cfgnt{L}\ \cfgnt{R}\ \phi_g\ \eta\ \cfgt{error}\ \cfgt{end}\rp 
	}
\end{mathpar}
\end{center}
\caption{GSE with lazy initialization indicated by $\rgse = \rightarrow_\ell \cup \rcom$.}
\label{fig:lazy}
\end{figure*}

