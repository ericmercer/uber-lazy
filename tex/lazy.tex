\section{Generalized Symbolic Execution}

Recall that in generalized symbolic execution (GSE) for
lazy initialization of symbolic references consists of three key
points of non-determinism where each symbolic reference can be
initialized non-deterministically to null, a new instance of the
symbolic reference, or aliases to symbolic references of the same type
previously initialized. The initialization in GSE consists of creating
branches in the execution tree for all the non-deterministic
choices. 

\begin{figure*}[t]
\begin{center}
\mprset{flushleft}
\begin{mathpar}
	\inferrule[Initialize (null)]{
	  \Lambda = \{ l \mid \exists \phi\ \lp \lp \phi\ l\rp  \in \cfgnt{L}\lp \cfgnt{r}\rp  \wedge  \cfgnt{R}\lp l,\cfgnt{f}\rp  = \bot\ \rp\}\\
      \Lambda \neq \emptyset\\\\
      \cfgnt{r}^\prime = \mathrm{fresh}_r\lp \rp\\ 
      \theta_\mathit{null} = \{ \lp \phi_T\ l_\mathit{null}\rp \} \\
      l_x = \mathrm{min}_l\lp \Lambda\rp \\\\
      \phi_g^\prime = \lp\phi_g \wedge \cfgnt{r}^\prime = \cfgnt{r}_\mathit{null}\rp
    }{
      \lp \cfgnt{L}\ \cfgnt{R}\ \phi_g\ \cfgnt{r}\ \cfgnt{f}\rp  \rightarrow_I 
      \lp \cfgnt{L}[\cfgnt{r}^\prime \mapsto \theta_\mathit{null}]\ \cfgnt{R}[ \lp l_x,\cfgnt{f}\rp  \mapsto \cfgnt{r}^\prime]\ \phi_g^\prime\ \cfgnt{r}\ \cfgnt{f}\rp 
	}
\and
	\inferrule[Initialize (new)]{
	  \Lambda  = \{ l \mid \exists \phi\ \lp \lp \phi\ l\rp  \in \cfgnt{L}\lp \cfgnt{r}\rp  \wedge  \cfgnt{R}\lp l,\cfgnt{f}\rp  = \bot\rp\}\\
      \Lambda \neq \emptyset\\\\
      \mathrm{C} = \mathrm{type}\lp \cfgnt{f}\rp\\
      \cfgnt{r}_f = \mathrm{init}_r\lp \rp\\
      l_f = \mathrm{init}_l\lp \mathrm{C}\rp \\\\
      \cfgnt{R}^\prime = \cfgnt{R}[\forall \cfgnt{f} \in \mathit{fields}\lp \mathrm{C}\rp \ \lp \lp l_f\ \cfgnt{f}\rp  \mapsto \bot \rp ] \\\\
      \rho = \{ \lp\cfgnt{r}_a\ l_a\rp \mid \mathrm{isInit}\lp \cfgnt{r}_a\rp  \wedge \mathrm{type}\lp l_a\rp  = \mathrm{C} \wedge \exists \phi_a\ \lp \lp \phi_a\ l_a\rp  \in \cfgnt{L}\lp \cfgnt{r}_a\rp\rp \}\\\\
      \theta_\mathit{new} = \{\lp \phi_T\ l_f\rp \} \\
      l_x = \mathrm{min}_l\lp \Lambda\rp \\\\
      \phi_g^\prime = \lp\phi_g \wedge \cfgnt{r}_f \neq \cfgnt{r}_\mathit{null} \wedge \lp \wedge_{\lp\cfgnt{r}_a\ l_a\rp \in \rho} \cfgnt{r}_f \ne \cfgnt{r}_a\rp\rp
    }{
      \lp \cfgnt{L}\ \cfgnt{R}\ \phi_g\ \cfgnt{r}\ \cfgnt{f}\rp  \rightarrow_I 
      \lp \cfgnt{L}[\cfgnt{r}_f \mapsto \theta_\mathit{new}]\ \cfgnt{R}^\prime[ \lp l_x,\cfgnt{f}\rp  \mapsto \cfgnt{r}_f ]\ \phi_g^\prime\ \cfgnt{r}\ \cfgnt{f}\rp 
	}
\and
	\inferrule[Initialize (alias)]{
	  \Lambda = \{ l \mid \exists \phi\ \lp \lp \phi\ l\rp  \in \cfgnt{L}\lp \cfgnt{r}\rp  \wedge  \cfgnt{R}\lp l,\cfgnt{f}\rp  = \bot\ \rp\}\\
      \Lambda \neq \emptyset\\\\
      \mathrm{C} = \mathrm{type}\lp \cfgnt{f}\rp\\
      \cfgnt{r}^\prime = \mathrm{fresh}_r\lp \rp\\\\
      \rho = \{ \lp\cfgnt{r}_a\ l_a\rp \mid \mathrm{isInit}\lp \cfgnt{r}_a\rp  \wedge \mathrm{type}\lp l_a\rp  = \mathrm{C} \wedge \exists \phi_a\ \lp \lp \phi_a\ l_a\rp  \in \cfgnt{L}\lp \cfgnt{r}_a\rp\rp \}\\\\
      \lp\cfgnt{r}_a\ l_a\rp \in \rho \\
      \theta_\mathit{alias} = \{ \lp \phi_T\ l_a\rp\}\\
      l_x = \mathrm{min}_l\lp \Lambda\rp\\\\
      \phi^\prime_g = \lp\phi_g \wedge \cfgnt{r}^\prime \neq \cfgnt{r}_\mathit{null} \wedge \cfgnt{r}^\prime = \cfgnt{r}_a \wedge \lp \wedge_{\lp \cfgnt{r}^{\prime}_a\ l_a\rp  \in \rho\ \lp \cfgnt{r}^{\prime}_a \neq \cfgnt{r}_a\rp } \cfgnt{r}^\prime \neq \cfgnt{r}^{\prime}_a \rp\rp
    }{
      \lp \cfgnt{L}\ \cfgnt{R}\ \phi_g\ \cfgnt{r}\ \cfgnt{f}\rp  \rightarrow_I 
      \lp \cfgnt{L}[\cfgnt{r}^\prime \mapsto \theta_\mathit{alias}]\ \cfgnt{R}[ \lp l_x,\cfgnt{f}\rp  \mapsto \cfgnt{r}^\prime ]\ \phi_g^\prime\ \cfgnt{r}\ \cfgnt{f}\rp 
	}
\and
	\inferrule[Initialize (end)]{
	  \Lambda = \{ l \mid \exists \phi\ \lp \lp \phi\ l\rp  \in \cfgnt{L}\lp \cfgnt{r}\rp  \wedge  \cfgnt{R}\lp l,\cfgnt{f}\rp  = \bot\ \rp\}\\
      \Lambda = \emptyset
    }{
      \lp \cfgnt{L}\ \cfgnt{R}\ \phi_g\ \cfgnt{r}\ \cfgnt{f}\rp  \rightarrow_I 
      \lp \cfgnt{L}\ \cfgnt{R}\ \phi_g\ \cfgnt{r}\ \cfgnt{f}\rp 
	}
\end{mathpar}
\end{center}
\caption{The initialization machine, $s ::= \lp\cfgnt{L}\ \cfgnt{R}\ \phi_g\ \cfgnt{r}\ \cfgnt{f}\rp$, with $s \rightarrow_I^* s^\prime$ indicating stepping the machine until the state does not change.}
\label{fig:lazyInit}
\end{figure*}

\begin{figure*}[t]
\begin{center}
\mprset{flushleft}
\begin{mathpar}
	\inferrule[Field Access]{
      \{\lp\phi\ l\rp\} = \cfgnt{L}\lp\cfgnt{r}\rp\\
      l \neq \cfgnt{l}_\mathit{null}\\
      \cfgnt{C} = \mathrm{type}\lp\cfgnt{l},\cfgnt{f}\rp\\\\
      \lp \cfgnt{L}\ \cfgnt{R}\ \cfgnt{r}\ \cfgnt{f}\ \cfgnt{C}\rp \rinit^*
      \lp \cfgnt{L}^\prime\ \cfgnt{R}^\prime\ \cfgnt{r}\ \cfgnt{f}\  \cfgnt{C}\rp \\\\ 
      \{\lp\phi^\prime\ l^\prime\rp\} = \cfgnt{L}^\prime\lp\cfgnt{R}^\prime\lp l,\cfgnt{f}\rp\rp \\
      \cfgnt{r}^\prime = \mathrm{stack}_r\lp\rp \\
    }{
      \lp \cfgnt{L}\ \cfgnt{R}\ \phi_g\ \eta\ \cfgnt{r}\ \lp \cfgt{*}\ \cfgt{\$}\ \cfgnt{f} \rightarrow \cfgnt{k}\rp \rp  \rightarrow_\ell \\\\
      \lp \cfgnt{L}^\prime[\cfgnt{r}^\prime \mapsto \lp\phi^\prime\ l^\prime\rp]\ \cfgnt{R}^\prime\ \phi_g^\prime\ \eta\ \cfgnt{r}^\prime\ \cfgnt{k}\rp 
	}
\and
	\inferrule[Field Access (NULL)]{
      l = \cfgnt{l}_\mathit{null}
    }{
      \lp \cfgnt{L}\ \cfgnt{R}\ \phi_g\ \eta\ \cfgnt{r}\ \lp \cfgt{*}\ \cfgt{\$}\ \cfgnt{f} \rightarrow \cfgnt{k}\rp \rp  \rightarrow_\ell \\\\
      \lp \cfgnt{L}\ \cfgnt{R}\ \phi_g\ \eta\ \cfgt{error}\ \cfgt{end}\rp 
	}
\end{mathpar}
\end{center}
\caption{GSE with lazy initialization indicated by $\rgse = \rightarrow_\ell \cup \rcom$.}
\label{fig:lazy}
\end{figure*}


\begin{figure*}[t]
\begin{center}
\mprset{flushleft}
\begin{mathpar}
    \inferrule[Equals (reference-true)]{
    \cfgnt{L}\lp \cfgnt{r}_0\rp = \cfgnt{L}\lp \cfgnt{r}_1\rp\\
    \phi^\prime = \lp\phi \wedge r_0 = r_1\rp
    }{
    \lp \cfgnt{L}\ \cfgnt{R}\ \phi\ \eta\ \cfgnt{r}_0\ \lp \cfgnt{r}_1\ \cfgt{=}\ \cfgt{*} \rightarrow \cfgnt{k}\rp \rp  \rightarrow_\ell \\\\
    \lp \cfgnt{L}\ \cfgnt{R}\ \phi^\prime\ \eta\ \cfgt{true}\ \cfgnt{k}\rp 
    }
\and
    \inferrule[Equals (reference-false)]{
    \cfgnt{L}\lp \cfgnt{r}_0\rp \neq \cfgnt{L}\lp \cfgnt{r}_1\rp\\
    \phi^\prime = \lp\phi \wedge r_0 \neq r_1\rp
   }{
    \lp \cfgnt{L}\ \cfgnt{R}\ \phi\ \eta\ \cfgnt{r}_0\ \lp \cfgnt{r}_1\ \cfgt{=}\ \cfgt{*} \rightarrow \cfgnt{k}\rp \rp  \rightarrow_\ell \\\\
    \lp \cfgnt{L}\ \cfgnt{R}\ \phi^\prime\ \eta\ \cfgt{false}\ \cfgnt{k}\rp 
    }
\end{mathpar}
\end{center}
\caption{GSE with lazy initialization indicated by $\rgse = \rightarrow_\ell \cup \rcom$.}
\label{fig:lazyeq}
\end{figure*}


\section{Generating Symbolic Heap Summaries}

\subsection{Initialization of Symbolic References}

In this section we present the Javalite rewrite rules for the summary
initialization of symbolic references. The initialization rules are
defined on the bi-partite graph consisting of references and
locations. In contrast, the heap summarization approach creates a
single branch that contains the summarization for all the
initialization in a single bi-partitate graph.


\newsavebox{\boxPi}
\savebox{\boxPi}{
%\begin{center}
\mprset{flushleft}
\begin{mathpar}
	\inferrule[Summarize]{
	\Lambda = \mathbb{UN}\lp \cfgnt{L}, \cfgnt{R}, \cfgnt{r}, \cfgnt{f}\rp \\
      \Lambda \neq \emptyset \\
      \lp\phi_x\ \cfgnt{l}_x\rp = \mathrm{min}_l\lp \Lambda\rp\\
      \cfgnt{r}_f = \mathrm{init}_r\lp \rp \\
      l_f  = \mathrm{fresh}_l\lp \mathrm{C}\rp\\\\
      \rho = \{ \lp \cfgnt{r}_a\ l_a\rp  \mid \mathrm{isInit}\lp \cfgnt{r}_a\rp  \wedge\cfgnt{r}_a = \mathrm{min}_r\lp \cfgnt{R}^{\leftarrow}[l_a]\rp \wedge \mathrm{type}\lp l_a\rp  = \mathrm{C} \} \\\\
      \theta_\mathit{null} = \{ \lp \phi\ l_\mathit{null}\rp  \mid \phi = \lp \phi_x \wedge \cfgnt{r}_f = \cfgnt{r}_\mathit{null} \rp  \} \\\\
      \theta_\mathit{new} = \{\lp \phi\ l_f\rp  \mid \phi = \lp \phi_x \wedge \cfgnt{r}_f \neq \cfgnt{r}_\mathit{null} \wedge \lp \wedge_{\lp \cfgnt{r}^\prime_a\ l^\prime_a\rp  \in \rho} \cfgnt{r}_f \ne \cfgnt{r}^\prime_a\rp \rp \}\\\\
      \theta_\mathit{alias} = \{ \lp \phi\ l_a\rp  \mid \exists\cfgnt{r}_a\ \lp\lp\cfgnt{r}_a\ l_a\rp  \in \rho \wedge \phi = \lp \phi_x \wedge \cfgnt{r}_f \neq \cfgnt{r}_\mathit{null} \wedge \cfgnt{r}_f = \cfgnt{r}_a \wedge \lp \wedge_{\lp \cfgnt{r}^{\prime}_a\ l^{\prime}_a\rp  \in \rho\ \lp \cfgnt{r}^\prime_a < \cfgnt{r}_a\rp } \cfgnt{r}_f \neq \cfgnt{r}^{\prime}_a \rp \rp \rp \} \\\\
      \theta_\mathit{orig} = \{\lp\phi\ \cfgnt{l}_\mathit{orig}\rp \mid \exists \phi_\mathit{orig} \lp \lp\phi_\mathit{orig}\ \cfgnt{l}_\mathit{orig}\rp \in \cfgnt{L}\lp\cfgnt{R}\lp\cfgnt{l}_x,\cfgnt{f}\rp\rp \wedge \phi = \lp\neg\phi_x \wedge \phi_\mathit{orig}\rp\}\\\\ 
      \theta = \theta_\mathit{null} \cup \theta_\mathit{new} \cup \theta_\mathit{alias} \cup \theta_\mathit{orig} \\
\cfgnt{R}^\prime = \cfgnt{R}[\forall \cfgnt{f} \in \mathit{fields}\lp \mathrm{C}\rp \ \lp \lp l_f\ \cfgnt{f}\rp  \mapsto \cfgnt{r}_\mathit{un} \rp ]
    }{
      \lp \cfgnt{L}\ \cfgnt{R}\ \cfgnt{r}\ \cfgnt{f}\ \cfgnt{C}\rp \rsum 
      \lp \cfgnt{L}[\cfgnt{r}_f \mapsto \theta]\ \cfgnt{R}^{\prime}[ \lp l_x,\cfgnt{f}\rp  \mapsto \cfgnt{r}_f ]\ \cfgnt{r}\ \cfgnt{f}\ \cfgnt{C}\rp
	}
\and
	\inferrule[Summarize-end]{
	  \Lambda = \mathbb{UN}\lp \cfgnt{L}, \cfgnt{R}, \cfgnt{r}, \cfgnt{f}\rp \\
      \Lambda = \emptyset
    }{
      \lp \cfgnt{L}\ \cfgnt{R}\ \cfgnt{r}\ \cfgnt{f}\ \cfgnt{C}\rp  \rsum
      \lp \cfgnt{L}\ \cfgnt{R}\ \cfgnt{r}\ \cfgnt{f}\ \cfgnt{C}\rp 
	}
\end{mathpar}}
%\end{center}
%\caption{The summary machine, $s ::= \lp\cfgnt{L}\ \cfgnt{R}\ \cfgnt{r}\ \cfgnt{f}\ \cfgnt{C}\rp$, with $s\rsum^*s^\prime =  s \rsum \cdots \rsum s^\prime \rsum s^\prime$.}
%\label{fig:symInit}
%\end{figure*}


The initialization rules are invoked when an uninitialized field in a
symbolic reference is accessed. The function $\mathbb{UN}(\cfgnt{L},
\cfgnt{R}, \cfgnt{r}, \cfgnt{f}) = \{\cfgnt{l}\ ...\}$ returns
constraint-location pairs in which the field $\cfgnt{f}$ is
uninitialized:
\[
\begin{array}{rcl}
\mathbb{UN}(\cfgnt{L}, \cfgnt{R}, \cfgnt{r}, \cfgnt{f}) & = &\{ \lp\phi\ \cfgnt{l}\rp \mid \lp \phi\ \cfgnt{l}\rp  \in \cfgnt{L}\lp \cfgnt{r}\rp  \wedge \\
& & \ \ \ \ \exists \phi^\prime \lp \lp \phi^\prime\ \cfgnt{l}_\mathit{un}\rp  \in \cfgnt{L}\lp \cfgnt{R}\lp l,\cfgnt{f}\rp\rp \wedge \\
& & \ \ \ \ \ \ \ \ \mathbb{S}\lp \phi \wedge \phi^\prime \rp\rp\}\\
\end{array}
\]
where $\mathbb{S}(\phi)$ returns true if $\phi$ is
satisfiable. Intutively, for the reference, $\cfgnt{r}$, it constructs
the set, $\theta$, that contains all contraint-location pairs that
point to the field $\cfgnt{f}$ and $\cfgnt{f}$ points to
$\cfgnt{l}_\mathit{un}$.

% The cardinality of the set, $\theta$ is never
%greater than one in GSE and the constraint is always satisfiable
%because all constraints are constant. This property is relaxed in GSE
%with heap summaries.

The rewrite rule to initialize a heap in the symbolic summarization
technique is shown in~\figref{fig:symInit}. Given the unitialized set
$\mathbb{UN}$ for field $f$, the $\mathit{min}_l$ function returns the
constraint location pair based on a lexical ordering of locations. The
init$_r$ method returns a new input reference $r_f$ while the
fresh$_f$ returns a new location of type $C$. The set $\rho$ contains
the set of reference location, $(r_a, l_a)$, pairs that the current
field may alias. The pair $(r_a, l_a)$ is contained in the set $ \rho
$ if the following conditions hold: (i) $r_a$ is part of the input
references initialized in the summary heap, (ii) $r_a$ is in the
pre-image of $l_a$, (iii) and the type of $l_a$ is the same as the
type of the field being initialized.

There are three constraint-location pair sets that correspond to
non-deterministic choices in GSE: (a) $\theta_\mathit{null}$ that
creates a constraint that states $r_f$ is null under which it points
to $l_\mathit{null}$, (b) $\theta_\mathit{new}$ the constraint that
$r_f$ is not null and it does not alias another reference in set
$\rho$ under which it points to the newly created location $l_f$, (c)
and $\theta_\mathit{alias}$ where for each element in the set $(r_a,
l_a) \in \rho$ it creates a constraints that $r_f$ is not null, points
to $r_a$, and does not point to any reference contained in
$(r_a^\prime, l_a^\prime) \in \rho$ such that $r_a \neq
r_a^\prime$. Additionally $\theta_\mathit{orig}$ allows for the
possibility that the field continues to remain uninitialized since the
heap is symbolic. We refer to this initialization in the summary
machine as conditional initialization. Finally, the fields of the
reference $r_f$ are marked as uninitialized.

In the post condition of the rule the reference $r_f$ points to
$\theta$ which contains the union of all the constraint location pairs
sets, and the field $f$ points to the new reference $r_f$ in the
summary heap.  Note that the summarize rule invoked repeatedly until
the set of unitialized constraint location pairs for field $f$ is
empty. In other words the set $\lambda$ is empty. This consititutes
the summarize end rule in~\figref{fig:symInit}.

\begin{figure*}[t]
\begin{center}
\begin{tabular}[c]{c|c|c|c}
\begin{tabular}[c]{c}
\scalebox{0.81}{\input{origHeap.pdf_t}} \\
\end{tabular} &
\begin{tabular}[c]{c}
\scalebox{0.81}{\input{summarizeXHeap.pdf_t}} \\
\end{tabular} &
\begin{tabular}[c]{c}
\scalebox{0.81}{\input{summarizeYHeap.pdf_t}} \\
\end{tabular} &
\begin{tabular}[c]{l}
$\rho := \{ (r_1^i, r_1^i \neq r_\mathit{null}, l_1 \}$ \\
$\theta_\mathit{null} := \{ ( r_2^i = r_\mathit{null}, l_\mathit{null}) \}$\\
$\theta_\mathit{new} := \{ ( r_2^i \neq r_\mathit{null} \wedge r_2^i \neq r_1^i, l_2) \}$\\
$\theta_\mathit{alias} := \{ ( r_1^i \neq r_\mathit{null} \wedge r_2^i \neq r_\mathit{null} \wedge r_2^i = r_1^i, l_1) \}$\\
$\theta_\mathit{orig} := \{ \}$ \\
$\phi_{\mathit{1a}} := r_1^i = r_\mathit{null} $ \\
$\phi_{\mathit{1b}} := r_1^i \neq r_\mathit{null} $  \\
$\phi_{\mathit{2a}} := r_2^i = r_\mathit{null}$ \\
$\phi_{\mathit{2b}} := r_2^i \neq r_\mathit{null} \wedge r_2^i \neq r_1^i$ \\
$\phi_{\mathit{2c}} :=  r_2^i \neq r_\mathit{null} \wedge r_2^i = r_1^i $ \\
\end{tabular} \\
(a) & (b) & (c) & (d) \\
\end{tabular}
\end{center}
\caption{initialize this.x and this.y}
\label{fig:initHeap}
\end{figure*}

We visualize the initialization process in the symbolic summarization
technique in~\figref{fig:initHeap}. The heap
in~\figref{fig:initHeap}(a) represents the initial heap. The
references with the superscript $s$ indicates that it is a local
reference. In~\figref{fig:initHeap} $r_0^s$ represents the reference
for the $\mathit{this}$ instance which has two fields $x$ and $y$ of
the same type. The reference $r_0^s$ points to the location
$l_0$. Note that when no constraint is specified, then there is a
implicit $\mathit{true}$ constraint. In~\figref{fig:initHeap} $r_0^s$
points to $l_0$ on $\mathit{true}$. The fields $x$ and $y$ point to
the uninitialized reference $r_\mathit{un}$. The reference
$r_\mathit{un}$ points to the uninitialized location $l_\mathit{un}$
on the $\mathit{true}$ constraint.

The graph in~\figref{fig:initHeap}(b) represents the summary heap
after the initialization of the $\mathit{this}.x$ field while the
graph in~\figref{fig:initHeap}(c) represents the summary heap after
the initialization of the $\mathit{this}.y$ field following the
initialization of $\mathit{this}.x$. The list
in~\figref{fig:initHeap}(d) represents the various sets constructed in
the rewrite system when the initialization takes places. We also
define the constraints of references to their corresponding labels in
the graph.

The access on $l_0.x$ creates a new input reference $r_1^i$ in the
summary heap shown in~\figref{fig:initHeap}(b). The reference $r_1^1$
points to the $l_\mathit{null}$ location under the constraint that
$r_1^i$ is null: $\phi_{1a} := r_1^i = r_\mathit{null}$. The reference
$r_1^i$ points to the location $l_1$ under the constraint that $r_1^i$
is not $\mathit{null}$ $\phi_{1b} :=r_1^i \neq r_\mathit{null}$. The
location $l_1$ represents a fresh location of type $C$ is created on
the heap such that $C$ is type of $\mathit{this}.x$.

Following the initialization to the $\mathit{this}.x$, when we
initialize $\mathit{this}.y$ the resulting summary heap is shown
in~\figref{fig:initHeap}(c). The set $\rho$ shown
in~\figref{fig:initHeap}(d) contains the potential aliases for
$\mathit{this}.y$ which is location $l_1$ ($\mathit{this}.x$). The
constraint location $\theta$ sets represents the initialization
choices for $\mathit{this}.y$. The new input reference $r_2^i$ for
$this.y$ points to (a) location $l_\mathit{null}$ under the constraint
$\phi_{2a} := r_2^i = r_\mathit{null}$, (b) location $l_2$ under the
constraint $\phi_{2b} := r_2^i \neq r_\mathit{null} \wedge r_2^i \neq
r_1^i$ that states that $r_2^i$ is not null and does not alias
$r_1^i$, and (c) location $l_1$ under the constraint $\phi_{2c} :=
r_2^i \neq r_\mathit{null} \wedge r_2^i = r_1^i$ that states $r_2^i$
is not null and \emph{does} alias $r_1^i$.





