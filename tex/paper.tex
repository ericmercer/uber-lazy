%-----------------------------------------------------------------------------
%
%               Template for sigplanconf LaTeX Class
%
% Name:         sigplanconf-template.tex
%
% Purpose:      A template for sigplanconf.cls, which is a LaTeX 2e class
%               file for SIGPLAN conference proceedings.
%
% Guide:        Refer to "Author's Guide to the ACM SIGPLAN Class,"
%               sigplanconf-guide.pdf
%
% Author:       Paul C. Anagnostopoulos
%               Windfall Software
%               978 371-2316
%               paul@windfall.com
%
% Created:      15 February 2005
%
%-----------------------------------------------------------------------------


\documentclass{sigplanconf}

% The following \documentclass options may be useful:

% preprint      Remove this option only once the paper is in final form.
% 10pt          To set in 10-point type instead of 9-point.
% 11pt          To set in 11-point type instead of 9-point.
% authoryear    To obtain author/year citation style instead of numeric.

\usepackage{amsmath}
\usepackage{comment}
\usepackage{mathpartir}
\usepackage{amssymb}
\usepackage{amsfonts}

\hyphenation{op-tical net-works semi-conduc-tor}

%% Commands For the syntax EBNF
\newcommand{\cfgnt}[1]{\emph{#1}}
\newcommand{\cfgq}[1]{\texttt{#1}}
\newcommand{\cfgt}[1]{\textbf{#1}}
\newcommand{\cfglhs}[1]{\cfgnt{#1} & $::=$}
\newcommand{\cfgrule}[2]{\cfglhs{#1} & #2 \\}
\newcommand{\cfgor}{\textbar\ }
\newcommand{\cfgstart}{\begin{tabular}{r@{\hspace{1mm}}r@{\hspace{2mm}}l}}
\newcommand{\cfgend}{\end{tabular}}
\newcommand{\cfgline}[1]{ ~ && #1 \\ }
\newcommand{\cfglinetab}[1]{ ~ && \hspace{1cm} #1 \\ }
\newcommand{\cfgorline}[1]{ ~ & \cfgor & #1 \\ }
\newcommand{\lp}{\cfgq{(}}
\newcommand{\lb}{\cfgq{[}}
\newcommand{\rp}{\cfgq{)}}
\newcommand{\rb}{\cfgq{]}}
\newcommand{\hsp}[0]{\hspace{1mm}}


\newcommand{\figref}[1]{Figure~\ref{#1}}
\newcommand{\secref}[1]{Section~\ref{#1}}

\usepackage{url}
\usepackage{listings}
\usepackage{color}
\usepackage[T1]{fontenc}

\definecolor{dkgreen}{rgb}{0,0.6,0}
\definecolor{gray}{rgb}{0.5,0.5,0.5}
\definecolor{mauve}{rgb}{0.58,0,0.82}

\lstset{frame=tb,
  language=Java,
  aboveskip=3mm,
  belowskip=3mm,
  showstringspaces=false,
  columns=flexible,
  basicstyle={\small\ttfamily},
  numbers=none,
  numberstyle=\tiny\color{gray},
  keywordstyle=\color{blue},
  commentstyle=\color{dkgreen},
  stringstyle=\color{mauve},
  breaklines=true,
  breakatwhitespace=true
  tabsize=3
}


\usepackage{graphicx}
\usepackage{graphics}
\usepackage{epsfig}
\usepackage{comment}

% used for inline lists
\usepackage{paralist}

%% This enables the xfig overlays to use the same font family as the document
%% (i.e., font family and size is the same in the figure as it is
%% in the text).

\gdef\SetFigFont#1#2#3#4#5{}


\usepackage{algorithm} 
\RequirePackage[noend]{algorithmic}
\renewenvironment{algorithm}[1][\textwidth]%  
{\begin{minipage}[t][\totalheight][c]{#1}\begin{algorithmic}[1]}  %%% change [1] to [0] to turn off line numbers
{\end{algorithmic}\end{minipage}}

%% Use for each in ``FOR'' constructs

\renewcommand{\algorithmicfor}{\textbf{for each}}

%% All comments are in italics

\renewcommand{\algorithmiccomment}[1]{\textit{${/\ast}$~#1~${\ast/}$}}

%% Use ``procedure'' instead of ``Algorithm'' for off set

\renewcommand{\algorithmicensure}{\textbf{procedure}}

\newcommand{\algoname}[1]{\ENSURE #1}
\newcommand*{\algobox}[1]{\framebox{#1}}

\newcommand{\bo}[1]{\textbf{#1}}
\newcommand{\cc}[1]{\cellcolor[gray]{.6}{#1}}
\newcommand{\negspace}{\hspace{-.40cm}}

%define theorem environments
\usepackage{amsthm}
\newtheorem{theorem}{Theorem}
\newtheorem{lemma}[theorem]{Lemma}
\newtheorem{proposition}[theorem]{Proposition}
\newtheorem{corollary}[theorem]{Corollary}

\newtheorem{definition}{Definition}
%\newenvironment{definition}[1][Definition]{\begin{trivlist}
%\item[\hskip \labelsep {\bfseries #1}]}{\end{trivlist}}

%% This enables the xfig overlays to use the same font family as the document
%% (i.e., font family and size is the same in the figure as it is
%% in the text).

\gdef\SetFigFont#1#2#3#4#5{}

\begin{document}

\special{papersize=8.5in,11in}
\setlength{\pdfpageheight}{\paperheight}
\setlength{\pdfpagewidth}{\paperwidth}

\conferenceinfo{CONF 'yy}{Month d--d, 20yy, City, ST, Country} 
\copyrightyear{20yy} 
\copyrightdata{978-1-nnnn-nnnn-n/yy/mm} 
\doi{nnnnnnn.nnnnnnn}

% Uncomment one of the following two, if you are not going for the 
% traditional copyright transfer agreement.

%\exclusivelicense                % ACM gets exclusive license to publish, 
                                  % you retain copyright

%\permissiontopublish             % ACM gets nonexclusive license to publish
                                  % (paid open-access papers, 
                                  % short abstracts)

%\titlebanner{banner above paper title}        % These are ignored unless
%\preprintfooter{short description of paper}   % 'preprint' option specified.

%\title{Contraints-based Reasoning of Heaps in Symbolic Execution}

%\title{Using Constraints to Characterize Heaps in Symbolic Execution}

%\title{Symbolic Execution with precise heap constraints}

%\title{Precise Heap Summaries from Symbolic Execution}

\title{Exact Heap Summaries from Symbolic Execution}

%\subtitle{Subtitle Text, if any}

\authorinfo{Anonymous}
          {}
    %       {Email1}
%\authorinfo{Name2\and Name3}
 %         {Affiliation2/3}
  %       {Email2/3}

\maketitle

\begin{abstract}
A recent trend in the analysis of object-oriented programs is the
modeling of references as sets of guarded values, enabling multiple
heap shapes to be represented in a single state. A fundamental problem
with using these guarded value sets is the creation of test inputs for
programs accepting symbolic reference input parameters. Although
several solutions have been proposed, none have been proven to be
sound and complete with respect to the properties provable by
generalized symbolic execution (GSE). This work presents a method for
initializing reference inputs in a symbolic input heap using guarded
value sets that exactly preserves GSE semantics. A correctness proof
for the initialization scheme is provided, as well as the results of
an empirical evaluation of a proof-of-concept implementation. The
initialization technique can be used to ensure that guarded value set
based symbolic execution engines operate in a provably correct manner
with regards to symbolic references.


%A fundamental challenge of using symbolic execution for software analysis is the treatment of dynamically allocated data. Existing techniques either underapproximate the space of possible inputs or are computationally infeasible. For example, dynamic symbolic execution (DSE) handles symbolic dereferencing by substituting in a value from a valid concrete execution. Generalized symbolic execution (GSE) initiates a new search path for every possible aliasing configuration. This paper introduces a method for de-referencing and manipulating values in a true block-box symbolic input heap that overcomes the limitations of previous methods. The symbolic heap supports arbitrary recursive data structures and captures all possible heaps that follow a common control flow path. Computation of complex preconditions and postconditions is supported, as well as automatic generation of test inputs. An evaluation of a proof-of-concept implementation in the Java Pathfinder framework is presented to demonstrate the computational feasibility of the approach over several classical symbolic execution benchmarks.


\end{abstract}

\category{CR-number}{subcategory}{third-level}

% general terms are not compulsory anymore, 
% you may leave them out
\terms
term1, term2

\keywords
keyword1, keyword2

\section{Introduction}

% SymExe is cool because for reasons x,y, and z

In recent years symbolic execution has provided the basis for various
software testing and analysis techniques. Symbolic execution
systematically explores the program execution space using symbolic
input values, and for each explored path, computes constraints on the
symbolic inputs to create a \emph{ path condition}.  The path
conditions computed by symbolic execution characterize the observed
program execution behaviors and have been used as an enabling
technology for various applications, e.g., regression
analysis~\cite{backes:2012,Godefroid:SAS11,Person:FSE08,person:pldi2011,Ramos:2011,Yang:ISSTA12},
data structure repair~\cite{KhurshidETAL05RepairingStructurally},
dynamic discovery of
invariants~\cite{CsallnerETAL08DySy,Zhang:ISSTA14}, and
debugging~\cite{Ma:2011}.

%
A major reason that path conditions are so useful is that each one represents exactly the set of concrete program inputs that would result in a given execution path. For any given program, a concrete execution will follow the same path as a symbolic execution if and only if the concrete inputs satisfy the path condition. Furthermore, because path conditions are logical predicates, they allow precise reasoning over potentially unbounded sets of program inputs. 

Symbolic execution's reliance on path conditions is both it's greatest strength, and a significant challenge. Since the path condition must be encoded in the form of an SMT problem, symbolic execution is limited by the capabilities of the underlying constraint solver. If the solver does not include a theory for reasoning about a given program operation, then the symbolic execution cannot proceed. Thus, extending the capabilities of symbolic execution to reason about new theories is an area of active research. 

One area of research involves symbolic reasoning for referencing operations. Of primary interest is dereferencing symbolic input references. There have been many proposed solutions for reasoning about symbolic input references, however to date none of them has been able to do so in a manner that preserves symbolic execution's most desirable property, namely the ability to produce a path condition that exactly represents all possible inputs for a given execution path.

% A couple of symExe�s big problems are path explosion and references. 

%Two of the main challenges facing current symbolic execution techniques 
%are path explosion and programs accepting references as inputs~\cite{Qu:2011,Chen:2013}.
% The path explosion problem stems from the way that 
%symbolic execution explores a program execution on a per-path basis. For 
%those programs for which there is an exponential number of possible 
%program paths, symbolic execution can be extremely inefficient. 
%References are a problem because symbolic execution requires that the 
%program state be represented by predicates formulated in terms of the 
%program inputs. Formulating such predicates for referencing operations 
%over potentially unbounded inputs has until now remained an open 
%question. 


% You can solve dereferencing by concretization, but that�s incomplete. 
%Here is an example of why this stinks.
One possible way to cope with the reference problem is to use a technique 
known as dynamic symbolic execution (DSE). When a DSE tool encounters 
an operation that is beyond the capabilities of its associated constraint 
solvers, such as dereferencing a symbolic reference, it substitutes a valid 
concrete result in place of the symbolic expression. However, this process 
of concretization introduces approximation into the formerly exact path condition,
and may even cause the analysis to miss valid program behaviors. 

%\begin{figure}
%void example( container a,b,c,d,x,y){
%a.f = x;
%b.f = c;
%c.f = d;
%d.f = a;
%if(x.f == y.f.f)
%     abort();\\error!
%}
%\caption{completeness example}
%\label{fig:DSEtest}
%\end{figure}
%
%Consider the example in \ref{fig:DSEtest}. During DSE, both symbolic and 
%concrete executions occur in parallel. Suppose that for the first pass, the 
%concrete execution picks the following points-to relationships: $a\rightarrow 
%loc_a, b\rightarrow loc_b, c\rightarrow loc_c, d\rightarrow loc_d ,x
%\rightarrow loc_a,  y\rightarrow loc_b$. When the program reaches the if() 
%statement, symbolic dereferencing of x and y are required. In order to 
%dereference symbol x to get x.f, x is concretized to point to $loc_a$, and 
%symbol x from field f is returned. Dereferencing y to find y.f requires 
%concretizing y to point to $loc_b$, From there, dereferencing b.f gives us 
%the reference d as the final value for y.f.f . Assuming the �false" branch is 
%executed first, the branch condition (x.f != y.f.f) reduces to x!=d, which 
%evaluates to �true" and DSE proceeds until program termination.
%After following the false branch, DSE will attempt to follow the true branch, 
%but the branch condition x==d will evaluate to false, and so DSE will 
%attempt a second pass starting from the beginning with a modified path 
%condition. For the second pass, DSE adds x==d to the path condition from 
%the beginning, to try to get the �true� branch. This time, the concrete 
%execution will choose x to point to $loc_d$. When evaluating the branch 
%condition, x.f will evaluate to a. Since the value for y is unchanged, the read 
%from y.f.f evaluates to d. The branch condition reduces to a==d, which 
%evaluates to false, so the �true� branch cannot be followed. In this instance, 
%DSE will fail to find a path to the error condition. 

% You can model references by forking the system state, but that makes 
%path explosion worse.
An alternative to concretization is to construct the input heap in a lazy manner,
deferring materialization of objects on the concrete heap until they
are needed for the analysis to proceed. The materialization creates
additional non-deterministic choice points in the symbolic execution
tree by representing the feasible heap configurations as (i) null,
(ii) an instance of a new reference of a compatible class, and (iii)
an alias to a previously initialized symbolic reference.  Symbolic
execution then follows concrete program semantics for materialized
heap locations. Although this approach enables the analysis of heap
manipulating programs, a large number of feasible concrete heap
configurations are created. Since each configuration requires a separate
path, GSE induces a path explosion problem while at the 
same time splitting the symbolic input space and decreasing the 
utility of the path condition.

% You can solve path explosion by bounding the execution and rolling 
%everything into one big equation, but then you�re incomplete. Here is an 
%example why that stinks.
%-maybe point out that finding the right bounds is a hard problem
One solution to the path explosion problem is to model the input heap as a 
predicate over a bounded set of locations. This allows the analysis to 
maintain a single representation for a set of possible heap configurations 
on each execution path, without case splitting for dereferencing operations. 
However, by placing an arbitrary bounds on the size of the input heap, neither
the path condition, nor the analysis performed by these techniques is complete.

% This paper introduces a technique that is, to the best knowledge of the 
%authors, the first sound and complete heap for SymEXE
This paper introduces a technique for modeling references and their 
associated operations that is both sound and complete\footnote{with 
respect to the properties provable by symbolic execution}, and models all 
possible heap configurations along a given execution path. To the 
knowledge of the authors, it is the first technique to do so. 

THIS NEXT BIT NEEDS SOME WORK, STILL:

% Creating this technique required the following key insights:
These advances were enabled by the following key insights: 

First, that a complete analysis requires an unconstrained input heap. 
However, it is unclear from prior work how this might be accomplished.

Second, that GSE techniques needlessly split during dereferencing 
operations. If GSE was performing a concretization, as in DSE, then case 
splitting makes more sense. However, lazy initialization is not a 
concretization, and is in fact only a partitioning of the input space. 

Third, that we can combine lazy initialization with a compositional heap 
abstraction to form constraints on a potentially unbounded input heap.


% This paper makes the following contributions.


\noindent{This paper makes the following contributions:}

\begin{compactdesc}

\item\textbf{-} The first sound and complete system for reasoning symbolically about the 
set of input heaps along any valid program path. 

\item\textbf{-} A bisimulation proof establishing the soundness and 
completeness of the heap summary approach with respect to
properties provable by GSE.

\item\textbf{-} A proof-of-concept implementation and empirical study 
demonstrating the scalability of the summary heap approach
compared to other GSE approaches.

\end{compactdesc}


%Initial work on symbolic execution largely focused on checking
%properties of programs with primitive
%types~\cite{clarke76TSE,King:76}.  With the advent of object-oriented
%languages, recent work has generalized the core ideas of symbolic
%execution to enable analysis of programs containing complex data
%structures with unbounded domains, i.e., data stored on the
%heap~\cite{Kiasan06,Kiasan07,GSE03}.  These \emph{Generalized 
%Symbolic
%  Execution} (GSE) techniques construct the heap in a lazy manner,..

%The goal of this work is to mitigate the path explosion problem in GSE
%by grouping multiple heaps together and only partitioning the heaps at
%points of divergence in the control flow graph. Our inspiration is
%found in the domain of static analysis that uses sets of constraints
%over heap locations to encode multiple heaps in a single
%representation. These sets, sometimes known as \emph{value sets} or
%\emph{value summaries}, allow multiple heaps to be represented
%simultaneously with a higher degree of precision than afforded by
%traditional techniques for shape analysis. Some of these previous
%attempts, however, are unable to handle aliasing in heaps due to a
%recursive definition of objects~\cite{Xie:2005}, while others require
%a set of heaps as input and are unable to initialize heaps
%on-the-fly~\cite{Dillig:2011,Tillmann:2008}.  As is typical in static
%analysis, over approximation of the heap representations alleviates
%some of these limitations but often leads to infeasible heaps. Also,
%most static analysis techniques for heap updates require a rewrite of
%the constraint system making it prohibitive for use in the context of
%symbolic execution.

%In this work, to effectively represent multiple heaps simultaneously
%in the context of symbolic execution, we define a novel heap
%representation and an algorithm to initialize and update the heap
%on-the-fly. In essence, our summary heap approach supports aliasing,
%does not require constraint rewriting on heap updates, and reduces
%non-determinism in the search space during symbolic execution.
%
%We represent the heap as a bipartite graph consisting of references
%and locations. Each reference is able to point to multiple locations,
%where each edge is predicated on constraints over aliasing between
%references. Each location points to a single reference for a given
%field. The use of a bipartite graph affords two key advantages: (i) it
%allows for a non-recursive definition of objects which enables us to
%support aliasing, and (ii) it does not introduce auxiliary variables
%or require rewriting of non-local constraints during updates to the
%heap.
%
%The summary heap algorithm defines how the bipartite graph is updated
%during lazy initialization. Unlike GSE, however, the summary heap
%algorithm introduces non-determinism in the search only at points of
%divergence in the control flow graph. These points of non-determinism
%are at field accesses that lead to null-pointer exceptions and at
%comparisons of references. The former represents a divergence due to
%exceptional control flow while the latter is due to program structure.
%The combination of the heap representation and the summary heap
%algorithm enables us to improve the efficiency of symbolic execution
%of heap manipulating programs over state-of-the-art techniques.
%
%The summary heap algorithm is sound and complete with respect to
%properties that are provable using GSE. This proof is accomplished by
%showing the existence of a bisimulation between states in GSE and
%states in the summary heap algorithm.  A preliminary implementation in
%Java Pathfinder (JPF) shows that in general, the heap summary
%algorithm improves over other state-of-the-art techniques, and in some
%instances, the improvement is remarkable: two-orders of magnitude
%reduction in running time. More importantly though, the heap summary
%enables the initialization of larger more complex heaps than
%previously possible as shown in our results.


\begin{figure}
void example( container a,b,c,d,x,y){
a.f = x;
b.f = c;
c.f = d;
d.f = a;
if(x.f == y.f.f)
     abort();\\error!
}
\caption{completeness example}
\label{fig:DSEtest}
\end{figure}

\subsection{Properties of Reference Theories}

We posit that an ideal symbolic model should be sound, complete, and efficient. However, achieving these three goals simultaneously has proven difficult for symbolic reference analyses.

\begin{compactdesc}
\item[Sound]
During a sound program analysis, all reported program properties are in fact true for the given program. However, ensuring soundness when reasoning about references can be computationally expensive. In order to speed things up, some analysis techniques overapproximate the set of possible program inputs. While this can ease the computational burden, it creates the practical problem of sorting out the incorrect results, which is at least as hard as the original problem in the first place.

\item[Complete] 
A complete program analysis should support a turing-complete model of computing, and will evaluate all possible program behaviors. However, most proposed symbolic reference models are not complete. Incomplete theories may be classified in one of two ways. First, a theory may underapproximate the space of possible referencing relationships. Second, techniques can underapproximate the domain of all possible programs.

Techniques that underapproximate references only explore a fragment of the possible input space. This may cause the analysis to miss valid program behaviors. One example of a technique that underapproximates the input space is DSE. Consider the example in \ref{fig:DSEtest}. During DSE, both symbolic and concrete executions occur in parallel. Suppose that for the first pass, the concrete execution picks the following points-to relationships: $a\rightarrow loc_a, b\rightarrow loc_b, c\rightarrow loc_c, d\rightarrow loc_d ,x\rightarrow loc_a,  y\rightarrow loc_b$. When the program reaches the if() statement, symbolic dereferencing of x and y are required. In order to dereference symbol x to get x.f, x is concretized to point to $loc_a$, and symbol x from field f is returned. Dereferencing y to find y.f requires concretizing y to point to $loc_b$, From there, dereferencing b.f gives us the reference d as the final value for y.f.f . Assuming the �false" branch is executed first, the branch condition (x.f != y.f.f) reduces to x!=d, which evaluates to �true" and DSE proceeds until program termination.
After following the false branch, DSE will attempt to follow the true branch, but the branch condition x==d will evaluate to false, and so DSE will attempt a second pass starting from the beginning with a modified path condition. For the second pass, DSE adds x==d to the path condition from the beginning, to try to get the �true� branch. This time, the concrete execution will choose x to point to $loc_d$. When evaluating the branch condition, x.f will evaluate to a. Since the value for y is unchanged, the read from y.f.f evaluates to d. The branch condition reduces to a==d, which evaluates to false, so the �true� branch cannot be followed. In this instance, DSE will fail to find a path to the error condition. Other common underapproximations include disallowing heaps with aliasing~\cite{Xie:2005,Babic:2007} or recursive data structures~\cite{Dillig:2011}.

Other techniques are incomplete in the types of programs they can evaluate. For example, some methods only accept programs where the size of memory required is independent from the program inputs ~\cite{Barnett:2006,Elkarablieh:2009}. Other restrictions include excluding programs with unbounded loops/recursion~\cite{Clarke:2004,Torlak:2014}. Since the computing models supported by these techniques are not Turing-complete, they exclude a wide class of interesting and useful programs. 

\item[Efficient]
An efficient model of references should support analysis of sufficient complexity to represent real-world programs. Unfortunately, many reference theories are too computationally inefficient to be useful for problems of practical size. For example, when dereferencing an uninitialized symbolic reference, a number of methods generate separate execution paths for each possible aliasing relationship~\cite{GSE03,KiasanKunit,Cadar:2008}. By branching at each new dereferencing operation, the number of paths created by grows exponentially with the size of the input heap.

\end{compactdesc}

%
%\paragraph{Why would you want to do lazy initialization, instead of using pre-generated input heaps?}
%\begin{compactdesc}
%\item[Generate a minimal set of independent test inputs:] 
%If you generate a fixed set of 1000 data structures at the beginning,
%you don’t know which of those 1000 data structures will actually
%result in different execution paths. Since many inputs may result in
%identical execution paths, the “path-independence” of those inputs is
%uncertain. Consider the case of a linked list, where each node
%contains two references, one for a data object, and one for the next
%node in the list. Suppose you want to test a function that makes
%structural transformations on linked lists. Perhaps your function
%removes loops, counts the number of nodes, or pops the first node off
%the list. If you tried every possible linked list as an input to your
%function, you would find that many of your lists led to identical
%execution paths, simply because the “data” reference was never
%accessed. Without lazy initialization, you would have to know
%beforehand which parts of your data structure will get accessed, and
%which will not. With lazy initialization, we test only independent
%sets of inputs, without any prior knowledge of the data structure. In
%fact, we generate a minimal set of independent test inputs within the
%given bounds.
%
%\item[Make per-path search decisions:] 
%With lazy initialization, we can make search decisions dynamically
%based on the machine state. For example, suppose you want to limit the
%number of nodes that any given execution path might attempt to
%access. Note that this is very different than bounding the input heap
%to a set size. Without lazy initialization, you would have to know
%exactly which reference chains any given execution path will follow
%beforehand. However, with lazy initialization, it is simple to keep
%track of the number of nodes that have been accessed at run-time, and
%limit the size of the lazy-initialized input heap dynamically. Some
%examples of types of per-path bounding metrics which are possible
%using lazy initialization include: length of reference chains, number
%of cycles in the graph, number of nodes accessed by the program, the
%amount of memory available to the test machine, total number of memory
%reads/writes. (Note that some of these things may be difficult to
%relate exactly back to the concrete heaps)
%
%\item[Generate test inputs based on per-path execution properties:] 
%Per-path search bounds can be combined with the test-input generation
%capability to answer questions like “Which test inputs will generate
%fewer than 20 memory read operations?”.
%
%\item[Cope with certain types of unbounded inputs:] 
%In some cases, lazy initialization can exactly model program behavior
%even when the size of the input heap is unbounded. For example,
%suppose we have a function that pushes a node on to the front of a
%linked list. In this case, lazy initialization will correctly model
%all program behaviors for linked lists of arbitrary size.
%
%\item[Less work for the programmer:] 
%You don’t need to generate any special data invariants specific to the
%analysis. They are generated on-the-fly from invariants written in
%normal program code.
%
%\item[Delay choices about data structures:] 
%If you are exploring a program with 1000 possible input structures, an
%execution tree with 1000 leaves will have fewer states that a set of
%1000 separate paths. Even in the case where all paths are merged into
%a single path, the constraints on the earlier states will be simpler
%if you don’t have to distinguish all 1000 possible inputs.
%\end{compactdesc}
%
%\paragraph{There are lots of symbolic heap representations. Why hasn’t anyone done lazy initialization before?}
%\begin{compactitem}
%\item[Inductively Defined Symbolic Locations:] 
%Earlier techniques using similar representations (SATURN, the Dilligs)
%have been limited by inductively defined symbolic locations. Suppose
%you were to define symbolic locations as being either a method
%parameter or a dereferenced symbolic location. This works well in
%heaps without loops, because dereferenced symbolic location are
%guaranteed to be “smaller” (closer to the root of the heap) than their
%targets. However, this inductive definition breaks down in the case of
%loops. Some recent work uses with non-recursive data structures,
%(e.g. Koushik’s stuff), but they don’t do lazy initialization.
%
%\item[Flat memory models:] 
%Many prior techniques make no special treatment for object-oriented
%memory models (Boogie, PEX, etc.). Typically the memory is modeled as
%a flat address space instead of a more high-level heap. There are many
%types of interactions that are possible in this memory model, such as
%pointer arithmetic and array indexing, which are not possible in an
%object-oriented system. Accounting for these interactions make lazy
%initialization challenging to implement.
%
%\item[Path splitting during memory access:] 
%Several existing methods for doing lazy initialization during symbolic
%execution introduce nondeterminism during memory access operations
%(GSE, Lazier\#, etc.). These new points of nondeterminism contribute
%to path explosion, rendering these methods infeasible for all but the
%smallest heap fragments.
%\end{compactitem}
%
%\paragraph{What’s special about using a graph representation? Why not represent the heap as an equation?}
%\begin{compactdesc}
%\item[Fewer Solver Calls:] 
%With a graph-based representation, we can answer structural questions
%about the heap without consulting the solver. For example, when
%comparing two references, if the intersection of their value sets is
%empty, we can surmise that they are in all cases unequal. Some heap
%properties that may be resolvable from structural properties alone
%are: equivalence of references, connectivity of heap fragments, upper
%bounds on pointer chain length, upper bounds on number of objects in
%the heap, and the absence of loops / aliasing.
%
%\item[Non-inductive definition of references:] 
%A graph-based representation can provide a non-inductive definition of
%the heap. Equation-based systems for modeling memories tend to
%introduce lots of auxiliary variables, which can complicate
%dereferencing and may place restrictions on heap morphology.  
%\end{compactdesc}
%
%\paragraph{What’s special about the “bipartite graph” representation?}
%\begin{compactdesc}
%\item[Object-oriented heap:] 
%Our bipartite graph is a non-recursive model more suited to OO
%programming than one based on pointers. In many OO programming
%environments, you don’t have to worry about nasty things like pointer
%arithmetic, so a cleaner model reduces wasted effort. You might say
%the bipartite graph is how we’ve modeled a “theory of objects” for
%heap constraint problems.
%\end{compactdesc}

\section{Preliminaries}

\figref{fig:surface-syntax} defines the surface syntax for the
Javalite language \cite{saints-MS}. \figref{fig:machine-syntax} is the
machine syntax. Javalite is syntactic machine defined as rewrites on a
string. The semantics use a CEKS model with a (C)ontrol string
representing the expression being evaluated, an (E)nvironment for
local variables, a (K)ontinuation for what is to be executed next, and
a (S)tore for the heap.

\begin{figure}
\begin{center}
\cfgstart
\cfgrule{P}{\lp $\mu$ \lp \cfgnt{C} \cfgnt{m}\rp\rp}
\cfgrule{$\mu$}{(\cfgnt{CL} ...)}
\cfgrule{T}{\cfgt{bool} \cfgor \cfgnt{C}}
\cfgrule{CL}{\lp\cfgt{class} \cfgnt{C} \lp\lb\cfgnt{T} \cfgnt{f}\rb ...\rp \lp\cfgnt{M} ...\rp}
\cfgrule{M}{\lp\cfgnt{T} \cfgnt{m} \lb\cfgnt{T} \cfgnt{x}\rb\  e\rp}
\cfgrule{e}{\cfgnt{x}
\cfgor{\lp\cfgt{new} \cfgnt{C}\rp}
\cfgor{\lp\cfgnt{e} \cfgt{\$} \cfgnt{f}\rp}
\cfgor{\lp\cfgnt{x} \cfgt{\$} \cfgnt{f} \cfgt{:=} \cfgnt{e}\rp}
\cfgor{\lp\cfgnt{e} \cfgt{=} \cfgnt{e}\rp}}
\cfgorline{\lp\cfgt{if} \cfgnt{e} \cfgnt{e} \cfgt{else} \cfgnt{e}\rp 
\cfgor {\lp\cfgt{var} \cfgnt{T} \cfgnt{x} \cfgt{:=} \cfgnt{e} \cfgt{in} \cfgnt{e}\rp}
\cfgor {\lp\cfgnt{e} \cfgt{@} \cfgnt{m} \cfgnt{e} \rp}}
\cfgorline{\lp\cfgnt{x} \cfgt{:=} \cfgnt{e}\rp
\cfgor{\lp\cfgt{begin} \cfgnt{e} ...\rp}
\cfgor{\cfgnt{v}}}
\cfgrule{x}{\cfgt{this} \cfgor \cfgnt{id}}
\cfgrule{f,m,C}{\cfgnt{id}}
%\cfgrule{m}{\cfgnt{id}}
%\cfgrule{C}{\cfgnt{id}}
\cfgrule{v}{\cfgnt{r} \cfgor \cfgt{null} \cfgor \cfgt{true} \cfgor \cfgt{false} \cfgor \cfgt{error}}
\cfgrule{r}{\cfgt{number}}
\cfgrule{id}{\cfgt{variable-not-otherwise-mentioned}}
\cfgend
\end{center}
\caption{The Javalite surface syntax.}
\label{fig:surface-syntax}
\end{figure}


\subsection{Environment}
The environment, $\eta$, associates a variable $\cfgnt{x}$ with a
value $\cfgnt{v}$. The value can be a reference, $\cfgnt{r}$ or one of
the special values $\cfgt{null}$, $\cfgt{true}$, or
$\cfgt{false}$. Although the Javalite machine is purely syntactic, for
clarity and brevity in the presentation, the more complex structures
such as the environment are treated as partial functions. As such,
$\eta(\cfgnt{x}) = \cfgnt{r}$ is the reference mapped to the variable
in the environment. The notation $\eta^\prime = \eta[\cfgnt{x} \mapsto
  \cfgnt{v}]$ defines a new partial function $\eta^\prime$ that is
just like $\eta$ only the variable $\cfgnt{x}$ now maps to
$\cfgnt{v}$.




\section{Generating Heap Summaries}

\subsection{Initialization of Symbolic References}

In this section we present the Javalite rewrite rules for the concrete
as well as summary initialization of symbolic references. The
initialization rules are defined on the bi-partite graph consisting of
references and locations. The lazy initialization of symbolic
references consists of three key points of non-determinism where each
symbolic reference can be initialized non-deterministically to null, a
new instance of the symbolic reference, or aliases to symbolic
references of the same type previously initialized. The initialization
in GSE consists of creating branches in the execution tree for all the
non-deterministic choices. On the other hand, the heap summarization
approach creates a single branch that contains the summarization for
all the initialization in a single bi-partitate graph.

\begin{figure*}[t]
\begin{center}
\mprset{flushleft}
\begin{mathpar}
	\inferrule[Initialize (null)]{
	  \Lambda = \{ l \mid \exists \phi\ \lp \lp \phi\ l\rp  \in \cfgnt{L}\lp \cfgnt{r}\rp  \wedge  \cfgnt{R}\lp l,\cfgnt{f}\rp  = \bot\ \rp\}\\
      \Lambda \neq \emptyset\\\\
      \cfgnt{r}^\prime = \mathrm{fresh}_r\lp \rp\\ 
      \theta_\mathit{null} = \{ \lp \phi_T\ l_\mathit{null}\rp \} \\
      l_x = \mathrm{min}_l\lp \Lambda\rp \\\\
      \phi_g^\prime = \lp\phi_g \wedge \cfgnt{r}^\prime = \cfgnt{r}_\mathit{null}\rp
    }{
      \lp \cfgnt{L}\ \cfgnt{R}\ \phi_g\ \cfgnt{r}\ \cfgnt{f}\rp  \rightarrow_I 
      \lp \cfgnt{L}[\cfgnt{r}^\prime \mapsto \theta_\mathit{null}]\ \cfgnt{R}[ \lp l_x,\cfgnt{f}\rp  \mapsto \cfgnt{r}^\prime]\ \phi_g^\prime\ \cfgnt{r}\ \cfgnt{f}\rp 
	}
\and
	\inferrule[Initialize (new)]{
	  \Lambda  = \{ l \mid \exists \phi\ \lp \lp \phi\ l\rp  \in \cfgnt{L}\lp \cfgnt{r}\rp  \wedge  \cfgnt{R}\lp l,\cfgnt{f}\rp  = \bot\rp\}\\
      \Lambda \neq \emptyset\\\\
      \mathrm{C} = \mathrm{type}\lp \cfgnt{f}\rp\\
      \cfgnt{r}_f = \mathrm{init}_r\lp \rp\\
      l_f = \mathrm{init}_l\lp \mathrm{C}\rp \\\\
      \cfgnt{R}^\prime = \cfgnt{R}[\forall \cfgnt{f} \in \mathit{fields}\lp \mathrm{C}\rp \ \lp \lp l_f\ \cfgnt{f}\rp  \mapsto \bot \rp ] \\\\
      \rho = \{ \lp\cfgnt{r}_a\ l_a\rp \mid \mathrm{isInit}\lp \cfgnt{r}_a\rp  \wedge \mathrm{type}\lp l_a\rp  = \mathrm{C} \wedge \exists \phi_a\ \lp \lp \phi_a\ l_a\rp  \in \cfgnt{L}\lp \cfgnt{r}_a\rp\rp \}\\\\
      \theta_\mathit{new} = \{\lp \phi_T\ l_f\rp \} \\
      l_x = \mathrm{min}_l\lp \Lambda\rp \\\\
      \phi_g^\prime = \lp\phi_g \wedge \cfgnt{r}_f \neq \cfgnt{r}_\mathit{null} \wedge \lp \wedge_{\lp\cfgnt{r}_a\ l_a\rp \in \rho} \cfgnt{r}_f \ne \cfgnt{r}_a\rp\rp
    }{
      \lp \cfgnt{L}\ \cfgnt{R}\ \phi_g\ \cfgnt{r}\ \cfgnt{f}\rp  \rightarrow_I 
      \lp \cfgnt{L}[\cfgnt{r}_f \mapsto \theta_\mathit{new}]\ \cfgnt{R}^\prime[ \lp l_x,\cfgnt{f}\rp  \mapsto \cfgnt{r}_f ]\ \phi_g^\prime\ \cfgnt{r}\ \cfgnt{f}\rp 
	}
\and
	\inferrule[Initialize (alias)]{
	  \Lambda = \{ l \mid \exists \phi\ \lp \lp \phi\ l\rp  \in \cfgnt{L}\lp \cfgnt{r}\rp  \wedge  \cfgnt{R}\lp l,\cfgnt{f}\rp  = \bot\ \rp\}\\
      \Lambda \neq \emptyset\\\\
      \mathrm{C} = \mathrm{type}\lp \cfgnt{f}\rp\\
      \cfgnt{r}^\prime = \mathrm{fresh}_r\lp \rp\\\\
      \rho = \{ \lp\cfgnt{r}_a\ l_a\rp \mid \mathrm{isInit}\lp \cfgnt{r}_a\rp  \wedge \mathrm{type}\lp l_a\rp  = \mathrm{C} \wedge \exists \phi_a\ \lp \lp \phi_a\ l_a\rp  \in \cfgnt{L}\lp \cfgnt{r}_a\rp\rp \}\\\\
      \lp\cfgnt{r}_a\ l_a\rp \in \rho \\
      \theta_\mathit{alias} = \{ \lp \phi_T\ l_a\rp\}\\
      l_x = \mathrm{min}_l\lp \Lambda\rp\\\\
      \phi^\prime_g = \lp\phi_g \wedge \cfgnt{r}^\prime \neq \cfgnt{r}_\mathit{null} \wedge \cfgnt{r}^\prime = \cfgnt{r}_a \wedge \lp \wedge_{\lp \cfgnt{r}^{\prime}_a\ l_a\rp  \in \rho\ \lp \cfgnt{r}^{\prime}_a \neq \cfgnt{r}_a\rp } \cfgnt{r}^\prime \neq \cfgnt{r}^{\prime}_a \rp\rp
    }{
      \lp \cfgnt{L}\ \cfgnt{R}\ \phi_g\ \cfgnt{r}\ \cfgnt{f}\rp  \rightarrow_I 
      \lp \cfgnt{L}[\cfgnt{r}^\prime \mapsto \theta_\mathit{alias}]\ \cfgnt{R}[ \lp l_x,\cfgnt{f}\rp  \mapsto \cfgnt{r}^\prime ]\ \phi_g^\prime\ \cfgnt{r}\ \cfgnt{f}\rp 
	}
\and
	\inferrule[Initialize (end)]{
	  \Lambda = \{ l \mid \exists \phi\ \lp \lp \phi\ l\rp  \in \cfgnt{L}\lp \cfgnt{r}\rp  \wedge  \cfgnt{R}\lp l,\cfgnt{f}\rp  = \bot\ \rp\}\\
      \Lambda = \emptyset
    }{
      \lp \cfgnt{L}\ \cfgnt{R}\ \phi_g\ \cfgnt{r}\ \cfgnt{f}\rp  \rightarrow_I 
      \lp \cfgnt{L}\ \cfgnt{R}\ \phi_g\ \cfgnt{r}\ \cfgnt{f}\rp 
	}
\end{mathpar}
\end{center}
\caption{The initialization machine, $s ::= \lp\cfgnt{L}\ \cfgnt{R}\ \phi_g\ \cfgnt{r}\ \cfgnt{f}\rp$, with $s \rightarrow_I^* s^\prime$ indicating stepping the machine until the state does not change.}
\label{fig:lazyInit}
\end{figure*}

\newsavebox{\boxPi}
\savebox{\boxPi}{
%\begin{center}
\mprset{flushleft}
\begin{mathpar}
	\inferrule[Summarize]{
	\Lambda = \mathbb{UN}\lp \cfgnt{L}, \cfgnt{R}, \cfgnt{r}, \cfgnt{f}\rp \\
      \Lambda \neq \emptyset \\
      \lp\phi_x\ \cfgnt{l}_x\rp = \mathrm{min}_l\lp \Lambda\rp\\
      \cfgnt{r}_f = \mathrm{init}_r\lp \rp \\
      l_f  = \mathrm{fresh}_l\lp \mathrm{C}\rp\\\\
      \rho = \{ \lp \cfgnt{r}_a\ l_a\rp  \mid \mathrm{isInit}\lp \cfgnt{r}_a\rp  \wedge\cfgnt{r}_a = \mathrm{min}_r\lp \cfgnt{R}^{\leftarrow}[l_a]\rp \wedge \mathrm{type}\lp l_a\rp  = \mathrm{C} \} \\\\
      \theta_\mathit{null} = \{ \lp \phi\ l_\mathit{null}\rp  \mid \phi = \lp \phi_x \wedge \cfgnt{r}_f = \cfgnt{r}_\mathit{null} \rp  \} \\\\
      \theta_\mathit{new} = \{\lp \phi\ l_f\rp  \mid \phi = \lp \phi_x \wedge \cfgnt{r}_f \neq \cfgnt{r}_\mathit{null} \wedge \lp \wedge_{\lp \cfgnt{r}^\prime_a\ l^\prime_a\rp  \in \rho} \cfgnt{r}_f \ne \cfgnt{r}^\prime_a\rp \rp \}\\\\
      \theta_\mathit{alias} = \{ \lp \phi\ l_a\rp  \mid \exists\cfgnt{r}_a\ \lp\lp\cfgnt{r}_a\ l_a\rp  \in \rho \wedge \phi = \lp \phi_x \wedge \cfgnt{r}_f \neq \cfgnt{r}_\mathit{null} \wedge \cfgnt{r}_f = \cfgnt{r}_a \wedge \lp \wedge_{\lp \cfgnt{r}^{\prime}_a\ l^{\prime}_a\rp  \in \rho\ \lp \cfgnt{r}^\prime_a < \cfgnt{r}_a\rp } \cfgnt{r}_f \neq \cfgnt{r}^{\prime}_a \rp \rp \rp \} \\\\
      \theta_\mathit{orig} = \{\lp\phi\ \cfgnt{l}_\mathit{orig}\rp \mid \exists \phi_\mathit{orig} \lp \lp\phi_\mathit{orig}\ \cfgnt{l}_\mathit{orig}\rp \in \cfgnt{L}\lp\cfgnt{R}\lp\cfgnt{l}_x,\cfgnt{f}\rp\rp \wedge \phi = \lp\neg\phi_x \wedge \phi_\mathit{orig}\rp\}\\\\ 
      \theta = \theta_\mathit{null} \cup \theta_\mathit{new} \cup \theta_\mathit{alias} \cup \theta_\mathit{orig} \\
\cfgnt{R}^\prime = \cfgnt{R}[\forall \cfgnt{f} \in \mathit{fields}\lp \mathrm{C}\rp \ \lp \lp l_f\ \cfgnt{f}\rp  \mapsto \cfgnt{r}_\mathit{un} \rp ]
    }{
      \lp \cfgnt{L}\ \cfgnt{R}\ \cfgnt{r}\ \cfgnt{f}\ \cfgnt{C}\rp \rsum 
      \lp \cfgnt{L}[\cfgnt{r}_f \mapsto \theta]\ \cfgnt{R}^{\prime}[ \lp l_x,\cfgnt{f}\rp  \mapsto \cfgnt{r}_f ]\ \cfgnt{r}\ \cfgnt{f}\ \cfgnt{C}\rp
	}
\and
	\inferrule[Summarize-end]{
	  \Lambda = \mathbb{UN}\lp \cfgnt{L}, \cfgnt{R}, \cfgnt{r}, \cfgnt{f}\rp \\
      \Lambda = \emptyset
    }{
      \lp \cfgnt{L}\ \cfgnt{R}\ \cfgnt{r}\ \cfgnt{f}\ \cfgnt{C}\rp  \rsum
      \lp \cfgnt{L}\ \cfgnt{R}\ \cfgnt{r}\ \cfgnt{f}\ \cfgnt{C}\rp 
	}
\end{mathpar}}
%\end{center}
%\caption{The summary machine, $s ::= \lp\cfgnt{L}\ \cfgnt{R}\ \cfgnt{r}\ \cfgnt{f}\ \cfgnt{C}\rp$, with $s\rsum^*s^\prime =  s \rsum \cdots \rsum s^\prime \rsum s^\prime$.}
%\label{fig:symInit}
%\end{figure*}



The initialization rules are invoked when an uninitialized field in a
symbolic reference is accessed. The function $\mathbb{UN}(\cfgnt{L},
\cfgnt{R}, \cfgnt{r}, \cfgnt{f}) = \{\cfgnt{l}\ ...\}$ returns
constraint-location pairs in which the field $\cfgnt{f}$ is
uninitialized:
\[
\begin{array}{rcl}
\mathbb{UN}(\cfgnt{L}, \cfgnt{R}, \cfgnt{r}, \cfgnt{f}) & = &\{ \lp\phi\ \cfgnt{l}\rp \mid \lp \phi\ \cfgnt{l}\rp  \in \cfgnt{L}\lp \cfgnt{r}\rp  \wedge \\
& & \ \ \ \ \exists \phi^\prime \lp \lp \phi^\prime\ \cfgnt{l}_\mathit{un}\rp  \in \cfgnt{L}\lp \cfgnt{R}\lp l,\cfgnt{f}\rp\rp \wedge \\
& & \ \ \ \ \ \ \ \ \mathbb{S}\lp \phi \wedge \phi^\prime \rp\rp\}\\
\end{array}
\]
where $\mathbb{S}(\phi)$ returns true if $\phi$ is
satisfiable. Intutively, for the reference, $\cfgnt{r}$, it constructs
the set, $\theta$, that contains all contraint-location pairs that
point to the field $\cfgnt{f}$ and $\cfgnt{f}$ points to
$\cfgnt{l}_\mathit{un}$. The cardinality of the set, $\theta$ is never
greater than one in GSE and the constraint is always satisfiable
because all constraints are constant. This property is relaxed in GSE
with heap summaries.

The rules in~\figref{fig:lazyInit} present the rewrite rules for the
concrete initialization of symbolic heap objects.  These rules are
invoked until a fix pointed is reached. 

The initialize (null) rewrite rule in~\figref{fig:lazyInit} first
checks that the field, $\cfgnt{r}$ is uninitialized. The fresh method
returns a new input heap reference from the partition 




\section{GSE with Heap Summaries}

The function $\mathbb{VS}(L,R,\phi_g,r,f)$ constructs the value-set given a
heap, reference, and desired field:
\[
\begin{array}{rcl}
  \mathbb{VS}(L,R,\phi_g,r,f) & = & \{(\phi\wedge\phi^\prime\ l^\prime) \mid \\
  & & \ \ \ \ \exists l\ ((\phi\ l) \in L(r)\ \wedge \\
  & & \ \ \ \ \ \ \ \ \exists r^\prime \in R(l,f) ( \\
  & & \ \ \ \ \ \ \ \ \ \ \ \ (\phi^\prime\ l^\prime) \in L(r^\prime)\ \wedge\\
  & & \ \ \ \ \ \ \ \ \ \ \ \ \mathbb{S}(\phi\wedge\phi^\prime\wedge \phi_g)))\}
\end{array}
\]
where $\mathbb{S}(\phi)$ returns true if $\phi$ is satisfiable.

The strengthen function $\mathbb{ST}(L,r,\phi^\prime)$ strengthens every
constraint from the reference $r$ with $\phi^\prime$ and keeps only location-constraint
pairs that are satisfiable after this strengthening:
\[
\begin{array}{rcl} 
\mathbb{ST}(L,r,\phi) & = & \{ (\phi\wedge\phi^\prime\ l) \mid 
(\phi^\prime\ l)\in L(r)\wedge\mathbb{S}(\phi\wedge\phi^\prime\wedge\phi_g) \}
\end{array}
\]

\newsavebox{\boxPi}
\savebox{\boxPi}{
%\begin{center}
\mprset{flushleft}
\begin{mathpar}
	\inferrule[Summarize]{
	\Lambda = \mathbb{UN}\lp \cfgnt{L}, \cfgnt{R}, \cfgnt{r}, \cfgnt{f}\rp \\
      \Lambda \neq \emptyset \\
      \lp\phi_x\ \cfgnt{l}_x\rp = \mathrm{min}_l\lp \Lambda\rp\\
      \cfgnt{r}_f = \mathrm{init}_r\lp \rp \\
      l_f  = \mathrm{fresh}_l\lp \mathrm{C}\rp\\\\
      \rho = \{ \lp \cfgnt{r}_a\ l_a\rp  \mid \mathrm{isInit}\lp \cfgnt{r}_a\rp  \wedge\cfgnt{r}_a = \mathrm{min}_r\lp \cfgnt{R}^{\leftarrow}[l_a]\rp \wedge \mathrm{type}\lp l_a\rp  = \mathrm{C} \} \\\\
      \theta_\mathit{null} = \{ \lp \phi\ l_\mathit{null}\rp  \mid \phi = \lp \phi_x \wedge \cfgnt{r}_f = \cfgnt{r}_\mathit{null} \rp  \} \\\\
      \theta_\mathit{new} = \{\lp \phi\ l_f\rp  \mid \phi = \lp \phi_x \wedge \cfgnt{r}_f \neq \cfgnt{r}_\mathit{null} \wedge \lp \wedge_{\lp \cfgnt{r}^\prime_a\ l^\prime_a\rp  \in \rho} \cfgnt{r}_f \ne \cfgnt{r}^\prime_a\rp \rp \}\\\\
      \theta_\mathit{alias} = \{ \lp \phi\ l_a\rp  \mid \exists\cfgnt{r}_a\ \lp\lp\cfgnt{r}_a\ l_a\rp  \in \rho \wedge \phi = \lp \phi_x \wedge \cfgnt{r}_f \neq \cfgnt{r}_\mathit{null} \wedge \cfgnt{r}_f = \cfgnt{r}_a \wedge \lp \wedge_{\lp \cfgnt{r}^{\prime}_a\ l^{\prime}_a\rp  \in \rho\ \lp \cfgnt{r}^\prime_a < \cfgnt{r}_a\rp } \cfgnt{r}_f \neq \cfgnt{r}^{\prime}_a \rp \rp \rp \} \\\\
      \theta_\mathit{orig} = \{\lp\phi\ \cfgnt{l}_\mathit{orig}\rp \mid \exists \phi_\mathit{orig} \lp \lp\phi_\mathit{orig}\ \cfgnt{l}_\mathit{orig}\rp \in \cfgnt{L}\lp\cfgnt{R}\lp\cfgnt{l}_x,\cfgnt{f}\rp\rp \wedge \phi = \lp\neg\phi_x \wedge \phi_\mathit{orig}\rp\}\\\\ 
      \theta = \theta_\mathit{null} \cup \theta_\mathit{new} \cup \theta_\mathit{alias} \cup \theta_\mathit{orig} \\
\cfgnt{R}^\prime = \cfgnt{R}[\forall \cfgnt{f} \in \mathit{fields}\lp \mathrm{C}\rp \ \lp \lp l_f\ \cfgnt{f}\rp  \mapsto \cfgnt{r}_\mathit{un} \rp ]
    }{
      \lp \cfgnt{L}\ \cfgnt{R}\ \cfgnt{r}\ \cfgnt{f}\ \cfgnt{C}\rp \rsum 
      \lp \cfgnt{L}[\cfgnt{r}_f \mapsto \theta]\ \cfgnt{R}^{\prime}[ \lp l_x,\cfgnt{f}\rp  \mapsto \cfgnt{r}_f ]\ \cfgnt{r}\ \cfgnt{f}\ \cfgnt{C}\rp
	}
\and
	\inferrule[Summarize-end]{
	  \Lambda = \mathbb{UN}\lp \cfgnt{L}, \cfgnt{R}, \cfgnt{r}, \cfgnt{f}\rp \\
      \Lambda = \emptyset
    }{
      \lp \cfgnt{L}\ \cfgnt{R}\ \cfgnt{r}\ \cfgnt{f}\ \cfgnt{C}\rp  \rsum
      \lp \cfgnt{L}\ \cfgnt{R}\ \cfgnt{r}\ \cfgnt{f}\ \cfgnt{C}\rp 
	}
\end{mathpar}}
%\end{center}
%\caption{The summary machine, $s ::= \lp\cfgnt{L}\ \cfgnt{R}\ \cfgnt{r}\ \cfgnt{f}\ \cfgnt{C}\rp$, with $s\rsum^*s^\prime =  s \rsum \cdots \rsum s^\prime \rsum s^\prime$.}
%\label{fig:symInit}
%\end{figure*}

\newsavebox{\boxPFAFW}
\savebox{\boxPFAFW}{
%\begin{figure}[t]
%\begin{center}
\mprset{flushleft}
\begin{mathpar}
	\inferrule[Field Access]{
      \exists \lp \phi\ l\rp \in \cfgnt{L}\lp \cfgnt{r}\rp\ \lp l \neq l_{\mathit{null}} \wedge \mathbb{S}\lp \phi \wedge \phi_g\rp \rp \\\\
      \theta = \{ \phi \mid \lp \phi\ l_\mathit{null} \rp \wedge \mathbb{S}\lp \phi \wedge \phi_g\rp \} \\\\
      \phi_g^\prime = \phi_g \wedge (\wedge_{\phi \in \theta} \neg \phi) \\\\
      \{\cfgnt{C}\} = \{\cfgnt{C} \mid \exists \lp \phi\ l\rp  \in \cfgnt{L}\lp \cfgnt{r}\rp\ \lp\cfgnt{C} = \mathrm{type}\lp \cfgnt{l},\cfgnt{f}\rp\rp\} \\\\
      \lp \cfgnt{L}\ \cfgnt{R}\ \cfgnt{r}\ \cfgnt{f}\ \cfgnt{C}\rp \rsum^* \lp \cfgnt{L}^\prime\ \cfgnt{R}^\prime\ \cfgnt{r}\ \cfgnt{f}\ \cfgnt{C}\rp \\
      \cfgnt{r}^\prime = \mathrm{stack}_r\lp \rp
    }{
      \lp \cfgnt{L}\ \cfgnt{R}\ \phi_g\ \eta\ \cfgnt{r}\ \lp \cfgt{*}\ \cfgt{\$}\ \cfgnt{f} \rightarrow \cfgnt{k}\rp \rp  \rightarrow_\mathit{FA}
      \lp \cfgnt{L}^\prime[\cfgnt{r}^\prime \mapsto \mathbb{VS}\lp \cfgnt{L}^\prime,\cfgnt{R}^\prime,\cfgnt{r},\cfgnt{f},\phi_g^\prime\rp ]\ \cfgnt{R}^\prime\ \phi_g^\prime\ \eta\ \cfgnt{r}^\prime\ \cfgnt{k}\rp 
	}
\and
	\inferrule[Field Access (NULL)]{
      \exists \lp \phi\ l\rp \in \cfgnt{L}\lp \cfgnt{r}\rp\ \lp l = l_{\mathit{null}} \wedge \mathbb{S}\lp \phi \wedge \phi_g\rp \rp
    }{
      \lp \cfgnt{L}\ \cfgnt{R}\ \phi_g\ \eta\ \cfgnt{r}\ \lp \cfgt{*}\ \cfgt{\$}\ \cfgnt{f} \rightarrow \cfgnt{k}\rp \rp  \rightarrow_\mathit{FA}
      \lp \cfgnt{L}\ \cfgnt{R}\ \phi_g\ \eta\ \cfgt{error}\ \lp \cfgt{*}\ \cfgt{\$}\ \cfgnt{f} \rightarrow \cfgnt{k}\rp \rp
	}
\and
	\inferrule[Field Write]{
      \exists \lp \phi\ l\rp \in \cfgnt{L}\lp \cfgnt{r}\rp\ \lp l \neq l_{\mathit{null}} \wedge \mathbb{S}\lp \phi \wedge \phi_g\rp \rp \\\\
      \theta = \{ \phi \mid \lp \phi\ l_\mathit{null} \rp \wedge \mathbb{S}\lp \phi \wedge \phi_g\rp \} \\\\
      \phi_g^\prime = \phi_g \wedge (\wedge_{\phi \in \theta} \neg \phi) \\\\
      \cfgnt{r}_x = \eta\lp \cfgnt{x}\rp \\
      \Psi_x =\{\lp \phi\ l\ \cfgnt{r}_\mathit{cur} \rp  \mid \lp \phi\ \cfgnt{l}\rp  \in \cfgnt{L}\lp \cfgnt{r}_x\rp  \wedge \cfgnt{r}_\mathit{cur} = \cfgnt{R}\lp l,\cfgnt{f}\rp  \}\\\\
      X = \{ \lp l\ \theta \rp  \mid \exists \phi\ \lp \lp \phi\ l\ \cfgnt{r}_\mathit{cur} \rp \in \Psi_x \wedge \theta = \mathbb{ST}\lp \cfgnt{L},\cfgnt{r},\phi,\phi_g^\prime\rp  \cup \mathbb{ST}\lp \cfgnt{L},\cfgnt{r}_\mathit{cur},\neg\phi,\phi_g^\prime\rp \rp  \}\\\\
      \cfgnt{R}^{\prime} = \cfgnt{R}[\forall \lp l\ \theta \rp  \in X\ \lp \lp l\ \cfgnt{f}\rp  \mapsto \mathrm{fresh}_r\lp \rp \rp ]\\\\
      \cfgnt{L}^{\prime} = \cfgnt{L}[\forall \lp l\ \theta \rp  \in X\ \lp \exists \cfgnt{r}_\mathit{targ}\ \lp \cfgnt{r}_\mathit{targ} = \cfgnt{R}^\prime\lp l,\cfgnt{f}\rp \wedge \lp\cfgnt{r}_\mathit{targ} \mapsto \theta\rp  \rp \rp ]
    }{
      \lp \cfgnt{L}\ \cfgnt{R}\ \phi_g\ \eta\ \cfgnt{r}\ \lp \cfgnt{x}\ \cfgt{\$}\ \cfgnt{f}\ \cfgt{:=}\ \cfgt{*}\ \rightarrow\ \cfgnt{k}\rp \rp  \rightarrow_\mathit{FW}
      \lp \cfgnt{L}^{\prime}\ \cfgnt{R}^{\prime}\ \phi_g^\prime\ \eta\ \cfgnt{r}\ \cfgnt{k}\rp 
	}	
\and
	\inferrule[Field Write (NULL)]{
      \exists \lp \phi\ l\rp \in \cfgnt{L}\lp \cfgnt{r}\rp\ \lp l \neq l_{\mathit{null}} \wedge \mathbb{S}\lp \phi \wedge \phi_g\rp \rp
    }{
      \lp \cfgnt{L}\ \cfgnt{R}\ \phi_g\ \eta\ \cfgnt{r}\ \lp \cfgnt{x}\ \cfgt{\$}\ \cfgnt{f}\ \cfgt{:=}\ \cfgt{*}\ \rightarrow\ \cfgnt{k}\rp \rp  \rightarrow_\mathit{FW}
      \lp \cfgnt{L}\ \cfgnt{R}\ \phi_g\ \eta\ \cfgt{error}\ \lp \cfgnt{x}\ \cfgt{\$}\ \cfgnt{f}\ \cfgt{:=}\ \cfgt{*}\ \rightarrow\ \cfgnt{k}\rp \rp
	}	
\end{mathpar}}
%\end{center}
%\caption{Precise symbolic heap summaries from symbolic execution indicated by $\rsym = \rightarrow_\mathit{FA} \cup \rightarrow_\mathit{FW} \cup \rightarrow_\mathit{EQ} \cup \rcom$.}
%\label{fig:symfield}
%\end{figure}



\section{Proofs}

\subsection{Definitions}

\begin{definition}
A \textbf{state transition function} $\rightarrow_{\Phi}$ is a mapping $\rightarrow_{\Phi} : s \mapsto s$ , which takes one machine state and transforms it into another machine state. The state transition function with superscript represents composition of the state transition function: $$ s_a \rightarrow_{\Phi} s_b \rightarrow_{\Phi} s_c \implies s_a \rightarrow_{\Phi}^2 s_c $$
\end{definition}

\begin{definition}
A \textbf{feasible state sequence} is defined as a sequence of states resulting from repeated application of the state transition relation to some initial state $s_0$: $$\Pi_n = s_0,s_1,...,s_n$$ where the relation $s_i \rightarrow_{\Phi} s_{i+1}$ holds for all $i \in \{ i | 0 \leq i < n \}$
\end{definition}

\begin{definition}
Given a sequence of states $$\Pi_n = s_0,s_1,...,s_n$$ where $$s_i = ( \mu_i\ \cfgnt{L}_i\ \cfgnt{R}_i\ \phi_i\ \eta_i\ \cfgnt{e}_i\ \cfgnt{k}_i )$$ the \textbf{control flow sequence} of $\Pi_n$ is the defined as the sequence of tuples $$ \pi_n = \mathbb{CF}(\Pi_n) = (\eta_0\ \cfgnt{e}_0\ \cfgnt{k}_0),(\eta_1\ \cfgnt{e}_1\ \cfgnt{k}_1),...,(\eta_n\ \cfgnt{e}_n\ \cfgnt{k}_n)$$
\end{definition}

\begin{definition}
Given a state transition function $\rightarrow_{\Phi}$, an initial state $s_0$ and a control flow sequence $\pi_n$, the \textbf{feasible state set} $\zeta = \mathbb{FS}(\rightarrow_{\Phi},s_0,\pi_n)$ is defined as
 $$\zeta = \{ \forall s | \pi_n = \mathbb{CF}(\Pi_n) \wedge s = max_s(\Pi) \wedge s_0 \rightarrow_{\Phi}^{n-1} s\} $$
\end{definition}

\begin{definition}
A \textbf{heap homomorphism} $(g\ h)$ between two states $s_x$ and $s_y$ is defined as a pair of functions $g:\cfgnt{r} \mapsto \cfgnt{r}$ and $h:\cfgnt{l} \mapsto \cfgnt{l}$ such that for any reference $r \in s_x$, location $l \in s_x$, and field $f$, $$ l \in \cfgnt{L}_x(r) \Leftrightarrow h(l)\in \cfgnt{L}_y(g(r))$$ and $$ r = \cfgnt{R}_x(l,f) \Leftrightarrow g(r) = \cfgnt{R}_y(h(l),f)$$ If a such a pair of functions exists from one state $s_x$ to another state $s_y$ we say that state $s_x$ is \textbf{heap homomorphic} to state $s_y$, indicated by the notation $(g\ f):\ s_x \rightarrow s_y$. 

Suppose we take the set of constraints from the image of $s_x$ in $s_y$ under $(g\ h)$ :
$$\chi = \{ \phi\ | \exists (r \in \cfgnt{R}_x,\  l \in \cfgnt{L}_x) ( (\phi\ h(l)) \in \cfgnt{L}_y (g(r))  \}$$ and we conjoin those constraints with the global invariant from $s_y$ :
$$\phi_G = \phi_y \bigwedge_{\phi_i \in \chi} \phi_i $$
 If the expression $\phi_G$ is satisfiable, we say that the heap homomorphism $(g\ h):\ s_x \rightarrow s_y $ is \textbf{valid}.
\end{definition}
\begin{definition}
The \textbf{representation relation} is defined as follows: given two states $s_\ell$ and $s_s$, $s_\ell \sqsubset s_s $ if and only if $\eta_{\ell} = \eta_{s} ,\ \cfgnt{e}_{\ell} = \cfgnt{e}_{s} ,\ \cfgnt{k}_{\ell} = \cfgnt{k}_{s}$, and there exists a valid heap homomorphism $(g\ h):\ s_\ell \rightarrow s_s $. The expression $s_\ell \sqsubset s_s $ can be read as "state $s_\ell$ is represented by state $s_s$ . 
\end{definition}

\begin{definition}
A state $s_s$ is \textbf{congruent} to a set of states $\mathcal{S}$ if and only if $s$ represents every state in $\mathcal{S}$ and represents no other state: 
$$ s_s \equiv \mathcal{S} : s_i \in \mathcal{S} \Leftrightarrow s_i \sqsubset s_s $$
\end{definition}

\begin{definition}
A symbolic state $s_s$ is \textbf{exact} with respect to an initial state $s_0$ and control flow sequence $\pi$ if and only if it is congruent the set of feasible lazy states on the same control flow path:
$$ s_s \equiv \mathbb{FS}(\rightarrow_{\ell},s_0,\pi)$$
\end{definition}

\subsection{Theorems}

\begin{lemma}
If symbolic state $s_s = \lp \cfgnt{L}_{\mathcal{S}}\ \cfgnt{R}_{\mathcal{S}}\ \phi_g\ \eta\ \cfgnt{r}\ \lp \cfgt{*}\ \cfgt{\$}\ \cfgnt{f} \rightarrow \cfgnt{k}\rp \rp$ is exact with respect to some initial state $s_0$ and control flow path $\pi_n$, then the state $s_s^\prime : s_s \rightarrow_s s_s^\prime$ is exact with respect to $s_0$ and $\pi_{n+1}.
\end{lemma}

\begin{proof}
We will consider two cases for this proof. In the first case, we assume that all of the fields involved in the read are initialized. In the second case we consider the case of uninitialized fields. 

Case 1: suppose all of the pertinent fields in $s_s$ are initialized. Take an arbitrary lazy state $s_\ell \sqsubset s_s$. Since $s_s$ is exact,  $s_\ell = \lp \cfgnt{L}_{\ell}\ \cfgnt{R}_{\ell}\ \phi_L\ \eta\ \cfgnt{r}\ \lp \cfgt{*}\ \cfgt{\$}\ \cfgnt{f} \rightarrow \cfgnt{k} \rp \rp$. If we apply the state transition functions to achieve states $s_\ell^\prime : s_\ell \rightarrow_\ell s_\ell^\prime$ and $s_s^\prime : s_s \rightarrow_s s_s^\prime$, we find that:
$$s_\ell^\prime = \lp \cfgnt{L}_{\ell} [\cfgnt{r}^\prime \mapsto \lp\phi^\prime\ l^\prime\rp]\ \cfgnt{R}_{\ell}\ \phi_L\ \eta\ \cfgnt{r}^\prime\ \cfgnt{k}\rp $$
 and 
 $$ s_s^\prime = \lp \cfgnt{L}_{s}[\cfgnt{r}^\prime \mapsto \mathbb{VS}\lp \cfgnt{L}_{\mathcal{S}},\cfgnt{R}_{s},\cfgnt{r},\cfgnt{f},\phi_g\rp ]\ \cfgnt{R}_{\mathcal{S}}\ \phi_g\ \eta\ \cfgnt{r}^\prime\ \cfgnt{k}\rp $$

We now show that $s_\ell^\prime \sqsubset s_s^\prime$. Since $\eta$, $e$, and $k$ are identical between $s_s^\prime$ and $s_\ell^\prime $, the first condition is met by default. To find a valid heap homomorphism, first construct functions $g^\prime : g^\prime = g[ r^\prime \mapsto r^\prime]$ and $h^\prime : h^\prime = h$. 

%we now need to show that we have a heap homomorphism here, and that the homomorphism is valid. once we've done that , we've proved the soundness case

Now, we show that no infeasible lazy states are allowed by $\mathcal{S}^\prime$. NEED TO FILL IN THIS BIT, TOO

$\mathbb{S}(\bigwedge_{(\phi l) \in L_s^i} \phi)) \wedge \forall l \in L_L (\exists l \in L_s^i)$

Since $s_\ell \subseteq s_s$
\end{proof}



%\section{Uber-lazy Operational Semantics}
\label{semantics}

The operational semantics for Uber-lazy
symbolic execution are specified using the Javalite 
language~\cite{saints-MS}. Javalite is an imperative model of Java
that includes many features of the Java language. In this work, we 
present only the salient features of the Javalite
language relevant to understanding the Uber-lazy symbolic execution
algorithm. A detailed explanation of Javalite is available in~\cite{saints-MS}.

Javalite is an imperative model of the Java language
developed to facilitate rapid prototyping 
of model checking algorithms and proofs about the
algorithms. It is specified with a Java-like syntax and a set
of reduction rules for syntacticly executing Javalite programs.
Javalite is based on a variant of the CEKS syntactic 
machine~\cite{Felleisen:1987}. The operational semantics are 
defined using the structure (syntax)
of the language and a set of reduction rules for syntactically
executing Javalite programs. In a CEKS machine, the
(C)ontrol string represents a program, command, or instruction to be
evaluated; it is initialized to a string representing the entire progam. 
An (E)nvironment  maps (local) variables to their values. The (K)ontinuation 
specifies what is to be executed next, and the (S)tore is used to store 
dynamically allocated data, i.e., the heap. 

\subsection{Symbolic Store}

At the core of the Uber-lazy symbolic execution algorithm
is a fully symbolic heap, i.e., a heap in which objects
are never materialized as concrete objects during symbolic
execution, but instead are represented using constraints
charactering feasible heap shapes. To represent the heap
store ($S$), our algorithm defines a labeled bi-partite 
graph where $S = (R, L, E)$.  $R$ contains the set of nodes 
representing program \emph{references}.
$L$ contains the set of nodes representing \emph{locations} in the store. 
The store is initialized with two special locations: $null$ and $\bot$ representing a null object
and an uninitialized location respectively. Each edge in the set of labeled
edges, $E$, is uni-directional. An edge from a reference $r \in R$ to a location $l \in L$ is
labeled with a constraint $\phi$ indicating the conditions under which $r$ references
i.e., points to, that location in the store. This paper uses a standard definition
of constraints $\phi \in \Phi$ assuming all of the usual relation operators and connectives.
Reference nodes collect the the
feasible points-to relations for a given program execution path during symbolic execution.
Each edge from a location $l \in L$ to a reference $r \in R$ is labeled with the 
name of a field $f \in F$. 

More details here including we assume the input program is typesafe
or if there is a type and a mismatch occurs, then the machine halts.

The machine also halts if an exception is thrown.

\subsection{Syntax}

\begin{figure}
\begin{center}
\cfgstart
\cfgrule{P}{\lp $\mu$ \lp \cfgnt{C} \cfgnt{m}\rp\rp}
\cfgrule{$\mu$}{(\cfgnt{CL} ...)}
\cfgrule{T}{\cfgt{bool} \cfgor \cfgnt{C}}
\cfgrule{CL}{\lp\cfgt{class} \cfgnt{C} \lp\lb\cfgnt{T} \cfgnt{f}\rb ...\rp \lp\cfgnt{M} ...\rp}
\cfgrule{M}{\lp\cfgnt{T} \cfgnt{m} \lb\cfgnt{T} \cfgnt{x}\rb\  e\rp}
\cfgrule{e}{\cfgnt{x}
\cfgor{\lp\cfgt{new} \cfgnt{C}\rp}
\cfgor{\lp\cfgnt{e} \cfgt{\$} \cfgnt{f}\rp}
\cfgor{\lp\cfgnt{x} \cfgt{\$} \cfgnt{f} \cfgt{:=} \cfgnt{e}\rp}
\cfgor{\lp\cfgnt{e} \cfgt{=} \cfgnt{e}\rp}}
\cfgorline{\lp\cfgt{if} \cfgnt{e} \cfgnt{e} \cfgt{else} \cfgnt{e}\rp 
\cfgor {\lp\cfgt{var} \cfgnt{T} \cfgnt{x} \cfgt{:=} \cfgnt{e} \cfgt{in} \cfgnt{e}\rp}
\cfgor {\lp\cfgnt{e} \cfgt{@} \cfgnt{m} \cfgnt{e} \rp}}
\cfgorline{\lp\cfgnt{x} \cfgt{:=} \cfgnt{e}\rp
\cfgor{\lp\cfgt{begin} \cfgnt{e} ...\rp}
\cfgor{\cfgnt{v}}}
\cfgrule{x}{\cfgt{this} \cfgor \cfgnt{id}}
\cfgrule{f,m,C}{\cfgnt{id}}
%\cfgrule{m}{\cfgnt{id}}
%\cfgrule{C}{\cfgnt{id}}
\cfgrule{v}{\cfgnt{r} \cfgor \cfgt{null} \cfgor \cfgt{true} \cfgor \cfgt{false} \cfgor \cfgt{error}}
\cfgrule{r}{\cfgt{number}}
\cfgrule{id}{\cfgt{variable-not-otherwise-mentioned}}
\cfgend
\end{center}
\caption{The Javalite surface syntax.}
\label{fig:surface-syntax}
\end{figure}

\begin{figure}
\begin{center}
\cfgstart
\cfgrule{e}{\lp ... \cfgor \lp \cfgnt{v} \cfgt{@} \cfgnt{m} \cfgnt{v} \rp\rp}
%\cfgrule{object}{ (\cfgnt{C} [ \cfgnt{f} \cfgnt{loc} ] ...) }
%\cfgrule{hv}{ (\cfgnt{v} \cfgnt{object})}
\cfgrule{$\phi$}{\cfgnt{constraint}}
\cfgrule{l}{\cfgt{number}}
%\cfgrule{h}{(\cfgnt{mt}\ (\cfgnt{h}\ [\cfgnt{loc} $\rightarrow$ \cfgnt{hv}]) )}
\cfgrule{$\eta$}{(\cfgnt{mt}\ ($\eta$ [\cfgnt{x} $\rightarrow$ \cfgnt{loc}]))}
\cfgrule{s}{\lp$\mu$ \cfgnt{L} \cfgnt{R} \cfgnt{g} $\eta$ \cfgnt{e} \cfgnt{k}\rp}
\cfgrule{k}{\cfgt{end}}
\cfgorline{\lp \cfgt{*} \cfgt{\$} \cfgnt{f} $\rightarrow$ \cfgnt{k}\rp}
\cfgorline{\lp \cfgt{*} \cfgt{@} \cfgnt{m} \lp \cfgnt{e} ... \rp $\rightarrow$ \cfgnt{k} \rp}
\cfgorline{\lp \cfgnt{v} \cfgt{@} \cfgnt{m} \cfgnt{v} \cfgt{*} \lp \cfgnt{e} ... \rp $\rightarrow$ \cfgnt{k} \rp}
\cfgorline{\lp \cfgt{*} \cfgt{=} \cfgnt{e} $\rightarrow$ \cfgnt{k}\rp}
\cfgorline{\lp \cfgt{v} \cfgt{=} \cfgnt{*} $\rightarrow$ \cfgnt{k}\rp}
\cfgorline{\lp \cfgt{x} \cfgt{:=} \cfgnt{*} $\rightarrow$ \cfgnt{k}\rp}
\cfgorline{\lp \cfgt{x} \cfgt{\$} \cfgnt{f}  \cfgt{:=} \cfgnt{*} $\rightarrow$ \cfgnt{k}\rp}
\cfgorline{\lp \cfgt{if} \cfgnt{*} \cfgnt{e} \cfgt{else} \cfgnt{e} $\rightarrow$ \cfgnt{k} \rp}
\cfgorline{\lp\cfgt{var} \cfgnt{T} \cfgnt{x} \cfgt{:=} \cfgnt{*} \cfgt{in} \cfgnt{e}  $\rightarrow$ \cfgnt{k} \rp}
\cfgorline{\lp\cfgt{begin}  \cfgnt{*} \lp \cfgnt{e} ...\rp $\rightarrow$ \cfgnt{k} \rp}
\cfgorline{\lp\cfgt{pop} $\eta$ \cfgnt{k}\rp}
\cfgend
\end{center}
\caption{The machine syntax for Javalite.}
\label{fig:machine-syntax}
\end{figure}
 

Javalite programs and expressions are written in a syntax specified by the 
grammar in~\figref{fig:machine-syntax}. The production rules correspond
to the various features supported by Javalite such as classes, fields,
methods, and expressions. 

\figref{fig:machine-syntax} specifies
the machine syntax for Javalite. The expression $e$ is equivalent to
a CEKS machine's control string. The 

\subsection{Reduction Rules}

We first present the reduction rules 


\subsubsection{Basic Reduction Rules}

Most rules presented here - short discussion

\subsubsection{Store Update Rules}

Interesting rules presented here; more elaborate discussion

Start off with helper functions for rules that manipulate the store...

The function $\mathbb{VS}(L,R,r,f)$ constructs the value-set given a
heap, reference, and desired field such that
$(l^\prime\ \phi^\prime\wedge\phi) \in \mathbb{VS}(L,R,r,f)$ if and
only if
\[
\begin{array}{l}
  \exists (l\ \phi) \in L(r) ( \\
  \ \ \ \ \ \ \ \ \ \exists r^\prime \in R(l,f) ( \\
  \ \ \ \ \ \ \ \ \ \ \ \ \ \ \ \ \ \ \exists (l^\prime\ \phi^\prime) \in L(r^\prime) (\mathbb{S}(\phi\wedge\phi^\prime))))
\end{array}
\]
where $\mathbb{S}(\phi)$ returns true if $\phi$ is satisfiable.

The strengthen function $\mathbb{ST}(L,r,\phi^\prime)$ strengthens every
constraint from the reference $r$ and keeps only location-constraint
pairs that are satisfiable after strengthening. Formally,
$(l\ \phi\wedge\phi^\prime)\in\mathbb{ST}(L,r,\phi^\prime)$ if and
only if $\exists (l\ \phi)\in
L(r)\wedge\mathbb{S}(\phi\wedge\phi^\prime)$

The empty-reference function $\mathbb{ER}(L,\phi^\prime) = \{r\ |\ L(r) \neq
\emptyset \wedge \forall(l,\phi) \in L(r)(\neg \mathbb{S}(\phi \wedge
\phi^\prime))\}$ searches the heap for references that become
disconnected from all their locations after strengthening. These
references, if reachable, imply the heap is no longer valid on the
current search path. As such, the symbolic execution algorithm should
backtrack. This check is similar to a feasibility check in classic
symbolic execution with only primitive data types.

The consistency function is critical to the soundness of the algorithm
as it detects when a symbolic heap becomes invalid along a path,
similar to a feasibility check when doing classical symbolic execution
with just primitives. As the constraints in the heap are strengthened
with different aliasing requirements, it is possible to reach a point
where the heap is no longer connected. Meaning, a valid reference is
live, either in the local environment or the continuation, the reaches
another reference that is no longer connected to any locations due to
strengthening. The function relies on the empty-reference function to
identify disconnected references. The function in essence checks every
reference in the local environment and every reference found in the
continuation, as these are all considered live. This operation similar
to garbage collection where the local environment and stack are
inspected to find the roots of the heap for the scan.


The consistency function relies on two auxiliary functions which are
informally defined. The function $\mathrm{ref}(\eta,\cfgnt{k})$
inspects the local environment and continuation for all live
references, and it returns those references in a set. The function
$\mathrm{reach}(L, R, r, r^\prime)$ returns true if $r^\prime$ is
reachable from $r$ in the heap and false otherwise. The consistency function $\mathbb{C}(L,R,X,\eta,k)$ is now defined as
\[
 \left\{ \begin{array}{rl} 
        0 & \exists r \in \mathrm{ref}(\eta, \cfgnt{k})\ (\exists r^\prime \in X\ (\mathrm{reach}(L, R, r,r^\prime))) \\ 
        1 & \mbox{otherwise}\end{array}\right .
\]


\begin{figure*}[t]
\begin{center}
\mprset{flushleft}
\begin{mathpar}
	\inferrule[Variable lookup]{}{
      (L\ R\ g\ \eta\ \cfgnt{x}\ k) \rightarrow (L\ R\ g\ \eta\ \eta(\cfgnt{x})\ k)
	}
\and
	\inferrule[Field Access(eval)]{}{
      (L\ R\ g\ \eta\ (e\ \$\ \cfgnt{f})\ k) \rightarrow (L\ R\ g\ \eta\ e\ (\cfgt{init}\ \cfgnt{f}\ (*\ \$\ \cfgnt{f} \rightarrow k)))
	}
\and
%	\inferrule[Field Access (NULL)]{
 %     L(r) = \emptyset
 %   }{
 %     (L\ R\ g\ \eta\ r\ (*\ \$\ \cfgnt{f} \rightarrow k)) \rightarrow 
%      (L[r \mapsto \{(\bot,\phi_T)\}]\ R\ \eta\ r\ (*\ \$\ \cfgnt{f} \rightarrow k))
%	}
%\and
	\inferrule[Lazy Initialization]{
	  \Lambda = \{ l \mid \exists \phi\ ((l,\phi) \in L(r) \wedge  R(l,f) = \emptyset\}\\
      l_x = \mathrm{min}_l(\Lambda) \\\\
      \Lambda \neq \emptyset \\
      \mathrm{type}(\cfgnt{f}) = \mathrm{C}_r\\
      \mathrm{init}_r() = r_f\\ 
      \mathrm{init}_l(\mathrm{C}_r) = l_f \\\\
      \rho = \{ (r^\prime,\ \phi^\prime,\ l^\prime) \mid \mathrm{isInit}(r^\prime) \wedge (l^\prime, \phi^\prime) \in L(r^\prime) \wedge \mathrm{type}(l^\prime) = \mathrm{C}_r \} \\\\
      \theta_\mathit{alias} = \{ ( l ,\ \phi) | (r^\prime,\ \phi^\prime,\ \l) \in \rho \wedge \phi = (\phi^\prime \wedge r^\prime \neq r_\mathit{null} \wedge r^\prime = r_f \wedge (\wedge_{(r^{\prime\prime},\ \phi^{\prime\prime},\ l^{\prime\prime}) \in \rho\ (r^{\prime\prime} \neq r^\prime)} r^{\prime\prime} \neq r_f )) \} \\\\
      \theta_\mathit{new} = \{(l_f,\ r_f \neq r_\mathit{null} \wedge (\wedge_{(r^\prime,\ \phi^\prime,\ l^\prime) \in \rho} r_f \ne r^\prime))\}\\\\
      \theta_\mathit{null} = \{ (l_\mathit{null},\ r_f = r_\mathit{null}) \}\\\\
      \theta = \theta_\mathit{alias} \cup \theta_\mathit{new} \cup \theta_\mathit{null}
    }{
      (L\ R\ g\ \eta\ r\ (\cfgt{init}\ \cfgnt{f}\ k)) \rightarrow 
      (L[r_f \mapsto \theta]\ R[ (l_x,f) \mapsto r_f ]\ g\ \eta\ r\ (\cfgt{init}\ \cfgnt{f}\ k))
	}
\and
	\inferrule[Lazy Initialization (end)]{
	\forall (l,\phi) \in L(r)\ (R(l,f) \neq \emptyset)
    }{
      (L\ R\ g\ \eta\ r\ (\cfgt{init}\ \cfgnt{f}\ k)) \rightarrow 
      (L\ R\ g\ \eta\ r\ k)
	}
\and
	\inferrule[New]{
      \mathrm{fresh}_r(\cfgnt{C}) = r \\
      \mathrm{fresh}_l(\cfgnt{C}) = l \\\\
      R^\prime = R[\forall \cfgnt{f} \in \mathrm{C}\ ((l\ \cfgnt{f}) \mapsto \mathrm{fresh}_r(\mathrm{type}(\cfgnt{f})))] \\\\
      L^\prime = L[r \mapsto \{(l\ \phi_T)\}]
    }{
      (L\ R\ g\ \eta\ (\cfgt{new}\ \cfgnt{C})\ k) \rightarrow 
      (L^\prime\ R^\prime\ g\ \eta\ r\ k)
	}
\and
	\inferrule[Field Access]{
      \forall (l,\phi) \in L(r)\ (l = l_{\mathit{null}} \rightarrow \neg \mathbb{S}(\phi \wedge g)) \\
      \mathrm{fresh}_r() = r_f
    }{
      (L\ R\ g\ \eta\ r\ (*\ \$\ \cfgnt{f} \rightarrow k)) \rightarrow 
      (L[r_f \mapsto \mathbb{VS}(L,R,r,f,g)]\ R\ g\ \eta\ r_f\ k)
	}
\and
    \inferrule[Equals (l-operand eval)]{}{
      (L\ R\ g\ \eta\ (\cfgnt{e}_0 = \cfgnt{e}) \ k) \rightarrow 
      (L\ R\ g\ \eta\ \cfgnt{e}_0\ (\cfgt{*}\; \cfgt{=}\; \cfgnt{e} \rightarrow \cfgnt{k}))
    }
\and
    \inferrule[Equals (r-operand eval)]{}{
    (L\ R\ g\ \eta\ \cfgnt{v}\ (\cfgt{*}\; \cfgt{=}\; \cfgnt{e} \rightarrow \cfgnt{k})) \rightarrow
    (L\ R\ g\ \eta\ \cfgnt{e}\ (\cfgnt{v}\; \cfgt{=}\; \cfgt{*} \rightarrow \cfgnt{k}))
    }
\and
    \inferrule[Equals (bool)]{
    \cfgnt{v}_0 \in \{\cfgt{true}, \cfgt{false}\} \\
    \cfgnt{v}_1 \in \{\cfgt{true}, \cfgt{false}\} \\ 
    \mathrm{eq?}(v_0, v_1) = \cfgnt{v}_r}{
    (L\ R\ g\ \eta\ \cfgnt{v}_0\ (\cfgnt{v}_1\; \cfgt{=}\; \cfgt{*} \rightarrow \cfgnt{k})) \rightarrow
    (L\ R\ g\ \eta\ \cfgnt{v}_r\ \cfgnt{k})
    }
\and
    \inferrule[Equals (references-true)]{
    \cfgnt{v}_0 \not\in \{\cfgt{true}, \cfgt{false}\} \\
    \cfgnt{v}_1 \not\in \{\cfgt{true}, \cfgt{false}\}\\\\
    \theta_\alpha = \{\phi_0 \wedge \phi_1\mid\exists (l\ \phi_0) \in L(v_0)(\exists (l\ \phi_1) \in L(v_1) ( \mathbb{S}(\phi_0 \wedge \phi_1)))\} \\\\
    \theta_0 = \{\phi_0 \mid \exists (l_0\ \phi_0) \in L(v_0) \wedge \forall (l_1\ \phi_1) \in L(v_1)(l_0 \neq l_1)\} \\\\
    \theta_1 = \{\phi_1 \mid \exists (l_1\ \phi_1) \in L(v_1) \wedge \forall (l_0\ \phi_0) \in L(v_0)(l_0 \neq l_1)\} \\\\
    \phi_g = (\vee_{\phi_\alpha\in\theta_\alpha}\phi_\alpha)\wedge(\wedge_{\phi_0 \in \theta_0} \neg \phi_0)  \wedge(\wedge_{\phi_1
    \in \theta_1} \neg \phi_1)\\ g^\prime = g \wedge \phi_g}{
     % L^\prime = L[\forall r (\exists (\cfgnt{l}\ \phi )\in L(r)(r \mapsto \mathbb{ST}(L,r,\phi_g)))]\\\\
      %X = \mathbb{ER}(L, \phi_g) \\ \mathbb{C}(L^\prime, R, X, \eta, k) = 1}
    (L\ R\ g\ \eta\ \cfgnt{v}_0\ (\cfgnt{v}_1\; \cfgt{=}\; \cfgt{*} \rightarrow \cfgnt{k})) \rightarrow
    (L^\prime\ R\ g^\prime\ \eta\ \cfgt{true}\ \cfgnt{k})
    }
\and
    \inferrule[Equals (references-false)]{
    \cfgnt{v}_0 \not\in \{\cfgt{true}, \cfgt{false}\} \\
    \cfgnt{v}_1 \not\in \{\cfgt{true}, \cfgt{false}\}\\\\
    \theta_\alpha = \{\phi_0 \rightarrow \neg \phi_1\mid\exists (l\ \phi_0) \in L(v_0)(\exists (l\ \phi_1) \in L(v_1) ( \mathbb{S}(\phi_0 \wedge \phi_1)))\} \\\\
    \theta_0 = \{\phi_0 \mid \exists (l_0\ \phi_0) \in L(v_0) \wedge \forall (l_1\ \phi_1) \in L(v_1)(l_0 \neq l_1)\} \\\\
    \theta_1 = \{\phi_1 \mid \exists (l_1\ \phi_1) \in L(v_1) \wedge \forall (l_0\ \phi_0) \in L(v_0)(l_0 \neq l_1)\} \\\\
    \phi_g = (\wedge_{\phi_\alpha\in\theta_\alpha}\phi_\alpha)\vee((\vee_{\phi_0 \in \theta_0} \phi_0)  \vee(\vee_{\phi_1
    \in \theta_1} \phi_1)) \\ g^\prime = g \wedge \phi_g}{
    (L\ R\ g\ \eta\ \cfgnt{v}_0\ (\cfgnt{v}_1\; \cfgt{=}\; \cfgt{*} \rightarrow \cfgnt{k})) \rightarrow
    (L^\prime\ R\ g^\prime\ \eta\ \cfgt{true}\ \cfgnt{k})
    }
    
  %  \\\\
  %    L^\prime = L[\forall r (\exists (\cfgnt{l}\ \phi )\in L(r)(r \mapsto \mathbb{ST}(L,r,\phi_g)))]\\\\
  %    X = \mathbb{ER}(L, \phi_g) \\ \mathbb{C}(L^\prime, R, X, \eta, k) = 1}{
 %   (L\ R\ g\ \eta\ \cfgnt{v}_0\ (\cfgnt{v}_1\; \cfgt{=}\; \cfgt{*} \rightarrow \cfgnt{k})) \rightarrow
 %   (L^\prime\ R\ \eta\ \cfgt{false}\ \cfgnt{k})
 %   }
\and
   \inferrule[Method Invocation (object eval)]{}{
    (L\ R\ g\ \eta\ \lp\cfgnt{e}_0\ \cfgt{@}\ \cfgnt{m}\ \cfgnt{e}_1\rp\ \cfgnt{k}) \rightarrow
    (L\ R\ g\ \eta\ \cfgnt{e}_0\ (\cfgt{*}\ \cfgt{@}\ \cfgnt{m}\ \cfgnt{e}_1\ \rightarrow \cfgnt{k}))
   }
   
\and
   \inferrule[Method Invocation (arg eval)]{}{
    (L\ R\ g\ \eta\ \cfgnt{v}_0\ (\cfgt{*}\ \cfgt{@}\ \cfgnt{m}\ \cfgnt{e}_1\ \rightarrow \cfgnt{k})) \rightarrow
    (L\ R\ g\ \eta\ \cfgnt{e}_1\ (\cfgnt{v}_0\ \cfgt{@}\ \cfgnt{m}\ \cfgt{*}\ \rightarrow \cfgnt{k}))
   }
%\and
%   \inferrule[Method Invocation (arg0 eval)]{}{
%    (L\ R\ \eta\ \cfgnt{v}_0\ (\cfgt{*}\ \cfgt{@}\ \cfgnt{m} \lp\cfgnt{e}_1\ \cfgnt{e}_2\ ...\rp \rightarrow \cfgnt{k})) \rightarrow
%    (L\ R\ \eta\ \cfgnt{e}_1\ (\cfgnt{v}_0\ \cfgt{@}\ \cfgnt{m}\ ()\ \cfgt{*}\ \lp\cfgnt{e}_2\ ...\rp \rightarrow \cfgnt{k}))
%   }
%\and
%   \inferrule[Method Invocation (argi eval)]{}{
%    (L\ R\ \eta\ \cfgnt{v}_i\ (\cfgnt{v}_0\ \cfgt{@}\ \cfgnt{m}\ (\cfgnt{v}_1\ ...)\ \cfgt{*}\ \lp\cfgnt{e}_{i+1}\ \cfgnt{e}_{i+2}\ ...\rp \rightarrow \cfgnt{k})) \rightarrow
%    (L\ R\ \eta\ \cfgnt{e}_{i+1}\ (\cfgnt{v}_0\ \cfgt{@}\ \cfgnt{m}\ (\cfgnt{v}_1\ ...\ v_i)\ \cfgt{*}\ \lp\cfgnt{e}_{i+2}\ ...\rp \rightarrow \cfgnt{k}))
%   }
%\and
%   \inferrule[Method Invocation (args)]{}{
%    (L\ R\ \eta\ \cfgnt{v}_n\ (\cfgnt{v}_0\ \cfgt{@}\ \cfgnt{m}\ (\cfgnt{v}_1\ ...)\ \cfgt{*}\ \lp\rp \rightarrow \cfgnt{k})) \rightarrow
%    (L\ R\ \eta\ (\cfgt{raw}\ \cfgnt{v}_0\ \cfgt{@}\ \cfgnt{m}\ (\cfgnt{v}_1\ ...\ v_n))\ \cfgnt{k})
%   }
%\and
%   \inferrule[Method Invocation (no args)]{}{
%    (L\ R\ \eta\ \cfgnt{e}_0\ (\cfgt{*}\ \cfgt{@}\ \cfgnt{m}\ \lp\rp \rightarrow \cfgnt{k})) \rightarrow
%    (L\ R\ \eta\ (\cfgt{raw}\ \cfgnt{v}_0\ \cfgt{@}\ \cfgnt{m}\ ()\ \cfgnt{k}))
%   }
\and
   \inferrule[Method Invocation (raw)]{
    \mathrm{lookup}(\cfgnt{m}) = \lp\cfgnt{T}\ \cfgnt{m}\ \lb\cfgnt{T}\ \cfgnt{x}\rb\ \ e_m\rp \\
    \eta_m = \eta[\cfgt{this} \mapsto \cfgnt{v}_0][\cfgnt{x} \mapsto \cfgnt{v}_1]\ }{
    (L\ R\ g\ \eta\ \cfgnt{v}_1\ (\cfgnt{v}_0\ \cfgt{@}\ \cfgnt{m}\ \cfgt{*}\ \rightarrow \cfgnt{k})) \rightarrow
    (L\ R\ g\ \eta\ (\cfgt{raw}\ \cfgnt{v}_0\ \cfgt{@}\ \cfgnt{m}\ \cfgnt{v}_1\ \cfgnt{k})
   }
\and
   \inferrule[Method Invocation]{
    \mathrm{lookup}(\cfgnt{m}) = \lp\cfgnt{T}\ \cfgnt{m}\ \lb\cfgnt{T}\ \cfgnt{x}\rb\ \ e_m\rp \\
    \eta_m = \eta[\cfgt{this} \mapsto \cfgnt{v}_0][\cfgnt{x} \mapsto \cfgnt{v}_1]\ }{
    (L\ R\ g\ \eta\ (\cfgt{raw}\ \cfgnt{v}_0\ \cfgt{@}\ \cfgnt{m}\ \cfgnt{v}_1\ \cfgnt{k}) \rightarrow
    (L^\prime\ R^\prime\ g\ \eta_m\ \cfgnt{e}_m\ (\cfgt{pop}\ \eta\ \cfgnt{k}))
   }

\end{mathpar}
\end{center}
\caption{Uber-lazy state reductions}
\label{fig:expr:red}
\end{figure*}

\begin{figure*}[t]
\begin{center}
\mprset{flushleft}
\begin{mathpar}
	\inferrule[ITE (expr-eval)]{}{
      (L\ R\ g\ \eta\ (\cfgt{if}\ \cfgnt{e}_0\ \cfgnt{e}_1\ \cfgt{else}\ \cfgnt{e}_2)\ k) \rightarrow (L\ R\ g\ \eta\ \cfgnt{e}_0\ (\cfgt{if}\ \cfgnt{*}\ \cfgnt{e}_1\ \cfgt{else}\ \cfgnt{e}_2) \rightarrow k)
	}
\and
	\inferrule[ITE (true)]{}{
       (L\ R\ g\ \eta\ \cfgt{true}\ (\cfgt{if}\ \cfgnt{*}\ \cfgnt{e}_1\ \cfgt{else}\ \cfgnt{e}_2) \rightarrow k) \rightarrow (L\ R\ g\ \eta\ \cfgnt{e}_1\  k)
	}
\and
	\inferrule[ITE (false)]{}{
       (L\ R\ g\ \eta\ \cfgt{false}\ (\cfgt{if}\ \cfgnt{*}\ \cfgnt{e}_1\ \cfgt{else}\ \cfgnt{e}_2) \rightarrow k) \rightarrow (L\ R\ g\ \eta\ \cfgnt{e}_2\  k)
	}
\and
	\inferrule[Field Write (eval)]{}{
       (L\ R\ g\ \eta\ (\cfgnt{x}\ \cfgt{\$}\ \cfgnt{f}\ \cfgt{:=}\ \cfgnt{e})\ k) \rightarrow (L\ R\ g\ \eta\ \cfgnt{e}\ (\cfgt{init}\ \cfgnt{f}\ (\cfgnt{x}\ \cfgt{\$}\ \cfgnt{f}\ \cfgt{:=}\ \cfgnt{*}\ \rightarrow\ k)))
	}
\and
	\inferrule[Field Write]{
	r_x = \eta(\cfgnt{x})\\
      	\Lambda = \{ l_i | (l_i,*) \in L(r_x) \wedge R(l_i,f)= \emptyset\}\\\\
      \Lambda = \emptyset \\
      %(\bot,*) \notin L(r_x)  \\\\
     \{(l_{null},\phi)|(l_{null},\phi) \in L(r) \wedge \mathbb{S}(\phi)\}=\emptyset \\\\
      \Psi_x =\{(\phi,l,r^\prime)\ |\ \exists(\cfgnt{l},\phi) \in L(r_x) \wedge \exists r^\prime \in R(\cfgnt{l,f}) \}\\\\
      X = \{ (r^\prime, l^\prime, \theta)| (\phi^\prime,l^\prime,r^\prime)\in \Psi_x \wedge \theta = \mathbb{ST}(L,r,\phi^\prime) \cup \mathbb{ST}(L,r^\prime,\neg\phi^\prime) \}\\\\
      Y= \{l\ |\ (r^\prime, l^\prime, \theta) \in X\}\\ R^\prime = R[\forall l \in Y ((l,f) \mapsto \mathrm{fresh}_r(\mathrm{C}_f)) ]\\\\
      L^\prime = L[\forall l^\prime\ (\exists (r^\prime,l^\prime,\theta) \in X\ (\exists r^{\prime\prime} = R (l^\prime,f) ( r^{\prime\prime} \mapsto \theta) ) )]
    }{
      (L\ R\ g\ \eta\ \cfgnt{r}\ (\cfgnt{x}\ \cfgt{\$}\ \cfgnt{f}\ \cfgt{:=}\ \cfgnt{*}\ \rightarrow\ k)) \rightarrow 
      (L^\prime R^\prime\ g\ \eta\ k)
	}	
\and
   \inferrule[Begin (no args)]{}{
    (L\ R\ g\ \eta\ (\cfgt{begin})\ \cfgnt{k}) \rightarrow
    (L\ R\ g\ \eta\ \cfgnt{k})
   }
\and
   \inferrule[Begin (arg0 eval)]{}{
    (L\ R\ g\ \eta\ (\cfgt{begin}\ \lp \cfgnt{e}_0\ \cfgnt{e}_1\ ...\rp) \rightarrow \cfgnt{k}) \rightarrow
    (L\ R\ g\ \eta\ \cfgnt{e}_0\ (\cfgt{begin}\ \cfgt{*}\ \lp\cfgnt{e}_1\ ...\rp) \rightarrow \cfgnt{k})
   }
\and
   \inferrule[Begin (argI eval)]{}{
    (L\ R\ g\ \eta\ v\ (\cfgt{begin}\ \cfgt{*}\ \lp\cfgnt{e}_i\ \cfgnt{e}_{i+1}\ ...\rp) \rightarrow \cfgnt{k}) \rightarrow
    (L\ R\ g\ \eta\ \cfgnt{e}_i\ (\cfgt{begin}\ \cfgt{*}\ \lp\cfgnt{e}_{i+1}\ ...\rp) \rightarrow \cfgnt{k})
   }
\and
   \inferrule[Begin (argN eval)]{}{
    (L\ R\ g\ \eta\ v\ (\cfgt{begin}\ \lp\cfgnt{e}_{n}\rp) \rightarrow \cfgnt{k}) \rightarrow
    (L\ R\ g\ \eta\ \cfgnt{e}_n\ (\cfgt{begin}\ \cfgt{*}\ \lp\rp) \rightarrow \cfgnt{k})
   }
\and
   \inferrule[Begin]{}{
    (L\ R\ g\ \eta\ v\ (\cfgt{begin}\ \cfgt{*}\ \lp\rp \rightarrow \cfgnt{k})) \rightarrow
    (L\ R\ g\ \eta\ v\ \cfgnt{k})
   }	
\and
   \inferrule[Variable Declaration (eval)]{}{
    (L\ R\ g\ \eta\ \lp\cfgt{var}\ \cfgnt{T}\ \cfgnt{x}\ \cfgt{:=}\ \cfgnt{e}_0\ \cfgt{in}\ \cfgnt{e}_1\rp\ \cfgnt{k})) \rightarrow
    (L\ R\ g\ \eta\ \cfgnt{e}_0\ \lp\cfgt{var}\ \cfgnt{T}\ \cfgnt{x}\ \cfgt{*}\ \cfgt{:=}\ \cfgt{in}\ \cfgnt{e}_1\rp \rightarrow \cfgnt{k})
   }	
\and
   \inferrule[Variable Declaration]{
   \mathrm{fresh}_r(\cfgnt{T}) = r \\
   \mathrm{fresh}_l(\cfgnt{T}) = l \\
   \eta^\prime = \eta[x \mapsto  r]\\\\
   R^\prime = R[\forall \cfgnt{f} \in \mathrm{C}\ ((l\ \cfgnt{f}) \mapsto \mathrm{fresh}_r(\mathrm{type}(\cfgnt{f})))] \\\\
   L^\prime = L[r \mapsto \{(l\ \phi_T)\}]
   }{
    (L\ R\ g\ \eta\ v\ \lp\cfgt{var}\ \cfgnt{T}\ \cfgnt{x}\ \cfgt{*}\ \cfgt{:=}\ \cfgt{in}\ \cfgnt{e}_1\rp \rightarrow \cfgnt{k}) \rightarrow
    (L^\prime\ R^\prime\ g\ \eta^
    \prime\ \cfgnt{e}_1\ (\cfgt{pop}\ \eta\ \cfgnt{k}))
   }	
\and
   \inferrule[Pop]{}{
    (L\ R\ g\ \eta\ (\cfgt{pop}\ \eta_0\ \cfgnt{k})) \rightarrow
    (L\ R\ g\ \eta_0\ \cfgnt{k})
   }
\end{mathpar}
\end{center}
\caption{Uber-lazy state reductions}
\label{fig:expr:red}
\end{figure*}

%% Expression syntax




\section{Related Work}
\label{related}

The related work goes here.
%\section{System State}
The program state is represented using a path condition, a program location, a stack, a symbolic heap, and a symbolic store. The path condition is a collection of predicates over the program inputs that indicates constraints on the values of those inputs at the present location. The program location indicates the instruction to be executed presently. The symbolic heap is a mapping from heap locations symbolic objects. 
\paragraph{Heap Symbols}
A heap symbol $\pi$ is a symbol which is created in the process of performing heap operations. The dynamic nature of heap symbols distinguishes them from statically-created symbols like those found in the path condition. Heap symbols may be symbolic primitives, symbolic references, symbolic locations, or symbolic types.
\paragraph{Symbolic References}
Symbolic references are symbolic points-to relations. The symbolic reference has a state parameter that indicates whether the reference is uninitialized, non-null, or initialized. The uninitialized and non-null states indicate that the location pointed to by the reference has yet to be resolved. The non-null state reflects the additional constraint that the reference does not point to the null location. References in the initialized state are associated with constraints reflecting which location the reference points to and under what condition it points to that location. References point to one and only one location at a time, so the conditions must be mutually exclusive, yet collectively exhaustive.
\paragraph{Symbolic Locations}
A symbolic location $h$ represents the index of a slot in the heap structure that holds a symbolic object. Each symbolic location represents a unique slot on the symbolic heap. There are two special symbolic locations null and non-null, which are never in the symbolic heap.
\paragraph{Symbolic Objects}
A symbolic object represents the contents of a heap slot. A symbolic object is characterized by a symbolic type, and contains a mapping from fields to heap symbols.
\paragraph{Constrained Symbol}
A constrained symbol $\psi$ is a symbol that assumes a value $\pi$ contingent upon the satisfiability of a constraint condition $\phi$.
\begin{equation}
\psi\models \phi\uparrow \Rightarrow\pi
\end{equation}
\paragraph{Symbolic Value Set}
A symbolic value set $\theta$ is a set of constrained symbols $\theta\colon \{\psi _1,\psi _2,...,\psi _n\}$, for which the constraints are mutually exclusive and collectively exhaustive.
\paragraph{Symbolic Store}
The symbolic store is a mapping from heap symbols to symbolic value sets $\mathbb{S}\colon \pi \mapsto \theta$. The abstract store represents a set of heaps common to the current program execution path. 
\paragraph{Heap}
The symbolic heap $\mathbb{H}$ is an indexed set of symbolic objects. Like the path condition, the symbolic heap contains heap state which is common to all heaps on the current execution path. 
\paragraph{Stack}
The Java Virtual Machine uses a system stack.
%\section{Semantic Model}
Program execution proceeds an in standard symbolic execution, with additional rules to handle heap constructs. The Java Virtual Macine (JVM) implements five classes of semantic rules, a categorization which we find relevant to symbolic execution: 1) load/store instructions, 2) arithmetic instructions, 3) object creation / manipulation instructions, 4) control transfer instructions, and 5) assume/assert instructions. We will deal with each class of instruction in sequence.
\subsection{Reference Target Resolution}
Load and store instructions take symbolic references as arguments. Since an uninitialized symbolic reference may point to any one of a number of symbolic locations, we need to be able to resolve which locations are feasible targets of a particular references.
Location resolution begins with an empty list of heap locations, and then adds locations by comparing locations on the symbolic heap to the constraints on the symbolic reference obtained from the abstract store. Locations are checked in the following order:
\begin{compactenum}
\item If the null location is feasible, return a null pointer exception and terminate execution.
\item If the non-null location is feasible, create a new symbolic location of the proper type and add it to the symbolic heap. Search the heap for type-compatible objects and add those to the symbolic value set for the reference, and remove the non-null location from the symbolic value set.
\item Check the symbolic heap for type-compatible objects, comparing those against the constraints in the symbolic value set. Add any feasible objects to the feasible location set.
\end{compactenum}

\subsection{Read}
The load instruction has two arguments: a reference $r$ and a field index $f$. If $r$ is a symbolic location, then load simply accesses the field and returns the value contained there. If $r$ is a symbolic reference, then the following steps are taken, in order.
\begin{compactenum}
\item Resolve a list of feasible targets as described above.
\item Access the fields of the feasible targets, gathering a set of the symbolic values contained therein. 
\item Form a new symbolic value, and create a new entry in the symbolic store mapping the value to the symbolic value set. Return the new symbolic value.
\end{compactenum}

\subsection{Write}
The write instruction takes three arguments: a reference $r$, a field $f$, and a value $v$. If $r$ is a symbolic location, then the target field is written with the value directly. If $r$ is a symbolic reference, write proceeds as follows:
\begin{compactenum}
\item Resolve a list of feasible targets.
\item Access the fields of the feasible targets, performing a conditional write. The conditional write works by modifying the constraint equation as follows:
\begin{equation}
  ((r\rightarrow h_n) \Rightarrow f\rightarrow v)  \wedge (\lnot (r\rightarrow h_n) \Rightarrow Eqn_o_l_d)
\end{equation}
\end{compactenum}

\subsection{Arithmetic Instructions}
\subsection{Object Creation/Manipulation Instructions}
\subsection{Control Transfer Instructions}
Control transfer instructions are those instructions that have a "branching" nature. They are instructions that have more than one possible program location successor. Control transfer instructions are executed by comparing the compare condition to the constraints contained in the symbolic store. Those values in the store for which the constraints are now infeasible are eliminated. 
\subsection{Assume / Assert Instructions}
%\section{Example}
%\section{Bytecodes}

PUTFIELD needs to remove path constraints from PC that enforcing equality between references.

\noindent \textbf{Assume}: all symbolic locations are concertized lazily. Although the algorithm is not specific to any particular initialization strategy, the presentation assumes a lazy initialization. Extending to lazier initialization may be non-trivial even though this reduction is orthogonal to the lazier reduction (i.e., this reduction should further improve the performance of the lazier algorithm).

\noindent \textbf{Assume}: all variables, symbolic or otherwise, are non-primitive (i.e. objects). \textit{Must relax this assumption because you need to do some interesting things with primitives as they relate to getfield}.

This papers uses subsumption, which is expressed as a subtyping relation $\leq$ over types $T$. For classes $C$ and $D$, $C \leq D$ iff either $C = D$ or the class declaration for $C$ is \texttt{class C extends B $\{\ldots\}$} for some $B \leq D$. For example, in $A \leq B \leq C \leq D$, $D$ is the supertype, and if you have something that is an instance of $A$ but currently viewed as $B$, then you can move it toward $D$ in a typecast (up the hierarchy).

Heap locations range over positive natural numbers $H \subseteq \mathbb{N}_{\geq 0}$. Every heap location has a special variable $T_h$ used in constraints over the type of the object stored in the heap location. The varable $\mathit{SH}$ is the set of heap locations created when concretizing symbolic variables. The set of constraints over the type stored in the heap location is given by the function $\mathtt{C}(h)$. Constraints are of the form $T_h \sim T$ or $T \sim T$ where $\sim\ \in \{\leq, =, \not =\}$. The initial type hierarchy for the program is expressed as a set of constraints $C_\mathrm{init}$. The set contains all relationships needed to describe the entire hierarchy.

For a set of constraints $C$, the function $\mathtt{SAT}(C) \mapsto \{0,1\}$ returns 1 if the constraints are satisfiable and 0 otherwise. The usual Boolean connectives are used as expected. The function $\mathtt{Type}(h)$ returns the actual type of the object in the heap location and the function $\mathtt{Obj}(h)$ returns the object at the location.

Each variable $v$ is associated with a set of heap locations $H(v) = \{h_0, h_1, \ldots, h_n\}$ that represents an equivalence class (i.e., each heap location yields the same execution path and behavior up to the current point of execution).  The representative object for a given variable (i.e., the one that is currently being used by the variable) is given by the function $I(v) \mapsto H(v)$. The set of heap locations and the representative location are part of the meta-deta for the variable. This meta-data follows the variable through the program execution and is appropriately copied on assignment to other variables such that each variable has its own copy of the meta-data that is separate from other copies.

Finally, there is a global variable $\mathit{PC}$ that represents the path constraint along the current path of exploration. This path constraint is used to track relationships between symbolic variables such as equality. Properties of symbolic variables are represented in the path constraint by creating special variables for the representative and the set of represented objects. For a variable $v$, the special variable $I_v$ is the representative heap opbject and the special variable $H_v$ is the set of associated heap locations. It is assumed the $v$ is alpha-renamed to be unique in the path constraint. Finally, we use the special value \texttt{SYM} to denote a symbolic variable that is yet to be initialized.

\subsection{Reference}
\noindent \textbf{\texttt{GETFIELD}}: the bytecode behavior depends on the field operand: concrete, concrete though initialized from symbolic, and symbolic. Each case is enumerated:
\begin{compactenum}
\item Referencing a concrete field: the bytecode has default behavior returning the field.
\item Referencing an initialized field from a symbolic variable (i.e., the base type for the field is initialized from a symbolic object): the bytecode may have multiple outcomes; it partitions the equiavalence class to group heap locations with objects that have common values for the field.
\item Referencing a symbolic variable that has yet to be initialized: the bytecode has two outcomes: one that returns \texttt{null} and another that builds the potential equivalence class, chooses a representative location, and returns that location.
\end{compactenum}
Consider the code
\begin{lstlisting}
// The declared type of f is F
T t = b.f;
\end{lstlisting}
For the case where \texttt{b.f} is an already initialized symbolic variable, $h = \mathtt{I}(\mathtt{b})$, $\mathtt{H}(\mathtt{b})$ is partitioned into disjoint sets, $S_0, P_1, \ldots, P_n$, with $n+1$ partitions, or choices. The first set $S_0$ is a special set that includes $h$, the representative object for \texttt{b}, and any members of the equivalence class that either have the same field value for \texttt{b.f} or the field value is a symbolic variable that has yet to be initialized:
\begin{eqnarray*}
S_0 &=& \{h_i \mid h_i \in \mathtt{H}(\mathtt{b})\ \wedge \\
    & & (\mathtt{Obj}(h).\mathtt{f} = \mathtt{Obj}(h_i).\mathtt{f}\ \vee\ \mathtt{Obj}(h_i).\mathtt{f} = \mathtt{SYM})\}
\end{eqnarray*}
For this special case of $S_0$,  $\mathtt{H}_0(\mathtt{b}) = S_0$, and $I_0(\mathtt{b}) = h$ where the subscript indicates the choice number in the choice generator (i.e., the partition size may change but not the representative heap location). The other partitions group common values of the field such that
\begin{compactitem}
\item $\forall h_i, h_j \in P_i,\ \mathtt{Obj}(h_i).f = \mathtt{Obj}(h_j).f$ 
\item $\mathtt{H}_i(\mathtt{b}) = P_i$ 
\item $\exists h \in P_i,\ \mathtt{I}_i(\mathtt{b}) = h$
\end{compactitem}
The partitions are maximal and represent a unique value that has been created thus far in the program execution. The non-initialized symbolic members of the partition all belong to $S_0$ as the original representative heap location $h$ captures that these other aliases were intended to have the same field value for field \texttt{f} before the split (i.e., the value is assigned programatically but the change was only reflected in the representative heap location). Once the choice generator is created over the different partitions, the bytecode returns the requested field value of the representative as expected.

Returning to the third behavior of the bytecode, the case in which the accessed field has yet to be initialized, then the bytecode follows lazy initialization creating a \texttt{null} instance, a new instance that is the representative, and the alias set. When creating the alias set, the new instance should be included in the set, as well as any prior object created in concretization of symbolic variables that is type compatible with the new instance. Recall that $\mathit{SH}$ is set of locations in the symbolic heap and $C_\mathrm{init}$ is the set of constraints describing the type hierarchy, assuming $h$ is the heap location of the new instance of the type, then 
\begin{compactitem}
\item $\mathtt{I}(\mathtt{b.f}) = h$
\item $\forall h_i \in \mathit{SH}, \mathtt{SAT}(C_\mathrm{init} \cup \{T_h \leq \mathtt{Type}(h_i)\}) \rightarrow h_i \in \mathtt{H}(\mathtt{b.f})$  
\item $\mathtt{C}(\mathtt{b.f}) = C_\mathrm{init} \cup \{T_h \leq \mathtt{Type}(h)\}$  
\end{compactitem}
The $C_\mathrm{init}$ set constains relationships in the class hierarchy with the correct sub-types and super-types as they relate to the delcared type of the object.

\noindent \textbf{\texttt{GETSTATIC}}: the bytecode is handled similarly to \texttt{GETFIELD}. 

\noindent \textbf{\texttt{ALOAD}}: the bytecode is handled similarly to \texttt{GETFIELD}. 

\subsection{Comparison}

\noindent \textbf{\texttt{IF\_ACMPEQ}}: the bytecode may return both the \texttt{true} and \texttt{false} values, and it must possibly refine the set of represented concretizations and mutate the heap location of the object involved in the bytecode according to the returned outcome. Consider the code
\begin{lstlisting}
if (a == b) {
   // code...
}
\end{lstlisting}
There are two cases that need to be considered to determine the outcome of the bytecode:
\begin{compactenum}
\item $\mathtt{H}(a) \cap \mathtt{H}(b) = \emptyset$: the bytecode returns \texttt{false} and nothing further is requred.
\item $\mathtt{H}(a) \cap \mathtt{H}(b) \not = \emptyset$ $\wedge$ $\mathtt{SAT}(PC \cup \{I_a = I_b, H_a = H_b\})$: the bytecode may return either \texttt{true} or \texttt{false} and a choice generator needs to be created.
\end{compactenum}
The choice generator for the compare bytecode is more complex than for other bytecodes because it must create representative sets without enumerating all possible outcomes using the path constraint. For the case \texttt{true} outcome
\begin{compactitem}
\item $\mathit{PC} = \mathit{PC} \cup \{\mathtt{I}(a) = \mathtt{I}(b),\mathtt{H}(a) = \mathtt{H}(b)\}$
\item $\mathtt{H}(a) = \mathtt{H}(b) = \mathtt{H}(a) \cap \mathtt{H}(b)$
\item $\mathtt{I}(a) \in \mathtt{H}(a) \cap \mathtt{H}(b) \rightarrow \mathtt{I}(b) = \mathtt{I}(a)$ $\vee$ $\exists h \in \mathtt{H}(a) \cap \mathtt{H}(b)\ .\ \mathtt{I}(b) = \mathtt{I}(a) = h$
\end{compactitem}
In essence, in the case where two variable reference the same object, the path constraint and sets are modified to represent the new restriction. The \texttt{false} outcome is handled similarly with a few notable exceptions on the path constraint and the represented set.
\begin{compactitem}
\item $\mathit{PC} = \mathit{PC} \cup \{\mathtt{I}(a) \not = \mathtt{I}(b)\}$
\item $\mathtt{I}(a) = \mathtt{I}(b) \rightarrow \exists h \in \mathtt{H}(b)\ .\ h \not = \mathtt{I}(a) \wedge \mathtt{I}(b) = h$
\end{compactitem}

\noindent \textbf{\texttt{IF\_ACMPNE}}: the bytecode is handled similarly to \texttt{IF\_ACMPEQ}. 

\subsection{Invocation}
\textbf{\texttt{INVOKEVIRTUAL}}

When we come to an invoke virtual you have to look for all the specialized implementations
of the method, creating choices with symbolic locations of various "actual types". The number
of choices will be equal to the number of specialized implementations of the method. When you create a choice on a specialization, you need to update the "actual type" field in the symbolic location. The "current cast" does not need to change. The number of types that the symbolic location cannot be will also be updated according to the "actual type" field. The number of types that the symbolic location cannot be will be updated with the types of the other specializations since invoking a specialization associated with a type implies that the object cannot be the types containing the other specializations.

\subsection{Checking Types and Casting}
\noindent\textbf{\texttt{INSTANCEOF}}: the bytecode may return both the \texttt{true} and \texttt{false} values when dealing with initialized symbolic variables, and it must possibly refine the equivalence class for the represented object referenced by the variable and mutate the contents of the heap location of the object involved in the bytecode according to the returned outcome. The bytecode implements the default bahvior when the operand is concrete and not an initialized symbolic variable. For the rest of the discussion, assume the operand is an initialized symbolic variable.
Consider the code
\begin{lstlisting}
if (a instanceof C) {
   // code...
}
\end{lstlisting}
There are two cases that need to be considered to determine the outcome of the bytecode where $h = \mathtt{I}(a)$ is the representative object of the equivalence class:
\begin{compactenum}
\item $\mathtt{Type}(h) = C$: the bytecode returns \texttt{true} and nothing further is required as the type stored in the heap location is $C$.
\item $\neg \mathtt{SAT}(\mathtt{C}(h) \cup \{T_h \leq C\})$: the bytecode returns \texttt{false} and nothing further is requred as the current constraints on what is in the heap location restrict it from being of type $C$.
\item $\mathtt{SAT}(\mathtt{C}(h) \cup \{T_h \leq C\})$: the bytecode can return either \texttt{true} or \texttt{false} requiring a choice generator.
\end{compactenum}
% // C <= B <= A
% A a; // C(a) = {(T_a <= A)}
%
%if (a instance of C) {
%    ** TRUE **
%    (T_h \leq C)
%    ...
%}
% ** FALSE **
% (C \leq T_h) \wedge (T_h \not = C)
% ** TRUE **
% (T_h \leq C)
\noindent The \texttt{true} outcome for the choice generator in clause (3) is
\begin{compactitem}
    \item $\mathtt{C}(h) = \mathtt{C}(h) \cup \{T_h \leq C\}$
    \item $\mathtt{Type}(h) = C$
    \item $\mathtt{H}(a) = H^\prime$ where $H^\prime = \{h_i \mid h_i \in \mathtt{H}(a) \wedge \mathtt{SAT}(\mathtt{C}(h_i) \cup \{T_{h_i} \leq C\})\}$ 
\end{compactitem}
The second statement indicates that the actual type in the heap location $h$ needs to change. As such, the object is mutated to be an instance of $C$. This mutation retains all fields and values from the previous object and only adds new fields for type $C$. The last statement refines the equivalence class to exclude any heap locations that cannot be considered something of type $C$. 

For the \texttt{false} outcome of the generator, $\mathtt{C}(h) = \mathtt{C}(h) \cup \{C \leq T_h, T_h \not = C\}$. Unlike the \texttt{true} outcome, the \texttt{false} outcome retains the entire equivalence class and does not need to mutate any heap entries.

\noindent\textbf{\texttt{CHECKCAST}}: the bytecode is syntactic sugar for 
\begin{lstlisting}
if (! (obj == null  ||  obj instanceof <class>)) {
    throw new ClassCastException();
}
// if this point is reached, then object 
// is either null, or an instance of <class> 
// or one of its superclasses.
\end{lstlisting}
Please see the \texttt{IFNULL} and \texttt{INSTANCEOF} bytecodes for details. If the exception is thrown, then JPF will catch the unhandled exception as per its normal behavior.

\subsection{Programs to consider}
\begin{compactitem}
\item \texttt{TestGetfieldSplit.java}: checks alias equivalence classes when assigning to initialized values.
\end{compactitem}
%\section{Conclusion}

\acks

Acknowledgments, if needed.

% We recommend abbrvnat bibliography style.

\bibliographystyle{abbrvnat}
\bibliography{../bib/paper}


\end{document}

%                       Revision History
%                       -------- -------
%  Date         Person  Ver.    Change
%  ----         ------  ----    ------

%  2013.06.29   TU      0.1--4  comments on permission/copyright notices

