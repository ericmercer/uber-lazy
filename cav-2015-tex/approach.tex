\subsection{Generating and Updating Symbolic Heaps}
\label{sec:precise}

The symbolic heap algorithm only partitions the heap at comparison
operations or for exception handling.  There are three sets of rewrite
rules specific to the symbolic heap algorithm: (i) rules to initialize
symbolic references, (ii) rules to perform field dereferences and
writes, and (iii) rules to check equality and inequality of
references.


\begin{comment}
NOTE: It may be required to add back into the constraint on new the
conditions under which the original location was created. In other
words, look at the preimage on the location used for new, find the
least reference (the one that attached to the first creation), pull
the constraint off of that reference, and add it to the constraint for
nw. This is required for the isClone() to work on the example that
tests equivalence.
\end{comment}

\newsavebox{\boxPI}
\savebox{\boxPI}{
\mprset{flushleft}
\begin{mathpar}
	\inferrule[Summarize]{
	\Lambda = \mathbb{UN}\lp \cfgnt{L}, \cfgnt{R}, \cfgnt{r}, \cfgnt{f}\rp \\
      \Lambda \neq \emptyset \\
      \lp\phi_x\ \cfgnt{l}_x\rp = \mathrm{min}_l\lp \Lambda\rp\\
      \cfgnt{r}_f = \mathrm{init}_r\lp \rp \\
      l_f  = \mathrm{fresh}_l\lp \mathrm{C}\rp\\\\
      \rho = \{ \lp \cfgnt{r}_a\ l_a\rp  \mid \mathrm{isInit}\lp \cfgnt{r}_a\rp  \wedge\cfgnt{r}_a = \mathrm{min}_r\lp \cfgnt{R}^{\leftarrow}[l_a]\rp \wedge \mathrm{type}\lp l_a\rp  = \mathrm{C} \} \\\\
      \theta_\mathit{null} = \{ \lp \phi\ l_\mathit{null}\rp  \mid \phi = \lp \phi_x \wedge \cfgnt{r}_f = \cfgnt{r}_\mathit{null} \rp  \} \\\\
      \theta_\mathit{new} = \{\lp \phi\ l_f\rp  \mid \phi = \lp \phi_x \wedge \cfgnt{r}_f \neq \cfgnt{r}_\mathit{null} \wedge \lp \wedge_{\lp \cfgnt{r}^\prime_a\ l^\prime_a\rp  \in \rho} \cfgnt{r}_f \ne \cfgnt{r}^\prime_a\rp \rp \}\\\\
      \theta_\mathit{alias} = \{ \lp \phi\ l_a\rp  \mid \exists\cfgnt{r}_a\ \lp\lp\cfgnt{r}_a\ l_a\rp  \in \rho \wedge \phi = \lp \phi_x \wedge \cfgnt{r}_f \neq \cfgnt{r}_\mathit{null} \wedge \cfgnt{r}_f = \cfgnt{r}_a \wedge \lp \wedge_{\lp \cfgnt{r}^{\prime}_a\ l^{\prime}_a\rp  \in \rho\ \lp \cfgnt{r}^\prime_a < \cfgnt{r}_a\rp } \cfgnt{r}_f \neq \cfgnt{r}^{\prime}_a \rp \rp \rp \} \\\\
      \theta_\mathit{orig} = \{\lp\phi\ \cfgnt{l}_\mathit{orig}\rp \mid \exists \phi_\mathit{orig} \lp \lp\phi_\mathit{orig}\ \cfgnt{l}_\mathit{orig}\rp \in \cfgnt{L}\lp\cfgnt{R}\lp\cfgnt{l}_x,\cfgnt{f}\rp\rp \wedge \phi = \lp\neg\phi_x \wedge \phi_\mathit{orig}\rp\}\\\\ 
      \theta = \theta_\mathit{null} \cup \theta_\mathit{new} \cup \theta_\mathit{alias} \cup \theta_\mathit{orig} \\
\cfgnt{R}^\prime = \cfgnt{R}[\forall \cfgnt{f} \in \mathit{fields}\lp \mathrm{C}\rp \ \lp \lp l_f\ \cfgnt{f}\rp  \mapsto \cfgnt{r}_\mathit{un} \rp ]
    }{
      \lp \cfgnt{L}\ \cfgnt{R}\ \cfgnt{r}\ \cfgnt{f}\ \cfgnt{C}\rp \rsum 
      \lp \cfgnt{L}[\cfgnt{r}_f \mapsto \theta]\ \cfgnt{R}^{\prime}[ \lp l_x,\cfgnt{f}\rp  \mapsto \cfgnt{r}_f ]\ \cfgnt{r}\ \cfgnt{f}\ \cfgnt{C}\rp
	}
\and
	\inferrule[Summarize-end]{
	  \Lambda = \mathbb{UN}\lp \cfgnt{L}, \cfgnt{R}, \cfgnt{r}, \cfgnt{f}\rp \\
      \Lambda = \emptyset
    }{
      \lp \cfgnt{L}\ \cfgnt{R}\ \cfgnt{r}\ \cfgnt{f}\ \cfgnt{C}\rp  \rsum
      \lp \cfgnt{L}\ \cfgnt{R}\ \cfgnt{r}\ \cfgnt{f}\ \cfgnt{C}\rp 
	}
\end{mathpar}}
%\end{center}
%\caption{The summary machine, $s ::= \lp\cfgnt{L}\ \cfgnt{R}\ \cfgnt{r}\ \cfgnt{f}\ \cfgnt{C}\rp$, with $s\rsum^*s^\prime =  s \rsum \cdots \rsum s^\prime \rsum s^\prime$.}
%\label{fig:symInit}
%\end{figure*}


\begin{figure*}
\begin{tabular}[c]{l}
\scalebox{1.0}{\usebox{\boxPI}} \\
\end{tabular}
\caption{Initializing fields, $s ::= \lp\cfgnt{L}\ \cfgnt{R}\ \cfgnt{r}\ \cfgnt{f}\ \cfgnt{C}\rp$, with $s\rsum^*s^\prime =  s \rsum \cdots \rsum s^\prime \rsum s^\prime$.}
\label{fig:symInit}
\end{figure*}

The initialization rule in \figref{fig:symInit} is invoked whenever a
field within an uninitialized reference is accessed. The interaction
with the solver in the definition of the rules is denoted by
$\mathbb{S}(\phi)$ which returns true if $\phi$ is satisfiable.  The
check for uninitialized references is determined by the function
$\mathbb{UN}(\cfgnt{L}, \cfgnt{R}, \cfgnt{r}, \cfgnt{f}) =
\{\lp\phi\ \cfgnt{l}\rp\ ...\}$ which returns constraint-location
pairs where the field $\cfgnt{f}$ is uninitialized.
\[
\begin{array}{rcl}
\mathbb{UN}(\cfgnt{L}, \cfgnt{R}, \cfgnt{r}, \cfgnt{f}) & = &\{ \lp\phi\ \cfgnt{l}\rp \mid \lp \phi\ \cfgnt{l}\rp  \in \cfgnt{L}\lp \cfgnt{r}\rp  \wedge \exists \phi^\prime \lp \lp \phi^\prime\ \cfgnt{l}_\mathit{un}\rp  \in \cfgnt{L}\lp \cfgnt{R}\lp l,\cfgnt{f}\rp\rp \wedge \mathbb{S}\lp \phi \wedge \phi^\prime \rp\rp\}\\
\end{array}
\]

In~\figref{fig:symInit}, given the uninitialized set $\Lambda$ for
field $f$, the $\mathit{min}_l$ function returns
$(\phi_x\ \cfgnt{l}_x)$ based on a lexicographical ordering of
uninitialized locations in $\Lambda$ to make initialization
deterministic.  The set $\rho$ contains reference-location pairs that
represent potential aliases. There are four cases encoded in the
symbolic heap: (i) $\theta_\mathit{null}$ represents the condition
where $\cfgnt{l}_\mathit{null}$ is possible, (ii)
$\theta_\mathit{new}$ represents the case where $\cfgnt{r}_f$ points
to a fresh location, (iii) each member of $\theta_\mathit{alias}$
restricts $\cfgnt{r}_f$ to a particular alias in $\rho$, and at the
same time, not alias any member of $\rho$ that was initialized earlier
than the current choice, and (iv) $\theta_\mathit{orig}$ implements
conditional initialization to preserve the original heap structure.
These sets are added into the heap on $\cfgnt{r}_f$ after the fields
for $\cfgnt{l}_f$ are initialized to $\cfgnt{r}_\mathit{un}$.

\begin{figure*}[t]
\begin{center}
\begin{tabular}[c]{c|c|c|c}
\begin{tabular}[c]{c}
\scalebox{0.81}{\input{origHeap.pdf_t}} \\
\end{tabular} &
\begin{tabular}[c]{c}
\scalebox{0.81}{\input{summarizeXHeap.pdf_t}} \\
\end{tabular} &
\begin{tabular}[c]{c}
\scalebox{0.81}{\input{summarizeYHeap.pdf_t}} \\
\end{tabular} &
\begin{tabular}[c]{l}
$\rho := \{ (\cfgnt{r}_1^i\ \cfgnt{l}_1 \}$ \\
$\theta_\mathit{null} := \{ ( \cfgnt{r}_2^i = \cfgnt{r}_\mathit{null}\ \cfgnt{l}_\mathit{null}) \}$\\
$\theta_\mathit{new} := \{ ( \cfgnt{r}_2^i \neq \cfgnt{r}_\mathit{null} \wedge \cfgnt{r}_2^i \neq \cfgnt{r}_1^i\ \cfgnt{l}_2) \}$\\
$\theta_\mathit{alias} := \{ ( \cfgnt{r}_1^i \neq \cfgnt{r}_\mathit{null} \wedge \cfgnt{r}_2^i \neq \cfgnt{r}_\mathit{null} \wedge \cfgnt{r}_2^i = \cfgnt{r}_1^i\ \cfgnt{l}_1) \}$\\
$\theta_\mathit{orig} := \{ \}$ \\
$\phi_{\mathit{1a}} := \cfgnt{r}_1^i = \cfgnt{r}_\mathit{null} $ \\
$\phi_{\mathit{1b}} := \cfgnt{r}_1^i \neq \cfgnt{r}_\mathit{null} $  \\
$\phi_{\mathit{2a}} := \cfgnt{r}_2^i = \cfgnt{r}_\mathit{null}$ \\
$\phi_{\mathit{2b}} := \cfgnt{r}_2^i \neq \cfgnt{r}_\mathit{null} \wedge \cfgnt{r}_2^i \neq \cfgnt{r}_1^i$ \\
$\phi_{\mathit{2c}} :=  \cfgnt{r}_2^i \neq \cfgnt{r}_\mathit{null} \wedge \cfgnt{r}_2^i = \cfgnt{r}_1^i $ \\
\end{tabular} \\
(a) & (b) & (c) & (d) \\
\end{tabular}
\end{center}
\caption{An example that initializes $\lp\cfgt{this}\  \cfgt{\$}\ \cfgnt{x}\rp$ and $\lp\cfgt{this}\  \cfgt{\$}\ \cfgnt{y}\rp$. (a) Initial heap structure. (b) After $\lp\cfgt{this}\  \cfgt{\$}\ \cfgnt{x}\rp$ is initialized. (c) After $\lp\cfgt{this}\  \cfgt{\$}\ \cfgnt{y}\rp$ is initialized. (d) Sets in the summarize rule and constraints on the edges.}
\label{fig:initHeap}
\end{figure*}

We visualize the initialization process in~\figref{fig:initHeap}. The
graph in~\figref{fig:initHeap}(a) represents the initial heap. The
reference superscripts $s$ and $i$ indicate the partition containing
the reference.  In~\figref{fig:initHeap}, $\cfgnt{r}_0^s$ represents a
stack reference for the $\cfgt{this}$ variable which has two fields
$x$ and $y$ of the same type. Note that when no constraint is
specified for a location, there is an implicit $\mathit{true}$
constraint. For example, $\cfgnt{r}_0^s$ points to $\cfgnt{l}_0$ on
the constraint $\mathit{true}$. The fields $x$ and $y$ point to the
uninitialized reference $\cfgnt{r}_\mathit{un}$. The graph
in~\figref{fig:initHeap}(b) represents the symbolic heap after the
initialization of the $\lp\cfgt{this}\ \cfgt{\$}\ \cfgnt{x}\rp$ field
while the graph in~\figref{fig:initHeap}(c) represents the symbolic
heap after the initialization of the
$\lp\cfgt{this}\ \cfgt{\$}\ \cfgnt{y}\rp$ field following the
initialization of $\lp\cfgt{this}\ \cfgt{\$}\ \cfgnt{x}\rp$. The list
in~\figref{fig:initHeap}(d) represents the various sets constructed in
initialization rule for $\lp\cfgt{this}\ \cfgt{\$}\ \cfgnt{y}\rp$. We
also define values of constraints used as labels in the graph.


\newsavebox{\boxPFAFW}
\savebox{\boxPFAFW}{
%\begin{figure}[t]
%\begin{center}
\mprset{flushleft}
\begin{mathpar}
	\inferrule[Field Access]{
      \exists \lp \phi\ l\rp \in \cfgnt{L}\lp \cfgnt{r}\rp\ \lp l \neq l_{\mathit{null}} \wedge \mathbb{S}\lp \phi \wedge \phi_g\rp \rp \\\\
      \theta = \{ \phi \mid \lp \phi\ l_\mathit{null} \rp \wedge \mathbb{S}\lp \phi \wedge \phi_g\rp \} \\\\
      \phi_g^\prime = \phi_g \wedge (\wedge_{\phi \in \theta} \neg \phi) \\\\
      \{\cfgnt{C}\} = \{\cfgnt{C} \mid \exists \lp \phi\ l\rp  \in \cfgnt{L}\lp \cfgnt{r}\rp\ \lp\cfgnt{C} = \mathrm{type}\lp \cfgnt{l},\cfgnt{f}\rp\rp\} \\\\
      \lp \cfgnt{L}\ \cfgnt{R}\ \cfgnt{r}\ \cfgnt{f}\ \cfgnt{C}\rp \rsum^* \lp \cfgnt{L}^\prime\ \cfgnt{R}^\prime\ \cfgnt{r}\ \cfgnt{f}\ \cfgnt{C}\rp \\
      \cfgnt{r}^\prime = \mathrm{stack}_r\lp \rp
    }{
      \lp \cfgnt{L}\ \cfgnt{R}\ \phi_g\ \eta\ \cfgnt{r}\ \lp \cfgt{*}\ \cfgt{\$}\ \cfgnt{f} \rightarrow \cfgnt{k}\rp \rp  \rsym^\mathit{A}
      \lp \cfgnt{L}^\prime[\cfgnt{r}^\prime \mapsto \mathbb{VS}\lp \cfgnt{L}^\prime,\cfgnt{R}^\prime,\cfgnt{r},\cfgnt{f},\phi_g^\prime\rp ]\ \cfgnt{R}^\prime\ \phi_g^\prime\ \eta\ \cfgnt{r}^\prime\ \cfgnt{k}\rp 
	}
\and
	\inferrule[Field Write]{
      \cfgnt{r}_x = \eta\lp\cfgnt{x}\rp \\
      \exists \lp \phi\ l\rp \in \cfgnt{L}\lp \cfgnt{r}_x\rp\ \lp l \neq l_{\mathit{null}} \wedge \mathbb{S}\lp \phi \wedge \phi_g\rp \rp \\\\
      \theta = \{ \phi \mid \lp \phi\ l_\mathit{null} \rp \wedge \mathbb{S}\lp \phi \wedge \phi_g\rp \} \\\\
      \phi_g^\prime = \phi_g \wedge (\wedge_{\phi \in \theta} \neg \phi) \\\\
      \Psi_x =\{\lp \phi\ l\ \cfgnt{r}_\mathit{cur} \rp  \mid \lp \phi\ \cfgnt{l}\rp  \in \cfgnt{L}\lp \cfgnt{r}_x\rp  \wedge \cfgnt{r}_\mathit{cur} = \cfgnt{R}\lp l,\cfgnt{f}\rp  \}\\\\
      X = \{ \lp l\ \theta \rp  \mid \exists \phi\ \lp \lp \phi\ l\ \cfgnt{r}_\mathit{cur} \rp \in \Psi_x \wedge \theta = \mathbb{ST}\lp \cfgnt{L},\cfgnt{r},\phi,\phi_g^\prime\rp  \cup \mathbb{ST}\lp \cfgnt{L},\cfgnt{r}_\mathit{cur},\neg\phi,\phi_g^\prime\rp \rp  \}\\\\
      \cfgnt{R}^{\prime} = \cfgnt{R}[\forall \lp l\ \theta \rp  \in X\ \lp \lp l\ \cfgnt{f}\rp  \mapsto \mathrm{fresh}_r\lp \rp \rp ]\\\\
      \cfgnt{L}^{\prime} = \cfgnt{L}[\forall \lp l\ \theta \rp  \in X\ \lp \exists \cfgnt{r}_\mathit{targ}\ \lp \cfgnt{r}_\mathit{targ} = \cfgnt{R}^\prime\lp l,\cfgnt{f}\rp \wedge \lp\cfgnt{r}_\mathit{targ} \mapsto \theta\rp  \rp \rp ]
    }{
      \lp \cfgnt{L}\ \cfgnt{R}\ \phi_g\ \eta\ \cfgnt{r}\ \lp \cfgnt{x}\ \cfgt{\$}\ \cfgnt{f}\ \cfgt{:=}\ \cfgt{*}\ \rightarrow\ \cfgnt{k}\rp \rp  \rsym^\mathit{W}
      \lp \cfgnt{L}^{\prime}\ \cfgnt{R}^{\prime}\ \phi_g^\prime\ \eta\ \cfgnt{r}\ \cfgnt{k}\rp 
	}	
\end{mathpar}}
%\end{center}
%\caption{Precise symbolic heap summaries from symbolic execution indicated by $\rsym = \rightarrow_\mathit{FA} \cup \rightarrow_\mathit{FW} \cup \rightarrow_\mathit{EQ} \cup \rcom$.}
%\label{fig:symfield}
%\end{figure}



There are two rewrite rules in~\figref{fig:fHeap}(a), one for reading
the value of a field (field-access) and the other when we write to a
field (field-write). Both rules first check that the operations can be
performed on a non-null location. The field-access rewrite rule
in~\figref{fig:fHeap} dereferences a field of type $\cfgnt{C}$ and
uses \figref{fig:symInit}, to get a new symbolic heap that is
initialized on the field $\cfgnt{f}$. The symbolic heap is further
modified to include a new stack reference pointing to the value set
(possible values of the field) returned during the dereferencing; the
new stack reference is the return value from the operation.


 \begin{definition}
\label{def:VS}
The $\mathbb{VS}$ function constructs the value set given a
heap, reference, and desired field where $\mathbb{S}(\phi)$ returns true if $\phi$ is satisfiable.
\[
\begin{array}{rcl}
  \mathbb{VS}(\cfgnt{L},\cfgnt{R},\phi_g,\cfgnt{r},\cfgnt{f}) & = &
  \{(\phi\wedge\phi^\prime\ \cfgnt{l}^\prime) \mid \exists
  \cfgnt{l}\ ((\phi\ l) \in L(r)\ \wedge \\ & &
  \ \ \ \ \ \ \ \ \exists \cfgnt{r}^\prime ( \cfgnt{r}^\prime =
  R(\cfgnt{l},\cfgnt{f})\ \wedge (\phi^\prime\ l^\prime) \in
  \cfgnt{L}(\cfgnt{r}^\prime)\ \wedge
  \mathbb{S}(\phi\wedge\phi^\prime\wedge \phi_g)))\}
\end{array}
\]
\end{definition}
 

The value set computation is local: it only considers constraint
locations on the reference $\cfgnt{r}$. Further, it only includes
constraints that are feasible under the current path condition
$\phi_g$. It also ensures that the access path is valid. There is a
constraint $\phi$ from the base reference $r$ and a given
location. Once a location $l$ is selected its field $\cfgnt{f}$ is
accessed to get another reference $r^\prime$, and from that reference
there is a set of locations each with a constraint. The first
constraint $\phi$ from the base reference plus this second constraint
to the actual location of the field $\phi^\prime$ comprise the access
path. For each member of the value set, these constraints together
with the global path constraint must be satisfied on some aliasing
assignment. % \nsr{neha: this para needs work -- flow and content}


For field-write in~\figref{fig:fHeap}, after the non-null check and
strengthening of the global heap constraint, it computes the current
references associated with the field in every location in
$\Psi_\cfgnt{x}$. Note that the reference $\cfgnt{r}_x$ is the base
reference whose field, $\cfgnt{r}_\mathit{cur}$ is being written to,
while the reference $r$ is the target reference. The set $\Psi_x$
contains tuples $(\phi\ l\ \cfgnt{r}_\mathit{cur})$ of constraints,
locations, and references. These tuples represent access chains
leading from $\cfgnt{r}_x$ to the reference of the field,
$\cfgnt{r}_\mathit{cur}$. The goal is to change the fields to no
longer point to $\cfgnt{r}_\mathit{cur}$, but rather fresh references
that have both the original locations before the write and the
locations from the write in the value sets (i.e., conditional
write). Since the target of the write is $r$, the rule checks that the
location constraint pairs of $\cfgnt{L}(\cfgnt{r})$ are satisfiable
when accessed through the $\cfgnt{r}_x$ chain in the strengthening
function.  \begin{definition}
\label{def:ST}
The strengthen function $\mathbb{ST}(\cfgnt{L},\cfgnt{r},\phi,\phi_g)$ strengthens every
constraint from the reference $\cfgnt{r}$ with $\phi$ and keeps only location-constraint
pairs that are satisfiable after this strengthening with the inclusion of the global heap constraint $\phi_g$:
\[
\begin{array}{rcl} 
\mathbb{ST}(\cfgnt{L},\cfgnt{r},\phi,\phi_g) & = & \{ (\phi\wedge\phi^\prime\ \cfgnt{l}^\prime) \mid  \\
& & \ \ \ \ (\phi^\prime\ \cfgnt{l}^\prime)\in \cfgnt{L}(\cfgnt{r})\wedge\mathbb{S}(\phi\wedge\phi^\prime\wedge\phi_g) \}
\end{array}
\]
\end{definition}
 Constraints on locations are strengthened
with new aliasing conditions and those that are feasible with the
current path constraint are retained.

Strengthening in the field write creates a value set that contains two
types of locations: those for the case where the write is feasible
(the first call), and those for the case where it is not (the second
call). In the case that $\phi$ is true then $\cfgnt{r}_\mathit{cur}$
will point to the constraint location pairs of $\cfgnt{L}(\cfgnt{r})$.
Whereas, if $\phi$ is false then $\cfgnt{r}_\mathit{cur}$ will
continue to point to the constraint location pair it currently
references.  As the references are immutable, the rule creates fresh
references for each $\cfgnt{r}_\mathit{cur}$ and points them to the
appropriate value sets.

\begin{figure*}[t]
\begin{center}
\mprset{flushleft}
\begin{mathpar}
    \inferrule[Equals (references-true)]{
    \theta_\alpha = \{\lp\phi_0 \wedge \phi_1\rp \mid \exists l\ \lp \lp \phi_0\ l\rp  \in \cfgnt{L}\lp \cfgnt{r}_0\rp  \wedge \lp \phi_1\ l\rp  \in \cfgnt{L}\lp \cfgnt{r}_1\rp \rp \} \\\\
    \theta_0 = \{\phi_0 \mid \exists l_0\ \lp \lp \phi_0\ l_0\rp  \in \cfgnt{L}\lp \cfgnt{r}_0\rp  \wedge \forall \lp \phi_1\ l_1\rp  \in \cfgnt{L}\lp \cfgnt{r}_1\rp \ \lp l_0 \neq l_1\rp \rp \} \\\\
    \theta_1 = \{\phi_1 \mid \exists l_1\ \lp \lp \phi_1\ l_1\rp  \in \cfgnt{L}\lp \cfgnt{r}_1\rp  \wedge \forall \lp \phi_0\ l_0\rp  \in \cfgnt{L}\lp \cfgnt{r}_0\rp \ \lp l_0 \neq l_1\rp \rp \} \\\\
    \phi^\prime =  \phi \wedge \lp \vee_{\phi_\alpha\in\theta_\alpha}\phi_\alpha\rp \wedge\lp \wedge_{\phi_0 \in \theta_0} \neg \phi_0\rp \wedge\lp \wedge_{\phi_1
    \in \theta_1} \neg \phi_1\rp \\\\ 
    \mathbb{S}(\phi^\prime)}{
    \lp \cfgnt{L}\ \cfgnt{R}\ \phi\ \eta\ \cfgnt{r}_0\ \lp \cfgnt{r}_1\; \cfgt{=}\; \cfgt{*} \rightarrow \cfgnt{k}\rp \rp  \rightarrow_\mathit{EQ}
    \lp \cfgnt{L}\ \cfgnt{R}\ \phi^\prime\ \eta\ \cfgt{true}\ \cfgnt{k}\rp 
    }
\and
    \inferrule[Equals (references-false)]{
    \theta_\alpha = \{\lp\phi_0 \Rightarrow \neg \phi_1\rp \mid \exists l\ \lp \lp \phi_0\ l\rp  \in \cfgnt{L}\lp \cfgnt{r}_0\rp  \wedge \lp \phi_1\ l\rp  \in \cfgnt{L}\lp \cfgnt{r}_1\rp \rp \} \\\\
    \theta_0 = \{\phi_0 \mid \exists l_0\ \lp \lp \phi_0\ l_0\rp  \in \cfgnt{L}\lp \cfgnt{r}_0\rp  \wedge \forall \lp \phi_1\ l_1\rp  \in \cfgnt{L}\lp \cfgnt{r}_1\rp \ \lp l_0 \neq l_1\rp \rp \} \\\\
    \theta_1 = \{\phi_1 \mid \exists l_1\ \lp \lp \phi_1\ l_1\rp  \in \cfgnt{L}\lp \cfgnt{r}_1\rp  \wedge \forall \lp \phi_0\ l_0\rp  \in \cfgnt{L}\lp \cfgnt{r}_0\rp \ \lp l_0 \neq l_1\rp \rp \} \\\\
    \phi^\prime = \phi \wedge \lp \wedge_{\phi_\alpha\in\theta_\alpha}\phi_\alpha\rp \vee\lp \lp \vee_{\phi_0 \in \theta_0} \phi_0\rp   \vee\lp \vee_{\phi_1
    \in \theta_1} \phi_1\rp \rp  \\\\ 
    \mathbb{S}(\phi^\prime)}{
    \lp \cfgnt{L}\ \cfgnt{R}\ \phi\ \eta\ \cfgnt{r}_0\ \lp \cfgnt{r}_1\; \cfgt{=}\; \cfgt{*} \rightarrow \cfgnt{k}\rp \rp  \rightarrow_\mathit{EQ}
    \lp \cfgnt{L}\ \cfgnt{R}\ \phi^\prime\ \eta\ \cfgt{false}\ \cfgnt{k}\rp 
    }
\end{mathpar}
\end{center}
\caption{Precise symbolic heap summaries from symbolic execution indicated by $\rsym = \rightarrow_\mathit{FA} \cup \rightarrow_\mathit{FW} \cup \rightarrow_\mathit{EQ} \cup \rcom$.}
\label{fig:symeq}
\end{figure*}


\begin{figure*}
\begin{center}
\begin{tabular}[c]{c}
\scalebox{1.0}{\usebox{\boxPEQ}} \\
% & \usebox{\boxPEX} \\ \\
%(a) & (b) \\
\end{tabular}
\end{center}
\caption{The reference compare rewrite rule for true, $\rsym^\mathit{E}$ outcomes.}
\label{fig:eqs}
\end{figure*}


The rewrite rule to compare two references in the symbolic heap is
shown in~\figref{fig:eqs}. The equals references-true rewrite rule
returns true when two references $\cfgnt{r}_0$ and $\cfgnt{r}_1$
\emph{can} be equal. Intutively, $\Phi_\alpha$ contains all
constraints under which $\cfgnt{r}_0$ and $\cfgnt{r}_1$ may point to
the same location in the symbolic heap. The second set, $\Phi_0$,
contains constraints under which the reference $\cfgnt{r}_0$ points to
corresponding locations such that the reference $\cfgnt{r}_1$
\emph{does not} point to those locations under any
constraint. Finally, the set, $\Phi_1$, contains constraints under
which $\cfgnt{r}_1$ points to corresponding locations and
$\cfgnt{r}_0$ \emph{does not} point to those locations under any
constraint. We use the three sets of constraints to update the current
global heap constraint $\phi_g$ and create a new global heap
constraint $\phi_g^\prime$. We add to $\phi_g$ the disjunction of the
constraints in $\Phi_\alpha$ to indicate that if any of the
constraints are satisfiable, then references $\cfgnt{r}_0$ and
$\cfgnt{r}_1$ can be equal. Furthermore, we conjoin to the global heap
constraint $\phi_g^\prime$, the conjunctions of negations to the
constraints in $\Phi_0$ and $\Phi_1$. This indicates for locations
that are not common to the references, the negations of their
constraints are satisfiable. The rule does not complete (i.e., is not
feasible) if the new global constraint is not satisfied on any
aliasing assignment. In such a case, the $\cfgt{true}$ outcome is not
possible on any symbolic heap. Before the rewrite rule returns
$\cfgt{true}$ we check the satisfiability of the updated global heap
constraint. The reference-false is the logical dual of the rule.


