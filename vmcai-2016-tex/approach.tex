\section{Symbolic Initialization}
\label{sec:precise}
%Although the lazy initialization method used in~\gsetxt{} can be applied
%to guarded value sets, a direct case-splitting implementation would run into the same
%path explosion problems as~\gstetxt{}. A more desirable approach would be to create a new~\symtxt{} scheme that uses the state merging properties inherent to guarded value sets in order to avoid creating new paths during dereferencing operations. To accomplish this in a manner equivalent to~\gsetxt{}, uninitialized reference values must be initialized to guarded value sets that exactly reproduce the behavior of the individual~\gsetxt{} paths. Since~\gsetxt{} either chooses a new reference, a null reference, or an alias to an existing reference during lazy initialization, it seems obvious that these should be the locations represented in the guarded value set produced by~\symtxt{}. Although choosing the appropriate locations is straightforward, choosing the guards is more complicated. Since lazy initialization does not alter the path condition, it provides few clues as to what guards might be used. Thus, choosing the appropriate guards is the key obstacle in creating the desired~\symtxt{} scheme.
%
%If the exact nature of the guards is opaque, a few of the requirements may be divined by observing the properties of references. Since in a concrete heap, no reference points to two locations at the same time, the guards in any value set must be mutually exclusive. Also, since every reference points somewhere~\footnote{Even so-called dangling pointers can be thought of as pointing to unallocated memory.}, the set of guards must be collectively exhaustive. Furthermore, the fact that the input heap may contain a practically unbounded number of distinct references indicates that the value domain of the variables used in the guards should be similarly unbounded. 

%We don't really care about the locations where the references are stored, only whether references alias or not.

This work presents a new initialization scheme which avoids the
nondeterminism introduced by~\gsetxt{}. Called \emph{\symtxt{}}, this scheme leverages the core idea in generalized symbolic execution
with lazy initialization, using on-the-fly reasoning to model a
black-box input heap during symbolic execution. Unlike~\gsetxt{}, \symtxt{} constructs a single
symbolic heap and polynomially-sized path condition for each control
flow path.

There are three sets of rewrite
rules specific to the~\symtxt{} algorithm: (i) rules to initialize
symbolic references, (ii) rules to perform field dereferences and
writes, and (iii) rules to check equality and inequality of
references. Rules relating to (ii) and (iii) are similar to previously proposed methods utilizing guarded value sets \cite{Sen:2014,Dillig:2011}. The rules for (i) are novel in how they  preserve \gsetxt\ semantics.

\input{summarize-precise-incomplete}

\begin{figure*}[t]
\centering
%\begin{center}
\begin{tabular}[c]{l}
\scalebox{1.0}{\usebox{\boxPI}} \\

\end{tabular}
%\end{center}

\caption{Initializing fields, $s ::= \lp\cfgnt{L}\ \cfgnt{R}\ \cfgnt{r}\ \cfgnt{f}\ \cfgnt{C}\rp$, with $s\rsum^*s^\prime =  s \rsum \cdots \rsum s^\prime$.}
\label{fig:symInit}
\end{figure*}

In~\figref{fig:symInit}, similar to before, given the uninitialized set $\Lambda$ for
field $f$, the $\mathrm{min}_l$ function returns
$(\phi_x\ \cfgnt{l}_x)$ which represents the earliest created uninitialized location in that set.  The set $\rho$ contains reference-location pairs that
represent potential aliases, where $\mathrm{isInit}\lp\rp$ ensures that the references are initialized. 
There are four cases encoded in the
symbolic heap. The first three correspond to the three types of choices made during lazy initialization: (i) $\theta_\mathit{null}$ represents the condition
where $\cfgnt{l}_\mathit{null}$ is possible, (ii)
$\theta_\mathit{new}$ represents the case where $\cfgnt{r}_f$ points
to a fresh location, (iii) each member of $\theta_\mathit{alias}$
restricts $\cfgnt{r}_f$ to a particular alias in $\rho$, and at the
same time, not alias any member of $\rho$ that was initialized earlier
than the current choice. 

Unlike the first three cases, which correspond directly to~\gsetxt{} initialization rules, $\theta_\mathit{orig}$, case (iv), is unique to~\symtxt{}. In this case, $\theta_\mathit{orig}$ implements conditional initialization to preserve the original heap structure. This step is necessary in order to maintain \emph{homomorphism} (i.e., equivalent shapes) between symbolic heaps created using~\symtxt{} and the~\gsetxt{} heaps they are intended to represent. 
The sets from each of the four cases are added into the heap on $\cfgnt{r}_f$ after the fields
for $\cfgnt{l}_f$ are initialized to $\cfgnt{r}_\mathit{un}$.

\begin{figure*}[t]
\begin{center}
\begin{tabular}[c]{c|c|c}
\begin{tabular}[c]{c c}
\scalebox{0.81}{\input{origHeap.pdf_t}}&  \\
\end{tabular} &
\begin{tabular}[c]{c}
\scalebox{0.81}{\input{summarizeXHeap.pdf_t}} \\
\end{tabular} &
\begin{tabular}[c]{c}
\scalebox{0.81}{\input{summarizeYHeap.pdf_t}} \\
\end{tabular}\\
(a) & (b) & (c)\\
\end{tabular}
\begin{tabular}[c]{c}
\begin{tabular}[c]{c c}
\hline
\begin{tabular}[c]{l}
$\rho := \{ (\cfgnt{r}_1^i\ \cfgnt{l}_1) \}$ \\
$\theta_\mathit{null} := \{ ( \cfgnt{r}_2^i = \cfgnt{r}_\mathit{null}\ \cfgnt{l}_\mathit{null}) \}$\\
$\theta_\mathit{new} := \{ ( \cfgnt{r}_2^i \neq \cfgnt{r}_\mathit{null} \wedge \cfgnt{r}_2^i \neq \cfgnt{r}_1^i\ \cfgnt{l}_2) \}$\\
$\theta_\mathit{alias} := \{ ( \cfgnt{r}_1^i \neq \cfgnt{r}_\mathit{null} \wedge \cfgnt{r}_2^i \neq \cfgnt{r}_\mathit{null} \wedge \cfgnt{r}_2^i = \cfgnt{r}_1^i\ \cfgnt{l}_1) \}$\\
$\theta_\mathit{orig} := \{ \}$ \\
\end{tabular} &
\begin{tabular}[c]{l}
$\phi_{\mathit{1a}} := \cfgnt{r}_1^i = \cfgnt{r}_\mathit{null} $ \\
$\phi_{\mathit{1b}} := \cfgnt{r}_1^i \neq \cfgnt{r}_\mathit{null} $  \\
$\phi_{\mathit{2a}} := \cfgnt{r}_2^i = \cfgnt{r}_\mathit{null}$ \\
$\phi_{\mathit{2b}} := \cfgnt{r}_2^i \neq \cfgnt{r}_\mathit{null} \wedge \cfgnt{r}_2^i \neq \cfgnt{r}_1^i$ \\
$\phi_{\mathit{2c}} :=  \cfgnt{r}_2^i \neq \cfgnt{r}_\mathit{null} \wedge \cfgnt{r}_2^i = \cfgnt{r}_1^i $ \\
\end{tabular}\\
\end{tabular} \\
(d)
\end{tabular}
\end{center}
\caption{An example that initializes $\lp\cfgt{this}\  \cfgt{\$}\ \cfgnt{x}\rp$ and $\lp\cfgt{this}\  \cfgt{\$}\ \cfgnt{y}\rp$. (a) Initial heap structure. (b) After $\lp\cfgt{this}\  \cfgt{\$}\ \cfgnt{x}\rp$ is initialized. (c) After $\lp\cfgt{this}\  \cfgt{\$}\ \cfgnt{y}\rp$ is initialized. (d) Sets in the initialization rule and constraints on the edges.}
\label{fig:initHeap}
\end{figure*}

\figref{fig:initHeap} illustrates the initialization process. The
graph in~\figref{fig:initHeap}(a) represents the initial heap. The
reference superscripts $s$ and $i$ indicate the partition containing
the reference: input, auxiliary, or stack.  In~\figref{fig:initHeap}, $\cfgnt{r}_0^s$ represents a
stack reference for the $\cfgt{this}$ variable which has two fields
$x$ and $y$ of the same type. Note that when no constraint is
specified for a location, there is an implicit $\mathit{true}$
constraint. For example, $\cfgnt{r}_0^s$ points to $\cfgnt{l}_0$ on
the constraint $\mathit{true}$. The fields $x$ and $y$ point to the
uninitialized reference $\cfgnt{r}_\mathit{un}$. The graph
in~\figref{fig:initHeap}(b) represents the symbolic heap after the
initialization of the $\lp\cfgt{this}\ \cfgt{\$}\ \cfgnt{x}\rp$ field
while the graph in~\figref{fig:initHeap}(c) represents the symbolic
heap after the initialization of the
$\lp\cfgt{this}\ \cfgt{\$}\ \cfgnt{y}\rp$ field following the
initialization of $\lp\cfgt{this}\ \cfgt{\$}\ \cfgnt{x}\rp$. The list
in~\figref{fig:initHeap}(d) represents the various sets constructed in
the initialization for $\lp\cfgt{this}\ \cfgt{\$}\ \cfgnt{y}\rp$. 


\input{precise-fig-incomplete}

\begin{figure*}[t]
\begin{center}
\begin{tabular}[c]{l}
\scalebox{1.0}{\usebox{\boxPFAFW}} \\
\end{tabular}
\end{center}
\caption{Field read and write relations: Field-access, $\rsym^\mathit{A}$, and field-write, $\rsym^\mathit{W}$, rewrite rules for the $\rsym$ relation.}
\label{fig:fHeap}
\end{figure*}


There are two rewrite rules in~\figref{fig:fHeap}, one for reading
the value of a field (field-access) and the other for writing to a
field (field-write). Both rules first check that the operations can be
performed on a non-null location. The field-access rewrite rule
in~\figref{fig:fHeap} dereferences a field of type $\cfgnt{C}$, recall that the heaps are type consistent and programs are type safe, and
uses the $\rsum^*$ relation from \figref{fig:symInit}, to get a new symbolic heap that is
initialized on the field $\cfgnt{f}$. The symbolic heap is further
modified to include a new stack reference pointing to the guarded value set
(possible values of the field) returned during the dereferencing; the
new stack reference is the return value from the $\mathbb{VS}$ operation.

 \begin{definition}
\label{def:VS}
The $\mathbb{VS}$ function constructs the value set given a
heap, reference, and desired field where $\mathbb{S}(\phi)$ returns true if $\phi$ is satisfiable.
\[
\begin{array}{rcl}
  \mathbb{VS}(L,R,\phi_g,\cfgnt{r},\cfgnt{f}) & \defeq &
  \{(\phi\wedge\phi^\prime\ l^\prime) \mid \exists
  l\ ((\phi\ l) \in L(\cfgnt{r})\ \wedge \\ & &
  \ \ (\phi^\prime\ l^\prime) \in L(R(l,\cfgnt{f}))\ \wedge \\ & &
  \ \ \mathbb{S}(\phi\wedge\phi^\prime\wedge \phi_g))\}
\end{array}
\]
\end{definition}
 

For a reference $\cfgnt{r}$ and field $\cfgnt{f}$, the value set function computes 
the guarded value set of locations and access path constraints
that are feasible under the current path condition $\phi_g$.
The access path constraint is the union of two local constraints: 
the constraint $\phi$ from dereferencing $\cfgnt{r}$ to location $\cfgnt{l}$, 
and the constraint $\phi^\prime$ from dereferencing the field $\cfgnt{f}$ of the location $\cfgnt{l}$ to the actual location of the field, $\cfgnt{l}^\prime$. 
This access path constraint, paired with location $\cfgnt{l}^\prime$, is a member of the value set only if its union with the path condition is satisfiable, ensuring that the access path is valid and feasible under the path condition.
 % \nsr{neha: this para needs work -- flow and content tao: worked on the flow}


For field-write in~\figref{fig:fHeap}, after the non-null check and
strengthening of the global heap constraint, it computes the current
references associated with the field in every location in
$\Psi_\cfgnt{x}$. Note that the reference $\cfgnt{r}_x$ is the base
reference whose field, $\cfgnt{r}_\mathit{cur}$ is being written to,
while the reference $r$ is the target reference. The set $\Psi_x$
contains tuples $(\phi\ l\ \cfgnt{r}_\mathit{cur})$ of constraints,
locations, and references. These tuples represent access chains
leading from $\cfgnt{r}_x$ to the reference of the field,
$\cfgnt{r}_\mathit{cur}$. The goal is to change the fields to no
longer point to $\cfgnt{r}_\mathit{cur}$, but rather fresh references
that have both the original locations before the write and the
locations from the write in the value sets (i.e., conditional
write). Since the target of the write is $r$, the rule checks that the
constraints of the guarded value set $\cfgnt{L}(\cfgnt{r})$ are satisfiable
when accessed through the $\cfgnt{r}_x$ chain in the strengthening
function.  \begin{definition}
\label{def:ST}
The function $\mathbb{ST}(\cfgnt{L},\cfgnt{r},\phi,\phi_g)$
strengthens every constraint in $\cfgnt{L}\lp\cfgnt{r}\rp$ with
$\phi$ and retains strengthened location-constraint pairs that are satisfiable
with the path condition $\phi_g$:
\[
\begin{array}{rcl} 
\mathbb{ST}(\cfgnt{L},\cfgnt{r},\phi,\phi_g) & = & \{ (\phi\wedge\phi^\prime\ \cfgnt{l}^\prime) \mid (\phi^\prime\ \cfgnt{l}^\prime)\in \cfgnt{L}(\cfgnt{r})\wedge \\ & & 
\ \ \mathbb{S}(\phi\wedge\phi^\prime\wedge\phi_g) \}
\end{array}
\]
\end{definition}
 Constraints on locations are strengthened
with new aliasing conditions and those that are feasible with the
current path condition are retained.

Strengthening in the field write creates a value set, \cfgnt{X},  that contains two
types of locations: those for the case where the write is feasible
(the first call in \cfgnt{X}), and those for the case where it is not (the second
call in \cfgnt{X}). In the case that $\phi$ is true then $\cfgnt{r}_\mathit{cur}$
will point to the guarded value set $\cfgnt{L}(\cfgnt{r})$.
Whereas, if $\phi$ is false then $\cfgnt{r}_\mathit{cur}$ will
continue to point to the constraint location pair it currently
references.  As the references are immutable, the rule creates fresh
references for each $\cfgnt{r}_\mathit{cur}$ and points them to the
appropriate value sets.

\newsavebox{\boxPEQ}
\savebox{\boxPEQ}{
%\begin{figure}[t]
%\begin{center}
\mprset{flushleft}
\begin{tabular}[c]{c}
\begin{mathpar}
    \inferrule[Equals (references-true)]{
    \Phi_\alpha = \{\lp\phi_0 \wedge \phi_1\rp \mid \exists l\ \lp \lp \phi_0\ l\rp  \in \cfgnt{L}\lp \cfgnt{r}_0\rp  \wedge \lp \phi_1\ l\rp  \in \cfgnt{L}\lp \cfgnt{r}_1\rp \rp \} \\\\
    \Phi_0 = \{\phi_0 \mid \exists l_0\ \lp \lp \phi_0\ l_0\rp  \in \cfgnt{L}\lp \cfgnt{r}_0\rp  \wedge \forall \lp \phi_1\ l_1\rp  \in \cfgnt{L}\lp \cfgnt{r}_1\rp \ \lp l_0 \neq l_1\rp \rp \} \\\\
    \Phi_1 = \{\phi_1 \mid \exists l_1\ \lp \lp \phi_1\ l_1\rp  \in \cfgnt{L}\lp \cfgnt{r}_1\rp  \wedge \forall \lp \phi_0\ l_0\rp  \in \cfgnt{L}\lp \cfgnt{r}_0\rp \ \lp l_0 \neq l_1\rp \rp \} \\\\
    \phi^\prime =  \phi \wedge \lp \vee_{\phi_\alpha\in\Phi_\alpha}\phi_\alpha\rp \wedge\lp \wedge_{\phi_0 \in \Phi_0} \neg \phi_0\rp \wedge\lp \wedge_{\phi_1
    \in \Phi_1} \neg \phi_1\rp \\\\ 
    \mathbb{S}(\phi^\prime)}{
    \lp \cfgnt{L}\ \cfgnt{R}\ \phi\ \eta\ \cfgnt{r}_0\ \lp \cfgnt{r}_1\; \cfgt{=}\; \cfgt{*} \rightarrow \cfgnt{k}\rp \rp  \rsym^\mathit{E}
    \lp \cfgnt{L}\ \cfgnt{R}\ \phi^\prime\ \eta\ \cfgt{true}\ \cfgnt{k}\rp 
    }
%\and
%    \inferrule[Equals (references-false)]{
%    \Phi_\alpha = \{\lp\phi_0 \Rightarrow \neg \phi_1\rp \mid \exists l\ \lp \lp \phi_0\ l\rp  \in \cfgnt{L}\lp \cfgnt{r}_0\rp  \wedge \lp \phi_1\ l\rp  \in \cfgnt{L}\lp \cfgnt{r}_1\rp \rp \} \\\\
%    \Phi_0 = \{\phi_0 \mid \exists l_0\ \lp \lp \phi_0\ l_0\rp  \in \cfgnt{L}\lp \cfgnt{r}_0\rp  \wedge \forall \lp \phi_1\ l_1\rp  \in \cfgnt{L}\lp \cfgnt{r}_1\rp \ \lp l_0 \neq l_1\rp \rp \} \\\\
%    \Phi_1 = \{\phi_1 \mid \exists l_1\ \lp \lp \phi_1\ l_1\rp  \in \cfgnt{L}\lp \cfgnt{r}_1\rp  \wedge \forall \lp \phi_0\ l_0\rp  \in \cfgnt{L}\lp \cfgnt{r}_0\rp \ \lp l_0 \neq l_1\rp \rp \} \\\\
%    \phi^\prime = \phi \wedge \lp \wedge_{\phi_\alpha\in\theta_\alpha}\phi_\alpha\rp \vee\lp \lp \vee_{\phi_0 \in \theta_0} \phi_0\rp   \vee\lp \vee_{\phi_1
%    \in \theta_1} \phi_1\rp \rp  \\\\ 
%    \mathbb{S}(\phi^\prime)}{
%    \lp \cfgnt{L}\ \cfgnt{R}\ \phi\ \eta\ \cfgnt{r}_0\ \lp \cfgnt{r}_1\; \cfgt{%=}\; \cfgt{*} \rightarrow \cfgnt{k}\rp \rp  \rsym^\mathit{E^\prime}
%    \lp \cfgnt{L}\ \cfgnt{R}\ \phi^\prime\ \eta\ \cfgt{false}\ \cfgnt{k}\rp 
%    }
\end{mathpar}
\end{tabular}}
%\end{center}
%\caption{FIX THIS CAPTION AND MOVE $\rsym$ DEFINITION (MAY NOT BE NEEDED BECAUSE OF \defref{def:meta}: Precise symbolic heap summaries from symbolic execution indicated by $\rsym = \rightarrow_\mathit{FA} \cup \rightarrow_\mathit{FW} \cup \rightarrow_\mathit{EQ}^T \cup \rightarrow_\mathit{EQ}^F \cup \rcom$.}
%\label{fig:symeq}
%\end{figure}


\begin{figure*}
\begin{center}
\begin{tabular}[c]{c}
\scalebox{1.0}{\usebox{\boxPEQ}} \\
% & \usebox{\boxPEX} \\ \\
%(a) & (b) \\
\end{tabular}
\end{center}
\caption{The reference compare rewrite rule for true, $\rsym^\mathit{E}$ outcomes.}
\label{fig:eqs}
\end{figure*}


The rewrite rule to compare two references in the symbolic heap is
shown in~\figref{fig:eqs}. The equals references-true rewrite rule
returns true when two references $\cfgnt{r}_0$ and $\cfgnt{r}_1$
\emph{can be equal}. Intuitively, $\Phi_\alpha$ contains all
constraints under which $\cfgnt{r}_0$ and $\cfgnt{r}_1$ may point to
the same location in the symbolic heap. The second set, $\Phi_0$,
contains constraints under which the reference $\cfgnt{r}_0$ points to
corresponding locations such that the reference $\cfgnt{r}_1$
\emph{does not} point to those locations under any
constraint. Finally, the set, $\Phi_1$, contains constraints under
which $\cfgnt{r}_1$ points to corresponding locations and
$\cfgnt{r}_0$ \emph{does not} point to those locations under any
constraint. The three sets of constraints are used to create a new path condition $\phi_g^\prime$
as an update of the current path condition $\phi_g$.  The update is accomplished by first taking the disjunction of the
constraints in $\Phi_\alpha$ to indicate that if any of the
constraints are satisfiable, then references $\cfgnt{r}_0$ and
$\cfgnt{r}_1$ can be equal. This disjunction is then conjoined with $\phi_g$ to form $\phi_g^\prime$. Furthermore, the conjunctions of negations to the
constraints in $\Phi_0$ and $\Phi_1$ is conjoined with $\phi_g^\prime$. This indicates for locations
that are not common to the references, the negations of their
constraints are satisfiable. The rule does not complete (i.e., is not
feasible) if the new global constraint is not satisfied on any
aliasing assignment. In such a case, the $\cfgt{true}$ outcome is not
possible on any symbolic heap. Before the rewrite rule returns
$\cfgt{true}$, it verifies the satisfiability of the updated global heap
constraint. The reference-false is the logical dual of the rule.


Consider the example in~\figref{fig:initHeap}(c). In order to compare
$\cfgnt{r}_1^i$ and $\cfgnt{r}_2^i$, the Equals rule gets the 
guarded value set associated with each of the references:
\[
\cfgnt{L}(\cfgnt{r}_1^i) = \{ (\phi_{1a}\; \cfgnt{l}_\mathit{null})\; (\phi_{1b}\; \cfgnt{l}_1) \} 
\]
\[
\cfgnt{L}(\cfgnt{r}_2^i) = \{ (\phi_{2a}\; \cfgnt{l}_\mathit{null})\; (\phi_{2b}\; \cfgnt{l}_2)\; (\phi_{2c}\; \cfgnt{l}_1) \} \\
\]
\noindent{The three constraint sets are:} 
\[
\Phi_\alpha = \{ (\phi_{1a}\; \wedge\; \phi_{2a} ) (\phi_{1b}\; \wedge\; \phi_{2c} ) \}\;
\Phi_0 = \{ \}\; \Phi_1 = \{ \phi_{2b}\} \\
\]
\noindent{Finally the global constraint is} 
\[
\phi^\prime = \mathit{true} \wedge [ (\phi_{1a}\; \wedge\; \phi_{2a} )\vee (\phi_{1b}\; \wedge\; \phi_{2c} ) ] \wedge \neg\phi_{2b} 
\]





