\section{Uber-lazy Operational Semantics}
\label{semantics}

The operational semantics for Uber-lazy
symbolic execution are specified using the Javalite 
language~\cite{saints-MS}. Javalite is an imperative model of Java
that includes many features of the Java language. In this work, we 
present only the salient features of the Javalite
language relevant to understanding the Uber-lazy symbolic execution
algorithm. A detailed explanation of Javalite is available in~\cite{saints-MS}.

Javalite is an imperative model of the Java language
developed to facilitate rapid prototyping 
of model checking algorithms and proofs about the
algorithms. It is specified with a Java-like syntax and a set
of reduction rules for syntacticly executing Javalite programs.
Javalite is based on a variant of the CEKS syntactic 
machine~\cite{Felleisen:1987}. The operational semantics are 
defined using the structure (syntax)
of the language and a set of reduction rules for syntactically
executing Javalite programs. In a CEKS machine, the
(C)ontrol string represents a program, command, or instruction to be
evaluated; it is initialized to a string representing the entire progam. 
An (E)nvironment  maps (local) variables to their values. The (K)ontinuation 
specifies what is to be executed next, and the (S)tore is used to store 
dynamically allocated data, i.e., the heap. 

\subsection{Symbolic Store}

At the core of the Uber-lazy symbolic execution algorithm
is a fully symbolic heap, i.e., a heap in which objects
are never materialized as concrete objects during symbolic
execution, but instead are represented using constraints
charactering feasible heap shapes. To represent the heap
store ($S$), our algorithm defines a labeled bi-partite 
graph where $S = (R, L, E)$.  $R$ contains the set of nodes 
representing program \emph{references}.
$L$ contains the set of nodes representing \emph{locations} in the store. 
The store is initialized with two special locations: $null$ and $\bot$ representing a null object
and an uninitialized location respectively. Each edge in the set of labeled
edges, $E$, is uni-directional. An edge from a reference $r \in R$ to a location $l \in L$ is
labeled with a constraint $\phi$ indicating the conditions under which $r$ references
i.e., points to, that location in the store. This paper uses a standard definition
of constraints $\phi \in \Phi$ assuming all of the usual relation operators and connectives.
Reference nodes collect the the
feasible points-to relations for a given program execution path during symbolic execution.
Each edge from a location $l \in L$ to a reference $r \in R$ is labeled with the 
name of a field $f \in F$. 

More details here including we assume the input program is typesafe
or if there is a type and a mismatch occurs, then the machine halts.

The machine also halts if an exception is thrown.

\subsection{Syntax}

Javalite programs and expressions are written in a syntax specified by the 
grammar in~\figref{fig:machine-syntax}. The production rules correspond
to the various features supported by Javalite such as classes, fields,
methods, and expressions. 

\figref{fig:machine-syntax} specifies
the machine syntax for Javalite. The expression $e$ is equivalent to
a CEKS machine's control string. The 

\subsection{Reduction Rules}

We first present the reduction rules 


\subsubsection{Basic Reduction Rules}

Most rules presented here - short discussion

\subsubsection{Store Update Rules}

Interesting rules presented here; more elaborate discussion

Start off with helper functions for rules that manipulate the store...

The function $\mathbb{VS}(L,R,r,f)$ constructs the value-set given a
heap, reference, and desired field such that
$(l^\prime\ \phi^\prime\wedge\phi) \in \mathbb{VS}(L,R,r,f)$ if and
only if
\[
\begin{array}{l}
  \exists (l\ \phi) \in L(r) ( \\
  \ \ \ \ \ \ \ \ \ \exists r^\prime \in R(l,f) ( \\
  \ \ \ \ \ \ \ \ \ \ \ \ \ \ \ \ \ \ \exists (l^\prime\ \phi^\prime) \in L(r^\prime) (\mathbb{S}(\phi\wedge\phi^\prime))))
\end{array}
\]
where $\mathbb{S}(\phi)$ returns true if $\phi$ is satisfiable.

The strengthen function $\mathbb{ST}(L,r,\phi^\prime)$ strengthens every
constraint from the reference $r$ and keeps only location-constraint
pairs that are satisfiable after strengthening. Formally,
$(l\ \phi\wedge\phi^\prime)\in\mathbb{ST}(L,r,\phi^\prime)$ if and
only if $\exists (l\ \phi)\in
L(r)\wedge\mathbb{S}(\phi\wedge\phi^\prime)$

The empty-reference function $\mathbb{ER}(L,\phi^\prime) = \{r\ |\ L(r) \neq
\emptyset \wedge \forall(l,\phi) \in L(r)(\neg \mathbb{S}(\phi \wedge
\phi^\prime))\}$ searches the heap for references that become
disconnected from all their locations after strengthening. These
references, if reachable, imply the heap is no longer valid on the
current search path. As such, the symbolic execution algorithm should
backtrack. This check is similar to a feasibility check in classic
symbolic execution with only primitive data types.

The consistency function is critical to the soundness of the algorithm
as it detects when a symbolic heap becomes invalid along a path,
similar to a feasibility check when doing classical symbolic execution
with just primitives. As the constraints in the heap are strengthened
with different aliasing requirements, it is possible to reach a point
where the heap is no longer connected. Meaning, a valid reference is
live, either in the local environment or the continuation, the reaches
another reference that is no longer connected to any locations due to
strengthening. The function relies on the empty-reference function to
identify disconnected references. The function in essence checks every
reference in the local environment and every reference found in the
continuation, as these are all considered live. This operation similar
to garbage collection where the local environment and stack are
inspected to find the roots of the heap for the scan.


The consistency function relies on two auxiliary functions which are
informally defined. The function $\mathrm{ref}(\eta,\cfgnt{k})$
inspects the local environment and continuation for all live
references, and it returns those references in a set. The function
$\mathrm{reach}(L, R, r, r^\prime)$ returns true if $r^\prime$ is
reachable from $r$ in the heap and false otherwise. The consistency function $\mathbb{C}(L,R,X,\eta,k)$ is now defined as
\[
 \left\{ \begin{array}{rl} 
        0 & \exists r \in \mathrm{ref}(\eta, \cfgnt{k})\ (\exists r^\prime \in X\ (\mathrm{reach}(L, R, r,r^\prime))) \\ 
        1 & \mbox{otherwise}\end{array}\right .
\]
