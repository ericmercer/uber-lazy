\subsection{Transition System}

In Javalite, states are strings that match a certain pattern. 

\begin{figure}
\begin{center}
\cfgstart
\cfgrule{e}{\lp .... \cfgor \lp\cfgt{raw} \cfgnt{v} \cfgt{@} \cfgnt{m} \cfgnt{v} \rp\rp}
%    ;; eval syntax
%  (object (C [f loc] ...))
%  (hv v
%      object)
\cfgrule{$\phi$}{\cfgnt{constraint}}
\cfgrule{l}{\cfgt{number}}
%  (h mt
%     (h [loc -> hv]))
%  (η mt
%     (η [x -> loc]))
\cfgrule{s}{\lp$\mu$ \cfgnt{L} \cfgnt{R} $\eta$ \cfgnt{e} \cfgnt{k}\rp}
\cfgrule{k}{\cfgt{end}}
\cfgorline{\lp \cfgt{*} \cfgt{\$} \cfgnt{f} $\rightarrow$ \cfgnt{k}\rp}
%     (* @ m (e ...) -> k)
%     (v @ m (v ...) * (e ...) -> k)
%     (* == e -> k)
%     (v == * -> k)
%     (x := * -> k)
%     (x $ f := * -> k)
%     (if * e else e -> k)
%     (var T x := * in e -> k)
%     (begin * (e ...) -> k)
%     (pop η k))
\cfgend
\end{center}
\caption{The machine syntax for Javalite.}
\label{fig:machine-syntax}
\end{figure}


\begin{definition}
\label{def:state}
The set of \textbf{states} $\mathcal{S}$ is defined as the set of strings matching the pattern $s$ in \ref{fig:machine-syntax}.
\end{definition}

\begin{definition}
\label{def:initstate}
$\mathcal{S}_0$ is defined as the set of \textbf{initial states}. An initial state is a state meeting the following conditions:  The range of L has exactly three locations: $l_{null}$, $l_{un}$, and $l_0$, the function R is defined only for location $l_0$, and for any field $f$, $R(l_0,f)$ returns $r_{un}$. 
\end{definition}

The rewrite rules that define the Javalite semantics are presented in
the supplemental document accompanying the paper.


\begin{definition}
A \textbf{state transition relation} $\rightarrow_{\Phi}$ is a binary relation $\rightarrow_{\Phi}\ \subseteq\ \mathcal{S} \times \mathcal{S} $, which relates machine states with successor states. Two important state transition relations are \textbf{GSE} $\rgse$ and \textbf{symbolic} $\rsym$. Each of these use a separate relation for initialization: $\rinit$ for GSE and $\rsum$ for symbolic. All of these transition relations are defined in \figref{fig:lazy}, \figref{fig:lazyInit}, \figref{fig:sym}, and \figref{fig:symInit}.
\end{definition}

