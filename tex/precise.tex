\section{Generating Summary Heaps}
\label{sec:precise}
There are three sets of rewrite rules specific to the summary heap
algorithm: (i) rules to initialize symbolic references, (ii) rules to complete field
access and field write, and (iii) rules to check equality and
inequality of references. The other rules are common to both $\rsym$
and $\rgse$.

\subsection{Initialization of Symbolic References}

Initialization of uninitialized references in the $\rgse$ relation is
a non-deterministic choice that creates a heap either for
the initialization of the reference to $\mathit{null}$, a new instance of
the object, or to any type compatible object in the input heap (e.g.,
previously initialized by $\rgse$). In contrast, initialization in the
$\rsym$ relation creates a single summary heap containing all of the
possible initialization choices.

\begin{comment}
NOTE: It may be required to add back into the constraint on new the
conditions under which the original location was created. In other
words, look at the preimage on the location used for new, find the
least reference (the one that attached to the first creation), pull
the constraint off of that reference, and add it to the constraint for
nw. This is required for the isClone() to work on the example that
tests equivalence.
\end{comment}

\newsavebox{\boxPI}
\savebox{\boxPI}{
\mprset{flushleft}
\begin{mathpar}
	\inferrule[Summarize]{
	\Lambda = \mathbb{UN}\lp \cfgnt{L}, \cfgnt{R}, \cfgnt{r}, \cfgnt{f}\rp \\
      \Lambda \neq \emptyset \\
      \lp\phi_x\ \cfgnt{l}_x\rp = \mathrm{min}_l\lp \Lambda\rp\\
      \cfgnt{r}_f = \mathrm{init}_r\lp \rp \\
      l_f  = \mathrm{fresh}_l\lp \mathrm{C}\rp\\\\
      \rho = \{ \lp \cfgnt{r}_a\ l_a\rp  \mid \mathrm{isInit}\lp \cfgnt{r}_a\rp  \wedge\cfgnt{r}_a = \mathrm{min}_r\lp \cfgnt{R}^{\leftarrow}[l_a]\rp \wedge \mathrm{type}\lp l_a\rp  = \mathrm{C} \} \\\\
      \theta_\mathit{null} = \{ \lp \phi\ l_\mathit{null}\rp  \mid \phi = \lp \phi_x \wedge \cfgnt{r}_f = \cfgnt{r}_\mathit{null} \rp  \} \\\\
      \theta_\mathit{new} = \{\lp \phi\ l_f\rp  \mid \phi = \lp \phi_x \wedge \cfgnt{r}_f \neq \cfgnt{r}_\mathit{null} \wedge \lp \wedge_{\lp \cfgnt{r}^\prime_a\ l^\prime_a\rp  \in \rho} \cfgnt{r}_f \ne \cfgnt{r}^\prime_a\rp \rp \}\\\\
      \theta_\mathit{alias} = \{ \lp \phi\ l_a\rp  \mid \exists\cfgnt{r}_a\ \lp\lp\cfgnt{r}_a\ l_a\rp  \in \rho \wedge \phi = \lp \phi_x \wedge \cfgnt{r}_f \neq \cfgnt{r}_\mathit{null} \wedge \cfgnt{r}_f = \cfgnt{r}_a \wedge \lp \wedge_{\lp \cfgnt{r}^{\prime}_a\ l^{\prime}_a\rp  \in \rho\ \lp \cfgnt{r}^\prime_a < \cfgnt{r}_a\rp } \cfgnt{r}_f \neq \cfgnt{r}^{\prime}_a \rp \rp \rp \} \\\\
      \theta_\mathit{orig} = \{\lp\phi\ \cfgnt{l}_\mathit{orig}\rp \mid \exists \phi_\mathit{orig} \lp \lp\phi_\mathit{orig}\ \cfgnt{l}_\mathit{orig}\rp \in \cfgnt{L}\lp\cfgnt{R}\lp\cfgnt{l}_x,\cfgnt{f}\rp\rp \wedge \phi = \lp\neg\phi_x \wedge \phi_\mathit{orig}\rp\}\\\\ 
      \theta = \theta_\mathit{null} \cup \theta_\mathit{new} \cup \theta_\mathit{alias} \cup \theta_\mathit{orig} \\
\cfgnt{R}^\prime = \cfgnt{R}[\forall \cfgnt{f} \in \mathit{fields}\lp \mathrm{C}\rp \ \lp \lp l_f\ \cfgnt{f}\rp  \mapsto \cfgnt{r}_\mathit{un} \rp ]
    }{
      \lp \cfgnt{L}\ \cfgnt{R}\ \cfgnt{r}\ \cfgnt{f}\ \cfgnt{C}\rp \rsum 
      \lp \cfgnt{L}[\cfgnt{r}_f \mapsto \theta]\ \cfgnt{R}^{\prime}[ \lp l_x,\cfgnt{f}\rp  \mapsto \cfgnt{r}_f ]\ \cfgnt{r}\ \cfgnt{f}\ \cfgnt{C}\rp
	}
\and
	\inferrule[Summarize-end]{
	  \Lambda = \mathbb{UN}\lp \cfgnt{L}, \cfgnt{R}, \cfgnt{r}, \cfgnt{f}\rp \\
      \Lambda = \emptyset
    }{
      \lp \cfgnt{L}\ \cfgnt{R}\ \cfgnt{r}\ \cfgnt{f}\ \cfgnt{C}\rp  \rsum
      \lp \cfgnt{L}\ \cfgnt{R}\ \cfgnt{r}\ \cfgnt{f}\ \cfgnt{C}\rp 
	}
\end{mathpar}}
%\end{center}
%\caption{The summary machine, $s ::= \lp\cfgnt{L}\ \cfgnt{R}\ \cfgnt{r}\ \cfgnt{f}\ \cfgnt{C}\rp$, with $s\rsum^*s^\prime =  s \rsum \cdots \rsum s^\prime \rsum s^\prime$.}
%\label{fig:symInit}
%\end{figure*}


\begin{figure*}
\begin{tabular}[c]{l}
\scalebox{1.0}{\usebox{\boxPI}} \\
\end{tabular}
\caption{The summary machine, $s ::= \lp\cfgnt{L}\ \cfgnt{R}\ \cfgnt{r}\ \cfgnt{f}\ \cfgnt{C}\rp$, with $s\rsum^*s^\prime =  s \rsum \cdots \rsum s^\prime \rsum s^\prime$.}
\label{fig:symInit}
\end{figure*}

The initialization rules, \figref{fig:symInit}, are invoked whenever
an uninitialized field in a symbolic reference is accessed. The
interaction with the solver in the definition of the rules is denoted
by $\mathbb{S}(\phi)$ where it returns true if $\phi$ is
satisfiable. The end rule is active when there is nothing to
initialize. This condition is determined by the function
$\mathbb{UN}(\cfgnt{L}, \cfgnt{R}, \cfgnt{r}, \cfgnt{f}) =
\{\lp\phi\ \cfgnt{l}\rp\ ...\}$ which returns constraint-location pairs
where the field $\cfgnt{f}$ is uninitialized.
\[
\begin{array}{rcl}
\mathbb{UN}(\cfgnt{L}, \cfgnt{R}, \cfgnt{r}, \cfgnt{f}) & = &\{ \lp\phi\ \cfgnt{l}\rp \mid \lp \phi\ \cfgnt{l}\rp  \in \cfgnt{L}\lp \cfgnt{r}\rp  \wedge \\
& & \ \ \ \ \exists \phi^\prime \lp \lp \phi^\prime\ \cfgnt{l}_\mathit{un}\rp  \in \cfgnt{L}\lp \cfgnt{R}\lp l,\cfgnt{f}\rp\rp \wedge \\
& & \ \ \ \ \ \ \ \ \mathbb{S}\lp \phi \wedge \phi^\prime \rp\rp\}\\
\end{array}
\]
For reference $\cfgnt{r}$, this function constructs the set, $\theta$,
of constraint location pairs reachable from $\cfgnt{r}$ on the field
$\cfgnt{f}$ that are valid under the access path from $\cfgnt{r}$.

% The cardinality of the set, $\theta$ is never
%greater than one in GSE and the constraint is always satisfiable
%because all constraints are constant. This property is relaxed in GSE
%with heap summaries.

In~\figref{fig:symInit}, given the unitialized set $\mathbb{UN}$ for
field $f$, the $\mathit{min}_l$ function returns
$(\phi_x\ \cfgnt{l}_x)$ based on a lexicographical ordering of
uninitialized locations in $\Lambda$ to make initialization
deterministic (the same heap is always initialized in the same
way). 
%The $\mathrm{fresh}_\cfgnt{l}(\cfgnt{C})$ call returns a new location of type $C$. 
The set $\rho$ contains the set of
reference-location pairs from the input heap that are potential
alias. Minimizing over the set of references in the pre-image ensures
that any pair in $\rho$ is the reference $\cfgnt{r}_a$ that birthed
the location $\cfgnt{l}_a$ in some prior initialization (i.e., it was
$\cfgnt{r}_f$ and $\cfgnt{l}_f$ at one time). Aliasing constraint
in the new summary heap reason over these original creators; otherwise aliasing and local value set
computation breaks down.  The reference partitions with their
monotonicity make this determination possible.

There are three sets correspond to the non-deterministic choices in
GSE with a fourth that preserves the pre-initialized heap structure
(the constraints in each preserve the access path to the location):
(i) $\theta_\mathit{null}$ asserts the condition under which
$\cfgnt{l}_\mathit{null}$ is possible (any time the solver assigns
values to reference variables in such a way that $\cfgnt{r}_f =
\cfgnt{r}_\mathit{null}$ holds as does access path $\phi_x$). (ii)
$\theta_\mathit{new}$, meaning $\cfgnt{r}_f$ refers to the fresh
location, occurs when $\cfgnt{r}_f$ is not new or any other alias. (iii) any member of $\theta_\mathit{alias}$ restricts $\cfgnt{r}_f$ to
only alias one member of $\rho$ and not alias any member of $\rho$
that was initialized earlier than the current choice (a later
initialization may indeed alias the current alias choice in a valid
heap but the current choice cannot alias both the choice and a
previous initialization). And (iv) $\theta_\mathit{orig}$  implements conditional initialization to preserve the
original heap structure. This property is needed for the bisumulation. 
These sets are added into the heap on the field after initializing the fields in the new location to $\cfgnt{l}_\mathit{un}$.


\begin{comment} where for each element in the set $(\cfgnt{r}_a\ \cfgnt{l}_a)
\in \rho$ it creates a constraint stating that $\cfgnt{r}_f$ is not null, points to
$\cfgnt{r}_a$, and does not point to any reference contained in $(\cfgnt{r}_a^\prime\ 
\cfgnt{l}_a^\prime) \in \rho$ such that $\cfgnt{r}_a < \cfgnt{r}_a^\prime$ based on a lexical
ordering of initialized references. Additionally,
$\theta_\mathit{orig}$ allows for the possibility that the field
continues to remain uninitialized since the heap is symbolic.
\sjp{The last statement seems to jump out since we have not been
talking about the heap being symbolic (have we?)} We refer
to initialization in the summary machine as conditional
initialization \sjp{Not sure why we say this - what does it mean
to the reader?}. Finally, the fields of the reference $\cfgnt{r}_f$ are marked
as uninitialized. In the post condition of the rule, the reference
$\cfgnt{r}_f$ points to $\theta$ which contains the union of all the
constraint location pairs sets, and the field $f$ points to the new
reference $\cfgnt{r}_f$ in the summary heap.
\end{comment}

\begin{comment}
 Note that the summarize rule invoked repeatedly until
the set of unitialized constraint location pairs for field $f$ is
empty. In other words the set $\Lambda$ is empty. This consititutes
the summarize end rule in~\figref{fig:symInit}.
\end{comment}

\begin{figure*}[t]
\begin{center}
\begin{tabular}[c]{c|c|c|c}
\begin{tabular}[c]{c}
\scalebox{0.81}{\input{origHeap.pdf_t}} \\
\end{tabular} &
\begin{tabular}[c]{c}
\scalebox{0.81}{\input{summarizeXHeap.pdf_t}} \\
\end{tabular} &
\begin{tabular}[c]{c}
\scalebox{0.81}{\input{summarizeYHeap.pdf_t}} \\
\end{tabular} &
\begin{tabular}[c]{l}
$\rho := \{ (\cfgnt{r}_1^i\ \cfgnt{l}_1 \}$ \\
$\theta_\mathit{null} := \{ ( \cfgnt{r}_2^i = \cfgnt{r}_\mathit{null}\ \cfgnt{l}_\mathit{null}) \}$\\
$\theta_\mathit{new} := \{ ( \cfgnt{r}_2^i \neq \cfgnt{r}_\mathit{null} \wedge \cfgnt{r}_2^i \neq \cfgnt{r}_1^i\ \cfgnt{l}_2) \}$\\
$\theta_\mathit{alias} := \{ ( \cfgnt{r}_1^i \neq \cfgnt{r}_\mathit{null} \wedge \cfgnt{r}_2^i \neq \cfgnt{r}_\mathit{null} \wedge \cfgnt{r}_2^i = \cfgnt{r}_1^i\ \cfgnt{l}_1) \}$\\
$\theta_\mathit{orig} := \{ \}$ \\
$\phi_{\mathit{1a}} := \cfgnt{r}_1^i = \cfgnt{r}_\mathit{null} $ \\
$\phi_{\mathit{1b}} := \cfgnt{r}_1^i \neq \cfgnt{r}_\mathit{null} $  \\
$\phi_{\mathit{2a}} := \cfgnt{r}_2^i = \cfgnt{r}_\mathit{null}$ \\
$\phi_{\mathit{2b}} := \cfgnt{r}_2^i \neq \cfgnt{r}_\mathit{null} \wedge \cfgnt{r}_2^i \neq \cfgnt{r}_1^i$ \\
$\phi_{\mathit{2c}} :=  \cfgnt{r}_2^i \neq \cfgnt{r}_\mathit{null} \wedge \cfgnt{r}_2^i = \cfgnt{r}_1^i $ \\
\end{tabular} \\
(a) & (b) & (c) & (d) \\
\end{tabular}
\end{center}
\caption{An example that initializes $\lp\cfgt{this}\  \cfgt{\$}\ \cfgnt{x}\rp$ and $\lp\cfgt{this}\  \cfgt{\$}\ \cfgnt{y}\rp$. (a) Initial heap structure. (b) After $\lp\cfgt{this}\  \cfgt{\$}\ \cfgnt{x}\rp$ is initialized. (c) After $\lp\cfgt{this}\  \cfgt{\$}\ \cfgnt{y}\rp$ is initialized. (d) Constraint sets from the summary rule and constraints on the edges.}
\label{fig:initHeap}
\end{figure*}

We visualize the initialization process in~\figref{fig:initHeap}. The
graph in~\figref{fig:initHeap}(a) represents the initial heap. The
references with the superscript $s$ indicates that it is a reference from the stack partition.
 In~\figref{fig:initHeap} $\cfgnt{r}_0^s$ represents a stack
reference for the $\cfgt{this}$ variable which has two fields $x$
and $y$ of the same type; reference $\cfgnt{r}_0^s$ points to location
$\cfgnt{l}_0$. Note that when no constraint is specified in the graphs, then
there is an implicit $\mathit{true}$ constraint, for example, $\cfgnt{r}_0^s$
points to $\cfgnt{l}_0$ on the constraint $\mathit{true}$. The fields $x$ and
$y$ point to the uninitialized reference $\cfgnt{r}_\mathit{un}$.

% The reference $\cfgnt{r}_\mathit{un}$ points to the uninitialized location
%$\cfgnt{l}_\mathit{un}$ on the $\mathit{true}$ constraint.

The graph in~\figref{fig:initHeap}(b) represents the summary heap
after the initialization of the $\lp\cfgt{this}\  \cfgt{\$}\ \cfgnt{x}\rp$ field while
the graph in~\figref{fig:initHeap}(c) represents the symbolic heap
summary after the initialization of the $\lp\cfgt{this}\  \cfgt{\$}\ \cfgnt{y}\rp$ field
following the initialization of $\lp\cfgt{this}\  \cfgt{\$}\ \cfgnt{x}\rp$. The list
in~\figref{fig:initHeap}(d) represents the various sets constructed in
the summarize rule during the initialization of $\lp\cfgt{this}\  \cfgt{\$}\ \cfgnt{y}\rp$. We
also define values of constraints used as labels in the graph.

The access on $\cfgnt{l}_0.x$ creates a new input heap reference $\cfgnt{r}_1^i$ in the
summary heap shown in~\figref{fig:initHeap}(b). The reference $\cfgnt{r}_1^1$
points to the $\cfgnt{l}_\mathit{null}$ location under the constraint that
$\cfgnt{r}_1^i$ is null: $\phi_{1a} := \cfgnt{r}_1^i = \cfgnt{r}_\mathit{null}$. The reference
$\cfgnt{r}_1^i$ points to the location $\cfgnt{l}_1$ under the constraint that $\cfgnt{r}_1^i$
is not $\mathit{null}$ $\phi_{1b} :=\cfgnt{r}_1^i \neq \cfgnt{r}_\mathit{null}$. The
location $\cfgnt{l}_1$ represents a fresh location of type $C$ is created on
the heap such that $C$ is type of $\lp\cfgt{this}\  \cfgt{\$}\ \cfgnt{x}\rp$.

Following initialization to $\lp\cfgt{this}\  \cfgt{\$}\ \cfgnt{x}\rp$, when
$\lp\cfgt{this}\  \cfgt{\$}\ \cfgnt{y}\rp$ is initialized, the resulting summary heap is shown
in~\figref{fig:initHeap}(c). The set $\rho$ shown
in~\figref{fig:initHeap}(d) contains the potential aliases for
 $\lp\cfgt{this}\  \cfgt{\$}\ \cfgnt{y}\rp$ which is location $\cfgnt{l}_1$ ($\lp\cfgt{this}\  \cfgt{\$}\ \cfgnt{x}\rp$). The
constraint location $\theta$ sets represents the initialization
choices for $\lp\cfgt{this}\  \cfgt{\$}\ \cfgnt{y}\rp$. The new input reference $\cfgnt{r}_2^i$ for
$\lp\cfgt{this}\  \cfgt{\$}\ \cfgnt{y}\rp$ points to (a) location $\cfgnt{l}_\mathit{null}$ under the constraint
$\phi_{2a} := \cfgnt{r}_2^i = \cfgnt{r}_\mathit{null}$, (b) location $\cfgnt{l}_2$ under the
constraint $\phi_{2b} := \cfgnt{r}_2^i \neq \cfgnt{r}_\mathit{null} \wedge \cfgnt{r}_2^i \neq
\cfgnt{r}_1^i$ that states that $\cfgnt{r}_2^i$ is not null and does not alias
$\cfgnt{r}_1^i$, and (c) location $\cfgnt{l}_1$ under the constraint $\phi_{2c} :=
\cfgnt{r}_2^i \neq \cfgnt{r}_\mathit{null} \wedge \cfgnt{r}_2^i = \cfgnt{r}_1^i$ that states $\cfgnt{r}_2^i$
is not null and \emph{does} alias $\cfgnt{r}_1^i$.


