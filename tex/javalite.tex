\section{Javalite}
The algorithm for generalized symbolic execution with symbolic heaps
is presented using Javalite. Javalite is a syntactic machine defined
as rewrites on a string \cite{saints-MS}. The semantics are small-step
using a CESK model with a (C)ontrol string representing the expression
being evaluated, an (E)nvironment for local variables, a (S)tore for
the heap, and a (K)ontinuation for what is to be executed next. The
semantics of the original Javalite machine, and its machine syntax,
are changed to reflect generalized symbolic execution and then
generalized execution with symbolic heaps.

The surface syntax (input) for Javalite is in
\figref{fig:surface-syntax} with the machine syntax (state) in
\figref{fig:machine-syntax}. Terminals are in bold face while
non-terminals are italicized. Ellipses indicate zero or more
repetitions. Tuples omit the commas for compactness. The language
itself uses s-expressions for convenience.

\begin{figure}
\begin{center}
\cfgstart
\cfgrule{P}{\lp $\mu$ \lp \cfgnt{C} \cfgnt{m}\rp\rp}
\cfgrule{$\mu$}{(\cfgnt{CL} ...)}
\cfgrule{T}{\cfgt{bool} \cfgor \cfgnt{C}}
\cfgrule{CL}{\lp\cfgt{class} \cfgnt{C} \lp\lb\cfgnt{T} \cfgnt{f}\rb ...\rp \lp\cfgnt{M} ...\rp}
\cfgrule{M}{\lp\cfgnt{T} \cfgnt{m} \lb\cfgnt{T} \cfgnt{x}\rb\  e\rp}
\cfgrule{e}{\cfgnt{x}
\cfgor{\lp\cfgt{new} \cfgnt{C}\rp}
\cfgor{\lp\cfgnt{e} \cfgt{\$} \cfgnt{f}\rp}
\cfgor{\lp\cfgnt{x} \cfgt{\$} \cfgnt{f} \cfgt{:=} \cfgnt{e}\rp}
\cfgor{\lp\cfgnt{e} \cfgt{=} \cfgnt{e}\rp}}
\cfgorline{\lp\cfgt{if} \cfgnt{e} \cfgnt{e} \cfgt{else} \cfgnt{e}\rp 
\cfgor {\lp\cfgt{var} \cfgnt{T} \cfgnt{x} \cfgt{:=} \cfgnt{e} \cfgt{in} \cfgnt{e}\rp}
\cfgor {\lp\cfgnt{e} \cfgt{@} \cfgnt{m} \cfgnt{e} \rp}}
\cfgorline{\lp\cfgnt{x} \cfgt{:=} \cfgnt{e}\rp
\cfgor{\lp\cfgt{begin} \cfgnt{e} ...\rp}
\cfgor{\cfgnt{v}}}
\cfgrule{x}{\cfgt{this} \cfgor \cfgnt{id}}
\cfgrule{f,m,C}{\cfgnt{id}}
%\cfgrule{m}{\cfgnt{id}}
%\cfgrule{C}{\cfgnt{id}}
\cfgrule{v}{\cfgnt{r} \cfgor \cfgt{null} \cfgor \cfgt{true} \cfgor \cfgt{false} \cfgor \cfgt{error}}
\cfgrule{r}{\cfgt{number}}
\cfgrule{id}{\cfgt{variable-not-otherwise-mentioned}}
\cfgend
\end{center}
\caption{The Javalite surface syntax.}
\label{fig:surface-syntax}
\end{figure}

\begin{figure}
\begin{center}
\cfgstart
\cfgrule{e}{\lp .... \cfgor \lp\cfgt{raw} \cfgnt{v} \cfgt{@} \cfgnt{m} \cfgnt{v} \rp\rp}
%    ;; eval syntax
%  (object (C [f loc] ...))
%  (hv v
%      object)
\cfgrule{$\phi$}{\cfgnt{constraint}}
\cfgrule{l}{\cfgt{number}}
%  (h mt
%     (h [loc -> hv]))
%  (η mt
%     (η [x -> loc]))
\cfgrule{s}{\lp$\mu$ \cfgnt{L} \cfgnt{R} $\eta$ \cfgnt{e} \cfgnt{k}\rp}
\cfgrule{k}{\cfgt{end}}
\cfgorline{\lp \cfgt{*} \cfgt{\$} \cfgnt{f} $\rightarrow$ \cfgnt{k}\rp}
%     (* @ m (e ...) -> k)
%     (v @ m (v ...) * (e ...) -> k)
%     (* == e -> k)
%     (v == * -> k)
%     (x := * -> k)
%     (x $ f := * -> k)
%     (if * e else e -> k)
%     (var T x := * in e -> k)
%     (begin * (e ...) -> k)
%     (pop η k))
\cfgend
\end{center}
\caption{The machine syntax for Javalite.}
\label{fig:machine-syntax}
\end{figure}


A Javalite program, $\cfgnt{P}$, is a registry of classes, $\mu$, with
a tuple indicating a class, $\cfgnt{C}$, and a method, $\cfgnt{m}$,
where execution starts. The only primitive type in Javalite is
Boolean. Classes have fields, $([\cfgnt{T}\ \cfgnt{f}]\ ...)$, and
methods, $(\cfgnt{M}\ ...)$. Methods only take a single
parameter. Expressions, $\cfgnt{e}$, are somewhat typical with
$\cfgt{:=}$ indicating assignment, $\cfgt{=}$ indicating comparison,
$\cfgt{\$}$ as the dot-operator for field access, and $\cfgt{@}$ as
the dot-operator for method invocation. There is no explicit return
statement in Javalite; rather, the value of the last expression is
used as a return value. A variable is always indicated by \cfgnt{x}
and a value by \cfgnt{v}. A value can be a reference in the heap,
$\cfgnt{r}$, or any of the special values. It is assumed that only
type correct programs are given as input to the machine.

The state of the Javalite machine (\figref{fig:machine-syntax}),
$\cfgnt{s}$, includes the program registry\footnote{The registry $\mu$
  is implied in the rewrite rules rather that states for
  compactness.}, a pair of functions defining the heap
$\lp\cfgnt{L}\ \cfgnt{R}\rp$, which is the store, a constraint,
$\phi$, defined over references in the heap, the environment, $\eta$,
for local variables, and the continuation $\cfgnt{k}$. The more
complex structures such as the environment are defined as lists which
start with empty, \cfgnt{mt}. The rewrite rules that define the
semantics treat these lists as partial functions. As such,
$\eta(\cfgnt{x}) = \cfgnt{r}$ is the reference mapped to the variable
$\cfgnt{x}$. The notation $\eta^\prime = \eta[\cfgnt{x} \mapsto
  \cfgnt{v}]$ defines a new partial function $\eta^\prime$ that is
just like $\eta$ only the variable $\cfgnt{x}$ now maps to
$\cfgnt{v}$.

The continuation $\cfgnt{k}$ helps accomplish the small-step semantics
by indicating where the hole, $\cfgt{*}$, is located, by storing
temporary computations, and by keeping track of the next
computation. For example, the continuation
$\lp\cfgt{*}\ \cfgt{\$}\ \cfgnt{f} \rightarrow \cfgnt{k}\rp$ indicates
that the machine is currently computing the expression for the object
reference on which the field $\cfgnt{f}$ is going to be accessed. Once
the field access is complete, the machine continues with $\cfgnt{k}$.




