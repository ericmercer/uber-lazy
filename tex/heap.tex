\subsection{Summary Heaps as Bipartite Graphs}
A novel contribution of summary heaps for symbolic execution is the
labeled bipartite graph representation of the heap which enables a
summary heap to group heaps together that follow the same control flow
to a given point of execution. The graph structure is able to capture
group membership and compute value sets using entirely non-recursive
local look up. Additionally, updates on the heap do not require
rewriting the path condition or any constraints in summary heap.

The bipartite graph itself consists of references, $\cfgnt{r}$, and
constraint-location pairs, $\lp\phi\ \cfgnt{l}\rp$. The graph is
expressed in the location map, $\cfgnt{L}$, and the reference map,
$\cfgnt{R}$. As done with the environment, $\cfgnt{L}$ and $\cfgnt{R}$
are treated as partial functions where $\cfgnt{L}(r) =
\{(\phi\ \cfgnt{l})\ ...\}$ is the set of constraint-location pairs in
the heap associated with the given reference, and
$\cfgnt{R}(\cfgnt{l},\cfgnt{f}) = \cfgnt{r}$ is the reference
associated with the given location-field pair in the heap. Predicate
calculus is used to describe the more complex updates to the heap in
the semantics.

There are several properties of the summary heap enforced by the
rewrite rules that are invariant and important to the correctness
proofs presented in \secref{sec:bisim}: \emph{reference partitions},
\emph{immutable references}, \emph{determinism}, and \emph{type
  consistency}.

Reference partitions: new references are drawn from
  three distinct partitions (i) $\mathrm{init}_\cfgnt{r}()$ for references for the \emph{input heap}, (ii) $\mathrm{fresh}_\cfgnt{r}()$ for references for \emph{auxiliary literals}, and
  (iii) $\mathrm{stack}_\cfgnt{r}()$ for references for the \emph{stack literals} in the expressions and
  continuations. The references strictly increase in value on each
  call, and modular arithmetic is used to determine the partition to
  which a reference belongs.
\begin{compactdesc}
\item[Input heap] these references have two
  interpretations: one as a literal and another as a variable. First,
  an input heap reference is a literal, and that literal is a node in
  the summary heap. Second, an input heap reference is a variable, and
  that variable is a term in any number of constraints that label
  edges in the summary heap. Only input heap references and the
  special reference $\cfgnt{r}_\mathit{null}$ appear in constraints
\item[Auxiliary literals] these references are only used to define additional structure in
  the bipartite graph and are not part of any constraint and do not appear
  in any environment, expression, or continuation.
\item[Stack references] these references are also used to define additional structure in the bipartite graph but may also appear in the environment, expression, or continuation. These are not part of any constraint.
\end{compactdesc}
Reference partitions are a critical piece to the existence proof of a
bisimilation between the summary heap algorithm and generalized
symbolic execution (see \secref{sec:bisim}).

Immutable references: a reference does not change in the
location map. It always points to the same set of constraint location
pairs. This property ensures that constraints never need to be
rewritten as the program evolves its state.

Determinism: a reference in  $(\cfgnt{L}\ \cfgnt{R})$ cannot point multiple locations simultaneously.
$$
\begin{array}{l}
\forall \cfgnt{r} \in \cfgnt{L}^\leftarrow\ (\forall (\phi\ \cfgnt{l}),(\phi^\prime\ \cfgnt{l}^\prime) \in \cfgnt{L}(r)\ (\\
\ \ \ \ (\cfgnt{l} \neq \cfgnt{l}^\prime \vee \phi \neq \phi^\prime) \Rightarrow (\phi \wedge \phi^\prime = \cfgt{false}))
\end{array}
$$
$\cfgnt{L}^\leftarrow$ is the pre-image of the location
map. Informally, any two constraint location pairs connected to a
common reference form a contradiction: only one can be true in any
satisfying assignment. The property means that a satisfying assignment
resolving aliasing relationships results in a deterministic heap: a
field access returns a single location.

Type consistency: all locations associated with a reference have the same type.
\[
\begin{array}{l}
\forall \cfgnt{r} \in \cfgnt{L}^\leftarrow\ (\forall (\phi\ \cfgnt{l}),(\phi^\prime\ \cfgnt{l}^\prime) \in \cfgnt{L}(r)\ (\\
\ \ \ \ (\mathrm{Type}\lp\cfgnt{l}\rp = \mathrm{Type}\lp\cfgnt{l}^\prime\rp)))
\end{array}
\]

Null and Uninitialized: every summary heap contains 
a special location for null ($\cfgnt{l}_\mathit{null}$), and a special
location for uninitialized
  ($\cfgnt{l}_\mathit{un}$), with corresponding references
  $\cfgnt{r}_\mathit{null}$ and
  $\cfgnt{r}_\mathit{un}$, respectively. $\cfgnt{L}(\cfgnt{r}_\mathit{null}) =
  \{(\cfgt{true}\ \cfgnt{l}_\mathit{null})\}$ and
  $\cfgnt{L}(\cfgnt{r}_\mathit{un}) =
  \{(\cfgt{true}\ \cfgnt{l}_\mathit{un})\}$. These special references can be used in constraints.

The reference partitions merit further discussion as they are a
central to the proofs. The heap summary algorithm creates
many references that are auxiliary literals or stack literals to
define the structure of the bipartite graph.  It only uses input heap
references for constraints expressing potential aliasing. A solver
reasons over the aliasing relationships, and a satisfying assignment
uniquely identifies active edges in the graph. Those active edges
describe a concrete heap and its shape.

For example, if the solver assigns input heap variables $\cfgnt{r}$ and $\cfgnt{r}^\prime$
such that $\cfgnt{r} = \cfgnt{r}^\prime$ is true, then there exists
some location $\cfgnt{l}$ and constraints $\phi$ and $\phi^\prime$ such
that
\begin{compactitem}
\item $\lp\phi\ \cfgnt{l}\rp \in \cfgnt{L}(\cfgnt{r})$; and
\item $\lp\phi^\prime\ \cfgnt{l}\rp \in \cfgnt{L}(\cfgnt{r}^\prime)$; and
\item $\phi$ and $\phi^\prime$ are true under aliasing assignments from the solver.
\end{compactitem}
In this way, the graph groups heaps with the conditions under which
those heaps exist. This representation, by virtue of the reference
partitions, separates the constraint problem from the morphology
problem: it is possible to update the heap without having to rewrite
constraints. Without this separation, it would not be possible to
lazily initialize heap locations along a program path.

\subsection{State Transition Relation}

The initial state of a Javalite program is constructed from its
surface syntax on the $\cfgnt{P}$ production:
$\lp\mu\ \lp\cfgnt{C}\ \cfgnt{m}\rp\rp$. Assuming that
$\lp\cfgnt{L}\ \cfgnt{R}\rp$ is an empty heap with no references
or locations and $\eta$ is an empty environment with no
variables, the initial state is
$$
\begin{array}{l}
\lp\mu 
\ \cfgnt{L}[\cfgnt{r}_\mathit{null} \mapsto \{\lp\cfgt{true}\ \cfgnt{l}_\mathit{null})\}\rp] 
           [\cfgnt{r}_\mathit{un} \mapsto \{\lp\cfgt{true}\ \cfgnt{l}_\mathit{un}\rp\}] \\
\ \cfgnt{R}
\ \cfgt{true}\ \eta\  \lp\lp\cfgt{new}\ \cfgnt{C}\rp\ \cfgt{@}\ \cfgnt{m}\ \cfgt{true}\rp\rp\ \cfgt{end}\rp
\end{array}
$$
The first expression in the initial state is to construct a new object
in the summary heap of type $\cfgnt{C}$. Note that the resulting object is not referenced
by the input heap since it comes from the new-expression.  Once an
instance of $\cfgnt{C}$ exists in the summary heap, the method
$\cfgnt{m}$ of the class is invoked.

In general, the semantics of Javalite are expressed as
rewrites on strings using pattern matching. Consider the rewrite rule
for the beginning of a field access instruction in the Javalite
machine:
$$
\mprset{flushleft}
	\inferrule[Field Access(eval)]{}{
      \lp \cfgnt{L}\ \cfgnt{R}\ \phi\ \eta\ \lp \cfgnt{e}\ \cfgt{\$}\ \cfgnt{f}\rp \ \cfgnt{k}\rp  \rcom 
      \lp \cfgnt{L}\ \cfgnt{R}\ \phi\ \eta\ \cfgnt{e}\ \lp \cfgt{*}\ \cfgt{\$}\ \cfgnt{f} \rightarrow \cfgnt{k}\rp \rp 
	}
$$
If the string representing the current state matches the left side, then it
creates the new string on the right. In this example, the new string
on the right is now evaluating the expression $\cfgnt{e}$ in the field
access, and it includes the continuation indicating that it still
needs to complete the actual field access once the expression is
evaluated.

Other more complex rules such as the one to create a new instance of a
class define constraints on the rewrites and more complex
transformations on the heap.
$$
\mprset{flushleft}
	\inferrule[New]{
      \cfgnt{r} = \mathrm{stack}_r\lp \rp\\
      l = \mathrm{fresh}_l\lp \cfgnt{C}\rp\\\\
      \cfgnt{R}^\prime = \cfgnt{R}[\forall \cfgnt{f} \in \mathit{fields}\lp \mathrm{C}\rp \ \lp \lp l\ \cfgnt{f}\rp  \mapsto \cfgnt{r}_\mathit{null} \rp ] \\\\
      \cfgnt{L}^\prime = \cfgnt{L}[\cfgnt{r} \mapsto \{\lp \cfgt{true}\ l\rp \}]
    }{
      \lp \cfgnt{L}\ \cfgnt{R}\ \phi\ \eta\ \lp \cfgt{new}\ \cfgnt{C}\rp \ \cfgnt{k}\rp  \rcom
      \lp \cfgnt{L}^\prime\ \cfgnt{R}^\prime\ \phi\ \eta\ \cfgnt{r}\ \cfgnt{k}\rp 
	}
$$
In this rule, when the string matches the new-expression, it is rewritten to use
a new heap location where all of the fields for the new object point to
$\cfgnt{r}_\mathit{null}$
and the location map points a new stack reference to that new object.

To show the heap summary algorithm is sound and complete with respect
to properties proved by generalized symbolic execution, we refer to
the generalized symbolic execution with lazy initialization machine by
the relation $\rgse$ or $p \rgse p^\prime$, and to the new summary
heap algorithm machine by $\rsym$ or $q \rsym q^\prime$. Both machines
use the summary heap structure, and for simplicity,
both use lazy initialization as opposed to lazier or lazier\#.

Due to space restrictions, the full operational semantics for
generalized symbolic execution with lazy initialization for Javalite
are presented in the supplemental document. We also present the
portion of the Javalite rewrite rules which are common to both
machines in the supplementary document. In the next section, we
present the rewrite rules specific to the new summary heap algorithm
for symbolic execution.
