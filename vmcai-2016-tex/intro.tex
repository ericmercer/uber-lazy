\section{Introduction}

In symbolic execution, the values of program variables are represented
as constraints over the program's inputs. During the course of an
execution, a program variable may assume a number of possible values
under a variety of different conditions. One way to represent this
behavior is by pairing each possible value with the constraints under
which it is assumed. A set of such value-constraint pairs, called a
\emph{guarded value set}\footnote{Guarded value sets are variously
  referred to in the literature as \emph{guarded location sets},
  \emph{symbolic value sets}, or \emph{value summaries}. The term
  guarded value set is sometimes abbreviated in this text
  as~\emph{value set}.}, represents the state of a program variable.

Guarded value sets are rising in importance in the representation of
reference values. First appearing in Verification-Condition Generator
(VCG) style techniques~\cite{Xie:2005,Dillig:2011}, guarded value sets
are a convenient way to represent sets of heap shapes. More recently,
guarded value sets are being used in symbolic execution for the
purposes of modularization~\cite{Yorsh:2008}, state
merging~\cite{Sen:2014}, and for determining
invariants~\cite{Ferrara:2014,Torlak:2014}.

A common hurdle to using guarded value sets is the treatment of
\emph{symbolic} heap inputs. A symbolic heap input refers to a portion of the heap that is yet unconstrained, meaning that it is able to take on any shape. For many applications, such a symbolic heap input is desirable for modeling sets of heaps in the analysis: the operations on the heap discovered in the analysis further constrain the symbolic heap input
structure.

Precisely modeling operations on unconstrained heaps
is challenging because it requires formulating logical predicates over
an input domain that contains a potentially unbounded number of hidden
references~\cite{Chen:2013,Qu:2011}. Generalized symbolic execution
(\gsetxt{}), and its variants, provide an accurate
solution, but they quickly produce an overwhelming number of execution
paths for all but the simplest heap shapes \cite{GSE03,Deng:2007}. Other efforts to 
create heap inputs for guarded value set-based analysis 
techniques \cite{Dillig:2011,Xie:2005} have yet to
produce results provably equivalent to~\gsetxt{}. Thus, despite
the advantages of using guarded value sets, it remains an open
question to how they can be used in symbolic execution to automatically
model all possible program behaviors in the case of an arbitrary input
heap.

The contribution of this work is \emph{symbolic initialization} for 
uninitialized references in a fully symbolic heap. Where \gsetxt\ lazily
instantiates uninitialized references to either NULL, a new instance
of the correct type, or an alias to a previously initialized object, \symtxt\ creates a guarded value set
expressing all of these eventualities in a single symbolic heap. Where
\gsetxt\ branches the search space on each choice, \symtxt{} does not. Symbolic initialization only branches on reference compares; thus it creates equivalence classes over heaps and only partitions those classes at compares to indicate heaps where the references
alias and heaps where the references do not.

This paper includes a proof that \symtxt\ is sound and
complete with respect to properties provable by \gsetxt. Such a result is
important because it means that \symtxt\ can be used to
create test inputs for other analyses in a way that is provably
correct with regards to \gsetxt: there are no missing \gsetxt\ heaps and there are
no extra GSE heaps.

A proof-of-concept implementation of symbolic initialization is
reported in this paper. The implementation is for Java in the Java
Pathfinder tool. It demonstrates on a common set of data structure benchmarks for GSE evaluation~\cite{Deng:2006,Deng:2007,boyapati2002korat,Ferrara:2014,Rosner:2015} an increase in the number of heaps that can be analyzed when compared to existing ~\gsetxt{} methods. Although guarded value
sets require extra interaction with an SMT solver, the
savings in representing multiple heap shapes in a single
structure can overcome the cost of the solver calls. For the tree structures in the standard benchmarks, the
comparison shows an exponential increase in the number of heaps that
can be analyzed, meaning that other
approaches based on~\gsetxt\ fail to complete in the allotted time.

In summary, this paper presents a new initialization technique that
enables the use of guarded value sets in heap representations in a
sound and complete manner. Such a result means that any analysis using
this initialization technique can be assured of precision with regard
to \gsetxt{} semantics \cite{THINGSTHATBENEFIT}. This includes generating concrete heaps for test input generation.
