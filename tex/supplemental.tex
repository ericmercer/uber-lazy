
\documentclass[pldi]{sigplanconf-pldi15}

\usepackage{amsmath}
\usepackage{comment}
\usepackage{mathpartir}
\usepackage{amssymb}
\usepackage{amsfonts}

\hyphenation{op-tical net-works semi-conduc-tor}

%% Commands For the syntax EBNF
\newcommand{\cfgnt}[1]{\emph{#1}}
\newcommand{\cfgq}[1]{\texttt{#1}}
\newcommand{\cfgt}[1]{\textbf{#1}}
\newcommand{\cfglhs}[1]{\cfgnt{#1} & $::=$}
\newcommand{\cfgrule}[2]{\cfglhs{#1} & #2 \\}
\newcommand{\cfgor}{\textbar\ }
\newcommand{\cfgstart}{\begin{tabular}{r@{\hspace{1mm}}r@{\hspace{2mm}}l}}
\newcommand{\cfgend}{\end{tabular}}
\newcommand{\cfgline}[1]{ ~ && #1 \\ }
\newcommand{\cfglinetab}[1]{ ~ && \hspace{1cm} #1 \\ }
\newcommand{\cfgorline}[1]{ ~ & \cfgor & #1 \\ }
\newcommand{\lp}{\cfgq{(}}
\newcommand{\lb}{\cfgq{[}}
\newcommand{\rp}{\cfgq{)}}
\newcommand{\rb}{\cfgq{]}}
\newcommand{\hsp}[0]{\hspace{1mm}}


\newcommand{\figref}[1]{Figure~\ref{#1}}
\newcommand{\secref}[1]{Section~\ref{#1}}
\newcommand{\thref}[1]{Theorem~\ref{#1}}
\newcommand{\lemref}[1]{Lemma~\ref{#1}}
\newcommand{\defref}[1]{Definition~\ref{#1}}

\newcommand{\sym}{\ensuremath{\varsigma}}
\newcommand{\gse}{\ensuremath{g}}
\newcommand{\rsym}{\ensuremath{\rightarrow_\sym}}
\newcommand{\rssym}{\ensuremath{\leadsto_\sym}}
\newcommand{\rgse}{\ensuremath{\rightarrow_\gse}}
\newcommand{\rsgse}{\ensuremath{\leadsto_\gse}}
\newcommand{\rsum}{\ensuremath{\rightarrow_S}}
\newcommand{\rinit}{\ensuremath{\rightarrow_I}}
\newcommand{\com}{\ensuremath{J}}
\newcommand{\rcom}{\ensuremath{\rightarrow_\com}}

\usepackage{url}
\usepackage{listings}
\usepackage{color}
\usepackage[T1]{fontenc}
%\usepackage{SIunits}            % typset units correctly
\usepackage{courier}            % standard fixed width font
\usepackage[scaled]{helvet} % see www.ctan.org/get/macros/latex/required/psnfss/psnfss2e.pdf

\newcommand{\doi}[1]{doi:~\href{http://dx.doi.org/#1}{\Hurl{#1}}}   % print a hyperlinked DOI

\definecolor{dkgreen}{rgb}{0,0.6,0}
\definecolor{gray}{rgb}{0.5,0.5,0.5}
\definecolor{mauve}{rgb}{0.58,0,0.82}

\lstset{frame=tb,
  language=Java,
  aboveskip=3mm,
  belowskip=3mm,
  showstringspaces=false,
  columns=flexible,
  basicstyle={\small\ttfamily},
  numbers=none,
  numberstyle=\tiny\color{gray},
  keywordstyle=\color{blue},
  commentstyle=\color{dkgreen},
  stringstyle=\color{mauve},
  breaklines=true,
  breakatwhitespace=true
  tabsize=3
}


\usepackage{graphicx}
\usepackage{graphics}
\usepackage{epsfig}
\usepackage{comment}

% used for inline lists
\usepackage{paralist}

%% This enables the xfig overlays to use the same font family as the document
%% (i.e., font family and size is the same in the figure as it is
%% in the text).

\gdef\SetFigFont#1#2#3#4#5{}

\usepackage{multirow}
\usepackage{algorithm} 
\RequirePackage[noend]{algorithmic}
\renewenvironment{algorithm}[1][\textwidth]%  
{\begin{minipage}[t][\totalheight][c]{#1}\begin{algorithmic}[1]}  %%% change [1] to [0] to turn off line numbers
{\end{algorithmic}\end{minipage}}

%% Use for each in ``FOR'' constructs

\renewcommand{\algorithmicfor}{\textbf{for each}}

%% All comments are in italics

\renewcommand{\algorithmiccomment}[1]{\textit{${/\ast}$~#1~${\ast/}$}}

%% Use ``procedure'' instead of ``Algorithm'' for off set

\renewcommand{\algorithmicensure}{\textbf{procedure}}

\newcommand{\algoname}[1]{\ENSURE #1}
\newcommand*{\algobox}[1]{\framebox{#1}}

\newcommand{\bo}[1]{\textbf{#1}}
\newcommand{\cc}[1]{\cellcolor[gray]{.6}{#1}}
\newcommand{\negspace}{\hspace{-.40cm}}

%define theorem environments
\usepackage{mathtools}
\usepackage{amsthm}
\newtheorem{theorem}{Theorem}
\newtheorem{lemma}[theorem]{Lemma}
\newtheorem{proposition}[theorem]{Proposition}
\newtheorem{corollary}[theorem]{Corollary}

\newtheorem{definition}{Definition}
%\newenvironment{definition}[1][Definition]{\begin{trivlist}
%\item[\hskip \labelsep {\bfseries #1}]}{\end{trivlist}}

%% This enables the xfig overlays to use the same font family as the document
%% (i.e., font family and size is the same in the figure as it is
%% in the text).

\gdef\SetFigFont#1#2#3#4#5{}




\begin{document}



%\title{Contraints-based Reasoning of Heaps in Symbolic Execution}

%\title{Using Constraints to Characterize Heaps in Symbolic Execution}

%\title{Symbolic Execution with precise heap constraints}

%\title{Precise Heap Summaries from Symbolic Execution}

\title{Supplemental Document}


\maketitle

\section{Proofs}

\subsection{Definitions}

\begin{definition}
A \textbf{state transition function} $\rightarrow_{\Phi}$ is a mapping $\rightarrow_{\Phi} : s \mapsto s$ , which takes one machine state and transforms it into another machine state. The state transition function with superscript represents composition of the state transition function: $$ s_a \rightarrow_{\Phi} s_b \rightarrow_{\Phi} s_c \implies s_a \rightarrow_{\Phi}^2 s_c $$
\end{definition}

\begin{definition}
A \textbf{feasible state sequence} is defined as a sequence of states resulting from repeated application of the state transition relation to some initial state $s_0$: $$\Pi_n = s_0,s_1,...,s_n$$ where the relation $s_i \rightarrow_{\Phi} s_{i+1}$ holds for all $i \in \{ i | 0 \leq i < n \}$
\end{definition}

\begin{definition}
Given a sequence of states $$\Pi_n = s_0,s_1,...,s_n$$ where $$s_i = ( \mu_i\ \cfgnt{L}_i\ \cfgnt{R}_i\ \phi_i\ \eta_i\ \cfgnt{e}_i\ \cfgnt{k}_i )$$ the \textbf{control flow sequence} of $\Pi_n$ is the defined as the sequence of tuples $$ \pi_n = \mathbb{CF}(\Pi_n) = (\eta_0\ \cfgnt{e}_0\ \cfgnt{k}_0),(\eta_1\ \cfgnt{e}_1\ \cfgnt{k}_1),...,(\eta_n\ \cfgnt{e}_n\ \cfgnt{k}_n)$$
\end{definition}

\begin{definition}
Given a state transition function $\rightarrow_{\Phi}$, an initial state $s_0$ and a control flow sequence $\pi_n$, the \textbf{feasible state set} $\zeta = \mathbb{FS}(\rightarrow_{\Phi},s_0,\pi_n)$ is defined as
 $$\zeta = \{ \forall s | \pi_n = \mathbb{CF}(\Pi_n) \wedge s = max_s(\Pi) \wedge s_0 \rightarrow_{\Phi}^{n-1} s\} $$
\end{definition}

\begin{definition}
A \textbf{heap homomorphism} $(g\ h)$ between two states $s_x$ and $s_y$ is defined as a pair of functions $g:\cfgnt{r} \mapsto \cfgnt{r}$ and $h:\cfgnt{l} \mapsto \cfgnt{l}$ such that for any reference $r \in s_x$, location $l \in s_x$, and field $f$, $$ l \in \cfgnt{L}_x(r) \Leftrightarrow h(l)\in \cfgnt{L}_y(g(r))$$ and $$ r = \cfgnt{R}_x(l,f) \Leftrightarrow g(r) = \cfgnt{R}_y(h(l),f)$$ If a such a pair of functions exists from one state $s_x$ to another state $s_y$ we say that state $s_x$ is \textbf{heap homomorphic} to state $s_y$, indicated by the notation $(g\ f):\ s_x \rightarrow s_y$. 

Suppose we take the set of constraints from the image of $s_x$ in $s_y$ under $(g\ h)$ :
$$\chi = \{ \phi\ | \exists (r \in \cfgnt{R}_x,\  l \in \cfgnt{L}_x) ( (\phi\ h(l)) \in \cfgnt{L}_y (g(r))  \}$$ and we conjoin those constraints with the global invariant from $s_y$ :
$$\phi_G = \phi_y \bigwedge_{\phi_i \in \chi} \phi_i $$
 If the expression $\phi_G$ is satisfiable, we say that the heap homomorphism $(g\ h):\ s_x \rightarrow s_y $ is \textbf{valid}.
\end{definition}
\begin{definition}
The \textbf{representation relation} is defined as follows: given two states $s_\ell$ and $s_s$, $s_\ell \sqsubset s_s $ if and only if $\eta_{\ell} = \eta_{s} ,\ \cfgnt{e}_{\ell} = \cfgnt{e}_{s} ,\ \cfgnt{k}_{\ell} = \cfgnt{k}_{s}$, and there exists a valid heap homomorphism $(g\ h):\ s_\ell \rightarrow s_s $. The expression $s_\ell \sqsubset s_s $ can be read as "state $s_\ell$ is represented by state $s_s$ . 
\end{definition}

\begin{definition}
A state $s_s$ is \textbf{congruent} to a set of states $\mathcal{S}$ if and only if $s$ represents every state in $\mathcal{S}$ and represents no other state: 
$$ s_s \equiv \mathcal{S} : s_i \in \mathcal{S} \Leftrightarrow s_i \sqsubset s_s $$
\end{definition}

\begin{definition}
A symbolic state $s_s$ is \textbf{exact} with respect to an initial state $s_0$ and control flow sequence $\pi$ if and only if it is congruent the set of feasible lazy states on the same control flow path:
$$ s_s \equiv \mathbb{FS}(\rightarrow_{\ell},s_0,\pi)$$
\end{definition}

\subsection{Theorems}

\begin{lemma}
If symbolic state $s_s = \lp \cfgnt{L}_{\mathcal{S}}\ \cfgnt{R}_{\mathcal{S}}\ \phi_g\ \eta\ \cfgnt{r}\ \lp \cfgt{*}\ \cfgt{\$}\ \cfgnt{f} \rightarrow \cfgnt{k}\rp \rp$ is exact with respect to some initial state $s_0$ and control flow path $\pi_n$, then the state $s_s^\prime : s_s \rightarrow_s s_s^\prime$ is exact with respect to $s_0$ and $\pi_{n+1}.
\end{lemma}

\begin{proof}
We will consider two cases for this proof. In the first case, we assume that all of the fields involved in the read are initialized. In the second case we consider the case of uninitialized fields. 

Case 1: suppose all of the pertinent fields in $s_s$ are initialized. Take an arbitrary lazy state $s_\ell \sqsubset s_s$. Since $s_s$ is exact,  $s_\ell = \lp \cfgnt{L}_{\ell}\ \cfgnt{R}_{\ell}\ \phi_L\ \eta\ \cfgnt{r}\ \lp \cfgt{*}\ \cfgt{\$}\ \cfgnt{f} \rightarrow \cfgnt{k} \rp \rp$. If we apply the state transition functions to achieve states $s_\ell^\prime : s_\ell \rightarrow_\ell s_\ell^\prime$ and $s_s^\prime : s_s \rightarrow_s s_s^\prime$, we find that:
$$s_\ell^\prime = \lp \cfgnt{L}_{\ell} [\cfgnt{r}^\prime \mapsto \lp\phi^\prime\ l^\prime\rp]\ \cfgnt{R}_{\ell}\ \phi_L\ \eta\ \cfgnt{r}^\prime\ \cfgnt{k}\rp $$
 and 
 $$ s_s^\prime = \lp \cfgnt{L}_{s}[\cfgnt{r}^\prime \mapsto \mathbb{VS}\lp \cfgnt{L}_{\mathcal{S}},\cfgnt{R}_{s},\cfgnt{r},\cfgnt{f},\phi_g\rp ]\ \cfgnt{R}_{\mathcal{S}}\ \phi_g\ \eta\ \cfgnt{r}^\prime\ \cfgnt{k}\rp $$

We now show that $s_\ell^\prime \sqsubset s_s^\prime$. Since $\eta$, $e$, and $k$ are identical between $s_s^\prime$ and $s_\ell^\prime $, the first condition is met by default. To find a valid heap homomorphism, first construct functions $g^\prime : g^\prime = g[ r^\prime \mapsto r^\prime]$ and $h^\prime : h^\prime = h$. 

%we now need to show that we have a heap homomorphism here, and that the homomorphism is valid. once we've done that , we've proved the soundness case

Now, we show that no infeasible lazy states are allowed by $\mathcal{S}^\prime$. NEED TO FILL IN THIS BIT, TOO

$\mathbb{S}(\bigwedge_{(\phi l) \in L_s^i} \phi)) \wedge \forall l \in L_L (\exists l \in L_s^i)$

Since $s_\ell \subseteq s_s$
\end{proof}



\section{Machine Syntax}
\begin{figure}
\begin{center}
\cfgstart
\cfgrule{e}{\lp ... \cfgor \lp \cfgnt{v} \cfgt{@} \cfgnt{m} \cfgnt{v} \rp\rp}
%\cfgrule{object}{ (\cfgnt{C} [ \cfgnt{f} \cfgnt{loc} ] ...) }
%\cfgrule{hv}{ (\cfgnt{v} \cfgnt{object})}
\cfgrule{$\phi$}{\cfgnt{constraint}}
\cfgrule{l}{\cfgt{number}}
%\cfgrule{h}{(\cfgnt{mt}\ (\cfgnt{h}\ [\cfgnt{loc} $\rightarrow$ \cfgnt{hv}]) )}
\cfgrule{$\eta$}{(\cfgnt{mt}\ ($\eta$ [\cfgnt{x} $\rightarrow$ \cfgnt{loc}]))}
\cfgrule{s}{\lp$\mu$ \cfgnt{L} \cfgnt{R} \cfgnt{g} $\eta$ \cfgnt{e} \cfgnt{k}\rp}
\cfgrule{k}{\cfgt{end}}
\cfgorline{\lp \cfgt{*} \cfgt{\$} \cfgnt{f} $\rightarrow$ \cfgnt{k}\rp}
\cfgorline{\lp \cfgt{*} \cfgt{@} \cfgnt{m} \lp \cfgnt{e} ... \rp $\rightarrow$ \cfgnt{k} \rp}
\cfgorline{\lp \cfgnt{v} \cfgt{@} \cfgnt{m} \cfgnt{v} \cfgt{*} \lp \cfgnt{e} ... \rp $\rightarrow$ \cfgnt{k} \rp}
\cfgorline{\lp \cfgt{*} \cfgt{=} \cfgnt{e} $\rightarrow$ \cfgnt{k}\rp}
\cfgorline{\lp \cfgt{v} \cfgt{=} \cfgnt{*} $\rightarrow$ \cfgnt{k}\rp}
\cfgorline{\lp \cfgt{x} \cfgt{:=} \cfgnt{*} $\rightarrow$ \cfgnt{k}\rp}
\cfgorline{\lp \cfgt{x} \cfgt{\$} \cfgnt{f}  \cfgt{:=} \cfgnt{*} $\rightarrow$ \cfgnt{k}\rp}
\cfgorline{\lp \cfgt{if} \cfgnt{*} \cfgnt{e} \cfgt{else} \cfgnt{e} $\rightarrow$ \cfgnt{k} \rp}
\cfgorline{\lp\cfgt{var} \cfgnt{T} \cfgnt{x} \cfgt{:=} \cfgnt{*} \cfgt{in} \cfgnt{e}  $\rightarrow$ \cfgnt{k} \rp}
\cfgorline{\lp\cfgt{begin}  \cfgnt{*} \lp \cfgnt{e} ...\rp $\rightarrow$ \cfgnt{k} \rp}
\cfgorline{\lp\cfgt{pop} $\eta$ \cfgnt{k}\rp}
\cfgend
\end{center}
\caption{The machine syntax for Javalite.}
\label{fig:machine-syntax}
\end{figure}


\section{Common Rules for Javalite}

\begin{figure*}[t]
\begin{center}
\mprset{flushleft}
\begin{mathpar}
	\inferrule[Variable lookup]{}{
      \lp \cfgnt{L}\ \cfgnt{R}\ \phi\ \eta\ \cfgnt{x}\ \cfgnt{k}\rp  \rcom \\\\
      \lp \cfgnt{L}\ \cfgnt{R}\ \phi\ \eta\ \eta\lp \cfgnt{x}\rp \ \cfgnt{k}\rp 
	}
\and
	\inferrule[New]{
      \cfgnt{r} = \mathrm{stack}_r\lp \rp\\
      l = \mathrm{fresh}_l\lp \cfgnt{C}\rp\\\\
      \cfgnt{R}^\prime = \cfgnt{R}[\forall \cfgnt{f} \in \mathit{fields}\lp \mathrm{C}\rp \ \lp \lp l\ \cfgnt{f}\rp  \mapsto \cfgnt{r}_\mathit{null} \rp ] \\\\
      \cfgnt{L}^\prime = \cfgnt{L}[\cfgnt{r} \mapsto \{\lp \cfgt{true}\ l\rp \}]
    }{
      \lp \cfgnt{L}\ \cfgnt{R}\ \phi\ \eta\ \lp \cfgt{new}\ \cfgnt{C}\rp \ \cfgnt{k}\rp  \rcom
      \lp \cfgnt{L}^\prime\ \cfgnt{R}^\prime\ \phi\ \eta\ \cfgnt{r}\ \cfgnt{k}\rp 
	}
\and
	\inferrule[Field Access(eval)]{}{
      \lp \cfgnt{L}\ \cfgnt{R}\ \phi\ \eta\ \lp \cfgnt{e}\ \cfgt{\$}\ \cfgnt{f}\rp \ \cfgnt{k}\rp  \rcom \\\\
      \lp \cfgnt{L}\ \cfgnt{R}\ \phi\ \eta\ \cfgnt{e}\ \lp \cfgt{*}\ \cfgt{\$}\ \cfgnt{f} \rightarrow \cfgnt{k}\rp \rp 
	}
\and
	\inferrule[Field Write (eval)]{}{
       \lp \cfgnt{L}\ \cfgnt{R}\ \phi\ \eta\ \lp \cfgnt{x}\ \cfgt{\$}\ \cfgnt{f}\ \cfgt{:=}\ \cfgnt{e}\rp \ \cfgnt{k}\rp  \rcom \\\\
       \lp \cfgnt{L}\ \cfgnt{R}\ \phi\ \eta\ \cfgnt{e}\ \lp \cfgnt{x}\ \cfgt{\$}\ \cfgnt{f}\ \cfgt{:=}\ \cfgt{*}\ \rightarrow\ \cfgnt{k}\rp \rp 
	}
\and
    \inferrule[Equals (l-operand eval)]{}{
      \lp \cfgnt{L}\ \cfgnt{R}\ \phi\ \eta\ \lp \cfgnt{e}_0\ \cfgt{=}\ \cfgnt{e}\rp  \ \cfgnt{k}\rp  \rcom \\\\
      \lp \cfgnt{L}\ \cfgnt{R}\ \phi\ \eta\ \cfgnt{e}_0\ \lp \cfgt{*}\ \cfgt{=}\; \cfgnt{e} \rightarrow \cfgnt{k}\rp \rp 
    }
\and
    \inferrule[Equals (r-operand eval)]{}{
    \lp \cfgnt{L}\ \cfgnt{R}\ \phi\ \eta\ \cfgnt{v}\ \lp \cfgt{*}\; \cfgt{=}\; \cfgnt{e} \rightarrow \cfgnt{k}\rp \rp  \rcom \\\\
    \lp \cfgnt{L}\ \cfgnt{R}\ \phi\ \eta\ \cfgnt{e}\ \lp \cfgnt{v}\; \cfgt{=}\; \cfgt{*} \rightarrow \cfgnt{k}\rp \rp 
    }
\and
    \inferrule[Equals (bool)]{
    \cfgnt{v}_0 \in \{\cfgt{true}, \cfgt{false}\} \\
    \cfgnt{v}_1 \in \{\cfgt{true}, \cfgt{false}\} \\\\
    \cfgnt{v}_r = \mathrm{eq?}\lp \cfgnt{v}_0, \cfgnt{v}_1\rp}{
    \lp \cfgnt{L}\ \cfgnt{R}\ \phi\ \eta\ \cfgnt{v}_0\ \lp \cfgnt{v}_1\; \cfgt{=}\; \cfgt{*} \rightarrow \cfgnt{k}\rp \rp  \rcom \\\\
    \lp \cfgnt{L}\ \cfgnt{R}\ \phi\ \eta\ \cfgnt{v}_r\ \cfgnt{k}\rp 
    }
\and
    \inferrule[If-then-else (eval)]{}{
      \lp \cfgnt{L}\ \cfgnt{R}\ \phi\ \eta\ \lp \cfgt{if}\ \cfgnt{e}_0\ \cfgnt{e}_1\ \cfgt{else}\ \cfgnt{e}_2\rp \ \cfgnt{k}\rp  \rcom \\\\
      \lp \cfgnt{L}\ \cfgnt{R}\ \phi\ \eta\ \cfgnt{e}_0\ \lp \cfgt{if}\ \cfgt{*}\ \cfgnt{e}_1\ \cfgt{else}\ \cfgnt{e}_2\rp  \rightarrow \cfgnt{k}\rp 
	}
\and
	\inferrule[If-then-else (true) ]{}{
       \lp \cfgnt{L}\ \cfgnt{R}\ \phi\ \eta\ \cfgt{true}\ \lp \cfgt{if}\ \cfgt{*}\ \cfgnt{e}_1\ \cfgt{else}\ \cfgnt{e}_2\rp  \rcom \cfgnt{k}\rp  \rightarrow \\\\
       \lp \cfgnt{L}\ \cfgnt{R}\ \phi\ \eta\ \cfgnt{e}_1\  \cfgnt{k}\rp 
	}
\and
	\inferrule[If-then-else (false)]{}{
       \lp \cfgnt{L}\ \cfgnt{R}\ \phi\ \eta\ \cfgt{false}\ \lp \cfgt{if}\ \cfgt{*}\ \cfgnt{e}_1\ \cfgt{else}\ \cfgnt{e}_2\rp  \rcom \cfgnt{k}\rp  \rightarrow \\\\
       \lp \cfgnt{L}\ \cfgnt{R}\ \phi\ \eta\ \cfgnt{e}_2\  \cfgnt{k}\rp 
	}
\and
   \inferrule[Variable Declaration (eval)]{}{
    \lp \cfgnt{L}\ \cfgnt{R}\ \phi\ \eta\ \lp\cfgt{var}\ \cfgnt{T}\ \cfgnt{x}\ \cfgt{:=}\ \cfgnt{e}_0\ \cfgt{in}\ \cfgnt{e}_1\rp\ \cfgnt{k}\rp  \rcom \\\\
    \lp \cfgnt{L}\ \cfgnt{R}\ \phi\ \eta\ \cfgnt{e}_0\ \lp\cfgt{var}\ \cfgnt{T}\ \cfgnt{x}\ \cfgt{:=}\ \cfgt{*}\ \cfgt{in}\ \cfgnt{e}_1 \rightarrow \cfgnt{k}\rp\rp 
   }	
\and
   \inferrule[Variable Declaration]{}{
    \lp \cfgnt{L}\ \cfgnt{R}\ \phi\ \eta\ \cfgnt{v}\ \lp\cfgt{var}\ \cfgnt{T}\ \cfgnt{x}\ \cfgt{*}\ \cfgt{:=}\ \cfgt{in}\ \cfgnt{e}_1 \rightarrow \cfgnt{k}\rp\rp  \rcom \\\\
    \lp \cfgnt{L}\ \cfgnt{R}\ \phi\ \eta[x \mapsto \cfgnt{v}]\ \cfgnt{e}_1\ \lp \cfgt{pop}\ \eta\ \cfgnt{k}\rp \rp 
   }	
\and
   \inferrule[Method Invocation (object eval)]{}{
    \lp \cfgnt{L}\ \cfgnt{R}\ \phi\ \eta\ \lp\cfgnt{e}_0\ \cfgt{@}\ \cfgnt{m}\ \cfgnt{e}_1\rp\ \cfgnt{k}\rp  \rcom \\\\
    \lp \cfgnt{L}\ \cfgnt{R}\ \phi\ \eta\ \cfgnt{e}_0\ \lp \cfgt{*}\ \cfgt{@}\ \cfgnt{m}\ \cfgnt{e}_1\ \rightarrow \cfgnt{k}\rp \rp 
   }
\and
   \inferrule[Method Invocation (arg eval)]{}{
    \lp \cfgnt{L}\ \cfgnt{R}\ \phi\ \eta\ \cfgnt{v}_0\ \lp \cfgt{*}\ \cfgt{@}\ \cfgnt{m}\ \cfgnt{e}_1\ \rightarrow \cfgnt{k}\rp \rp  \rcom \\\\
    \lp \cfgnt{L}\ \cfgnt{R}\ \phi\ \eta\ \cfgnt{e}_1\ \lp \cfgnt{v}_0\ \cfgt{@}\ \cfgnt{m}\ \cfgt{*}\ \rightarrow \cfgnt{k}\rp \rp 
   }
\and
   \inferrule[Method Invocation]{
    \cfgnt{C} = \mathrm{type}\lp\cfgnt{r}\rp \\\\
    \lp\cfgnt{T}_o\ \cfgnt{m}\ \lb\cfgnt{T}_1\ \cfgnt{x}\rb\ \ \cfgnt{e}_m\rp = \mathrm{lookup}\lp \cfgnt{C},\cfgnt{m}\rp\\\\
    \eta_m = \eta[\cfgt{this} \mapsto \cfgnt{r}][\cfgnt{x} \mapsto \cfgnt{v}]}{
    \lp \cfgnt{L}\ \cfgnt{R}\ \phi\ \eta\ \cfgnt{v}\ \lp \cfgnt{r}\ \cfgt{@}\ \cfgnt{m}\ \cfgt{*}\ \rightarrow \cfgnt{k}\rp \rp  \rcom \\\\
    \lp \cfgnt{L}\ \cfgnt{R}\ \phi\ \eta_m\ \cfgnt{e}_m\ \lp \cfgt{pop}\ \eta\ \cfgnt{k}\rp \rp 
   }
\and
   \inferrule[Variable Assignment (eval)]{}{
    \lp \cfgnt{L}\ \cfgnt{R}\ \phi\ \eta\ \lp \cfgnt{x}\ \cfgt{:=}\ \cfgnt{e}\rp \ \cfgnt{k}\rp  \rcom \\\\
    \lp \cfgnt{L}\ \cfgnt{R}\ \phi\ \eta\ \cfgnt{e}\ \lp \cfgnt{x}\ \cfgt{:=}\ \cfgt{*} \rightarrow\ \cfgnt{k}\rp \rp 
   }	
\and
   \inferrule[Variable Assignment]{}{
    \lp \cfgnt{L}\ \cfgnt{R}\ \phi\ \eta\ \cfgnt{v}\ \lp \cfgnt{x}\ \cfgt{:=}\ \cfgt{*} \rightarrow\ \cfgnt{k}\rp \rp  \rcom \\\\
    \lp \cfgnt{L}\ \cfgnt{R}\ \phi\ \eta[\cfgnt{x} \mapsto \cfgnt{v}]\ \cfgnt{v}\ \cfgnt{k}\rp 
   }	
\and
   \inferrule[Begin (no args)]{}{
    \lp \cfgnt{L}\ \cfgnt{R}\ \phi\ \eta\ \lp \cfgt{begin}\rp \ \cfgnt{k}\rp  \rightarrow \\\\
    \lp \cfgnt{L}\ \cfgnt{R}\ \phi\ \eta\ \cfgnt{k}\rp 
   }
\and
   \inferrule[Begin (arg0 eval)]{}{
    \lp \cfgnt{L}\ \cfgnt{R}\ \phi\ \eta\ \lp \cfgt{begin}\ \cfgnt{e}_0\ \cfgnt{e}_1\ ...\rp \ \cfgnt{k}\rp  \rcom \\\\
    \lp \cfgnt{L}\ \cfgnt{R}\ \phi\ \eta\ \cfgnt{e}_0\ \lp \cfgt{begin}\ \cfgt{*}\ \lp\cfgnt{e}_1\ ...\rp \rightarrow \cfgnt{k}\rp \rp 
   }
\and
   \inferrule[Begin (argi eval)]{}{
    \lp \cfgnt{L}\ \cfgnt{R}\ \phi\ \eta\ \cfgnt{v}\ \lp \cfgt{begin}\ \cfgt{*}\ \lp\cfgnt{e}_i\ \cfgnt{e}_{i+1}\ ...\rp \rightarrow \cfgnt{k}\rp \rp  \rcom \\\\
    \lp \cfgnt{L}\ \cfgnt{R}\ \phi\ \eta\ \cfgnt{e}_i\ \lp \cfgt{begin}\ \cfgt{*}\ \lp\cfgnt{e}_{i+1}\ ...\rp \rightarrow \cfgnt{k}\rp \rp 
   }
\and
   \inferrule[Begin (argN eval)]{}{
    \lp \cfgnt{L}\ \cfgnt{R}\ \phi\ \eta\ \cfgnt{v}\ \lp \cfgt{begin}\ \cfgt{*}\ \lp\cfgnt{e}_{n}\rp \rightarrow \cfgnt{k}\rp \rp  \rcom \\\\
    \lp \cfgnt{L}\ \cfgnt{R}\ \phi\ \eta\ \cfgnt{e}_n\ \cfgnt{k}\rp 
   }
\and
	\inferrule[NULL]{}{
      \lp \cfgnt{L}\ \cfgnt{R}\ \phi\ \eta\ \cfgt{null}\ \cfgnt{k}\rp \rcom \\\\ 
      \lp \cfgnt{L}\ \cfgnt{R}\ \phi\ \eta\ \cfgnt{r}_\mathit{null}\ \cfgnt{k}\rp 
	}
\and
   \inferrule[Pop]{}{
    \lp \cfgnt{L}\ \cfgnt{R}\ \phi\ \eta\ \cfgnt{v}\ \lp \cfgt{pop}\ \eta_0\ \cfgnt{k}\rp \rp  \rcom \\\\
    \lp \cfgnt{L}\ \cfgnt{R}\ \phi\ \eta_0\  \cfgnt{v}\ \cfgnt{k}\rp 
   }
\end{mathpar}
\end{center}
\caption{Javalite rewrite rules, indicated by $\rcom$, that are common to generalized symbolic execution and precise heap summaries.}
\label{fig:javalite-common}
\end{figure*}


\section{Generating Heap Summaries}

\subsection{Initialization of Symbolic References}

In this section we present the Javalite rewrite rules for the concrete
as well as summary initialization of symbolic references. The
initialization rules are defined on the bi-partite graph consisting of
references and locations. The lazy initialization of symbolic
references consists of three key points of non-determinism where each
symbolic reference can be initialized non-deterministically to null, a
new instance of the symbolic reference, or aliases to symbolic
references of the same type previously initialized. The initialization
in GSE consists of creating branches in the execution tree for all the
non-deterministic choices. On the other hand, the heap summarization
approach creates a single branch that contains the summarization for
all the initialization in a single bi-partitate graph.

\begin{figure*}[t]
\begin{center}
\mprset{flushleft}
\begin{mathpar}
	\inferrule[Initialize (null)]{
	  \Lambda = \{ l \mid \exists \phi\ \lp \lp \phi\ l\rp  \in \cfgnt{L}\lp \cfgnt{r}\rp  \wedge  \cfgnt{R}\lp l,\cfgnt{f}\rp  = \bot\ \rp\}\\
      \Lambda \neq \emptyset\\\\
      \cfgnt{r}^\prime = \mathrm{fresh}_r\lp \rp\\ 
      \theta_\mathit{null} = \{ \lp \phi_T\ l_\mathit{null}\rp \} \\
      l_x = \mathrm{min}_l\lp \Lambda\rp \\\\
      \phi_g^\prime = \lp\phi_g \wedge \cfgnt{r}^\prime = \cfgnt{r}_\mathit{null}\rp
    }{
      \lp \cfgnt{L}\ \cfgnt{R}\ \phi_g\ \cfgnt{r}\ \cfgnt{f}\rp  \rightarrow_I 
      \lp \cfgnt{L}[\cfgnt{r}^\prime \mapsto \theta_\mathit{null}]\ \cfgnt{R}[ \lp l_x,\cfgnt{f}\rp  \mapsto \cfgnt{r}^\prime]\ \phi_g^\prime\ \cfgnt{r}\ \cfgnt{f}\rp 
	}
\and
	\inferrule[Initialize (new)]{
	  \Lambda  = \{ l \mid \exists \phi\ \lp \lp \phi\ l\rp  \in \cfgnt{L}\lp \cfgnt{r}\rp  \wedge  \cfgnt{R}\lp l,\cfgnt{f}\rp  = \bot\rp\}\\
      \Lambda \neq \emptyset\\\\
      \mathrm{C} = \mathrm{type}\lp \cfgnt{f}\rp\\
      \cfgnt{r}_f = \mathrm{init}_r\lp \rp\\
      l_f = \mathrm{init}_l\lp \mathrm{C}\rp \\\\
      \cfgnt{R}^\prime = \cfgnt{R}[\forall \cfgnt{f} \in \mathit{fields}\lp \mathrm{C}\rp \ \lp \lp l_f\ \cfgnt{f}\rp  \mapsto \bot \rp ] \\\\
      \rho = \{ \lp\cfgnt{r}_a\ l_a\rp \mid \mathrm{isInit}\lp \cfgnt{r}_a\rp  \wedge \mathrm{type}\lp l_a\rp  = \mathrm{C} \wedge \exists \phi_a\ \lp \lp \phi_a\ l_a\rp  \in \cfgnt{L}\lp \cfgnt{r}_a\rp\rp \}\\\\
      \theta_\mathit{new} = \{\lp \phi_T\ l_f\rp \} \\
      l_x = \mathrm{min}_l\lp \Lambda\rp \\\\
      \phi_g^\prime = \lp\phi_g \wedge \cfgnt{r}_f \neq \cfgnt{r}_\mathit{null} \wedge \lp \wedge_{\lp\cfgnt{r}_a\ l_a\rp \in \rho} \cfgnt{r}_f \ne \cfgnt{r}_a\rp\rp
    }{
      \lp \cfgnt{L}\ \cfgnt{R}\ \phi_g\ \cfgnt{r}\ \cfgnt{f}\rp  \rightarrow_I 
      \lp \cfgnt{L}[\cfgnt{r}_f \mapsto \theta_\mathit{new}]\ \cfgnt{R}^\prime[ \lp l_x,\cfgnt{f}\rp  \mapsto \cfgnt{r}_f ]\ \phi_g^\prime\ \cfgnt{r}\ \cfgnt{f}\rp 
	}
\and
	\inferrule[Initialize (alias)]{
	  \Lambda = \{ l \mid \exists \phi\ \lp \lp \phi\ l\rp  \in \cfgnt{L}\lp \cfgnt{r}\rp  \wedge  \cfgnt{R}\lp l,\cfgnt{f}\rp  = \bot\ \rp\}\\
      \Lambda \neq \emptyset\\\\
      \mathrm{C} = \mathrm{type}\lp \cfgnt{f}\rp\\
      \cfgnt{r}^\prime = \mathrm{fresh}_r\lp \rp\\\\
      \rho = \{ \lp\cfgnt{r}_a\ l_a\rp \mid \mathrm{isInit}\lp \cfgnt{r}_a\rp  \wedge \mathrm{type}\lp l_a\rp  = \mathrm{C} \wedge \exists \phi_a\ \lp \lp \phi_a\ l_a\rp  \in \cfgnt{L}\lp \cfgnt{r}_a\rp\rp \}\\\\
      \lp\cfgnt{r}_a\ l_a\rp \in \rho \\
      \theta_\mathit{alias} = \{ \lp \phi_T\ l_a\rp\}\\
      l_x = \mathrm{min}_l\lp \Lambda\rp\\\\
      \phi^\prime_g = \lp\phi_g \wedge \cfgnt{r}^\prime \neq \cfgnt{r}_\mathit{null} \wedge \cfgnt{r}^\prime = \cfgnt{r}_a \wedge \lp \wedge_{\lp \cfgnt{r}^{\prime}_a\ l_a\rp  \in \rho\ \lp \cfgnt{r}^{\prime}_a \neq \cfgnt{r}_a\rp } \cfgnt{r}^\prime \neq \cfgnt{r}^{\prime}_a \rp\rp
    }{
      \lp \cfgnt{L}\ \cfgnt{R}\ \phi_g\ \cfgnt{r}\ \cfgnt{f}\rp  \rightarrow_I 
      \lp \cfgnt{L}[\cfgnt{r}^\prime \mapsto \theta_\mathit{alias}]\ \cfgnt{R}[ \lp l_x,\cfgnt{f}\rp  \mapsto \cfgnt{r}^\prime ]\ \phi_g^\prime\ \cfgnt{r}\ \cfgnt{f}\rp 
	}
\and
	\inferrule[Initialize (end)]{
	  \Lambda = \{ l \mid \exists \phi\ \lp \lp \phi\ l\rp  \in \cfgnt{L}\lp \cfgnt{r}\rp  \wedge  \cfgnt{R}\lp l,\cfgnt{f}\rp  = \bot\ \rp\}\\
      \Lambda = \emptyset
    }{
      \lp \cfgnt{L}\ \cfgnt{R}\ \phi_g\ \cfgnt{r}\ \cfgnt{f}\rp  \rightarrow_I 
      \lp \cfgnt{L}\ \cfgnt{R}\ \phi_g\ \cfgnt{r}\ \cfgnt{f}\rp 
	}
\end{mathpar}
\end{center}
\caption{The initialization machine, $s ::= \lp\cfgnt{L}\ \cfgnt{R}\ \phi_g\ \cfgnt{r}\ \cfgnt{f}\rp$, with $s \rightarrow_I^* s^\prime$ indicating stepping the machine until the state does not change.}
\label{fig:lazyInit}
\end{figure*}

\newsavebox{\boxPi}
\savebox{\boxPi}{
%\begin{center}
\mprset{flushleft}
\begin{mathpar}
	\inferrule[Summarize]{
	\Lambda = \mathbb{UN}\lp \cfgnt{L}, \cfgnt{R}, \cfgnt{r}, \cfgnt{f}\rp \\
      \Lambda \neq \emptyset \\
      \lp\phi_x\ \cfgnt{l}_x\rp = \mathrm{min}_l\lp \Lambda\rp\\
      \cfgnt{r}_f = \mathrm{init}_r\lp \rp \\
      l_f  = \mathrm{fresh}_l\lp \mathrm{C}\rp\\\\
      \rho = \{ \lp \cfgnt{r}_a\ l_a\rp  \mid \mathrm{isInit}\lp \cfgnt{r}_a\rp  \wedge\cfgnt{r}_a = \mathrm{min}_r\lp \cfgnt{R}^{\leftarrow}[l_a]\rp \wedge \mathrm{type}\lp l_a\rp  = \mathrm{C} \} \\\\
      \theta_\mathit{null} = \{ \lp \phi\ l_\mathit{null}\rp  \mid \phi = \lp \phi_x \wedge \cfgnt{r}_f = \cfgnt{r}_\mathit{null} \rp  \} \\\\
      \theta_\mathit{new} = \{\lp \phi\ l_f\rp  \mid \phi = \lp \phi_x \wedge \cfgnt{r}_f \neq \cfgnt{r}_\mathit{null} \wedge \lp \wedge_{\lp \cfgnt{r}^\prime_a\ l^\prime_a\rp  \in \rho} \cfgnt{r}_f \ne \cfgnt{r}^\prime_a\rp \rp \}\\\\
      \theta_\mathit{alias} = \{ \lp \phi\ l_a\rp  \mid \exists\cfgnt{r}_a\ \lp\lp\cfgnt{r}_a\ l_a\rp  \in \rho \wedge \phi = \lp \phi_x \wedge \cfgnt{r}_f \neq \cfgnt{r}_\mathit{null} \wedge \cfgnt{r}_f = \cfgnt{r}_a \wedge \lp \wedge_{\lp \cfgnt{r}^{\prime}_a\ l^{\prime}_a\rp  \in \rho\ \lp \cfgnt{r}^\prime_a < \cfgnt{r}_a\rp } \cfgnt{r}_f \neq \cfgnt{r}^{\prime}_a \rp \rp \rp \} \\\\
      \theta_\mathit{orig} = \{\lp\phi\ \cfgnt{l}_\mathit{orig}\rp \mid \exists \phi_\mathit{orig} \lp \lp\phi_\mathit{orig}\ \cfgnt{l}_\mathit{orig}\rp \in \cfgnt{L}\lp\cfgnt{R}\lp\cfgnt{l}_x,\cfgnt{f}\rp\rp \wedge \phi = \lp\neg\phi_x \wedge \phi_\mathit{orig}\rp\}\\\\ 
      \theta = \theta_\mathit{null} \cup \theta_\mathit{new} \cup \theta_\mathit{alias} \cup \theta_\mathit{orig} \\
\cfgnt{R}^\prime = \cfgnt{R}[\forall \cfgnt{f} \in \mathit{fields}\lp \mathrm{C}\rp \ \lp \lp l_f\ \cfgnt{f}\rp  \mapsto \cfgnt{r}_\mathit{un} \rp ]
    }{
      \lp \cfgnt{L}\ \cfgnt{R}\ \cfgnt{r}\ \cfgnt{f}\ \cfgnt{C}\rp \rsum 
      \lp \cfgnt{L}[\cfgnt{r}_f \mapsto \theta]\ \cfgnt{R}^{\prime}[ \lp l_x,\cfgnt{f}\rp  \mapsto \cfgnt{r}_f ]\ \cfgnt{r}\ \cfgnt{f}\ \cfgnt{C}\rp
	}
\and
	\inferrule[Summarize-end]{
	  \Lambda = \mathbb{UN}\lp \cfgnt{L}, \cfgnt{R}, \cfgnt{r}, \cfgnt{f}\rp \\
      \Lambda = \emptyset
    }{
      \lp \cfgnt{L}\ \cfgnt{R}\ \cfgnt{r}\ \cfgnt{f}\ \cfgnt{C}\rp  \rsum
      \lp \cfgnt{L}\ \cfgnt{R}\ \cfgnt{r}\ \cfgnt{f}\ \cfgnt{C}\rp 
	}
\end{mathpar}}
%\end{center}
%\caption{The summary machine, $s ::= \lp\cfgnt{L}\ \cfgnt{R}\ \cfgnt{r}\ \cfgnt{f}\ \cfgnt{C}\rp$, with $s\rsum^*s^\prime =  s \rsum \cdots \rsum s^\prime \rsum s^\prime$.}
%\label{fig:symInit}
%\end{figure*}



The initialization rules are invoked when an uninitialized field in a
symbolic reference is accessed. The function $\mathbb{UN}(\cfgnt{L},
\cfgnt{R}, \cfgnt{r}, \cfgnt{f}) = \{\cfgnt{l}\ ...\}$ returns
constraint-location pairs in which the field $\cfgnt{f}$ is
uninitialized:
\[
\begin{array}{rcl}
\mathbb{UN}(\cfgnt{L}, \cfgnt{R}, \cfgnt{r}, \cfgnt{f}) & = &\{ \lp\phi\ \cfgnt{l}\rp \mid \lp \phi\ \cfgnt{l}\rp  \in \cfgnt{L}\lp \cfgnt{r}\rp  \wedge \\
& & \ \ \ \ \exists \phi^\prime \lp \lp \phi^\prime\ \cfgnt{l}_\mathit{un}\rp  \in \cfgnt{L}\lp \cfgnt{R}\lp l,\cfgnt{f}\rp\rp \wedge \\
& & \ \ \ \ \ \ \ \ \mathbb{S}\lp \phi \wedge \phi^\prime \rp\rp\}\\
\end{array}
\]
where $\mathbb{S}(\phi)$ returns true if $\phi$ is
satisfiable. Intutively, for the reference, $\cfgnt{r}$, it constructs
the set, $\theta$, that contains all contraint-location pairs that
point to the field $\cfgnt{f}$ and $\cfgnt{f}$ points to
$\cfgnt{l}_\mathit{un}$. The cardinality of the set, $\theta$ is never
greater than one in GSE and the constraint is always satisfiable
because all constraints are constant. This property is relaxed in GSE
with heap summaries.

The rules in~\figref{fig:lazyInit} present the rewrite rules for the
concrete initialization of symbolic heap objects.  These rules are
invoked until a fix pointed is reached. 

The initialize (null) rewrite rule in~\figref{fig:lazyInit} first
checks that the field, $\cfgnt{r}$ is uninitialized. The fresh method
returns a new input heap reference from the partition 



	

\section{Summary Machine Rules}

\begin{definition}
\label{def:VS}
The $\mathbb{VS}$ function constructs the value set given a
heap, reference, and desired field where $\mathbb{S}(\phi)$ returns true if $\phi$ is satisfiable.
\[
\begin{array}{rcl}
  \mathbb{VS}(\cfgnt{L},\cfgnt{R},\phi_g,\cfgnt{r},\cfgnt{f}) & = &
  \{(\phi\wedge\phi^\prime\ \cfgnt{l}^\prime) \mid \exists
  \cfgnt{l}\ ((\phi\ l) \in L(r)\ \wedge \\ & &
  \ \ \ \ \ \ \ \ \exists \cfgnt{r}^\prime ( \cfgnt{r}^\prime =
  R(\cfgnt{l},\cfgnt{f})\ \wedge (\phi^\prime\ l^\prime) \in
  \cfgnt{L}(\cfgnt{r}^\prime)\ \wedge
  \mathbb{S}(\phi\wedge\phi^\prime\wedge \phi_g)))\}
\end{array}
\]
\end{definition}

\begin{definition}
\label{def:ST}
The strengthen function $\mathbb{ST}(\cfgnt{L},\cfgnt{r},\phi,\phi_g)$ strengthens every
constraint from the reference $\cfgnt{r}$ with $\phi$ and keeps only location-constraint
pairs that are satisfiable after this strengthening with the inclusion of the global heap constraint $\phi_g$:
\[
\begin{array}{rcl} 
\mathbb{ST}(\cfgnt{L},\cfgnt{r},\phi,\phi_g) & = & \{ (\phi\wedge\phi^\prime\ \cfgnt{l}^\prime) \mid  \\
& & \ \ \ \ (\phi^\prime\ \cfgnt{l}^\prime)\in \cfgnt{L}(\cfgnt{r})\wedge\mathbb{S}(\phi\wedge\phi^\prime\wedge\phi_g) \}
\end{array}
\]
\end{definition}



\newsavebox{\boxPi}
\savebox{\boxPi}{
%\begin{center}
\mprset{flushleft}
\begin{mathpar}
	\inferrule[Summarize]{
	\Lambda = \mathbb{UN}\lp \cfgnt{L}, \cfgnt{R}, \cfgnt{r}, \cfgnt{f}\rp \\
      \Lambda \neq \emptyset \\
      \lp\phi_x\ \cfgnt{l}_x\rp = \mathrm{min}_l\lp \Lambda\rp\\
      \cfgnt{r}_f = \mathrm{init}_r\lp \rp \\
      l_f  = \mathrm{fresh}_l\lp \mathrm{C}\rp\\\\
      \rho = \{ \lp \cfgnt{r}_a\ l_a\rp  \mid \mathrm{isInit}\lp \cfgnt{r}_a\rp  \wedge\cfgnt{r}_a = \mathrm{min}_r\lp \cfgnt{R}^{\leftarrow}[l_a]\rp \wedge \mathrm{type}\lp l_a\rp  = \mathrm{C} \} \\\\
      \theta_\mathit{null} = \{ \lp \phi\ l_\mathit{null}\rp  \mid \phi = \lp \phi_x \wedge \cfgnt{r}_f = \cfgnt{r}_\mathit{null} \rp  \} \\\\
      \theta_\mathit{new} = \{\lp \phi\ l_f\rp  \mid \phi = \lp \phi_x \wedge \cfgnt{r}_f \neq \cfgnt{r}_\mathit{null} \wedge \lp \wedge_{\lp \cfgnt{r}^\prime_a\ l^\prime_a\rp  \in \rho} \cfgnt{r}_f \ne \cfgnt{r}^\prime_a\rp \rp \}\\\\
      \theta_\mathit{alias} = \{ \lp \phi\ l_a\rp  \mid \exists\cfgnt{r}_a\ \lp\lp\cfgnt{r}_a\ l_a\rp  \in \rho \wedge \phi = \lp \phi_x \wedge \cfgnt{r}_f \neq \cfgnt{r}_\mathit{null} \wedge \cfgnt{r}_f = \cfgnt{r}_a \wedge \lp \wedge_{\lp \cfgnt{r}^{\prime}_a\ l^{\prime}_a\rp  \in \rho\ \lp \cfgnt{r}^\prime_a < \cfgnt{r}_a\rp } \cfgnt{r}_f \neq \cfgnt{r}^{\prime}_a \rp \rp \rp \} \\\\
      \theta_\mathit{orig} = \{\lp\phi\ \cfgnt{l}_\mathit{orig}\rp \mid \exists \phi_\mathit{orig} \lp \lp\phi_\mathit{orig}\ \cfgnt{l}_\mathit{orig}\rp \in \cfgnt{L}\lp\cfgnt{R}\lp\cfgnt{l}_x,\cfgnt{f}\rp\rp \wedge \phi = \lp\neg\phi_x \wedge \phi_\mathit{orig}\rp\}\\\\ 
      \theta = \theta_\mathit{null} \cup \theta_\mathit{new} \cup \theta_\mathit{alias} \cup \theta_\mathit{orig} \\
\cfgnt{R}^\prime = \cfgnt{R}[\forall \cfgnt{f} \in \mathit{fields}\lp \mathrm{C}\rp \ \lp \lp l_f\ \cfgnt{f}\rp  \mapsto \cfgnt{r}_\mathit{un} \rp ]
    }{
      \lp \cfgnt{L}\ \cfgnt{R}\ \cfgnt{r}\ \cfgnt{f}\ \cfgnt{C}\rp \rsum 
      \lp \cfgnt{L}[\cfgnt{r}_f \mapsto \theta]\ \cfgnt{R}^{\prime}[ \lp l_x,\cfgnt{f}\rp  \mapsto \cfgnt{r}_f ]\ \cfgnt{r}\ \cfgnt{f}\ \cfgnt{C}\rp
	}
\and
	\inferrule[Summarize-end]{
	  \Lambda = \mathbb{UN}\lp \cfgnt{L}, \cfgnt{R}, \cfgnt{r}, \cfgnt{f}\rp \\
      \Lambda = \emptyset
    }{
      \lp \cfgnt{L}\ \cfgnt{R}\ \cfgnt{r}\ \cfgnt{f}\ \cfgnt{C}\rp  \rsum
      \lp \cfgnt{L}\ \cfgnt{R}\ \cfgnt{r}\ \cfgnt{f}\ \cfgnt{C}\rp 
	}
\end{mathpar}}
%\end{center}
%\caption{The summary machine, $s ::= \lp\cfgnt{L}\ \cfgnt{R}\ \cfgnt{r}\ \cfgnt{f}\ \cfgnt{C}\rp$, with $s\rsum^*s^\prime =  s \rsum \cdots \rsum s^\prime \rsum s^\prime$.}
%\label{fig:symInit}
%\end{figure*}


\begin{figure*}
\begin{tabular}[c]{l}
\scalebox{1.00}{\usebox{\boxPI}} \\
\end{tabular}
\caption{The summary machine, $s ::= \lp\cfgnt{L}\ \cfgnt{R}\ \cfgnt{r}\ \cfgnt{f}\ \cfgnt{C}\rp$, with $s\rsum^*s^\prime =  s \rsum \cdots \rsum s^\prime \rsum s^\prime$.}
\label{fig:symInit}
\end{figure*}

\newsavebox{\boxPFAFW}
\savebox{\boxPFAFW}{
%\begin{figure}[t]
%\begin{center}
\mprset{flushleft}
\begin{mathpar}
	\inferrule[Field Access]{
      \exists \lp \phi\ l\rp \in \cfgnt{L}\lp \cfgnt{r}\rp\ \lp l \neq l_{\mathit{null}} \wedge \mathbb{S}\lp \phi \wedge \phi_g\rp \rp \\\\
      \theta = \{ \phi \mid \lp \phi\ l_\mathit{null} \rp \wedge \mathbb{S}\lp \phi \wedge \phi_g\rp \} \\\\
      \phi_g^\prime = \phi_g \wedge (\wedge_{\phi \in \theta} \neg \phi) \\\\
      \{\cfgnt{C}\} = \{\cfgnt{C} \mid \exists \lp \phi\ l\rp  \in \cfgnt{L}\lp \cfgnt{r}\rp\ \lp\cfgnt{C} = \mathrm{type}\lp \cfgnt{l},\cfgnt{f}\rp\rp\} \\\\
      \lp \cfgnt{L}\ \cfgnt{R}\ \cfgnt{r}\ \cfgnt{f}\ \cfgnt{C}\rp \rsum^* \lp \cfgnt{L}^\prime\ \cfgnt{R}^\prime\ \cfgnt{r}\ \cfgnt{f}\ \cfgnt{C}\rp \\
      \cfgnt{r}^\prime = \mathrm{stack}_r\lp \rp
    }{
      \lp \cfgnt{L}\ \cfgnt{R}\ \phi_g\ \eta\ \cfgnt{r}\ \lp \cfgt{*}\ \cfgt{\$}\ \cfgnt{f} \rightarrow \cfgnt{k}\rp \rp  \rightarrow_\mathit{FA}
      \lp \cfgnt{L}^\prime[\cfgnt{r}^\prime \mapsto \mathbb{VS}\lp \cfgnt{L}^\prime,\cfgnt{R}^\prime,\cfgnt{r},\cfgnt{f},\phi_g^\prime\rp ]\ \cfgnt{R}^\prime\ \phi_g^\prime\ \eta\ \cfgnt{r}^\prime\ \cfgnt{k}\rp 
	}
\and
	\inferrule[Field Access (NULL)]{
      \exists \lp \phi\ l\rp \in \cfgnt{L}\lp \cfgnt{r}\rp\ \lp l = l_{\mathit{null}} \wedge \mathbb{S}\lp \phi \wedge \phi_g\rp \rp
    }{
      \lp \cfgnt{L}\ \cfgnt{R}\ \phi_g\ \eta\ \cfgnt{r}\ \lp \cfgt{*}\ \cfgt{\$}\ \cfgnt{f} \rightarrow \cfgnt{k}\rp \rp  \rightarrow_\mathit{FA}
      \lp \cfgnt{L}\ \cfgnt{R}\ \phi_g\ \eta\ \cfgt{error}\ \lp \cfgt{*}\ \cfgt{\$}\ \cfgnt{f} \rightarrow \cfgnt{k}\rp \rp
	}
\and
	\inferrule[Field Write]{
      \exists \lp \phi\ l\rp \in \cfgnt{L}\lp \cfgnt{r}\rp\ \lp l \neq l_{\mathit{null}} \wedge \mathbb{S}\lp \phi \wedge \phi_g\rp \rp \\\\
      \theta = \{ \phi \mid \lp \phi\ l_\mathit{null} \rp \wedge \mathbb{S}\lp \phi \wedge \phi_g\rp \} \\\\
      \phi_g^\prime = \phi_g \wedge (\wedge_{\phi \in \theta} \neg \phi) \\\\
      \cfgnt{r}_x = \eta\lp \cfgnt{x}\rp \\
      \Psi_x =\{\lp \phi\ l\ \cfgnt{r}_\mathit{cur} \rp  \mid \lp \phi\ \cfgnt{l}\rp  \in \cfgnt{L}\lp \cfgnt{r}_x\rp  \wedge \cfgnt{r}_\mathit{cur} = \cfgnt{R}\lp l,\cfgnt{f}\rp  \}\\\\
      X = \{ \lp l\ \theta \rp  \mid \exists \phi\ \lp \lp \phi\ l\ \cfgnt{r}_\mathit{cur} \rp \in \Psi_x \wedge \theta = \mathbb{ST}\lp \cfgnt{L},\cfgnt{r},\phi,\phi_g^\prime\rp  \cup \mathbb{ST}\lp \cfgnt{L},\cfgnt{r}_\mathit{cur},\neg\phi,\phi_g^\prime\rp \rp  \}\\\\
      \cfgnt{R}^{\prime} = \cfgnt{R}[\forall \lp l\ \theta \rp  \in X\ \lp \lp l\ \cfgnt{f}\rp  \mapsto \mathrm{fresh}_r\lp \rp \rp ]\\\\
      \cfgnt{L}^{\prime} = \cfgnt{L}[\forall \lp l\ \theta \rp  \in X\ \lp \exists \cfgnt{r}_\mathit{targ}\ \lp \cfgnt{r}_\mathit{targ} = \cfgnt{R}^\prime\lp l,\cfgnt{f}\rp \wedge \lp\cfgnt{r}_\mathit{targ} \mapsto \theta\rp  \rp \rp ]
    }{
      \lp \cfgnt{L}\ \cfgnt{R}\ \phi_g\ \eta\ \cfgnt{r}\ \lp \cfgnt{x}\ \cfgt{\$}\ \cfgnt{f}\ \cfgt{:=}\ \cfgt{*}\ \rightarrow\ \cfgnt{k}\rp \rp  \rightarrow_\mathit{FW}
      \lp \cfgnt{L}^{\prime}\ \cfgnt{R}^{\prime}\ \phi_g^\prime\ \eta\ \cfgnt{r}\ \cfgnt{k}\rp 
	}	
\and
	\inferrule[Field Write (NULL)]{
      \exists \lp \phi\ l\rp \in \cfgnt{L}\lp \cfgnt{r}\rp\ \lp l \neq l_{\mathit{null}} \wedge \mathbb{S}\lp \phi \wedge \phi_g\rp \rp
    }{
      \lp \cfgnt{L}\ \cfgnt{R}\ \phi_g\ \eta\ \cfgnt{r}\ \lp \cfgnt{x}\ \cfgt{\$}\ \cfgnt{f}\ \cfgt{:=}\ \cfgt{*}\ \rightarrow\ \cfgnt{k}\rp \rp  \rightarrow_\mathit{FW}
      \lp \cfgnt{L}\ \cfgnt{R}\ \phi_g\ \eta\ \cfgt{error}\ \lp \cfgnt{x}\ \cfgt{\$}\ \cfgnt{f}\ \cfgt{:=}\ \cfgt{*}\ \rightarrow\ \cfgnt{k}\rp \rp
	}	
\end{mathpar}}
%\end{center}
%\caption{Precise symbolic heap summaries from symbolic execution indicated by $\rsym = \rightarrow_\mathit{FA} \cup \rightarrow_\mathit{FW} \cup \rightarrow_\mathit{EQ} \cup \rcom$.}
%\label{fig:symfield}
%\end{figure}


\begin{figure*}[t]
\begin{center}
\setlength{\tabcolsep}{50pt}
\begin{tabular}[c]{c}
\usebox{\boxPFAFW} \\
\end{tabular}
\end{center}
\caption{FIXME: Write the caption}
\label{fig:fHeap}
\end{figure*}

\begin{figure*}[t]
\begin{center}
\mprset{flushleft}
\begin{mathpar}
    \inferrule[Equals (references-true)]{
    \theta_\alpha = \{\lp\phi_0 \wedge \phi_1\rp \mid \exists l\ \lp \lp \phi_0\ l\rp  \in \cfgnt{L}\lp \cfgnt{r}_0\rp  \wedge \lp \phi_1\ l\rp  \in \cfgnt{L}\lp \cfgnt{r}_1\rp \rp \} \\\\
    \theta_0 = \{\phi_0 \mid \exists l_0\ \lp \lp \phi_0\ l_0\rp  \in \cfgnt{L}\lp \cfgnt{r}_0\rp  \wedge \forall \lp \phi_1\ l_1\rp  \in \cfgnt{L}\lp \cfgnt{r}_1\rp \ \lp l_0 \neq l_1\rp \rp \} \\\\
    \theta_1 = \{\phi_1 \mid \exists l_1\ \lp \lp \phi_1\ l_1\rp  \in \cfgnt{L}\lp \cfgnt{r}_1\rp  \wedge \forall \lp \phi_0\ l_0\rp  \in \cfgnt{L}\lp \cfgnt{r}_0\rp \ \lp l_0 \neq l_1\rp \rp \} \\\\
    \phi^\prime =  \phi \wedge \lp \vee_{\phi_\alpha\in\theta_\alpha}\phi_\alpha\rp \wedge\lp \wedge_{\phi_0 \in \theta_0} \neg \phi_0\rp \wedge\lp \wedge_{\phi_1
    \in \theta_1} \neg \phi_1\rp \\\\ 
    \mathbb{S}(\phi^\prime)}{
    \lp \cfgnt{L}\ \cfgnt{R}\ \phi\ \eta\ \cfgnt{r}_0\ \lp \cfgnt{r}_1\; \cfgt{=}\; \cfgt{*} \rightarrow \cfgnt{k}\rp \rp  \rightarrow_\mathit{EQ}
    \lp \cfgnt{L}\ \cfgnt{R}\ \phi^\prime\ \eta\ \cfgt{true}\ \cfgnt{k}\rp 
    }
\and
    \inferrule[Equals (references-false)]{
    \theta_\alpha = \{\lp\phi_0 \Rightarrow \neg \phi_1\rp \mid \exists l\ \lp \lp \phi_0\ l\rp  \in \cfgnt{L}\lp \cfgnt{r}_0\rp  \wedge \lp \phi_1\ l\rp  \in \cfgnt{L}\lp \cfgnt{r}_1\rp \rp \} \\\\
    \theta_0 = \{\phi_0 \mid \exists l_0\ \lp \lp \phi_0\ l_0\rp  \in \cfgnt{L}\lp \cfgnt{r}_0\rp  \wedge \forall \lp \phi_1\ l_1\rp  \in \cfgnt{L}\lp \cfgnt{r}_1\rp \ \lp l_0 \neq l_1\rp \rp \} \\\\
    \theta_1 = \{\phi_1 \mid \exists l_1\ \lp \lp \phi_1\ l_1\rp  \in \cfgnt{L}\lp \cfgnt{r}_1\rp  \wedge \forall \lp \phi_0\ l_0\rp  \in \cfgnt{L}\lp \cfgnt{r}_0\rp \ \lp l_0 \neq l_1\rp \rp \} \\\\
    \phi^\prime = \phi \wedge \lp \wedge_{\phi_\alpha\in\theta_\alpha}\phi_\alpha\rp \vee\lp \lp \vee_{\phi_0 \in \theta_0} \phi_0\rp   \vee\lp \vee_{\phi_1
    \in \theta_1} \phi_1\rp \rp  \\\\ 
    \mathbb{S}(\phi^\prime)}{
    \lp \cfgnt{L}\ \cfgnt{R}\ \phi\ \eta\ \cfgnt{r}_0\ \lp \cfgnt{r}_1\; \cfgt{=}\; \cfgt{*} \rightarrow \cfgnt{k}\rp \rp  \rightarrow_\mathit{EQ}
    \lp \cfgnt{L}\ \cfgnt{R}\ \phi^\prime\ \eta\ \cfgt{false}\ \cfgnt{k}\rp 
    }
\end{mathpar}
\end{center}
\caption{Precise symbolic heap summaries from symbolic execution indicated by $\rsym = \rightarrow_\mathit{FA} \cup \rightarrow_\mathit{FW} \cup \rightarrow_\mathit{EQ} \cup \rcom$.}
\label{fig:symeq}
\end{figure*}


\begin{figure*}
\begin{tabular}[c]{c}
\usebox{\boxPEQ} \\
\end{tabular}
\caption{equals true for this.x == this.y}
\label{fig:eqs}
\end{figure*}



\end{document}

