\section{Related Work}
\label{related}

Initial work on symbolic execution largely focused on checking
properties of imperative programs with primitive
types~\cite{Clarke:76,King:76}.
Several projects have generalized the core idea of symbolic execution,
enabling it to be applied to programs with more general types,
including references and
arrays~\cite{GSE03,KiasanKunit,Cadar:2008,Rosner:2015}. However, these
generalized symbolic execution (\gsetxt{}) techniques work by branching over
multiple copies of the system state during dereferencing operations,
exacerbating path explosion.
Our work leverages the core idea in generalized symbolic execution
with lazy initialization using on-the-fly reasoning to model a
black-box input heap during symbolic execution, but avoids the
nondeterminism introduced by~\gsetxt{} by constructing a single
symbolic heap and polynomially-sized path condition for each control
flow path.

Recent progress has been made towards mitigating path explosion via state merging~\cite{Kuznetsov:2012,Sen:2014,Torlak:2014}. While the independently developed techniques in~\cite{Sen:2014,Torlak:2014} use representations similar to the one presented here, none of these techniques address the fundamental issue concerning symbolic heaps, namely dereferencing symbolic input references. For these reasons, we believe these works to be both orthogonal and complimentary to our own.

A number of dynamic symbolic execution (\dsetxt{}) techniques either use
concretization when reasoning about references or require solvers that
support theories of
arrays~\cite{Godefroid:2005,Sen:2005,Godefroid:POPL07,Tillmann:2008}. Sometimes,
however, as demonstrated in~\cite{Elkarablieh:2009},~\dsetxt{}
under-approximates the possible heaps configurations when it
concretizes symbolic values. The SAGE~\dsetxt{}
engine~\cite{Elkarablieh:2009} includes more sophisticated methods for
reasoning about references, and can successfully reason about more
complicated examples. However, the heap model in SAGE prohibits
input-dependent memory allocation. The fixed-memory requirement means
that analyses of programs accepting indefinite input heaps may
terminate prematurely, even for the class of self-limiting
solver-compatible programs.

Many bounded model checking (BMC) methods include a model for
reasoning about references, including those described
in~\cite{Clarke:2004,Barnett:2006,Xie:2005,Babic:2007,Dillig:2011}. However,
BMC techniques generally need to predetermine bounds of a heap. Other
limitations include assuming that symbolic references do not
alias~\cite{Xie:2005,Babic:2007}, that data structures are not
recursive~\cite{Dillig:2011}, or prohibiting dereferencing of
black-box input references~\cite{Clarke:2004}.

A number of constraint-based programming techniques are capable of
reasoning about
heaps~\cite{Degrave:2010,Charreteur:2009,Albert:2013}. Most notably,
the technique in~\cite{Albert:2013} is capable of reasoning about
dereferencing operations in a per-path set of unknown input
heaps. However, no guarantees are made about the precision of the
analysis performed.

The symbolic heap value sets presented in this work build upon a
number of prior techniques that use sets of guarded values to
represent program
state~\cite{Sen:2014,Torlak:2014,Yorsh:2008,Xie:2005,Dillig:2011,Elkarablieh:2009}. The
methods presented in~\cite{Sen:2014,Torlak:2014,Yorsh:2008} use value
sets to represent all symbolic program state, including
references. While symbolic referencing operations could, in principle,
be supported with these methods, it is not clear from the published
material exactly how that might be accomplished, especially when
reasoning about dereferencing in a black-box input heap. All of the
methods presenting solutions to the dereferencing
problem~\cite{Xie:2005,Dillig:2011,Elkarablieh:2009} are not complete,
for a variety of reasons which have been previously enumerated.