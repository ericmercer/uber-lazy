A recent trend in the analysis of object-oriented programs is the
modeling of references as sets of guarded values, enabling multiple
heap shapes to be represented in a single state. A fundamental problem
with using these guarded value sets is the creation of test inputs for
programs accepting symbolic reference input parameters. Although
several solutions have been proposed, none have been proven to be
sound and complete with respect to the properties provable by
generalized symbolic execution (GSE). This work presents a method for
initializing reference inputs in a symbolic input heap using guarded
value sets that exactly preserves GSE semantics. A correctness proof
for the initialization scheme is provided, as well as the results of
an empirical evaluation of a proof-of-concept implementation. The
initialization technique can be used to ensure that guarded value set
based symbolic execution engines operate in a provably correct manner
with regards to symbolic references.


%A fundamental challenge of using symbolic execution for software analysis is the treatment of dynamically allocated data. Existing techniques either underapproximate the space of possible inputs or are computationally infeasible. For example, dynamic symbolic execution (DSE) handles symbolic dereferencing by substituting in a value from a valid concrete execution. Generalized symbolic execution (GSE) initiates a new search path for every possible aliasing configuration. This paper introduces a method for de-referencing and manipulating values in a true block-box symbolic input heap that overcomes the limitations of previous methods. The symbolic heap supports arbitrary recursive data structures and captures all possible heaps that follow a common control flow path. Computation of complex preconditions and postconditions is supported, as well as automatic generation of test inputs. An evaluation of a proof-of-concept implementation in the Java Pathfinder framework is presented to demonstrate the computational feasibility of the approach over several classical symbolic execution benchmarks.

