\section{Our Approach}
\label{sec:sh}

The symbolic heap algorithm is defined using
Javalite~\cite{saints-MS}. Note that the approach is not restricted to
this language or even Java in general; we use Javalite simply as
mechanism to describe our approach. The small-step semantics of
Javalite, defined as a syntactic machine using a CESK architecture,
are modified to perform symbolic execution on a symbolic heap. The
surface (input) syntax and machine (state) syntax are shown
in~\figref{fig:syntax}. Terminals are in bold face while non-terminals
are italicized. Ellipses indicate zero or more repetitions. Tuples
omit the commas for compactness.

\begin{figure}
\begin{center}
\cfgstart
\cfgrule{P}{\lp $\mu$ \lp \cfgnt{C} \cfgnt{m}\rp\rp}
\cfgrule{$\mu$}{(\cfgnt{CL} ...)}
\cfgrule{T}{\cfgt{bool} \cfgor \cfgnt{C}}
\cfgrule{CL}{\lp\cfgt{class} \cfgnt{C} \lp\lb\cfgnt{T} \cfgnt{f}\rb ...\rp \lp\cfgnt{M} ...\rp}
\cfgrule{M}{\lp\cfgnt{T} \cfgnt{m} \lb\cfgnt{T} \cfgnt{x}\rb\  e\rp}
\cfgrule{e}{\cfgnt{x}
\cfgor{\lp\cfgt{new} \cfgnt{C}\rp}
\cfgor{\lp\cfgnt{e} \cfgt{\$} \cfgnt{f}\rp}
\cfgor{\lp\cfgnt{x} \cfgt{\$} \cfgnt{f} \cfgt{:=} \cfgnt{e}\rp}
\cfgor{\lp\cfgnt{e} \cfgt{=} \cfgnt{e}\rp}}
\cfgorline{\lp\cfgt{if} \cfgnt{e} \cfgnt{e} \cfgt{else} \cfgnt{e}\rp 
\cfgor {\lp\cfgt{var} \cfgnt{T} \cfgnt{x} \cfgt{:=} \cfgnt{e} \cfgt{in} \cfgnt{e}\rp}
\cfgor {\lp\cfgnt{e} \cfgt{@} \cfgnt{m} \cfgnt{e} \rp}}
\cfgorline{\lp\cfgnt{x} \cfgt{:=} \cfgnt{e}\rp
\cfgor{\lp\cfgt{begin} \cfgnt{e} ...\rp}
\cfgor{\cfgnt{v}}}
\cfgrule{x}{\cfgt{this} \cfgor \cfgnt{id}}
\cfgrule{f,m,C}{\cfgnt{id}}
%\cfgrule{m}{\cfgnt{id}}
%\cfgrule{C}{\cfgnt{id}}
\cfgrule{v}{\cfgnt{r} \cfgor \cfgt{null} \cfgor \cfgt{true} \cfgor \cfgt{false} \cfgor \cfgt{error}}
\cfgrule{r}{\cfgt{number}}
\cfgrule{id}{\cfgt{variable-not-otherwise-mentioned}}
\cfgend
\end{center}
\caption{The Javalite surface syntax.}
\label{fig:surface-syntax}
\end{figure}

\begin{figure}
\begin{center}
\cfgstart
\cfgrule{e}{\lp ... \cfgor \lp \cfgnt{v} \cfgt{@} \cfgnt{m} \cfgnt{v} \rp\rp}
%\cfgrule{object}{ (\cfgnt{C} [ \cfgnt{f} \cfgnt{loc} ] ...) }
%\cfgrule{hv}{ (\cfgnt{v} \cfgnt{object})}
\cfgrule{$\phi$}{\cfgnt{constraint}}
\cfgrule{l}{\cfgt{number}}
%\cfgrule{h}{(\cfgnt{mt}\ (\cfgnt{h}\ [\cfgnt{loc} $\rightarrow$ \cfgnt{hv}]) )}
\cfgrule{$\eta$}{(\cfgnt{mt}\ ($\eta$ [\cfgnt{x} $\rightarrow$ \cfgnt{loc}]))}
\cfgrule{s}{\lp$\mu$ \cfgnt{L} \cfgnt{R} \cfgnt{g} $\eta$ \cfgnt{e} \cfgnt{k}\rp}
\cfgrule{k}{\cfgt{end}}
\cfgorline{\lp \cfgt{*} \cfgt{\$} \cfgnt{f} $\rightarrow$ \cfgnt{k}\rp}
\cfgorline{\lp \cfgt{*} \cfgt{@} \cfgnt{m} \lp \cfgnt{e} ... \rp $\rightarrow$ \cfgnt{k} \rp}
\cfgorline{\lp \cfgnt{v} \cfgt{@} \cfgnt{m} \cfgnt{v} \cfgt{*} \lp \cfgnt{e} ... \rp $\rightarrow$ \cfgnt{k} \rp}
\cfgorline{\lp \cfgt{*} \cfgt{=} \cfgnt{e} $\rightarrow$ \cfgnt{k}\rp}
\cfgorline{\lp \cfgt{v} \cfgt{=} \cfgnt{*} $\rightarrow$ \cfgnt{k}\rp}
\cfgorline{\lp \cfgt{x} \cfgt{:=} \cfgnt{*} $\rightarrow$ \cfgnt{k}\rp}
\cfgorline{\lp \cfgt{x} \cfgt{\$} \cfgnt{f}  \cfgt{:=} \cfgnt{*} $\rightarrow$ \cfgnt{k}\rp}
\cfgorline{\lp \cfgt{if} \cfgnt{*} \cfgnt{e} \cfgt{else} \cfgnt{e} $\rightarrow$ \cfgnt{k} \rp}
\cfgorline{\lp\cfgt{var} \cfgnt{T} \cfgnt{x} \cfgt{:=} \cfgnt{*} \cfgt{in} \cfgnt{e}  $\rightarrow$ \cfgnt{k} \rp}
\cfgorline{\lp\cfgt{begin}  \cfgnt{*} \lp \cfgnt{e} ...\rp $\rightarrow$ \cfgnt{k} \rp}
\cfgorline{\lp\cfgt{pop} $\eta$ \cfgnt{k}\rp}
\cfgend
\end{center}
\caption{The machine syntax for Javalite.}
\label{fig:machine-syntax}
\end{figure}

\begin{figure}[t]
\begin{center}
\begin{tabular}{c}
\scalebox{0.9}{\usebox{\boxSurface}} \\
(a) \\\\
\scalebox{0.9}{\usebox{\boxMachine}} \\
(b)
\end{tabular}
\end{center}
\caption{Syntax for Javalite. (a) The Javalite surface syntax. (b) The machine syntax with $\bowtie\ \in \{\wedge,\vee,\Rightarrow\}$.}
\label{fig:syntax}
\end{figure}

A Javalite program, $\cfgnt{P}$, is a registry of classes, $\mu$, with
a tuple indicating a class, $\cfgnt{C}$, and a method, $\cfgnt{m}$,
where execution starts. The only primitive type in Javalite is
Boolean. For simplicity in presentation, classes have only non-primitive fields, and methods have a single
parameter. Expressions, $\cfgnt{e}$,  
include statements, and they
 use '$\cfgt{:=}$' to indicate assignment and '$\cfgt{=}$' to indicate comparison.
The dot-operator for field access is replaced by '$\cfgt{\$}$', and the dot-operator
for method invocation is replaced by '$\cfgt{@}$' since '.' is reserved in PLT Redex. There is no explicit return
statement; rather, the value of the last expression is
used as the return value. A variable is always indicated by \cfgnt{x}
and a value by \cfgnt{v}. A value can be a reference in the heap, $\cfgnt{r}$, or any of the special values shown in~\figref{fig:syntax}(a). 
Only type correct programs are given as input to the machine.

The state of the Javalite machine,
$\cfgnt{s}$, includes the program registry, $\mu$, a pair of functions defining the heap
$\lp\cfgnt{L}\ \cfgnt{R}\rp$, a constraint,
$\phi$, defined over references in the heap, the environment, $\eta$,
for local variables, and the continuation $\cfgnt{k}$. The registry never changes so it is omitted from the state tuple in the rest of the presentation.  The more
complex structures, such as the environment and heap, are defined as lists that
start with empty, \cfgnt{mt}. These lists are treated as partial functions so
$\eta(\cfgnt{x}) = \cfgnt{r}$ is the reference $\cfgnt{r}$ associated with variable
$\cfgnt{x}$. The notation $\eta^\prime = \eta[\cfgnt{x} \mapsto
  \cfgnt{v}]$ defines a new partial function $\eta^\prime$ that is
the same as $\eta$ except that the variable $\cfgnt{x}$ now maps to
$\cfgnt{v}$. The continuation $\cfgnt{k}$ indicates where the hole, $\cfgt{*}$, is located, stores temporary computation, and keeps track of the next 
computation. For example, the continuation
$\lp\cfgt{*}\ \cfgt{\$}\ \cfgnt{f} \rightarrow \cfgnt{k}\rp$ indicates
that the machine is evaluating the expression for the object
reference on which the field $\cfgnt{f}$ is going to be accessed. Once
the field access is complete, the machine continues with $\cfgnt{k}$.

\subsection{Symbolic Heap}

Our symbolic heap is a field sensitive points-to graph with edges
annotated with disjunctive formulas describing aliasing
relationships. We represent the symbolic heap is a bipartite graph on
references, $\cfgnt{r}$, and constraint location pairs,
$\lp\phi\ \cfgnt{l}\rp$. The graph is expressed in the location map,
$\cfgnt{L}$, and the reference map, $\cfgnt{R}$. As done with the
environment, $\cfgnt{L}$ and $\cfgnt{R}$ are treated as partial
functions where $\cfgnt{L}(r) = \{(\phi\ \cfgnt{l})\ ...\}$ is the set
of constraint-location pairs in the heap associated with the given
reference, and $\cfgnt{R}(\cfgnt{l},\cfgnt{f}) = \cfgnt{r}$ is the
reference associated with the given location-field pair in the
heap. References in the graph are immutable. Locations, however, are
mutated on updates.  Immutability of references means that look-ups in
the heap are non-recursive and updates do not modify the current path
condition.

The soundness and completeness proofs \sjp{in Section...} rely on a set of 
invariants \sjp{properties of the symbolic heap?} that
are preserved by the small-step semantics during symbolic execution:
%\emph{reference partitions}, \emph{null and uninitialized}, \emph{immutability}, \emph{determinism}, and
%\emph{type consistency}.
\begin{compactdesc}
\item[Reference partitions] means that a reference belongs to one of three partitions: $\mathrm{init}_\cfgnt{r}()$ for the \emph{input
  heap} that includes $\cfgnt{r}_\mathit{null}$ (these are created by lazy initialization and are the only references that appear as variables in
  constraints); $\mathrm{fresh}_\cfgnt{r}()$ for \emph{auxiliary
  literals} that includes $\cfgnt{r}_\mathit{un}$ (these add structure but do not appear elsewhere); and $\mathrm{stack}_\cfgnt{r}()$ for \emph{stack
  literals} (these appear in environments, expressions, and continuations).
\item[Null and Uninitialized] means that every symbolic heap contains 
a special location for null ($\cfgnt{l}_\mathit{null}$), and a special
location for uninitialized
  ($\cfgnt{l}_\mathit{un}$) with $\cfgnt{L}(\cfgnt{r}_\mathit{null}) =
  \{(\cfgt{true}\ \cfgnt{l}_\mathit{null})\}$ and
  $\cfgnt{L}(\cfgnt{r}_\mathit{un}) =
  \{(\cfgt{true}\ \cfgnt{l}_\mathit{un})\}$. 
\item[Immutability] means that a reference does not change in the
location map, and as a consequence, constraints never need to be
rewritten as the program evolves its state.
\item[Determinism] means that a reference in $(\cfgnt{L}\ \cfgnt{R})$ cannot point to multiple locations simultaneously: 
$
\forall \cfgnt{r} \in \cfgnt{L}^\leftarrow\ (\forall (\phi\ \cfgnt{l}),(\phi^\prime\ \cfgnt{l}^\prime) \in \cfgnt{L}(r)\ (
(\cfgnt{l} \neq \cfgnt{l}^\prime \vee \phi \neq \phi^\prime) \Rightarrow (\phi \wedge \phi^\prime = \cfgt{false}))
$
where $\cfgnt{L}^\leftarrow$ is the pre-image of $\cfgnt{L}$. 
\item[Type consistency] means that all locations associated with a reference have the same type:
$\forall \cfgnt{r} \in \cfgnt{L}^\leftarrow\ (\forall (\phi\ \cfgnt{l}),(\phi^\prime\ \cfgnt{l}^\prime) \in \cfgnt{L}(r)\ (
(\mathrm{Type}\lp\cfgnt{l}\rp = \mathrm{Type}\lp\cfgnt{l}^\prime\rp)))$

\end{compactdesc}

%\sjp{Up to now, no discussion of lazy initialization - what it is or
%  how it is being used.}  \nsr{A key insight to the summary heap is
%  that although many auxiliary or stack literals are created during
%  symbolic execution with lazy initialization, only input heap
%  references appear in the constraints expressing potential aliasing
%  relationships.}
A solver is used to reason over the aliasing relationships, and a
satisfying assignment uniquely identifies the set of aliasing
relationships that are representative of a concrete heap and its
shape. For example, if the solver assigns input heap variables
$\cfgnt{r}$ and $\cfgnt{r}^\prime$ such that $\cfgnt{r} =
\cfgnt{r}^\prime$ is true, then there exists some location $\cfgnt{l}$
and constraints $\phi$ and $\phi^\prime$ such that (i)
$\lp\phi\ \cfgnt{l}\rp \in \cfgnt{L}(\cfgnt{r})$; (ii)
$\lp\phi^\prime\ \cfgnt{l}\rp \in \cfgnt{L}(\cfgnt{r}^\prime)$; and
(iii) $\phi$ and $\phi^\prime$ hold in the aliasing assignments from
the solver.  In this way, the graph groups heaps with the conditions
under which those heaps exist so look-ups are non-recursive and
updates do not modify the path condition or require rewrites on
constraints.

\subsection{Semantics of Javalite}

Given an input program $\lp\mu\ \lp\cfgnt{C}\ \cfgnt{m}\rp\rp$, the initial machine state is
$$
\begin{array}{l}
\lp\mu 
\ \cfgnt{L}[\cfgnt{r}_\mathit{null} \mapsto \{\lp\cfgt{true}\ \cfgnt{l}_\mathit{null})\}\rp] 
           [\cfgnt{r}_\mathit{un} \mapsto \{\lp\cfgt{true}\ \cfgnt{l}_\mathit{un}\rp\}] \\
\ \cfgnt{R}
\ \cfgt{true}\ \eta\  \lp\lp\cfgt{new}\ \cfgnt{C}\rp\ \cfgt{@}\ \cfgnt{m}\ \cfgt{true}\rp\rp\ \cfgt{end}\rp
\end{array}
$$
The expression in the initial state constructs a new object of type
$\cfgnt{C}$ and then invokes the method $\cfgnt{m}$ in that
object. 

The semantics are expressed as
rewrites on strings using pattern matching. Consider the rewrite rule
for the beginning of a field access instruction:
$$
\mprset{flushleft}
	\inferrule[Field Access(eval)]{}{
      \lp \cfgnt{L}\ \cfgnt{R}\ \phi\ \eta\ \lp \cfgnt{e}\ \cfgt{\$}\ \cfgnt{f}\rp \ \cfgnt{k}\rp  \rcom 
      \lp \cfgnt{L}\ \cfgnt{R}\ \phi\ \eta\ \cfgnt{e}\ \lp \cfgt{*}\ \cfgt{\$}\ \cfgnt{f} \rightarrow \cfgnt{k}\rp \rp 
	}
$$
If the string representing the current state matches the left side, then it
creates the new string on the right. In this example, the new string
on the right is now evaluating the expression $\cfgnt{e}$ in the field
access, and it includes the continuation indicating that it still
needs to complete the actual field access once the expression is
evaluated.

Other more complex rules, such as one to create a new instance of a
class, define constraints on the rewrites and more complex
transformations on the heap.
$$
\mprset{flushleft}
	\inferrule[New]{
      \cfgnt{r} = \mathrm{stack}_r\lp \rp\\
      l = \mathrm{fresh}_l\lp \cfgnt{C}\rp\\\\
      \cfgnt{R}^\prime = \cfgnt{R}[\forall \cfgnt{f} \in \mathit{fields}\lp \mathrm{C}\rp \ \lp \lp l\ \cfgnt{f}\rp  \mapsto \cfgnt{r}_\mathit{null} \rp ] \\\\
      \cfgnt{L}^\prime = \cfgnt{L}[\cfgnt{r} \mapsto \{\lp \cfgt{true}\ l\rp \}]
    }{
      \lp \cfgnt{L}\ \cfgnt{R}\ \phi\ \eta\ \lp \cfgt{new}\ \cfgnt{C}\rp \ \cfgnt{k}\rp  \rcom
      \lp \cfgnt{L}^\prime\ \cfgnt{R}^\prime\ \phi\ \eta\ \cfgnt{r}\ \cfgnt{k}\rp 
	}
$$ In this rule, when the string matches the new-expression, it is
        rewritten to use a new heap location where all of the fields
        for the new object point to $\cfgnt{r}_\mathit{null}$ and the
        location map points a new stack reference to that new object.
        Note that references for objects from the new-expression are
        auxiliary literals, so they do not appear in any constraint in
        the symbolic heap as an aliasing option.
