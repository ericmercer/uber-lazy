\section{Summary Heap Framework}
\subsection{Syntax}
The summary heap algorithm is presented using Javalite, an
executable formalization of a Java-like language~\cite{saints-MS}. Javalite retains 
the essence of Java and other object-oriented languages but with formalizations 
in both PLT Redex and Coq so that the outcome of any program has a precise
and deterministic meaning.

Javalite is a syntactic machine defined as rewrites on a string.  The
semantics are small-step using the CESK architecture with a (C)ontrol string
representing the expression being evaluated, an (E)nvironment for
local variables, a (S)tore for the heap, and a (K)ontinuation for what
is to be executed next.  In this work we change the semantics of the original
 Javalite machine and its machine syntax to represent (i) GSE~\cite{GSE03}, and (ii) the summary heap algorithm
presented in this work.

The surface syntax (input) for Javalite is shown 
in~\figref{fig:surface-syntax}, and the machine syntax (state)
in~\figref{fig:machine-syntax}. Terminals are in bold face while
non-terminals are italicized. Ellipses indicate zero or more
repetitions. Tuples omit the commas for compactness. The language
itself uses s-expressions for convenience.

\begin{figure}
\begin{center}
\cfgstart
\cfgrule{P}{\lp $\mu$ \lp \cfgnt{C} \cfgnt{m}\rp\rp}
\cfgrule{$\mu$}{(\cfgnt{CL} ...)}
\cfgrule{T}{\cfgt{bool} \cfgor \cfgnt{C}}
\cfgrule{CL}{\lp\cfgt{class} \cfgnt{C} \lp\lb\cfgnt{T} \cfgnt{f}\rb ...\rp \lp\cfgnt{M} ...\rp}
\cfgrule{M}{\lp\cfgnt{T} \cfgnt{m} \lb\cfgnt{T} \cfgnt{x}\rb\  e\rp}
\cfgrule{e}{\cfgnt{x}
\cfgor{\lp\cfgt{new} \cfgnt{C}\rp}
\cfgor{\lp\cfgnt{e} \cfgt{\$} \cfgnt{f}\rp}
\cfgor{\lp\cfgnt{x} \cfgt{\$} \cfgnt{f} \cfgt{:=} \cfgnt{e}\rp}
\cfgor{\lp\cfgnt{e} \cfgt{=} \cfgnt{e}\rp}}
\cfgorline{\lp\cfgt{if} \cfgnt{e} \cfgnt{e} \cfgt{else} \cfgnt{e}\rp 
\cfgor {\lp\cfgt{var} \cfgnt{T} \cfgnt{x} \cfgt{:=} \cfgnt{e} \cfgt{in} \cfgnt{e}\rp}
\cfgor {\lp\cfgnt{e} \cfgt{@} \cfgnt{m} \cfgnt{e} \rp}}
\cfgorline{\lp\cfgnt{x} \cfgt{:=} \cfgnt{e}\rp
\cfgor{\lp\cfgt{begin} \cfgnt{e} ...\rp}
\cfgor{\cfgnt{v}}}
\cfgrule{x}{\cfgt{this} \cfgor \cfgnt{id}}
\cfgrule{f,m,C}{\cfgnt{id}}
%\cfgrule{m}{\cfgnt{id}}
%\cfgrule{C}{\cfgnt{id}}
\cfgrule{v}{\cfgnt{r} \cfgor \cfgt{null} \cfgor \cfgt{true} \cfgor \cfgt{false} \cfgor \cfgt{error}}
\cfgrule{r}{\cfgt{number}}
\cfgrule{id}{\cfgt{variable-not-otherwise-mentioned}}
\cfgend
\end{center}
\caption{The Javalite surface syntax.}
\label{fig:surface-syntax}
\end{figure}

\begin{figure}
\begin{center}
\cfgstart
\cfgrule{e}{\lp ... \cfgor \lp \cfgnt{v} \cfgt{@} \cfgnt{m} \cfgnt{v} \rp\rp}
%\cfgrule{object}{ (\cfgnt{C} [ \cfgnt{f} \cfgnt{loc} ] ...) }
%\cfgrule{hv}{ (\cfgnt{v} \cfgnt{object})}
\cfgrule{$\phi$}{\cfgnt{constraint}}
\cfgrule{l}{\cfgt{number}}
%\cfgrule{h}{(\cfgnt{mt}\ (\cfgnt{h}\ [\cfgnt{loc} $\rightarrow$ \cfgnt{hv}]) )}
\cfgrule{$\eta$}{(\cfgnt{mt}\ ($\eta$ [\cfgnt{x} $\rightarrow$ \cfgnt{loc}]))}
\cfgrule{s}{\lp$\mu$ \cfgnt{L} \cfgnt{R} \cfgnt{g} $\eta$ \cfgnt{e} \cfgnt{k}\rp}
\cfgrule{k}{\cfgt{end}}
\cfgorline{\lp \cfgt{*} \cfgt{\$} \cfgnt{f} $\rightarrow$ \cfgnt{k}\rp}
\cfgorline{\lp \cfgt{*} \cfgt{@} \cfgnt{m} \lp \cfgnt{e} ... \rp $\rightarrow$ \cfgnt{k} \rp}
\cfgorline{\lp \cfgnt{v} \cfgt{@} \cfgnt{m} \cfgnt{v} \cfgt{*} \lp \cfgnt{e} ... \rp $\rightarrow$ \cfgnt{k} \rp}
\cfgorline{\lp \cfgt{*} \cfgt{=} \cfgnt{e} $\rightarrow$ \cfgnt{k}\rp}
\cfgorline{\lp \cfgt{v} \cfgt{=} \cfgnt{*} $\rightarrow$ \cfgnt{k}\rp}
\cfgorline{\lp \cfgt{x} \cfgt{:=} \cfgnt{*} $\rightarrow$ \cfgnt{k}\rp}
\cfgorline{\lp \cfgt{x} \cfgt{\$} \cfgnt{f}  \cfgt{:=} \cfgnt{*} $\rightarrow$ \cfgnt{k}\rp}
\cfgorline{\lp \cfgt{if} \cfgnt{*} \cfgnt{e} \cfgt{else} \cfgnt{e} $\rightarrow$ \cfgnt{k} \rp}
\cfgorline{\lp\cfgt{var} \cfgnt{T} \cfgnt{x} \cfgt{:=} \cfgnt{*} \cfgt{in} \cfgnt{e}  $\rightarrow$ \cfgnt{k} \rp}
\cfgorline{\lp\cfgt{begin}  \cfgnt{*} \lp \cfgnt{e} ...\rp $\rightarrow$ \cfgnt{k} \rp}
\cfgorline{\lp\cfgt{pop} $\eta$ \cfgnt{k}\rp}
\cfgend
\end{center}
\caption{The machine syntax for Javalite.}
\label{fig:machine-syntax}
\end{figure}


A Javalite program, $\cfgnt{P}$, is a registry of classes, $\mu$, with
a tuple indicating a class, $\cfgnt{C}$, and a method, $\cfgnt{m}$,
where execution starts. The only primitive type in Javalite is
Boolean. Classes have non-primitive fields, $([\cfgnt{C}\ \cfgnt{f}]\ ...)$, and
methods, $(\cfgnt{M}\ ...)$.\footnote{Summary heaps are able to support non-primitive fields (see \secref{sec:eval}). They are omitted to focus and simplify the presentation.} Methods are limited to a single
parameter in our Javalite machine. Expressions, $\cfgnt{e}$,  
include statements, and they
 use '$\cfgt{:=}$' to indicate assignment and '$\cfgt{=}$' to indicate comparison.
The dot-operator for field access is replaced by '$\cfgt{\$}$', and the dot-operator
for method invocation is replaced by '$\cfgt{@}$' since '.' is reserved in PLT Redex. There is no explicit return
statement in Javalite; rather, the value of the last expression is
used as the return value. A variable is always indicated by \cfgnt{x}
and a value by \cfgnt{v}. A value can be a reference in the heap, $\cfgnt{r}$, or any of the special values shown in~\figref{fig:surface-syntax}. 
We assume that only
type correct programs are given as input to the machine.

The state of the Javalite machine (\figref{fig:machine-syntax}),
$\cfgnt{s}$, includes the program registry, $\mu$, a pair of functions defining the heap
$\lp\cfgnt{L}\ \cfgnt{R}\rp$, a constraint,
$\phi$, defined over references in the heap, the environment, $\eta$,
for local variables, and the continuation $\cfgnt{k}$.\footnote{The registry $\mu$ never changes so it is omitted from the state tuple in the rest of the presentation for compactness.}  The more
complex structures, such as the environment and heap, are defined as lists that
start with empty, \cfgnt{mt}. The rewrite rules that define the
semantics treat these lists as partial functions. As such,
$\eta(\cfgnt{x}) = \cfgnt{r}$ is the reference $\cfgnt{r}$ associated with variable
$\cfgnt{x}$. The notation $\eta^\prime = \eta[\cfgnt{x} \mapsto
  \cfgnt{v}]$ defines a new partial function $\eta^\prime$ that is
the same as $\eta$ except that the variable $\cfgnt{x}$ now maps to
$\cfgnt{v}$.

The continuation $\cfgnt{k}$ helps accomplish the small-step semantics
by indicating where the hole, $\cfgt{*}$, is located, by storing
temporary computations, and by keeping track of the next
computation. For example, the continuation
$\lp\cfgt{*}\ \cfgt{\$}\ \cfgnt{f} \rightarrow \cfgnt{k}\rp$ indicates
that the machine is currently computing the expression for the object
reference on which the field $\cfgnt{f}$ is going to be accessed. Once
the field access is complete, the machine continues with $\cfgnt{k}$.




