
%% bare_conf.tex
%% V1.3
%% 2007/01/11
%% by Michael Shell
%% See:
%% http://www.michaelshell.org/
%% for current contact information.
%%
%% This is a skeleton file demonstrating the use of IEEEtran.cls
%% (requires IEEEtran.cls version 1.7 or later) with an IEEE conference paper.
%%
%% Support sites:
%% http://www.michaelshell.org/tex/ieeetran/
%% http://www.ctan.org/tex-archive/macros/latex/contrib/IEEEtran/
%% and
%% http://www.ieee.org/

%%*************************************************************************
%% Legal Notice:
%% This code is offered as-is without any warranty either expressed or
%% implied; without even the implied warranty of MERCHANTABILITY or
%% FITNESS FOR A PARTICULAR PURPOSE! 
%% User assumes all risk.
%% In no event shall IEEE or any contributor to this code be liable for
%% any damages or losses, including, but not limited to, incidental,
%% consequential, or any other damages, resulting from the use or misuse
%% of any information contained here.
%%
%% All comments are the opinions of their respective authors and are not
%% necessarily endorsed by the IEEE.
%%
%% This work is distributed under the LaTeX Project Public License (LPPL)
%% ( http://www.latex-project.org/ ) version 1.3, and may be freely used,
%% distributed and modified. A copy of the LPPL, version 1.3, is included
%% in the base LaTeX documentation of all distributions of LaTeX released
%% 2003/12/01 or later.
%% Retain all contribution notices and credits.
%% ** Modified files should be clearly indicated as such, including  **
%% ** renaming them and changing author support contact information. **
%%
%% File list of work: IEEEtran.cls, IEEEtran_HOWTO.pdf, bare_adv.tex,
%%                    bare_conf.tex, bare_jrnl.tex, bare_jrnl_compsoc.tex
%%*************************************************************************

% *** Authors should verify (and, if needed, correct) their LaTeX system  ***
% *** with the testflow diagnostic prior to trusting their LaTeX platform ***
% *** with production work. IEEE's font choices can trigger bugs that do  ***
% *** not appear when using other class files.                            ***
% The testflow support page is at:
% http://www.michaelshell.org/tex/testflow/



% Note that the a4paper option is mainly intended so that authors in
% countries using A4 can easily print to A4 and see how their papers will
% look in print - the typesetting of the document will not typically be
% affected with changes in paper size (but the bottom and side margins will).
% Use the testflow package mentioned above to verify correct handling of
% both paper sizes by the user's LaTeX system.
%
% Also note that the "draftcls" or "draftclsnofoot", not "draft", option
% should be used if it is desired that the figures are to be displayed in
% draft mode.
%
\documentclass[conference]{IEEEtran}
% Add the compsoc option for Computer Society conferences.
%
% If IEEEtran.cls has not been installed into the LaTeX system files,
% manually specify the path to it like:
% \documentclass[conference]{../sty/IEEEtran}





% Some very useful LaTeX packages include:
% (uncomment the ones you want to load)


% *** MISC UTILITY PACKAGES ***
%
%\usepackage{ifpdf}
% Heiko Oberdiek's ifpdf.sty is very useful if you need conditional
% compilation based on whether the output is pdf or dvi.
% usage:
% \ifpdf
%   % pdf code
% \else
%   % dvi code
% \fi
% The latest version of ifpdf.sty can be obtained from:
% http://www.ctan.org/tex-archive/macros/latex/contrib/oberdiek/
% Also, note that IEEEtran.cls V1.7 and later provides a builtin
% \ifCLASSINFOpdf conditional that works the same way.
% When switching from latex to pdflatex and vice-versa, the compiler may
% have to be run twice to clear warning/error messages.






% *** CITATION PACKAGES ***
%
%\usepackage{cite}
% cite.sty was written by Donald Arseneau
% V1.6 and later of IEEEtran pre-defines the format of the cite.sty package
% \cite{} output to follow that of IEEE. Loading the cite package will
% result in citation numbers being automatically sorted and properly
% "compressed/ranged". e.g., [1], [9], [2], [7], [5], [6] without using
% cite.sty will become [1], [2], [5]--[7], [9] using cite.sty. cite.sty's
% \cite will automatically add leading space, if needed. Use cite.sty's
% noadjust option (cite.sty V3.8 and later) if you want to turn this off.
% cite.sty is already installed on most LaTeX systems. Be sure and use
% version 4.0 (2003-05-27) and later if using hyperref.sty. cite.sty does
% not currently provide for hyperlinked citations.
% The latest version can be obtained at:
% http://www.ctan.org/tex-archive/macros/latex/contrib/cite/
% The documentation is contained in the cite.sty file itself.






% *** GRAPHICS RELATED PACKAGES ***
%
\ifCLASSINFOpdf
  % \usepackage[pdftex]{graphicx}
  % declare the path(s) where your graphic files are
  % \graphicspath{{../pdf/}{../jpeg/}}
  % and their extensions so you won't have to specify these with
  % every instance of \includegraphics
  % \DeclareGraphicsExtensions{.pdf,.jpeg,.png}
\else
  % or other class option (dvipsone, dvipdf, if not using dvips). graphicx
  % will default to the driver specified in the system graphics.cfg if no
  % driver is specified.
  % \usepackage[dvips]{graphicx}
  % declare the path(s) where your graphic files are
  % \graphicspath{{../eps/}}
  % and their extensions so you won't have to specify these with
  % every instance of \includegraphics
  % \DeclareGraphicsExtensions{.eps}
\fi
% graphicx was written by David Carlisle and Sebastian Rahtz. It is
% required if you want graphics, photos, etc. graphicx.sty is already
% installed on most LaTeX systems. The latest version and documentation can
% be obtained at: 
% http://www.ctan.org/tex-archive/macros/latex/required/graphics/
% Another good source of documentation is "Using Imported Graphics in
% LaTeX2e" by Keith Reckdahl which can be found as epslatex.ps or
% epslatex.pdf at: http://www.ctan.org/tex-archive/info/
%
% latex, and pdflatex in dvi mode, support graphics in encapsulated
% postscript (.eps) format. pdflatex in pdf mode supports graphics
% in .pdf, .jpeg, .png and .mps (metapost) formats. Users should ensure
% that all non-photo figures use a vector format (.eps, .pdf, .mps) and
% not a bitmapped formats (.jpeg, .png). IEEE frowns on bitmapped formats
% which can result in "jaggedy"/blurry rendering of lines and letters as
% well as large increases in file sizes.
%
% You can find documentation about the pdfTeX application at:
% http://www.tug.org/applications/pdftex





% *** MATH PACKAGES ***
%
%\usepackage[cmex10]{amsmath}
% A popular package from the American Mathematical Society that provides
% many useful and powerful commands for dealing with mathematics. If using
% it, be sure to load this package with the cmex10 option to ensure that
% only type 1 fonts will utilized at all point sizes. Without this option,
% it is possible that some math symbols, particularly those within
% footnotes, will be rendered in bitmap form which will result in a
% document that can not be IEEE Xplore compliant!
%
% Also, note that the amsmath package sets \interdisplaylinepenalty to 10000
% thus preventing page breaks from occurring within multiline equations. Use:
%\interdisplaylinepenalty=2500
% after loading amsmath to restore such page breaks as IEEEtran.cls normally
% does. amsmath.sty is already installed on most LaTeX systems. The latest
% version and documentation can be obtained at:
% http://www.ctan.org/tex-archive/macros/latex/required/amslatex/math/





% *** SPECIALIZED LIST PACKAGES ***
%
\usepackage{algorithm2e}
\usepackage{amssymb}
\usepackage{comment}
\usepackage{mathpartir}

%\usepackage{algorithmicx}
%\usepackage{algpseudocode}
% algorithmic.sty was written by Peter Williams and Rogerio Brito.
% This package provides an algorithmic environment fo describing algorithms.
% You can use the algorithmic environment in-text or within a figure
% environment to provide for a floating algorithm. Do NOT use the algorithm
% floating environment provided by algorithm.sty (by the same authors) or
% algorithm2e.sty (by Christophe Fiorio) as IEEE does not use dedicated
% algorithm float types and packages that provide these will not provide
% correct IEEE style captions. The latest version and documentation of
% algorithmic.sty can be obtained at:
% http://www.ctan.org/tex-archive/macros/latex/contrib/algorithms/
% There is also a support site at:
% http://algorithms.berlios.de/index.html
% Also of interest may be the (relatively newer and more customizable)
% algorithmicx.sty package by Szasz Janos:
% http://www.ctan.org/tex-archive/macros/latex/contrib/algorithmicx/




% *** ALIGNMENT PACKAGES ***
%
%\usepackage{array}
% Frank Mittelbach's and David Carlisle's array.sty patches and improves
% the standard LaTeX2e array and tabular environments to provide better
% appearance and additional user controls. As the default LaTeX2e table
% generation code is lacking to the point of almost being broken with
% respect to the quality of the end results, all users are strongly
% advised to use an enhanced (at the very least that provided by array.sty)
% set of table tools. array.sty is already installed on most systems. The
% latest version and documentation can be obtained at:
% http://www.ctan.org/tex-archive/macros/latex/required/tools/


%\usepackage{mdwmath}
%\usepackage{mdwtab}
% Also highly recommended is Mark Wooding's extremely powerful MDW tools,
% especially mdwmath.sty and mdwtab.sty which are used to format equations
% and tables, respectively. The MDWtools set is already installed on most
% LaTeX systems. The lastest version and documentation is available at:
% http://www.ctan.org/tex-archive/macros/latex/contrib/mdwtools/


% IEEEtran contains the IEEEeqnarray family of commands that can be used to
% generate multiline equations as well as matrices, tables, etc., of high
% quality.


%\usepackage{eqparbox}
% Also of notable interest is Scott Pakin's eqparbox package for creating
% (automatically sized) equal width boxes - aka "natural width parboxes".
% Available at:
% http://www.ctan.org/tex-archive/macros/latex/contrib/eqparbox/





% *** SUBFIGURE PACKAGES ***
%\usepackage[tight,footnotesize]{subfigure}
% subfigure.sty was written by Steven Douglas Cochran. This package makes it
% easy to put subfigures in your figures. e.g., "Figure 1a and 1b". For IEEE
% work, it is a good idea to load it with the tight package option to reduce
% the amount of white space around the subfigures. subfigure.sty is already
% installed on most LaTeX systems. The latest version and documentation can
% be obtained at:
% http://www.ctan.org/tex-archive/obsolete/macros/latex/contrib/subfigure/
% subfigure.sty has been superceeded by subfig.sty.



%\usepackage[caption=false]{caption}
%\usepackage[font=footnotesize]{subfig}
% subfig.sty, also written by Steven Douglas Cochran, is the modern
% replacement for subfigure.sty. However, subfig.sty requires and
% automatically loads Axel Sommerfeldt's caption.sty which will override
% IEEEtran.cls handling of captions and this will result in nonIEEE style
% figure/table captions. To prevent this problem, be sure and preload
% caption.sty with its "caption=false" package option. This is will preserve
% IEEEtran.cls handing of captions. Version 1.3 (2005/06/28) and later 
% (recommended due to many improvements over 1.2) of subfig.sty supports
% the caption=false option directly:
%\usepackage[caption=false,font=footnotesize]{subfig}
%
% The latest version and documentation can be obtained at:
% http://www.ctan.org/tex-archive/macros/latex/contrib/subfig/
% The latest version and documentation of caption.sty can be obtained at:
% http://www.ctan.org/tex-archive/macros/latex/contrib/caption/




% *** FLOAT PACKAGES ***
%
%\usepackage{fixltx2e}
% fixltx2e, the successor to the earlier fix2col.sty, was written by
% Frank Mittelbach and David Carlisle. This package corrects a few problems
% in the LaTeX2e kernel, the most notable of which is that in current
% LaTeX2e releases, the ordering of single and double column floats is not
% guaranteed to be preserved. Thus, an unpatched LaTeX2e can allow a
% single column figure to be placed prior to an earlier double column
% figure. The latest version and documentation can be found at:
% http://www.ctan.org/tex-archive/macros/latex/base/



%\usepackage{stfloats}
% stfloats.sty was written by Sigitas Tolusis. This package gives LaTeX2e
% the ability to do double column floats at the bottom of the page as well
% as the top. (e.g., "\begin{figure*}[!b]" is not normally possible in
% LaTeX2e). It also provides a command:
%\fnbelowfloat
% to enable the placement of footnotes below bottom floats (the standard
% LaTeX2e kernel puts them above bottom floats). This is an invasive package
% which rewrites many portions of the LaTeX2e float routines. It may not work
% with other packages that modify the LaTeX2e float routines. The latest
% version and documentation can be obtained at:
% http://www.ctan.org/tex-archive/macros/latex/contrib/sttools/
% Documentation is contained in the stfloats.sty comments as well as in the
% presfull.pdf file. Do not use the stfloats baselinefloat ability as IEEE
% does not allow \baselineskip to stretch. Authors submitting work to the
% IEEE should note that IEEE rarely uses double column equations and
% that authors should try to avoid such use. Do not be tempted to use the
% cuted.sty or midfloat.sty packages (also by Sigitas Tolusis) as IEEE does
% not format its papers in such ways.





% *** PDF, URL AND HYPERLINK PACKAGES ***
%
%\usepackage{url}
% url.sty was written by Donald Arseneau. It provides better support for
% handling and breaking URLs. url.sty is already installed on most LaTeX
% systems. The latest version can be obtained at:
% http://www.ctan.org/tex-archive/macros/latex/contrib/misc/
% Read the url.sty source comments for usage information. Basically,
% \url{my_url_here}.
\usepackage{listings}
\usepackage{paralist}
\usepackage{amssymb}

\usepackage{minted}

% *** Do not adjust lengths that control margins, column widths, etc. ***
% *** Do not use packages that alter fonts (such as pslatex).         ***
% There should be no need to do such things with IEEEtran.cls V1.6 and later.
% (Unless specifically asked to do so by the journal or conference you plan
% to submit to, of course. )


% correct bad hyphenation here
\hyphenation{op-tical net-works semi-conduc-tor}

%% Commands For the syntax EBNF
\newcommand{\cfgnt}[1]{\emph{#1}}
\newcommand{\cfgq}[1]{\texttt{#1}}
\newcommand{\cfgt}[1]{\textbf{#1}}
\newcommand{\cfglhs}[1]{\cfgnt{#1} & $::=$}
\newcommand{\cfgrule}[2]{\cfglhs{#1} & #2 \\}
\newcommand{\cfgor}{\textbar\ }
\newcommand{\cfgstart}{\begin{tabular}{r@{\hspace{1mm}}r@{\hspace{2mm}}l}}
\newcommand{\cfgend}{\end{tabular}}
\newcommand{\cfgline}[1]{ ~ && #1 \\ }
\newcommand{\cfglinetab}[1]{ ~ && \hspace{1cm} #1 \\ }
\newcommand{\cfgorline}[1]{ ~ & \cfgor & #1 \\ }
\newcommand{\lp}{\cfgq{(}}
\newcommand{\lb}{\cfgq{[}}
\newcommand{\rp}{\cfgq{)}}
\newcommand{\rb}{\cfgq{]}}

%%Commands for semantics math
%\newcommand{\eval}[1]{[\hspace{-0.5mm}[#1]\hspace{-0.5mm}]}
%\newcommand{\evalenv}[1]{\ensuremath{( \mu~\sigma~\rho~\eta~#1 )}}
%\newcommand{\inhole}[1]{E[#1]}

%\newcommand{\boldeta}{\ensuremath{\mathbf{\etaup\hspace{-0.49em}\etaup\hspace{1mm}}}}
%\newcommand{\boldsigma}{\ensuremath{\mathbf{\sigma\hspace{-0.49em}\sigma\hspace{1mm}}}}

\newcommand{\hsp}[0]{\hspace{1mm}}
%%\newcommand{\infrule}[2]{\frac{\displaystyle #1}{\displaystyle #2}}
%%\newcommand{\symexpr}[2]{\ensuremath{(#1 \hsp #2)}}
\newcommand{\expr}[2]{\ensuremath{(\mathtt{#1} \hsp #2)}}

\newcommand{\reduce}[1]{\ensuremath{\rightarrow_{#1}}}
\newcommand{\reduceK}[1]{\ensuremath{\rightarrow_{#1}^{*}}}
\newcommand{\reduceN}[1]{\ensuremath{\dashrightarrow_{#1}}}
\newcommand{\reduceNK}[1]{\ensuremath{\dashrightarrow_{#1}^{*}}}

\newcommand{\estate}[2]{\ensuremath{(\sigma~\eta~#1~#2)}}
\newcommand{\estatek}[1]{\ensuremath{(\sigma~\eta~#1~\mathit{k})}}
\newcommand{\ksplit}[2]{\ensuremath{(#1~\texttt{*}~#2 \rightarrow \mathit{k})}}
\newcommand{\kop}[3]{\ensuremath{(#1~(#2)~\texttt{*}~(#3) \rightarrow \mathit{k})}}
\newcommand{\ksetop}[2]{\ensuremath{(#1~(\mathit{pattern}~\mathtt{in}~\texttt{*})~#2 \rightarrow \mathit{k})}}
\newcommand{\setop}[3]{\ensuremath{(#1~(\mathit{pattern}~\mathtt{in}~#2)~#3)}}

\newcommand{\esstate}[2]{\ensuremath{(\sigma~\eta~(#1)~(#2))}}

\newcommand{\tstate}[4]{\ensuremath{(\mu~\sigma#1~\eta#2~#3~#4)}}
\newcommand{\tstateK}[3]{\ensuremath{(\mu~\sigma#1~\eta#2~#3~\mathit{ck})}}

\newcommand{\mstate}[3]{\ensuremath{(\mu~\sigma#1~#2~#3)}}
\newcommand{\thread}[3]{\ensuremath{(\eta#1~\mathit{tcmd}#2~\mathit{ck}#3)}}


\begin{document}
%
% paper title
% can use linebreaks \\ within to get better formatting as desired
\title{Uber-lazy Symbolic Execution}


% author names and affiliations
% use a multiple column layout for up to three different
% affiliations
\author{\IEEEauthorblockN{Neha Rungta}
\IEEEauthorblockA{NASA Ames}
\and
\IEEEauthorblockN{Eric Mercer and Benjamin Hillery}
\IEEEauthorblockA{Brigham Young University}}

% conference papers do not typically use \thanks and this command
% is locked out in conference mode. If really needed, such as for
% the acknowledgment of grants, issue a \IEEEoverridecommandlockouts
% after \documentclass

% for over three affiliations, or if they all won't fit within the width
% of the page, use this alternative format:
% 
%\author{\IEEEauthorblockN{Michael Shell\IEEEauthorrefmark{1},
%Homer Simpson\IEEEauthorrefmark{2},
%James Kirk\IEEEauthorrefmark{3}, 
%Montgomery Scott\IEEEauthorrefmark{3} and
%Eldon Tyrell\IEEEauthorrefmark{4}}
%\IEEEauthorblockA{\IEEEauthorrefmark{1}School of Electrical and Computer Engineering\\
%Georgia Institute of Technology,
%Atlanta, Georgia 30332--0250\\ Email: see http://www.michaelshell.org/contact.html}
%\IEEEauthorblockA{\IEEEauthorrefmark{2}Twentieth Century Fox, Springfield, USA\\
%Email: homer@thesimpsons.com}
%\IEEEauthorblockA{\IEEEauthorrefmark{3}Starfleet Academy, San Francisco, California 96678-2391\\
%Telephone: (800) 555--1212, Fax: (888) 555--1212}
%\IEEEauthorblockA{\IEEEauthorrefmark{4}Tyrell Inc., 123 Replicant Street, Los Angeles, California 90210--4321}}




% use for special paper notices
%\IEEEspecialpapernotice{(Invited Paper)}




% make the title area
\maketitle


\begin{abstract}
%\boldmath
The abstract goes here.
\end{abstract}
% IEEEtran.cls defaults to using nonbold math in the Abstract.
% This preserves the distinction between vectors and scalars. However,
% if the conference you are submitting to favors bold math in the abstract,
% then you can use LaTeX's standard command \boldmath at the very start
% of the abstract to achieve this. Many IEEE journals/conferences frown on
% math in the abstract anyway.

% no keywords




% For peer review papers, you can put extra information on the cover
% page as needed:
% \ifCLASSOPTIONpeerreview
% \begin{center} \bfseries EDICS Category: 3-BBND \end{center}
% \fi
%
% For peerreview papers, this IEEEtran command inserts a page break and
% creates the second title. It will be ignored for other modes.
\IEEEpeerreviewmaketitle



% An example of a floating figure using the graphicx package.
% Note that \label must occur AFTER (or within) \caption.
% For figures, \caption should occur after the \includegraphics.
% Note that IEEEtran v1.7 and later has special internal code that
% is designed to preserve the operation of \label within \caption
% even when the captionsoff option is in effect. However, because
% of issues like this, it may be the safest practice to put all your
% \label just after \caption rather than within \caption{}.
%
% Reminder: the "draftcls" or "draftclsnofoot", not "draft", class
% option should be used if it is desired that the figures are to be
% displayed while in draft mode.
%
%\begin{figure}[!t]
%\centering
%\includegraphics[width=2.5in]{myfigure}
% where an .eps filename suffix will be assumed under latex, 
% and a .pdf suffix will be assumed for pdflatex; or what has been declared
% via \DeclareGraphicsExtensions.
%\caption{Simulation Results}
%\label{fig_sim}
%\end{figure}

% Note that IEEE typically puts floats only at the top, even when this
% results in a large percentage of a column being occupied by floats.


% An example of a double column floating figure using two subfigures.
% (The subfig.sty package must be loaded for this to work.)
% The subfigure \label commands are set within each subfloat command, the
% \label for the overall figure must come after \caption.
% \hfil must be used as a separator to get equal spacing.
% The subfigure.sty package works much the same way, except \subfigure is
% used instead of \subfloat.
%
%\begin{figure*}[!t]
%\centerline{\subfloat[Case I]\includegraphics[width=2.5in]{subfigcase1}%
%\label{fig_first_case}}
%\hfil
%\subfloat[Case II]{\includegraphics[width=2.5in]{subfigcase2}%
%\label{fig_second_case}}}
%\caption{Simulation results}
%\label{fig_sim}
%\end{figure*}
%
% Note that often IEEE papers with subfigures do not employ subfigure
% captions (using the optional argument to \subfloat), but instead will
% reference/describe all of them (a), (b), etc., within the main caption.


% An example of a floating table. Note that, for IEEE style tables, the 
% \caption command should come BEFORE the table. Table text will default to
% \footnotesize as IEEE normally uses this smaller font for tables.
% The \label must come after \caption as always.
%
%\begin{table}[!t]
%% increase table row spacing, adjust to taste
%\renewcommand{\arraystretch}{1.3}
% if using array.sty, it might be a good idea to tweak the value of
% \extrarowheight as needed to properly center the text within the cells
%\caption{An Example of a Table}
%\label{table_example}
%\centering
%% Some packages, such as MDW tools, offer better commands for making tables
%% than the plain LaTeX2e tabular which is used here.
%\begin{tabular}{|c||c|}
%\hline
%One & Two\\
%\hline
%Three & Four\\
%\hline
%\end{tabular}
%\end{table}


% Note that IEEE does not put floats in the very first column - or typically
% anywhere on the first page for that matter. Also, in-text middle ("here")
% positioning is not used. Most IEEE journals/conferences use top floats
% exclusively. Note that, LaTeX2e, unlike IEEE journals/conferences, places
% footnotes above bottom floats. This can be corrected via the \fnbelowfloat
% command of the stfloats package.

\section{Pseudo-code}
\figref{fig:surface-syntax} defines the surface syntax for the
Javalite language \cite{saints-MS}. The \figref{fig:machine-syntax} is
the machine syntax. The semantics of Javalite is syntax based and
defined as rewrites on a string. The semantics use a CEKS machine
model with a (C)ontrol string representing the expression being
evaluated, an (E)nvironment for local variables, a (K)ontinuation for
what is to be executed next, and a (S)tore for the heap. This paper
only defines salient features of the language and machine relevant to
understanding the new algorithm.

\begin{figure}
\begin{center}
\cfgstart
\cfgrule{P}{\lp $\mu$ \lp \cfgnt{C} \cfgnt{m}\rp\rp}
\cfgrule{$\mu$}{(\cfgnt{CL} ...)}
\cfgrule{T}{\cfgt{bool} \cfgor \cfgnt{C}}
\cfgrule{CL}{\lp\cfgt{class} \cfgnt{C} \lp\lb\cfgnt{T} \cfgnt{f}\rb ...\rp \lp\cfgnt{M} ...\rp}
\cfgrule{M}{\lp\cfgnt{T} \cfgnt{m} \lb\cfgnt{T} \cfgnt{x}\rb\  e\rp}
\cfgrule{e}{\cfgnt{x}
\cfgor{\lp\cfgt{new} \cfgnt{C}\rp}
\cfgor{\lp\cfgnt{e} \cfgt{\$} \cfgnt{f}\rp}
\cfgor{\lp\cfgnt{x} \cfgt{\$} \cfgnt{f} \cfgt{:=} \cfgnt{e}\rp}
\cfgor{\lp\cfgnt{e} \cfgt{=} \cfgnt{e}\rp}}
\cfgorline{\lp\cfgt{if} \cfgnt{e} \cfgnt{e} \cfgt{else} \cfgnt{e}\rp 
\cfgor {\lp\cfgt{var} \cfgnt{T} \cfgnt{x} \cfgt{:=} \cfgnt{e} \cfgt{in} \cfgnt{e}\rp}
\cfgor {\lp\cfgnt{e} \cfgt{@} \cfgnt{m} \cfgnt{e} \rp}}
\cfgorline{\lp\cfgnt{x} \cfgt{:=} \cfgnt{e}\rp
\cfgor{\lp\cfgt{begin} \cfgnt{e} ...\rp}
\cfgor{\cfgnt{v}}}
\cfgrule{x}{\cfgt{this} \cfgor \cfgnt{id}}
\cfgrule{f,m,C}{\cfgnt{id}}
%\cfgrule{m}{\cfgnt{id}}
%\cfgrule{C}{\cfgnt{id}}
\cfgrule{v}{\cfgnt{r} \cfgor \cfgt{null} \cfgor \cfgt{true} \cfgor \cfgt{false} \cfgor \cfgt{error}}
\cfgrule{r}{\cfgt{number}}
\cfgrule{id}{\cfgt{variable-not-otherwise-mentioned}}
\cfgend
\end{center}
\caption{The Javalite surface syntax.}
\label{fig:surface-syntax}
\end{figure}

\begin{figure}
\begin{center}
\cfgstart
\cfgrule{e}{\lp ... \cfgor \lp \cfgnt{v} \cfgt{@} \cfgnt{m} \cfgnt{v} \rp\rp}
%\cfgrule{object}{ (\cfgnt{C} [ \cfgnt{f} \cfgnt{loc} ] ...) }
%\cfgrule{hv}{ (\cfgnt{v} \cfgnt{object})}
\cfgrule{$\phi$}{\cfgnt{constraint}}
\cfgrule{l}{\cfgt{number}}
%\cfgrule{h}{(\cfgnt{mt}\ (\cfgnt{h}\ [\cfgnt{loc} $\rightarrow$ \cfgnt{hv}]) )}
\cfgrule{$\eta$}{(\cfgnt{mt}\ ($\eta$ [\cfgnt{x} $\rightarrow$ \cfgnt{loc}]))}
\cfgrule{s}{\lp$\mu$ \cfgnt{L} \cfgnt{R} \cfgnt{g} $\eta$ \cfgnt{e} \cfgnt{k}\rp}
\cfgrule{k}{\cfgt{end}}
\cfgorline{\lp \cfgt{*} \cfgt{\$} \cfgnt{f} $\rightarrow$ \cfgnt{k}\rp}
\cfgorline{\lp \cfgt{*} \cfgt{@} \cfgnt{m} \lp \cfgnt{e} ... \rp $\rightarrow$ \cfgnt{k} \rp}
\cfgorline{\lp \cfgnt{v} \cfgt{@} \cfgnt{m} \cfgnt{v} \cfgt{*} \lp \cfgnt{e} ... \rp $\rightarrow$ \cfgnt{k} \rp}
\cfgorline{\lp \cfgt{*} \cfgt{=} \cfgnt{e} $\rightarrow$ \cfgnt{k}\rp}
\cfgorline{\lp \cfgt{v} \cfgt{=} \cfgnt{*} $\rightarrow$ \cfgnt{k}\rp}
\cfgorline{\lp \cfgt{x} \cfgt{:=} \cfgnt{*} $\rightarrow$ \cfgnt{k}\rp}
\cfgorline{\lp \cfgt{x} \cfgt{\$} \cfgnt{f}  \cfgt{:=} \cfgnt{*} $\rightarrow$ \cfgnt{k}\rp}
\cfgorline{\lp \cfgt{if} \cfgnt{*} \cfgnt{e} \cfgt{else} \cfgnt{e} $\rightarrow$ \cfgnt{k} \rp}
\cfgorline{\lp\cfgt{var} \cfgnt{T} \cfgnt{x} \cfgt{:=} \cfgnt{*} \cfgt{in} \cfgnt{e}  $\rightarrow$ \cfgnt{k} \rp}
\cfgorline{\lp\cfgt{begin}  \cfgnt{*} \lp \cfgnt{e} ...\rp $\rightarrow$ \cfgnt{k} \rp}
\cfgorline{\lp\cfgt{pop} $\eta$ \cfgnt{k}\rp}
\cfgend
\end{center}
\caption{The machine syntax for Javalite.}
\label{fig:machine-syntax}
\end{figure}


%\item $n$ are numbers, and the set of all positive natural numbers is $\mathbb{N}^+$.
This paper uses a standard definition of constraints $\phi \in \Phi$
assuming all the usual relational operators and connectives. The heap
is a labeled bipartite graph consisting of references $R$ and
locations $L$ in the store. The functions $R$ and $L$ are defined for convenience
in manipulating the labeled bipartite graph.
\begin{itemize}
\item $R(l,f)$ maps location-field pairs from the store to a reference in $R$; and
\item $L(r)$ maps references to a set of location-constraint pairs in the store.
\end{itemize}
A reference is a node that gathers the possible store locations for an
object during symbolic execution. Each store location is guarded by a
constraint that determines the aliasing in the heap. Intuitively, the
reference is a level of indirection between a variable and the store,
and the reference is used to group a set of possible store locations
each predicated on the possible aliasing in the associated constraint.
For a variable (or field) to access any particular store location
associated with its reference, the corresponding constraint must be
satisfied.

\begin{comment}
 defined on value sets $L$ and references
$R$. $R$ and $L$ include the special symbol $\bot$ to indicate an
uninitialized reference and location respectively. Additionally, $L$
includes a special location $\mathtt{NULL}$ for null references.
\end{comment}

Locations are boxes in the graphical representation and indicated with
the letter $l$ in the math. References are circles in the graphical
representation and indicated with the letter $r$ in the math. Edges
from locations are labeled with field names $f \in F$. Edges from the
references are labeled with constraints $\phi \in \Phi$ (we assume
$\Phi$ is a power set over individual constraints and $\phi$ is a set
of constraints for the edge).

The function $\mathbb{VS}(L,R,r,f)$ constructs the value-set given a
heap, reference, and desired field such that
$(l^\prime\ \phi^\prime\wedge\phi) \in \mathbb{VS}(L,R,r,f)$ if and
only if
\[
  \exists (l\ \phi) \in L(r) \left ( 
     \exists r^\prime \in R(l,f) \left ( 
        \exists (l^\prime\ \phi^\prime) \in L(r^\prime) \left ( 
           \mathbb{S}(\phi\wedge\phi^\prime) 
        \right ) 
     \right )
  \right )
\]
where $\mathbb{S}(\phi)$ returns true if $\phi$ is satisfiable.


\begin{figure*}[t]
\begin{center}
\mprset{flushleft}
\begin{mathpar}
	\inferrule[Variable lookup]{}{
      (L\ R\ g\ \eta\ \cfgnt{x}\ k) \rightarrow (L\ R\ g\ \eta\ \eta(\cfgnt{x})\ k)
	}
\and
	\inferrule[Field Access(eval)]{}{
      (L\ R\ g\ \eta\ (e\ \$\ \cfgnt{f})\ k) \rightarrow (L\ R\ g\ \eta\ e\ (\cfgt{init}\ \cfgnt{f}\ (*\ \$\ \cfgnt{f} \rightarrow k)))
	}
\and
%	\inferrule[Field Access (NULL)]{
 %     L(r) = \emptyset
 %   }{
 %     (L\ R\ g\ \eta\ r\ (*\ \$\ \cfgnt{f} \rightarrow k)) \rightarrow 
%      (L[r \mapsto \{(\bot,\phi_T)\}]\ R\ \eta\ r\ (*\ \$\ \cfgnt{f} \rightarrow k))
%	}
%\and
	\inferrule[Lazy Initialization]{
	  \Lambda = \{ l \mid \exists \phi\ ((l,\phi) \in L(r) \wedge  R(l,f) = \emptyset\}\\
      l_x = \mathrm{min}_l(\Lambda) \\\\
      \Lambda \neq \emptyset \\
      \mathrm{type}(\cfgnt{f}) = \mathrm{C}_r\\
      \mathrm{init}_r() = r_f\\ 
      \mathrm{init}_l(\mathrm{C}_r) = l_f \\\\
      \rho = \{ (r^\prime,\ \phi^\prime,\ l^\prime) \mid \mathrm{isInit}(r^\prime) \wedge (l^\prime, \phi^\prime) \in L(r^\prime) \wedge \mathrm{type}(l^\prime) = \mathrm{C}_r \} \\\\
      \theta_\mathit{alias} = \{ ( l ,\ \phi) | (r^\prime,\ \phi^\prime,\ \l) \in \rho \wedge \phi = (\phi^\prime \wedge r^\prime \neq r_\mathit{null} \wedge r^\prime = r_f \wedge (\wedge_{(r^{\prime\prime},\ \phi^{\prime\prime},\ l^{\prime\prime}) \in \rho\ (r^{\prime\prime} \neq r^\prime)} r^{\prime\prime} \neq r_f )) \} \\\\
      \theta_\mathit{new} = \{(l_f,\ r_f \neq r_\mathit{null} \wedge (\wedge_{(r^\prime,\ \phi^\prime,\ l^\prime) \in \rho} r_f \ne r^\prime))\}\\\\
      \theta_\mathit{null} = \{ (l_\mathit{null},\ r_f = r_\mathit{null}) \}\\\\
      \theta = \theta_\mathit{alias} \cup \theta_\mathit{new} \cup \theta_\mathit{null}
    }{
      (L\ R\ g\ \eta\ r\ (\cfgt{init}\ \cfgnt{f}\ k)) \rightarrow 
      (L[r_f \mapsto \theta]\ R[ (l_x,f) \mapsto r_f ]\ g\ \eta\ r\ (\cfgt{init}\ \cfgnt{f}\ k))
	}
\and
	\inferrule[Lazy Initialization (end)]{
	\forall (l,\phi) \in L(r)\ (R(l,f) \neq \emptyset)
    }{
      (L\ R\ g\ \eta\ r\ (\cfgt{init}\ \cfgnt{f}\ k)) \rightarrow 
      (L\ R\ g\ \eta\ r\ k)
	}
\and
	\inferrule[New]{
      \mathrm{fresh}_r(\cfgnt{C}) = r \\
      \mathrm{fresh}_l(\cfgnt{C}) = l \\\\
      R^\prime = R[\forall \cfgnt{f} \in \mathrm{C}\ ((l\ \cfgnt{f}) \mapsto \mathrm{fresh}_r(\mathrm{type}(\cfgnt{f})))] \\\\
      L^\prime = L[r \mapsto \{(l\ \phi_T)\}]
    }{
      (L\ R\ g\ \eta\ (\cfgt{new}\ \cfgnt{C})\ k) \rightarrow 
      (L^\prime\ R^\prime\ g\ \eta\ r\ k)
	}
\and
	\inferrule[Field Access]{
      \forall (l,\phi) \in L(r)\ (l = l_{\mathit{null}} \rightarrow \neg \mathbb{S}(\phi \wedge g)) \\
      \mathrm{fresh}_r() = r_f
    }{
      (L\ R\ g\ \eta\ r\ (*\ \$\ \cfgnt{f} \rightarrow k)) \rightarrow 
      (L[r_f \mapsto \mathbb{VS}(L,R,r,f,g)]\ R\ g\ \eta\ r_f\ k)
	}
\and
    \inferrule[Equals (l-operand eval)]{}{
      (L\ R\ g\ \eta\ (\cfgnt{e}_0 = \cfgnt{e}) \ k) \rightarrow 
      (L\ R\ g\ \eta\ \cfgnt{e}_0\ (\cfgt{*}\; \cfgt{=}\; \cfgnt{e} \rightarrow \cfgnt{k}))
    }
\and
    \inferrule[Equals (r-operand eval)]{}{
    (L\ R\ g\ \eta\ \cfgnt{v}\ (\cfgt{*}\; \cfgt{=}\; \cfgnt{e} \rightarrow \cfgnt{k})) \rightarrow
    (L\ R\ g\ \eta\ \cfgnt{e}\ (\cfgnt{v}\; \cfgt{=}\; \cfgt{*} \rightarrow \cfgnt{k}))
    }
\and
    \inferrule[Equals (bool)]{
    \cfgnt{v}_0 \in \{\cfgt{true}, \cfgt{false}\} \\
    \cfgnt{v}_1 \in \{\cfgt{true}, \cfgt{false}\} \\ 
    \mathrm{eq?}(v_0, v_1) = \cfgnt{v}_r}{
    (L\ R\ g\ \eta\ \cfgnt{v}_0\ (\cfgnt{v}_1\; \cfgt{=}\; \cfgt{*} \rightarrow \cfgnt{k})) \rightarrow
    (L\ R\ g\ \eta\ \cfgnt{v}_r\ \cfgnt{k})
    }
\and
    \inferrule[Equals (references-true)]{
    \cfgnt{v}_0 \not\in \{\cfgt{true}, \cfgt{false}\} \\
    \cfgnt{v}_1 \not\in \{\cfgt{true}, \cfgt{false}\}\\\\
    \theta_\alpha = \{\phi_0 \wedge \phi_1\mid\exists (l\ \phi_0) \in L(v_0)(\exists (l\ \phi_1) \in L(v_1) ( \mathbb{S}(\phi_0 \wedge \phi_1)))\} \\\\
    \theta_0 = \{\phi_0 \mid \exists (l_0\ \phi_0) \in L(v_0) \wedge \forall (l_1\ \phi_1) \in L(v_1)(l_0 \neq l_1)\} \\\\
    \theta_1 = \{\phi_1 \mid \exists (l_1\ \phi_1) \in L(v_1) \wedge \forall (l_0\ \phi_0) \in L(v_0)(l_0 \neq l_1)\} \\\\
    \phi_g = (\vee_{\phi_\alpha\in\theta_\alpha}\phi_\alpha)\wedge(\wedge_{\phi_0 \in \theta_0} \neg \phi_0)  \wedge(\wedge_{\phi_1
    \in \theta_1} \neg \phi_1)\\ g^\prime = g \wedge \phi_g}{
     % L^\prime = L[\forall r (\exists (\cfgnt{l}\ \phi )\in L(r)(r \mapsto \mathbb{ST}(L,r,\phi_g)))]\\\\
      %X = \mathbb{ER}(L, \phi_g) \\ \mathbb{C}(L^\prime, R, X, \eta, k) = 1}
    (L\ R\ g\ \eta\ \cfgnt{v}_0\ (\cfgnt{v}_1\; \cfgt{=}\; \cfgt{*} \rightarrow \cfgnt{k})) \rightarrow
    (L^\prime\ R\ g^\prime\ \eta\ \cfgt{true}\ \cfgnt{k})
    }
\and
    \inferrule[Equals (references-false)]{
    \cfgnt{v}_0 \not\in \{\cfgt{true}, \cfgt{false}\} \\
    \cfgnt{v}_1 \not\in \{\cfgt{true}, \cfgt{false}\}\\\\
    \theta_\alpha = \{\phi_0 \rightarrow \neg \phi_1\mid\exists (l\ \phi_0) \in L(v_0)(\exists (l\ \phi_1) \in L(v_1) ( \mathbb{S}(\phi_0 \wedge \phi_1)))\} \\\\
    \theta_0 = \{\phi_0 \mid \exists (l_0\ \phi_0) \in L(v_0) \wedge \forall (l_1\ \phi_1) \in L(v_1)(l_0 \neq l_1)\} \\\\
    \theta_1 = \{\phi_1 \mid \exists (l_1\ \phi_1) \in L(v_1) \wedge \forall (l_0\ \phi_0) \in L(v_0)(l_0 \neq l_1)\} \\\\
    \phi_g = (\wedge_{\phi_\alpha\in\theta_\alpha}\phi_\alpha)\vee((\vee_{\phi_0 \in \theta_0} \phi_0)  \vee(\vee_{\phi_1
    \in \theta_1} \phi_1)) \\ g^\prime = g \wedge \phi_g}{
    (L\ R\ g\ \eta\ \cfgnt{v}_0\ (\cfgnt{v}_1\; \cfgt{=}\; \cfgt{*} \rightarrow \cfgnt{k})) \rightarrow
    (L^\prime\ R\ g^\prime\ \eta\ \cfgt{true}\ \cfgnt{k})
    }
    
  %  \\\\
  %    L^\prime = L[\forall r (\exists (\cfgnt{l}\ \phi )\in L(r)(r \mapsto \mathbb{ST}(L,r,\phi_g)))]\\\\
  %    X = \mathbb{ER}(L, \phi_g) \\ \mathbb{C}(L^\prime, R, X, \eta, k) = 1}{
 %   (L\ R\ g\ \eta\ \cfgnt{v}_0\ (\cfgnt{v}_1\; \cfgt{=}\; \cfgt{*} \rightarrow \cfgnt{k})) \rightarrow
 %   (L^\prime\ R\ \eta\ \cfgt{false}\ \cfgnt{k})
 %   }
\and
   \inferrule[Method Invocation (object eval)]{}{
    (L\ R\ g\ \eta\ \lp\cfgnt{e}_0\ \cfgt{@}\ \cfgnt{m}\ \cfgnt{e}_1\rp\ \cfgnt{k}) \rightarrow
    (L\ R\ g\ \eta\ \cfgnt{e}_0\ (\cfgt{*}\ \cfgt{@}\ \cfgnt{m}\ \cfgnt{e}_1\ \rightarrow \cfgnt{k}))
   }
   
\and
   \inferrule[Method Invocation (arg eval)]{}{
    (L\ R\ g\ \eta\ \cfgnt{v}_0\ (\cfgt{*}\ \cfgt{@}\ \cfgnt{m}\ \cfgnt{e}_1\ \rightarrow \cfgnt{k})) \rightarrow
    (L\ R\ g\ \eta\ \cfgnt{e}_1\ (\cfgnt{v}_0\ \cfgt{@}\ \cfgnt{m}\ \cfgt{*}\ \rightarrow \cfgnt{k}))
   }
%\and
%   \inferrule[Method Invocation (arg0 eval)]{}{
%    (L\ R\ \eta\ \cfgnt{v}_0\ (\cfgt{*}\ \cfgt{@}\ \cfgnt{m} \lp\cfgnt{e}_1\ \cfgnt{e}_2\ ...\rp \rightarrow \cfgnt{k})) \rightarrow
%    (L\ R\ \eta\ \cfgnt{e}_1\ (\cfgnt{v}_0\ \cfgt{@}\ \cfgnt{m}\ ()\ \cfgt{*}\ \lp\cfgnt{e}_2\ ...\rp \rightarrow \cfgnt{k}))
%   }
%\and
%   \inferrule[Method Invocation (argi eval)]{}{
%    (L\ R\ \eta\ \cfgnt{v}_i\ (\cfgnt{v}_0\ \cfgt{@}\ \cfgnt{m}\ (\cfgnt{v}_1\ ...)\ \cfgt{*}\ \lp\cfgnt{e}_{i+1}\ \cfgnt{e}_{i+2}\ ...\rp \rightarrow \cfgnt{k})) \rightarrow
%    (L\ R\ \eta\ \cfgnt{e}_{i+1}\ (\cfgnt{v}_0\ \cfgt{@}\ \cfgnt{m}\ (\cfgnt{v}_1\ ...\ v_i)\ \cfgt{*}\ \lp\cfgnt{e}_{i+2}\ ...\rp \rightarrow \cfgnt{k}))
%   }
%\and
%   \inferrule[Method Invocation (args)]{}{
%    (L\ R\ \eta\ \cfgnt{v}_n\ (\cfgnt{v}_0\ \cfgt{@}\ \cfgnt{m}\ (\cfgnt{v}_1\ ...)\ \cfgt{*}\ \lp\rp \rightarrow \cfgnt{k})) \rightarrow
%    (L\ R\ \eta\ (\cfgt{raw}\ \cfgnt{v}_0\ \cfgt{@}\ \cfgnt{m}\ (\cfgnt{v}_1\ ...\ v_n))\ \cfgnt{k})
%   }
%\and
%   \inferrule[Method Invocation (no args)]{}{
%    (L\ R\ \eta\ \cfgnt{e}_0\ (\cfgt{*}\ \cfgt{@}\ \cfgnt{m}\ \lp\rp \rightarrow \cfgnt{k})) \rightarrow
%    (L\ R\ \eta\ (\cfgt{raw}\ \cfgnt{v}_0\ \cfgt{@}\ \cfgnt{m}\ ()\ \cfgnt{k}))
%   }
\and
   \inferrule[Method Invocation (raw)]{
    \mathrm{lookup}(\cfgnt{m}) = \lp\cfgnt{T}\ \cfgnt{m}\ \lb\cfgnt{T}\ \cfgnt{x}\rb\ \ e_m\rp \\
    \eta_m = \eta[\cfgt{this} \mapsto \cfgnt{v}_0][\cfgnt{x} \mapsto \cfgnt{v}_1]\ }{
    (L\ R\ g\ \eta\ \cfgnt{v}_1\ (\cfgnt{v}_0\ \cfgt{@}\ \cfgnt{m}\ \cfgt{*}\ \rightarrow \cfgnt{k})) \rightarrow
    (L\ R\ g\ \eta\ (\cfgt{raw}\ \cfgnt{v}_0\ \cfgt{@}\ \cfgnt{m}\ \cfgnt{v}_1\ \cfgnt{k})
   }
\and
   \inferrule[Method Invocation]{
    \mathrm{lookup}(\cfgnt{m}) = \lp\cfgnt{T}\ \cfgnt{m}\ \lb\cfgnt{T}\ \cfgnt{x}\rb\ \ e_m\rp \\
    \eta_m = \eta[\cfgt{this} \mapsto \cfgnt{v}_0][\cfgnt{x} \mapsto \cfgnt{v}_1]\ }{
    (L\ R\ g\ \eta\ (\cfgt{raw}\ \cfgnt{v}_0\ \cfgt{@}\ \cfgnt{m}\ \cfgnt{v}_1\ \cfgnt{k}) \rightarrow
    (L^\prime\ R^\prime\ g\ \eta_m\ \cfgnt{e}_m\ (\cfgt{pop}\ \eta\ \cfgnt{k}))
   }

\end{mathpar}
\end{center}
\caption{Uber-lazy state reductions}
\label{fig:expr:red}
\end{figure*}

\begin{figure*}[t]
\begin{center}
\mprset{flushleft}
\begin{mathpar}
	\inferrule[ITE (expr-eval)]{}{
      (L\ R\ g\ \eta\ (\cfgt{if}\ \cfgnt{e}_0\ \cfgnt{e}_1\ \cfgt{else}\ \cfgnt{e}_2)\ k) \rightarrow (L\ R\ g\ \eta\ \cfgnt{e}_0\ (\cfgt{if}\ \cfgnt{*}\ \cfgnt{e}_1\ \cfgt{else}\ \cfgnt{e}_2) \rightarrow k)
	}
\and
	\inferrule[ITE (true)]{}{
       (L\ R\ g\ \eta\ \cfgt{true}\ (\cfgt{if}\ \cfgnt{*}\ \cfgnt{e}_1\ \cfgt{else}\ \cfgnt{e}_2) \rightarrow k) \rightarrow (L\ R\ g\ \eta\ \cfgnt{e}_1\  k)
	}
\and
	\inferrule[ITE (false)]{}{
       (L\ R\ g\ \eta\ \cfgt{false}\ (\cfgt{if}\ \cfgnt{*}\ \cfgnt{e}_1\ \cfgt{else}\ \cfgnt{e}_2) \rightarrow k) \rightarrow (L\ R\ g\ \eta\ \cfgnt{e}_2\  k)
	}
\and
	\inferrule[Field Write (eval)]{}{
       (L\ R\ g\ \eta\ (\cfgnt{x}\ \cfgt{\$}\ \cfgnt{f}\ \cfgt{:=}\ \cfgnt{e})\ k) \rightarrow (L\ R\ g\ \eta\ \cfgnt{e}\ (\cfgt{init}\ \cfgnt{f}\ (\cfgnt{x}\ \cfgt{\$}\ \cfgnt{f}\ \cfgt{:=}\ \cfgnt{*}\ \rightarrow\ k)))
	}
\and
	\inferrule[Field Write]{
	r_x = \eta(\cfgnt{x})\\
      	\Lambda = \{ l_i | (l_i,*) \in L(r_x) \wedge R(l_i,f)= \emptyset\}\\\\
      \Lambda = \emptyset \\
      %(\bot,*) \notin L(r_x)  \\\\
     \{(l_{null},\phi)|(l_{null},\phi) \in L(r) \wedge \mathbb{S}(\phi)\}=\emptyset \\\\
      \Psi_x =\{(\phi,l,r^\prime)\ |\ \exists(\cfgnt{l},\phi) \in L(r_x) \wedge \exists r^\prime \in R(\cfgnt{l,f}) \}\\\\
      X = \{ (r^\prime, l^\prime, \theta)| (\phi^\prime,l^\prime,r^\prime)\in \Psi_x \wedge \theta = \mathbb{ST}(L,r,\phi^\prime) \cup \mathbb{ST}(L,r^\prime,\neg\phi^\prime) \}\\\\
      Y= \{l\ |\ (r^\prime, l^\prime, \theta) \in X\}\\ R^\prime = R[\forall l \in Y ((l,f) \mapsto \mathrm{fresh}_r(\mathrm{C}_f)) ]\\\\
      L^\prime = L[\forall l^\prime\ (\exists (r^\prime,l^\prime,\theta) \in X\ (\exists r^{\prime\prime} = R (l^\prime,f) ( r^{\prime\prime} \mapsto \theta) ) )]
    }{
      (L\ R\ g\ \eta\ \cfgnt{r}\ (\cfgnt{x}\ \cfgt{\$}\ \cfgnt{f}\ \cfgt{:=}\ \cfgnt{*}\ \rightarrow\ k)) \rightarrow 
      (L^\prime R^\prime\ g\ \eta\ k)
	}	
\and
   \inferrule[Begin (no args)]{}{
    (L\ R\ g\ \eta\ (\cfgt{begin})\ \cfgnt{k}) \rightarrow
    (L\ R\ g\ \eta\ \cfgnt{k})
   }
\and
   \inferrule[Begin (arg0 eval)]{}{
    (L\ R\ g\ \eta\ (\cfgt{begin}\ \lp \cfgnt{e}_0\ \cfgnt{e}_1\ ...\rp) \rightarrow \cfgnt{k}) \rightarrow
    (L\ R\ g\ \eta\ \cfgnt{e}_0\ (\cfgt{begin}\ \cfgt{*}\ \lp\cfgnt{e}_1\ ...\rp) \rightarrow \cfgnt{k})
   }
\and
   \inferrule[Begin (argI eval)]{}{
    (L\ R\ g\ \eta\ v\ (\cfgt{begin}\ \cfgt{*}\ \lp\cfgnt{e}_i\ \cfgnt{e}_{i+1}\ ...\rp) \rightarrow \cfgnt{k}) \rightarrow
    (L\ R\ g\ \eta\ \cfgnt{e}_i\ (\cfgt{begin}\ \cfgt{*}\ \lp\cfgnt{e}_{i+1}\ ...\rp) \rightarrow \cfgnt{k})
   }
\and
   \inferrule[Begin (argN eval)]{}{
    (L\ R\ g\ \eta\ v\ (\cfgt{begin}\ \lp\cfgnt{e}_{n}\rp) \rightarrow \cfgnt{k}) \rightarrow
    (L\ R\ g\ \eta\ \cfgnt{e}_n\ (\cfgt{begin}\ \cfgt{*}\ \lp\rp) \rightarrow \cfgnt{k})
   }
\and
   \inferrule[Begin]{}{
    (L\ R\ g\ \eta\ v\ (\cfgt{begin}\ \cfgt{*}\ \lp\rp \rightarrow \cfgnt{k})) \rightarrow
    (L\ R\ g\ \eta\ v\ \cfgnt{k})
   }	
\and
   \inferrule[Variable Declaration (eval)]{}{
    (L\ R\ g\ \eta\ \lp\cfgt{var}\ \cfgnt{T}\ \cfgnt{x}\ \cfgt{:=}\ \cfgnt{e}_0\ \cfgt{in}\ \cfgnt{e}_1\rp\ \cfgnt{k})) \rightarrow
    (L\ R\ g\ \eta\ \cfgnt{e}_0\ \lp\cfgt{var}\ \cfgnt{T}\ \cfgnt{x}\ \cfgt{*}\ \cfgt{:=}\ \cfgt{in}\ \cfgnt{e}_1\rp \rightarrow \cfgnt{k})
   }	
\and
   \inferrule[Variable Declaration]{
   \mathrm{fresh}_r(\cfgnt{T}) = r \\
   \mathrm{fresh}_l(\cfgnt{T}) = l \\
   \eta^\prime = \eta[x \mapsto  r]\\\\
   R^\prime = R[\forall \cfgnt{f} \in \mathrm{C}\ ((l\ \cfgnt{f}) \mapsto \mathrm{fresh}_r(\mathrm{type}(\cfgnt{f})))] \\\\
   L^\prime = L[r \mapsto \{(l\ \phi_T)\}]
   }{
    (L\ R\ g\ \eta\ v\ \lp\cfgt{var}\ \cfgnt{T}\ \cfgnt{x}\ \cfgt{*}\ \cfgt{:=}\ \cfgt{in}\ \cfgnt{e}_1\rp \rightarrow \cfgnt{k}) \rightarrow
    (L^\prime\ R^\prime\ g\ \eta^
    \prime\ \cfgnt{e}_1\ (\cfgt{pop}\ \eta\ \cfgnt{k}))
   }	
\and
   \inferrule[Pop]{}{
    (L\ R\ g\ \eta\ (\cfgt{pop}\ \eta_0\ \cfgnt{k})) \rightarrow
    (L\ R\ g\ \eta_0\ \cfgnt{k})
   }
\end{mathpar}
\end{center}
\caption{Uber-lazy state reductions}
\label{fig:expr:red}
\end{figure*}

%% Expression syntax




% Algorithm2e environment
% http://en.wikibooks.org/wiki/LaTeX/Algorithms#Typesetting_using_the_algorithm2e_package
\begin{comment}
\begin{algorithm}
 \SetAlgoLined
 \KwData{this text}
 \KwResult{how to write algorithm with \LaTeX2e }
 initialization\;
 \While{not at end of this document $\wedge x < 2$}{
  read current\;
  \eIf{understand}{
   go to next section\;
   current section becomes this one\;
   }{
   go back to the beginning of current section\;
  }
 }
 \caption{How to write algorithms}
\end{algorithm}
\end{comment}

%\section{Introduction}

% SymExe is cool because for reasons x,y, and z

In recent years symbolic execution has provided the basis for various
software testing and analysis techniques. Symbolic execution
systematically explores the program execution space using symbolic
input values, and for each explored path, computes constraints on the
symbolic inputs to create a \emph{ path condition}.  The path
conditions computed by symbolic execution characterize the observed
program execution behaviors and have been used as an enabling
technology for various applications, e.g., regression
analysis~\cite{backes:2012,Godefroid:SAS11,Person:FSE08,person:pldi2011,Ramos:2011,Yang:ISSTA12},
data structure repair~\cite{KhurshidETAL05RepairingStructurally},
dynamic discovery of
invariants~\cite{CsallnerETAL08DySy,Zhang:ISSTA14}, and
debugging~\cite{Ma:2011}.

%
A major reason that path conditions are so useful is that each one represents exactly the set of concrete program inputs that would result in a given execution path. For any given program, a concrete execution will follow the same path as a symbolic execution if and only if the concrete inputs satisfy the path condition. Furthermore, because path conditions are logical predicates, they allow precise reasoning over potentially unbounded sets of program inputs. 

Symbolic execution's reliance on path conditions is both it's greatest strength, and a significant challenge. Since the path condition must be encoded in the form of an SMT problem, symbolic execution is limited by the capabilities of the underlying constraint solver. If the solver does not include a theory for reasoning about a given program operation, then the symbolic execution cannot proceed. Thus, extending the capabilities of symbolic execution to reason about new theories is an area of active research. 

One area of research involves symbolic reasoning for referencing operations. Of primary interest is dereferencing symbolic input references. There have been many proposed solutions for reasoning about symbolic input references, however to date none of them has been able to do so in a manner that preserves symbolic execution's most desirable property, namely the ability to produce a path condition that exactly represents all possible inputs for a given execution path.

% A couple of symExe�s big problems are path explosion and references. 

%Two of the main challenges facing current symbolic execution techniques 
%are path explosion and programs accepting references as inputs~\cite{Qu:2011,Chen:2013}.
% The path explosion problem stems from the way that 
%symbolic execution explores a program execution on a per-path basis. For 
%those programs for which there is an exponential number of possible 
%program paths, symbolic execution can be extremely inefficient. 
%References are a problem because symbolic execution requires that the 
%program state be represented by predicates formulated in terms of the 
%program inputs. Formulating such predicates for referencing operations 
%over potentially unbounded inputs has until now remained an open 
%question. 


% You can solve dereferencing by concretization, but that�s incomplete. 
%Here is an example of why this stinks.
One possible way to cope with the reference problem is to use a technique 
known as dynamic symbolic execution (DSE). When a DSE tool encounters 
an operation that is beyond the capabilities of its associated constraint 
solvers, such as dereferencing a symbolic reference, it substitutes a valid 
concrete result in place of the symbolic expression. However, this process 
of concretization introduces approximation into the formerly exact path condition,
and may even cause the analysis to miss valid program behaviors. 

%\begin{figure}
%void example( container a,b,c,d,x,y){
%a.f = x;
%b.f = c;
%c.f = d;
%d.f = a;
%if(x.f == y.f.f)
%     abort();\\error!
%}
%\caption{completeness example}
%\label{fig:DSEtest}
%\end{figure}
%
%Consider the example in \ref{fig:DSEtest}. During DSE, both symbolic and 
%concrete executions occur in parallel. Suppose that for the first pass, the 
%concrete execution picks the following points-to relationships: $a\rightarrow 
%loc_a, b\rightarrow loc_b, c\rightarrow loc_c, d\rightarrow loc_d ,x
%\rightarrow loc_a,  y\rightarrow loc_b$. When the program reaches the if() 
%statement, symbolic dereferencing of x and y are required. In order to 
%dereference symbol x to get x.f, x is concretized to point to $loc_a$, and 
%symbol x from field f is returned. Dereferencing y to find y.f requires 
%concretizing y to point to $loc_b$, From there, dereferencing b.f gives us 
%the reference d as the final value for y.f.f . Assuming the �false" branch is 
%executed first, the branch condition (x.f != y.f.f) reduces to x!=d, which 
%evaluates to �true" and DSE proceeds until program termination.
%After following the false branch, DSE will attempt to follow the true branch, 
%but the branch condition x==d will evaluate to false, and so DSE will 
%attempt a second pass starting from the beginning with a modified path 
%condition. For the second pass, DSE adds x==d to the path condition from 
%the beginning, to try to get the �true� branch. This time, the concrete 
%execution will choose x to point to $loc_d$. When evaluating the branch 
%condition, x.f will evaluate to a. Since the value for y is unchanged, the read 
%from y.f.f evaluates to d. The branch condition reduces to a==d, which 
%evaluates to false, so the �true� branch cannot be followed. In this instance, 
%DSE will fail to find a path to the error condition. 

% You can model references by forking the system state, but that makes 
%path explosion worse.
An alternative to concretization is to construct the input heap in a lazy manner,
deferring materialization of objects on the concrete heap until they
are needed for the analysis to proceed. The materialization creates
additional non-deterministic choice points in the symbolic execution
tree by representing the feasible heap configurations as (i) null,
(ii) an instance of a new reference of a compatible class, and (iii)
an alias to a previously initialized symbolic reference.  Symbolic
execution then follows concrete program semantics for materialized
heap locations. Although this approach enables the analysis of heap
manipulating programs, a large number of feasible concrete heap
configurations are created. Since each configuration requires a separate
path, GSE induces a path explosion problem while at the 
same time splitting the symbolic input space and decreasing the 
utility of the path condition.

% You can solve path explosion by bounding the execution and rolling 
%everything into one big equation, but then you�re incomplete. Here is an 
%example why that stinks.
%-maybe point out that finding the right bounds is a hard problem
One solution to the path explosion problem is to model the input heap as a 
predicate over a bounded set of locations. This allows the analysis to 
maintain a single representation for a set of possible heap configurations 
on each execution path, without case splitting for dereferencing operations. 
However, by placing an arbitrary bounds on the size of the input heap, neither
the path condition, nor the analysis performed by these techniques is complete.

% This paper introduces a technique that is, to the best knowledge of the 
%authors, the first sound and complete heap for SymEXE
This paper introduces a technique for modeling references and their 
associated operations that is both sound and complete\footnote{with 
respect to the properties provable by symbolic execution}, and models all 
possible heap configurations along a given execution path. To the 
knowledge of the authors, it is the first technique to do so. 

THIS NEXT BIT NEEDS SOME WORK, STILL:

% Creating this technique required the following key insights:
These advances were enabled by the following key insights: 

First, that a complete analysis requires an unconstrained input heap. 
However, it is unclear from prior work how this might be accomplished.

Second, that GSE techniques needlessly split during dereferencing 
operations. If GSE was performing a concretization, as in DSE, then case 
splitting makes more sense. However, lazy initialization is not a 
concretization, and is in fact only a partitioning of the input space. 

Third, that we can combine lazy initialization with a compositional heap 
abstraction to form constraints on a potentially unbounded input heap.


% This paper makes the following contributions.


\noindent{This paper makes the following contributions:}

\begin{compactdesc}

\item\textbf{-} The first sound and complete system for reasoning symbolically about the 
set of input heaps along any valid program path. 

\item\textbf{-} A bisimulation proof establishing the soundness and 
completeness of the heap summary approach with respect to
properties provable by GSE.

\item\textbf{-} A proof-of-concept implementation and empirical study 
demonstrating the scalability of the summary heap approach
compared to other GSE approaches.

\end{compactdesc}


%Initial work on symbolic execution largely focused on checking
%properties of programs with primitive
%types~\cite{clarke76TSE,King:76}.  With the advent of object-oriented
%languages, recent work has generalized the core ideas of symbolic
%execution to enable analysis of programs containing complex data
%structures with unbounded domains, i.e., data stored on the
%heap~\cite{Kiasan06,Kiasan07,GSE03}.  These \emph{Generalized 
%Symbolic
%  Execution} (GSE) techniques construct the heap in a lazy manner,..

%The goal of this work is to mitigate the path explosion problem in GSE
%by grouping multiple heaps together and only partitioning the heaps at
%points of divergence in the control flow graph. Our inspiration is
%found in the domain of static analysis that uses sets of constraints
%over heap locations to encode multiple heaps in a single
%representation. These sets, sometimes known as \emph{value sets} or
%\emph{value summaries}, allow multiple heaps to be represented
%simultaneously with a higher degree of precision than afforded by
%traditional techniques for shape analysis. Some of these previous
%attempts, however, are unable to handle aliasing in heaps due to a
%recursive definition of objects~\cite{Xie:2005}, while others require
%a set of heaps as input and are unable to initialize heaps
%on-the-fly~\cite{Dillig:2011,Tillmann:2008}.  As is typical in static
%analysis, over approximation of the heap representations alleviates
%some of these limitations but often leads to infeasible heaps. Also,
%most static analysis techniques for heap updates require a rewrite of
%the constraint system making it prohibitive for use in the context of
%symbolic execution.

%In this work, to effectively represent multiple heaps simultaneously
%in the context of symbolic execution, we define a novel heap
%representation and an algorithm to initialize and update the heap
%on-the-fly. In essence, our summary heap approach supports aliasing,
%does not require constraint rewriting on heap updates, and reduces
%non-determinism in the search space during symbolic execution.
%
%We represent the heap as a bipartite graph consisting of references
%and locations. Each reference is able to point to multiple locations,
%where each edge is predicated on constraints over aliasing between
%references. Each location points to a single reference for a given
%field. The use of a bipartite graph affords two key advantages: (i) it
%allows for a non-recursive definition of objects which enables us to
%support aliasing, and (ii) it does not introduce auxiliary variables
%or require rewriting of non-local constraints during updates to the
%heap.
%
%The summary heap algorithm defines how the bipartite graph is updated
%during lazy initialization. Unlike GSE, however, the summary heap
%algorithm introduces non-determinism in the search only at points of
%divergence in the control flow graph. These points of non-determinism
%are at field accesses that lead to null-pointer exceptions and at
%comparisons of references. The former represents a divergence due to
%exceptional control flow while the latter is due to program structure.
%The combination of the heap representation and the summary heap
%algorithm enables us to improve the efficiency of symbolic execution
%of heap manipulating programs over state-of-the-art techniques.
%
%The summary heap algorithm is sound and complete with respect to
%properties that are provable using GSE. This proof is accomplished by
%showing the existence of a bisimulation between states in GSE and
%states in the summary heap algorithm.  A preliminary implementation in
%Java Pathfinder (JPF) shows that in general, the heap summary
%algorithm improves over other state-of-the-art techniques, and in some
%instances, the improvement is remarkable: two-orders of magnitude
%reduction in running time. More importantly though, the heap summary
%enables the initialization of larger more complex heaps than
%previously possible as shown in our results.


%\section{System State}
The program state is represented using a path condition, a program location, a stack, a symbolic heap, and a symbolic store. The path condition is a collection of predicates over the program inputs that indicates constraints on the values of those inputs at the present location. The program location indicates the instruction to be executed presently. The symbolic heap is a mapping from heap locations symbolic objects. 
\paragraph{Heap Symbols}
A heap symbol $\pi$ is a symbol which is created in the process of performing heap operations. The dynamic nature of heap symbols distinguishes them from statically-created symbols like those found in the path condition. Heap symbols may be symbolic primitives, symbolic references, symbolic locations, or symbolic types.
\paragraph{Symbolic References}
Symbolic references are symbolic points-to relations. The symbolic reference has a state parameter that indicates whether the reference is uninitialized, non-null, or initialized. The uninitialized and non-null states indicate that the location pointed to by the reference has yet to be resolved. The non-null state reflects the additional constraint that the reference does not point to the null location. References in the initialized state are associated with constraints reflecting which location the reference points to and under what condition it points to that location. References point to one and only one location at a time, so the conditions must be mutually exclusive, yet collectively exhaustive.
\paragraph{Symbolic Locations}
A symbolic location $h$ represents the index of a slot in the heap structure that holds a symbolic object. Each symbolic location represents a unique slot on the symbolic heap. There are two special symbolic locations null and non-null, which are never in the symbolic heap.
\paragraph{Symbolic Objects}
A symbolic object represents the contents of a heap slot. A symbolic object is characterized by a symbolic type, and contains a mapping from fields to heap symbols.
\paragraph{Constrained Symbol}
A constrained symbol $\psi$ is a symbol that assumes a value $\pi$ contingent upon the satisfiability of a constraint condition $\phi$.
\begin{equation}
\psi\models \phi\uparrow \Rightarrow\pi
\end{equation}
\paragraph{Symbolic Value Set}
A symbolic value set $\theta$ is a set of constrained symbols $\theta\colon \{\psi _1,\psi _2,...,\psi _n\}$, for which the constraints are mutually exclusive and collectively exhaustive.
\paragraph{Symbolic Store}
The symbolic store is a mapping from heap symbols to symbolic value sets $\mathbb{S}\colon \pi \mapsto \theta$. The abstract store represents a set of heaps common to the current program execution path. 
\paragraph{Heap}
The symbolic heap $\mathbb{H}$ is an indexed set of symbolic objects. Like the path condition, the symbolic heap contains heap state which is common to all heaps on the current execution path. 
\paragraph{Stack}
The Java Virtual Machine uses a system stack.
%\section{Semantic Model}
Program execution proceeds an in standard symbolic execution, with additional rules to handle heap constructs. The Java Virtual Macine (JVM) implements five classes of semantic rules, a categorization which we find relevant to symbolic execution: 1) load/store instructions, 2) arithmetic instructions, 3) object creation / manipulation instructions, 4) control transfer instructions, and 5) assume/assert instructions. We will deal with each class of instruction in sequence.
\subsection{Reference Target Resolution}
Load and store instructions take symbolic references as arguments. Since an uninitialized symbolic reference may point to any one of a number of symbolic locations, we need to be able to resolve which locations are feasible targets of a particular references.
Location resolution begins with an empty list of heap locations, and then adds locations by comparing locations on the symbolic heap to the constraints on the symbolic reference obtained from the abstract store. Locations are checked in the following order:
\begin{compactenum}
\item If the null location is feasible, return a null pointer exception and terminate execution.
\item If the non-null location is feasible, create a new symbolic location of the proper type and add it to the symbolic heap. Search the heap for type-compatible objects and add those to the symbolic value set for the reference, and remove the non-null location from the symbolic value set.
\item Check the symbolic heap for type-compatible objects, comparing those against the constraints in the symbolic value set. Add any feasible objects to the feasible location set.
\end{compactenum}

\subsection{Read}
The load instruction has two arguments: a reference $r$ and a field index $f$. If $r$ is a symbolic location, then load simply accesses the field and returns the value contained there. If $r$ is a symbolic reference, then the following steps are taken, in order.
\begin{compactenum}
\item Resolve a list of feasible targets as described above.
\item Access the fields of the feasible targets, gathering a set of the symbolic values contained therein. 
\item Form a new symbolic value, and create a new entry in the symbolic store mapping the value to the symbolic value set. Return the new symbolic value.
\end{compactenum}

\subsection{Write}
The write instruction takes three arguments: a reference $r$, a field $f$, and a value $v$. If $r$ is a symbolic location, then the target field is written with the value directly. If $r$ is a symbolic reference, write proceeds as follows:
\begin{compactenum}
\item Resolve a list of feasible targets.
\item Access the fields of the feasible targets, performing a conditional write. The conditional write works by modifying the constraint equation as follows:
\begin{equation}
  ((r\rightarrow h_n) \Rightarrow f\rightarrow v)  \wedge (\lnot (r\rightarrow h_n) \Rightarrow Eqn_o_l_d)
\end{equation}
\end{compactenum}

\subsection{Arithmetic Instructions}
\subsection{Object Creation/Manipulation Instructions}
\subsection{Control Transfer Instructions}
Control transfer instructions are those instructions that have a "branching" nature. They are instructions that have more than one possible program location successor. Control transfer instructions are executed by comparing the compare condition to the constraints contained in the symbolic store. Those values in the store for which the constraints are now infeasible are eliminated. 
\subsection{Assume / Assert Instructions}
%\section{Example}
%\section{Bytecodes}

PUTFIELD needs to remove path constraints from PC that enforcing equality between references.

\noindent \textbf{Assume}: all symbolic locations are concertized lazily. Although the algorithm is not specific to any particular initialization strategy, the presentation assumes a lazy initialization. Extending to lazier initialization may be non-trivial even though this reduction is orthogonal to the lazier reduction (i.e., this reduction should further improve the performance of the lazier algorithm).

\noindent \textbf{Assume}: all variables, symbolic or otherwise, are non-primitive (i.e. objects). \textit{Must relax this assumption because you need to do some interesting things with primitives as they relate to getfield}.

This papers uses subsumption, which is expressed as a subtyping relation $\leq$ over types $T$. For classes $C$ and $D$, $C \leq D$ iff either $C = D$ or the class declaration for $C$ is \texttt{class C extends B $\{\ldots\}$} for some $B \leq D$. For example, in $A \leq B \leq C \leq D$, $D$ is the supertype, and if you have something that is an instance of $A$ but currently viewed as $B$, then you can move it toward $D$ in a typecast (up the hierarchy).

Heap locations range over positive natural numbers $H \subseteq \mathbb{N}_{\geq 0}$. Every heap location has a special variable $T_h$ used in constraints over the type of the object stored in the heap location. The varable $\mathit{SH}$ is the set of heap locations created when concretizing symbolic variables. The set of constraints over the type stored in the heap location is given by the function $\mathtt{C}(h)$. Constraints are of the form $T_h \sim T$ or $T \sim T$ where $\sim\ \in \{\leq, =, \not =\}$. The initial type hierarchy for the program is expressed as a set of constraints $C_\mathrm{init}$. The set contains all relationships needed to describe the entire hierarchy.

For a set of constraints $C$, the function $\mathtt{SAT}(C) \mapsto \{0,1\}$ returns 1 if the constraints are satisfiable and 0 otherwise. The usual Boolean connectives are used as expected. The function $\mathtt{Type}(h)$ returns the actual type of the object in the heap location and the function $\mathtt{Obj}(h)$ returns the object at the location.

Each variable $v$ is associated with a set of heap locations $H(v) = \{h_0, h_1, \ldots, h_n\}$ that represents an equivalence class (i.e., each heap location yields the same execution path and behavior up to the current point of execution).  The representative object for a given variable (i.e., the one that is currently being used by the variable) is given by the function $I(v) \mapsto H(v)$. The set of heap locations and the representative location are part of the meta-deta for the variable. This meta-data follows the variable through the program execution and is appropriately copied on assignment to other variables such that each variable has its own copy of the meta-data that is separate from other copies.

Finally, there is a global variable $\mathit{PC}$ that represents the path constraint along the current path of exploration. This path constraint is used to track relationships between symbolic variables such as equality. Properties of symbolic variables are represented in the path constraint by creating special variables for the representative and the set of represented objects. For a variable $v$, the special variable $I_v$ is the representative heap opbject and the special variable $H_v$ is the set of associated heap locations. It is assumed the $v$ is alpha-renamed to be unique in the path constraint. Finally, we use the special value \texttt{SYM} to denote a symbolic variable that is yet to be initialized.

\subsection{Reference}
\noindent \textbf{\texttt{GETFIELD}}: the bytecode behavior depends on the field operand: concrete, concrete though initialized from symbolic, and symbolic. Each case is enumerated:
\begin{compactenum}
\item Referencing a concrete field: the bytecode has default behavior returning the field.
\item Referencing an initialized field from a symbolic variable (i.e., the base type for the field is initialized from a symbolic object): the bytecode may have multiple outcomes; it partitions the equiavalence class to group heap locations with objects that have common values for the field.
\item Referencing a symbolic variable that has yet to be initialized: the bytecode has two outcomes: one that returns \texttt{null} and another that builds the potential equivalence class, chooses a representative location, and returns that location.
\end{compactenum}
Consider the code
\begin{lstlisting}
// The declared type of f is F
T t = b.f;
\end{lstlisting}
For the case where \texttt{b.f} is an already initialized symbolic variable, $h = \mathtt{I}(\mathtt{b})$, $\mathtt{H}(\mathtt{b})$ is partitioned into disjoint sets, $S_0, P_1, \ldots, P_n$, with $n+1$ partitions, or choices. The first set $S_0$ is a special set that includes $h$, the representative object for \texttt{b}, and any members of the equivalence class that either have the same field value for \texttt{b.f} or the field value is a symbolic variable that has yet to be initialized:
\begin{eqnarray*}
S_0 &=& \{h_i \mid h_i \in \mathtt{H}(\mathtt{b})\ \wedge \\
    & & (\mathtt{Obj}(h).\mathtt{f} = \mathtt{Obj}(h_i).\mathtt{f}\ \vee\ \mathtt{Obj}(h_i).\mathtt{f} = \mathtt{SYM})\}
\end{eqnarray*}
For this special case of $S_0$,  $\mathtt{H}_0(\mathtt{b}) = S_0$, and $I_0(\mathtt{b}) = h$ where the subscript indicates the choice number in the choice generator (i.e., the partition size may change but not the representative heap location). The other partitions group common values of the field such that
\begin{compactitem}
\item $\forall h_i, h_j \in P_i,\ \mathtt{Obj}(h_i).f = \mathtt{Obj}(h_j).f$ 
\item $\mathtt{H}_i(\mathtt{b}) = P_i$ 
\item $\exists h \in P_i,\ \mathtt{I}_i(\mathtt{b}) = h$
\end{compactitem}
The partitions are maximal and represent a unique value that has been created thus far in the program execution. The non-initialized symbolic members of the partition all belong to $S_0$ as the original representative heap location $h$ captures that these other aliases were intended to have the same field value for field \texttt{f} before the split (i.e., the value is assigned programatically but the change was only reflected in the representative heap location). Once the choice generator is created over the different partitions, the bytecode returns the requested field value of the representative as expected.

Returning to the third behavior of the bytecode, the case in which the accessed field has yet to be initialized, then the bytecode follows lazy initialization creating a \texttt{null} instance, a new instance that is the representative, and the alias set. When creating the alias set, the new instance should be included in the set, as well as any prior object created in concretization of symbolic variables that is type compatible with the new instance. Recall that $\mathit{SH}$ is set of locations in the symbolic heap and $C_\mathrm{init}$ is the set of constraints describing the type hierarchy, assuming $h$ is the heap location of the new instance of the type, then 
\begin{compactitem}
\item $\mathtt{I}(\mathtt{b.f}) = h$
\item $\forall h_i \in \mathit{SH}, \mathtt{SAT}(C_\mathrm{init} \cup \{T_h \leq \mathtt{Type}(h_i)\}) \rightarrow h_i \in \mathtt{H}(\mathtt{b.f})$  
\item $\mathtt{C}(\mathtt{b.f}) = C_\mathrm{init} \cup \{T_h \leq \mathtt{Type}(h)\}$  
\end{compactitem}
The $C_\mathrm{init}$ set constains relationships in the class hierarchy with the correct sub-types and super-types as they relate to the delcared type of the object.

\noindent \textbf{\texttt{GETSTATIC}}: the bytecode is handled similarly to \texttt{GETFIELD}. 

\noindent \textbf{\texttt{ALOAD}}: the bytecode is handled similarly to \texttt{GETFIELD}. 

\subsection{Comparison}

\noindent \textbf{\texttt{IF\_ACMPEQ}}: the bytecode may return both the \texttt{true} and \texttt{false} values, and it must possibly refine the set of represented concretizations and mutate the heap location of the object involved in the bytecode according to the returned outcome. Consider the code
\begin{lstlisting}
if (a == b) {
   // code...
}
\end{lstlisting}
There are two cases that need to be considered to determine the outcome of the bytecode:
\begin{compactenum}
\item $\mathtt{H}(a) \cap \mathtt{H}(b) = \emptyset$: the bytecode returns \texttt{false} and nothing further is requred.
\item $\mathtt{H}(a) \cap \mathtt{H}(b) \not = \emptyset$ $\wedge$ $\mathtt{SAT}(PC \cup \{I_a = I_b, H_a = H_b\})$: the bytecode may return either \texttt{true} or \texttt{false} and a choice generator needs to be created.
\end{compactenum}
The choice generator for the compare bytecode is more complex than for other bytecodes because it must create representative sets without enumerating all possible outcomes using the path constraint. For the case \texttt{true} outcome
\begin{compactitem}
\item $\mathit{PC} = \mathit{PC} \cup \{\mathtt{I}(a) = \mathtt{I}(b),\mathtt{H}(a) = \mathtt{H}(b)\}$
\item $\mathtt{H}(a) = \mathtt{H}(b) = \mathtt{H}(a) \cap \mathtt{H}(b)$
\item $\mathtt{I}(a) \in \mathtt{H}(a) \cap \mathtt{H}(b) \rightarrow \mathtt{I}(b) = \mathtt{I}(a)$ $\vee$ $\exists h \in \mathtt{H}(a) \cap \mathtt{H}(b)\ .\ \mathtt{I}(b) = \mathtt{I}(a) = h$
\end{compactitem}
In essence, in the case where two variable reference the same object, the path constraint and sets are modified to represent the new restriction. The \texttt{false} outcome is handled similarly with a few notable exceptions on the path constraint and the represented set.
\begin{compactitem}
\item $\mathit{PC} = \mathit{PC} \cup \{\mathtt{I}(a) \not = \mathtt{I}(b)\}$
\item $\mathtt{I}(a) = \mathtt{I}(b) \rightarrow \exists h \in \mathtt{H}(b)\ .\ h \not = \mathtt{I}(a) \wedge \mathtt{I}(b) = h$
\end{compactitem}

\noindent \textbf{\texttt{IF\_ACMPNE}}: the bytecode is handled similarly to \texttt{IF\_ACMPEQ}. 

\subsection{Invocation}
\textbf{\texttt{INVOKEVIRTUAL}}

When we come to an invoke virtual you have to look for all the specialized implementations
of the method, creating choices with symbolic locations of various "actual types". The number
of choices will be equal to the number of specialized implementations of the method. When you create a choice on a specialization, you need to update the "actual type" field in the symbolic location. The "current cast" does not need to change. The number of types that the symbolic location cannot be will also be updated according to the "actual type" field. The number of types that the symbolic location cannot be will be updated with the types of the other specializations since invoking a specialization associated with a type implies that the object cannot be the types containing the other specializations.

\subsection{Checking Types and Casting}
\noindent\textbf{\texttt{INSTANCEOF}}: the bytecode may return both the \texttt{true} and \texttt{false} values when dealing with initialized symbolic variables, and it must possibly refine the equivalence class for the represented object referenced by the variable and mutate the contents of the heap location of the object involved in the bytecode according to the returned outcome. The bytecode implements the default bahvior when the operand is concrete and not an initialized symbolic variable. For the rest of the discussion, assume the operand is an initialized symbolic variable.
Consider the code
\begin{lstlisting}
if (a instanceof C) {
   // code...
}
\end{lstlisting}
There are two cases that need to be considered to determine the outcome of the bytecode where $h = \mathtt{I}(a)$ is the representative object of the equivalence class:
\begin{compactenum}
\item $\mathtt{Type}(h) = C$: the bytecode returns \texttt{true} and nothing further is required as the type stored in the heap location is $C$.
\item $\neg \mathtt{SAT}(\mathtt{C}(h) \cup \{T_h \leq C\})$: the bytecode returns \texttt{false} and nothing further is requred as the current constraints on what is in the heap location restrict it from being of type $C$.
\item $\mathtt{SAT}(\mathtt{C}(h) \cup \{T_h \leq C\})$: the bytecode can return either \texttt{true} or \texttt{false} requiring a choice generator.
\end{compactenum}
% // C <= B <= A
% A a; // C(a) = {(T_a <= A)}
%
%if (a instance of C) {
%    ** TRUE **
%    (T_h \leq C)
%    ...
%}
% ** FALSE **
% (C \leq T_h) \wedge (T_h \not = C)
% ** TRUE **
% (T_h \leq C)
\noindent The \texttt{true} outcome for the choice generator in clause (3) is
\begin{compactitem}
    \item $\mathtt{C}(h) = \mathtt{C}(h) \cup \{T_h \leq C\}$
    \item $\mathtt{Type}(h) = C$
    \item $\mathtt{H}(a) = H^\prime$ where $H^\prime = \{h_i \mid h_i \in \mathtt{H}(a) \wedge \mathtt{SAT}(\mathtt{C}(h_i) \cup \{T_{h_i} \leq C\})\}$ 
\end{compactitem}
The second statement indicates that the actual type in the heap location $h$ needs to change. As such, the object is mutated to be an instance of $C$. This mutation retains all fields and values from the previous object and only adds new fields for type $C$. The last statement refines the equivalence class to exclude any heap locations that cannot be considered something of type $C$. 

For the \texttt{false} outcome of the generator, $\mathtt{C}(h) = \mathtt{C}(h) \cup \{C \leq T_h, T_h \not = C\}$. Unlike the \texttt{true} outcome, the \texttt{false} outcome retains the entire equivalence class and does not need to mutate any heap entries.

\noindent\textbf{\texttt{CHECKCAST}}: the bytecode is syntactic sugar for 
\begin{lstlisting}
if (! (obj == null  ||  obj instanceof <class>)) {
    throw new ClassCastException();
}
// if this point is reached, then object 
// is either null, or an instance of <class> 
// or one of its superclasses.
\end{lstlisting}
Please see the \texttt{IFNULL} and \texttt{INSTANCEOF} bytecodes for details. If the exception is thrown, then JPF will catch the unhandled exception as per its normal behavior.

\subsection{Programs to consider}
\begin{compactitem}
\item \texttt{TestGetfieldSplit.java}: checks alias equivalence classes when assigning to initialized values.
\end{compactitem}
%\section{Conclusion}
The conclusion goes here.




% conference papers do not normally have an appendix


% use section* for acknowledgement
\section*{Acknowledgment}


The authors would like to thank...
Plus I just rule. What?




% trigger a \newpage just before the given reference
% number - used to balance the columns on the last page
% adjust value as needed - may need to be readjusted if
% the document is modified later
%\IEEEtriggeratref{8}
% The "triggered" command can be changed if desired:
%\IEEEtriggercmd{\enlargethispage{-5in}}

% references section

% can use a bibliography generated by BibTeX as a .bbl file
% BibTeX documentation can be easily obtained at:
% http://www.ctan.org/tex-archive/biblio/bibtex/contrib/doc/
% The IEEEtran BibTeX style support page is at:
% http://www.michaelshell.org/tex/ieeetran/bibtex/
%\bibliographystyle{IEEEtran}
% argument is your BibTeX string definitions and bibliography database(s)
%\bibliography{IEEEabrv,../bib/paper}
%
% <OR> manually copy in the resultant .bbl file
% set second argument of \begin to the number of references
% (used to reserve space for the reference number labels box)
\begin{thebibliography}{1}

\bibitem{IEEEhowto:kopka}
H.~Kopka and P.~W. Daly, \emph{A Guide to \LaTeX}, 3rd~ed.\hskip 1em plus
  0.5em minus 0.4em\relax Harlow, England: Addison-Wesley, 1999.

\end{thebibliography}




% that's all folks
\end{document}


