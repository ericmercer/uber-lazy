\section{Rules for Javalite}\label{app:javalite}

\begin{figure*}[t]
\begin{center}
\mprset{flushleft}
\begin{mathpar}
	\inferrule[Variable lookup]{}{
      \lp \cfgnt{L}\ \cfgnt{R}\ \phi\ \eta\ \cfgnt{x}\ \cfgnt{k}\rp  \rightarrow_J \\\\
      \lp \cfgnt{L}\ \cfgnt{R}\ \phi\ \eta\ \eta\lp \cfgnt{x}\rp \ \cfgnt{k}\rp 
	}
\and
	\inferrule[New]{
      \cfgnt{r} = \mathrm{stack}_r\lp \rp\\
      l = \mathrm{fresh}_l\lp \cfgnt{C}\rp\\\\
      \cfgnt{R}^\prime = \cfgnt{R}[\forall \cfgnt{f} \in \mathit{fields}\lp \mathrm{C}\rp \ \lp \lp l\ \cfgnt{f}\rp  \mapsto \cfgnt{r}_\mathit{null} \rp ] \\\\
      \cfgnt{L}^\prime = \cfgnt{L}[\cfgnt{r} \mapsto \{\lp \cfgt{true}\ l\rp \}]
    }{
      \lp \cfgnt{L}\ \cfgnt{R}\ \phi\ \eta\ \lp \cfgt{new}\ \cfgnt{C}\rp \ \cfgnt{k}\rp  \rightarrow_J
      \lp \cfgnt{L}^\prime\ \cfgnt{R}^\prime\ \phi\ \eta\ \cfgnt{r}\ \cfgnt{k}\rp 
	}
\and
	\inferrule[Field Access(eval)]{}{
      \lp \cfgnt{L}\ \cfgnt{R}\ \phi\ \eta\ \lp \cfgnt{e}\ \cfgt{\$}\ \cfgnt{f}\rp \ \cfgnt{k}\rp  \rightarrow_J \\\\
      \lp \cfgnt{L}\ \cfgnt{R}\ \phi\ \eta\ \cfgnt{e}\ \lp \cfgt{*}\ \cfgt{\$}\ \cfgnt{f} \rightarrow \cfgnt{k}\rp \rp 
	}
\and
	\inferrule[Field Write (eval)]{}{
       \lp \cfgnt{L}\ \cfgnt{R}\ \phi\ \eta\ \lp \cfgnt{x}\ \cfgt{\$}\ \cfgnt{f}\ \cfgt{:=}\ \cfgnt{e}\rp \ \cfgnt{k}\rp  \rightarrow_J \\\\
       \lp \cfgnt{L}\ \cfgnt{R}\ \phi\ \eta\ \cfgnt{e}\ \lp \cfgnt{x}\ \cfgt{\$}\ \cfgnt{f}\ \cfgt{:=}\ \cfgt{*}\ \rightarrow\ \cfgnt{k}\rp \rp 
	}
\and
    \inferrule[Equals (l-operand eval)]{}{
      \lp \cfgnt{L}\ \cfgnt{R}\ \phi\ \eta\ \lp \cfgnt{e}_0\ \cfgt{=}\ \cfgnt{e}\rp  \ \cfgnt{k}\rp  \rightarrow_J \\\\
      \lp \cfgnt{L}\ \cfgnt{R}\ \phi\ \eta\ \cfgnt{e}_0\ \lp \cfgt{*}\ \cfgt{=}\; \cfgnt{e} \rightarrow \cfgnt{k}\rp \rp 
    }
\and
    \inferrule[Equals (r-operand eval)]{}{
    \lp \cfgnt{L}\ \cfgnt{R}\ \phi\ \eta\ \cfgnt{v}\ \lp \cfgt{*}\; \cfgt{=}\; \cfgnt{e} \rightarrow \cfgnt{k}\rp \rp  \rightarrow_J \\\\
    \lp \cfgnt{L}\ \cfgnt{R}\ \phi\ \eta\ \cfgnt{e}\ \lp \cfgnt{v}\; \cfgt{=}\; \cfgt{*} \rightarrow \cfgnt{k}\rp \rp 
    }
\and
    \inferrule[Equals (bool)]{
    \cfgnt{v}_0 \in \{\cfgt{true}, \cfgt{false}\} \\
    \cfgnt{v}_1 \in \{\cfgt{true}, \cfgt{false}\} \\\\
    \cfgnt{v}_r = \mathrm{eq?}\lp \cfgnt{v}_0, \cfgnt{v}_1\rp}{
    \lp \cfgnt{L}\ \cfgnt{R}\ \phi\ \eta\ \cfgnt{v}_0\ \lp \cfgnt{v}_1\; \cfgt{=}\; \cfgt{*} \rightarrow \cfgnt{k}\rp \rp  \rightarrow_J \\\\
    \lp \cfgnt{L}\ \cfgnt{R}\ \phi\ \eta\ \cfgnt{v}_r\ \cfgnt{k}\rp 
    }
\and
    \inferrule[If-then-else (eval)]{}{
      \lp \cfgnt{L}\ \cfgnt{R}\ \phi\ \eta\ \lp \cfgt{if}\ \cfgnt{e}_0\ \cfgnt{e}_1\ \cfgt{else}\ \cfgnt{e}_2\rp \ \cfgnt{k}\rp  \rightarrow_J \\\\
      \lp \cfgnt{L}\ \cfgnt{R}\ \phi\ \eta\ \cfgnt{e}_0\ \lp \cfgt{if}\ \cfgt{*}\ \cfgnt{e}_1\ \cfgt{else}\ \cfgnt{e}_2\rp  \rightarrow \cfgnt{k}\rp 
	}
\and
	\inferrule[If-then-else (true) ]{}{
       \lp \cfgnt{L}\ \cfgnt{R}\ \phi\ \eta\ \cfgt{true}\ \lp \cfgt{if}\ \cfgt{*}\ \cfgnt{e}_1\ \cfgt{else}\ \cfgnt{e}_2\rp  \rightarrow_J \cfgnt{k}\rp  \rightarrow \\\\
       \lp \cfgnt{L}\ \cfgnt{R}\ \phi\ \eta\ \cfgnt{e}_1\  \cfgnt{k}\rp 
	}
\and
	\inferrule[If-then-else (false)]{}{
       \lp \cfgnt{L}\ \cfgnt{R}\ \phi\ \eta\ \cfgt{false}\ \lp \cfgt{if}\ \cfgt{*}\ \cfgnt{e}_1\ \cfgt{else}\ \cfgnt{e}_2\rp  \rightarrow_J \cfgnt{k}\rp  \rightarrow \\\\
       \lp \cfgnt{L}\ \cfgnt{R}\ \phi\ \eta\ \cfgnt{e}_2\  \cfgnt{k}\rp 
	}
\and
   \inferrule[Variable Declaration (eval)]{}{
    \lp \cfgnt{L}\ \cfgnt{R}\ \phi\ \eta\ \lp\cfgt{var}\ \cfgnt{T}\ \cfgnt{x}\ \cfgt{:=}\ \cfgnt{e}_0\ \cfgt{in}\ \cfgnt{e}_1\rp\ \cfgnt{k}\rp  \rightarrow_J \\\\
    \lp \cfgnt{L}\ \cfgnt{R}\ \phi\ \eta\ \cfgnt{e}_0\ \lp\cfgt{var}\ \cfgnt{T}\ \cfgnt{x}\ \cfgt{:=}\ \cfgt{*}\ \cfgt{in}\ \cfgnt{e}_1 \rightarrow \cfgnt{k}\rp\rp 
   }	
\and
   \inferrule[Variable Declaration]{}{
    \lp \cfgnt{L}\ \cfgnt{R}\ \phi\ \eta\ \cfgnt{v}\ \lp\cfgt{var}\ \cfgnt{T}\ \cfgnt{x}\ \cfgt{*}\ \cfgt{:=}\ \cfgt{in}\ \cfgnt{e}_1 \rightarrow \cfgnt{k}\rp\rp  \rightarrow_J \\\\
    \lp \cfgnt{L}\ \cfgnt{R}\ \phi\ \eta[x \mapsto \cfgnt{v}]\ \cfgnt{e}_1\ \lp \cfgt{pop}\ \eta\ \cfgnt{k}\rp \rp 
   }	
\and
   \inferrule[Method Invocation (object eval)]{}{
    \lp \cfgnt{L}\ \cfgnt{R}\ \phi\ \eta\ \lp\cfgnt{e}_0\ \cfgt{@}\ \cfgnt{m}\ \cfgnt{e}_1\rp\ \cfgnt{k}\rp  \rightarrow_J \\\\
    \lp \cfgnt{L}\ \cfgnt{R}\ \phi\ \eta\ \cfgnt{e}_0\ \lp \cfgt{*}\ \cfgt{@}\ \cfgnt{m}\ \cfgnt{e}_1\ \rightarrow \cfgnt{k}\rp \rp 
   }
\and
   \inferrule[Method Invocation (arg eval)]{}{
    \lp \cfgnt{L}\ \cfgnt{R}\ \phi\ \eta\ \cfgnt{v}_0\ \lp \cfgt{*}\ \cfgt{@}\ \cfgnt{m}\ \cfgnt{e}_1\ \rightarrow \cfgnt{k}\rp \rp  \rightarrow_J \\\\
    \lp \cfgnt{L}\ \cfgnt{R}\ \phi\ \eta\ \cfgnt{e}_1\ \lp \cfgnt{v}_0\ \cfgt{@}\ \cfgnt{m}\ \cfgt{*}\ \rightarrow \cfgnt{k}\rp \rp 
   }
\and
   \inferrule[Method Invocation]{
    \lp\cfgnt{T}\ \cfgnt{m}\ \lb\cfgnt{T}\ \cfgnt{x}\rb\ \ \cfgnt{e}_m\rp = \mathrm{lookup}\lp \cfgnt{m}\rp\\\\
    \eta_m = \eta[\cfgt{this} \mapsto \cfgnt{v}_0][\cfgnt{x} \mapsto \cfgnt{v}_1]}{
    \lp \cfgnt{L}\ \cfgnt{R}\ \phi\ \eta\ \cfgnt{v}_1\ \lp \cfgnt{v}_0\ \cfgt{@}\ \cfgnt{m}\ \cfgt{*}\ \rightarrow \cfgnt{k}\rp \rp  \rightarrow_J \\\\
    \lp \cfgnt{L}\ \cfgnt{R}\ \phi\ \eta_m\ \cfgnt{e}_m\ \lp \cfgt{pop}\ \eta\ \cfgnt{k}\rp \rp 
   }
\and
   \inferrule[Variable Assignment (eval)]{}{
    \lp \cfgnt{L}\ \cfgnt{R}\ \phi\ \eta\ \lp \cfgnt{x}\ \cfgt{:=}\ \cfgnt{e}\rp \ \cfgnt{k}\rp  \rightarrow_J \\\\
    \lp \cfgnt{L}\ \cfgnt{R}\ \phi\ \eta\ \cfgnt{e}\ \lp \cfgnt{x}\ \cfgt{:=}\ \cfgt{*} \rightarrow\ \cfgnt{k}\rp \rp 
   }	
\and
   \inferrule[Variable Assignment]{}{
    \lp \cfgnt{L}\ \cfgnt{R}\ \phi\ \eta\ \cfgnt{v}\ \lp \cfgnt{x}\ \cfgt{:=}\ \cfgt{*} \rightarrow\ \cfgnt{k}\rp \rp  \rightarrow_J \\\\
    \lp \cfgnt{L}\ \cfgnt{R}\ \phi\ \eta[\cfgnt{x} \mapsto \cfgnt{v}]\ \cfgnt{v}\ \cfgnt{k}\rp 
   }	
\and
   \inferrule[Begin (no args)]{}{
    \lp \cfgnt{L}\ \cfgnt{R}\ \phi\ \eta\ \lp \cfgt{begin}\rp \ \cfgnt{k}\rp  \rightarrow \\\\
    \lp \cfgnt{L}\ \cfgnt{R}\ \phi\ \eta\ \cfgnt{k}\rp 
   }
\and
   \inferrule[Begin (arg0 eval)]{}{
    \lp \cfgnt{L}\ \cfgnt{R}\ \phi\ \eta\ \lp \cfgt{begin}\ \cfgnt{e}_0\ \cfgnt{e}_1\ ...\rp \ \cfgnt{k}\rp  \rightarrow_J \\\\
    \lp \cfgnt{L}\ \cfgnt{R}\ \phi\ \eta\ \cfgnt{e}_0\ \lp \cfgt{begin}\ \cfgt{*}\ \lp\cfgnt{e}_1\ ...\rp \rightarrow \cfgnt{k}\rp \rp 
   }
\and
   \inferrule[Begin (argi eval)]{}{
    \lp \cfgnt{L}\ \cfgnt{R}\ \phi\ \eta\ \cfgnt{v}\ \lp \cfgt{begin}\ \cfgt{*}\ \lp\cfgnt{e}_i\ \cfgnt{e}_{i+1}\ ...\rp \rightarrow \cfgnt{k}\rp \rp  \rightarrow \\\\
    \lp \cfgnt{L}\ \cfgnt{R}\ \phi\ \eta\ \cfgnt{e}_i\ \lp \cfgt{begin}\ \cfgt{*}\ \lp\cfgnt{e}_{i+1}\ ...\rp \rightarrow \cfgnt{k}\rp \rp 
   }
\and
   \inferrule[Begin (argN eval)]{}{
    \lp \cfgnt{L}\ \cfgnt{R}\ \phi\ \eta\ \cfgnt{v}\ \lp \cfgt{begin}\ \cfgt{*}\ \lp\cfgnt{e}_{n}\rp \rightarrow \cfgnt{k}\rp \rp  \rightarrow \\\\
    \lp \cfgnt{L}\ \cfgnt{R}\ \phi\ \eta\ \cfgnt{e}_n\ \lp \cfgt{begin}\ \cfgt{*}\ \lp\rp \rightarrow \cfgnt{k}\rp \rp 
   }
\and
   \inferrule[Begin]{}{
    \lp \cfgnt{L}\ \cfgnt{R}\ \phi\ \eta\ \cfgnt{v}\ \lp \cfgt{begin}\ \cfgt{*}\ \lp\rp \rightarrow \cfgnt{k}\rp \rp  \rightarrow \\\\
    \lp \cfgnt{L}\ \cfgnt{R}\ \phi\ \eta\ \cfgnt{v}\ \cfgnt{k}\rp 
   }	
\and
	\inferrule[NULL]{}{
      \lp \cfgnt{L}\ \cfgnt{R}\ \phi\ \eta\ \cfgt{null}\ \cfgnt{k}\rp \rightarrow \\\\ 
      \lp \cfgnt{L}\ \cfgnt{R}\ \phi\ \eta\ \cfgnt{r}_\mathit{null}\ \cfgnt{k}\rp 
	}
\and
   \inferrule[Pop]{}{
    \lp \cfgnt{L}\ \cfgnt{R}\ \phi\ \eta\ \cfgnt{v}\ \lp \cfgt{pop}\ \eta_0\ \cfgnt{k}\rp \rp  \rightarrow \\\\
    \lp \cfgnt{L}\ \cfgnt{R}\ \phi\ \eta_0\  \cfgnt{v}\ \cfgnt{k}\rp 
   }
\end{mathpar}
\end{center}
\caption{Javalite rewrite rules that are common to generalized symbolic execution and precise heap summaries.}
\label{fig:javalite-common}
\end{figure*}


The surface syntax and the machine syntax for the Javalite syntactic
machine are in the main paper. \figref{fig:javalite-common} defines
the rewrite relations for the portion of the Javalite language
semantics that are common to both generalized symbolic execution and
the symbolic heap algorithm. The relation $\rcom$ includes all of these
rules. Excepting \textrm{N{\footnotesize EW}}, the rules do not update
the heap, and are largely concerned with argument evaluation in an
expected way. It is assumed that only type safe programs are input to
the machine so there is no type checking or error conditions. The
machine simply halts if no rewrite is enabled.

%% Includes its section definition
\section{Initialization of Symbolic References}

In this section we present the Javalite rewrite rules for the concrete
as well as summary initialization of symbolic references. The
initialization rules are defined on the bi-partite graph consisting of
references and locations. The lazy initialization of symbolic
references consists of three key points of non-determinism where each
symbolic reference can be initialized non-deterministically to null, a
new instance of the symbolic reference, or aliases to symbolic
references of the same type previously initialized. The initialization
in GSE consists of creating branches in the execution tree for all the
non-deterministic choices. On the other hand, the heap summarization
approach creates a single branch that contains the summarization for
all the initialization in a single bi-partitate graph.

\begin{figure*}[t]
\begin{center}
\mprset{flushleft}
\begin{mathpar}
	\inferrule[Initialize (null)]{
	  \Lambda = \mathbb{UN}\lp \cfgnt{L}, \cfgnt{R}, \cfgnt{r}, \cfgnt{f}\rp \\
      \Lambda \neq \emptyset\\\\
      \cfgnt{r}^\prime = \mathrm{fresh}_r\lp \rp\\ 
      \theta_\mathit{null} = \{ \lp \cfgt{true}\ l_\mathit{null}\rp \} \\\\
      l_x = \mathrm{min}_l\lp \Lambda\rp \\\\
      \phi_g^\prime = \lp\phi_g \wedge \cfgnt{r}^\prime = \cfgnt{r}_\mathit{null}\rp
    }{
      \lp \cfgnt{L}\ \cfgnt{R}\ \phi_g\ \cfgnt{r}\ \cfgnt{f}\ \cfgnt{C}\rp  \rightarrow_I 
      \lp \cfgnt{L}[\cfgnt{r}^\prime \mapsto \theta_\mathit{null}]\ \cfgnt{R}[ \lp l_x,\cfgnt{f}\rp  \mapsto \cfgnt{r}^\prime]\ \phi_g^\prime\ \cfgnt{r}\ \cfgnt{f}\ \cfgnt{C}\rp 
	}
\and
	\inferrule[Initialize (new)]{
	  \Lambda = \mathbb{UN}\lp \cfgnt{L}, \cfgnt{R}, \cfgnt{r}, \cfgnt{f}\rp \\
      \Lambda \neq \emptyset\\
      \lp\phi_x\ \cfgnt{l}_x\rp = \mathrm{min}_l\lp \Lambda\rp\\\\
      \cfgnt{r}_f = \mathrm{init}_r\lp \rp\\
      l_f = \mathrm{fresh}_l\lp \cfgnt{C}\rp \\\\
      \rho = \{ \lp\cfgnt{r}_a\ l_a\rp \mid \mathrm{isInit}\lp \cfgnt{r}_a\rp  \wedge \cfgnt{r}_a = \mathrm{min}_r\lp \cfgnt{R}^{-1}[l_a]\rp \wedge \mathrm{type}\lp l_a\rp  = \cfgnt{C} \}\\\\
      \theta_\mathit{new} = \{\lp \cfgt{true}\ l_f\rp \} \\\\
      \cfgnt{R}^\prime = \cfgnt{R}[\forall \cfgnt{f} \in \mathit{fields}\lp \mathrm{C}\rp \ \lp \lp l_f\ \cfgnt{f}\rp  \mapsto \cfgnt{r}_\mathit{un} \rp ] \\\\
      \phi_g^\prime = \lp\phi_g \wedge \cfgnt{r}_f \neq \cfgnt{r}_\mathit{null} \wedge \lp \wedge_{\lp\cfgnt{r}_a\ l_a\rp \in \rho} \cfgnt{r}_f \ne \cfgnt{r}_a\rp\rp
    }{
      \lp \cfgnt{L}\ \cfgnt{R}\ \phi_g\ \cfgnt{r}\ \cfgnt{f}\ \cfgnt{C}\rp  \rightarrow_I 
      \lp \cfgnt{L}[\cfgnt{r}_f \mapsto \theta_\mathit{new}]\ \cfgnt{R}^\prime[ \lp l_x,\cfgnt{f}\rp  \mapsto \cfgnt{r}_f ]\ \phi_g^\prime\ \cfgnt{r}\ \cfgnt{f}\ \cfgnt{C}\rp 
	}
\and
	\inferrule[Initialize (alias)]{
  	  \Lambda = \mathbb{UN}\lp \cfgnt{L}, \cfgnt{R}, \cfgnt{r}, \cfgnt{f}\rp \\
      \Lambda \neq \emptyset\\
      \lp\phi_x\ \cfgnt{l}_x\rp = \mathrm{min}_l\lp \Lambda\rp\\\\
      \cfgnt{r}^\prime = \mathrm{fresh}_r\lp \rp\\\\
      \rho = \{ \lp\cfgnt{r}_a\ l_a\rp \mid \mathrm{isInit}\lp \cfgnt{r}_a\rp  \wedge \cfgnt{r}_a = \mathrm{min}_r\lp \cfgnt{R}^{-1}[l_a]\rp \wedge \mathrm{type}\lp l_a\rp  = \cfgnt{C} \}\\\\
      \lp\cfgnt{r}_a\ l_a\rp \in \rho \\
      \theta_\mathit{alias} = \{ \lp \cfgt{true}\ l_a\rp\}\\\\
      \phi^\prime_g = \lp\phi_g \wedge \cfgnt{r}^\prime \neq \cfgnt{r}_\mathit{null} \wedge \cfgnt{r}^\prime = \cfgnt{r}_a \wedge \lp \wedge_{\lp \cfgnt{r}^{\prime}_a\ l_a\rp  \in \rho\ \lp \cfgnt{r}^{\prime}_a \neq \cfgnt{r}_a\rp } \cfgnt{r}^\prime \neq \cfgnt{r}^{\prime}_a \rp\rp
    }{
      \lp \cfgnt{L}\ \cfgnt{R}\ \phi_g\ \cfgnt{r}\ \cfgnt{f}\ \cfgnt{C}\rp  \rightarrow_I 
      \lp \cfgnt{L}[\cfgnt{r}^\prime \mapsto \theta_\mathit{alias}]\ \cfgnt{R}[ \lp l_x,\cfgnt{f}\rp  \mapsto \cfgnt{r}^\prime ]\ \phi_g^\prime\ \cfgnt{r}\ \cfgnt{f}\ \cfgnt{C}\rp 
	}
\and
	\inferrule[Initialize (end)]{
	  \Lambda = \mathbb{UN}\lp \cfgnt{L}, \cfgnt{R}, \cfgnt{r}, \cfgnt{f}\rp \\
      \Lambda = \emptyset
    }{
      \lp \cfgnt{L}\ \cfgnt{R}\ \phi_g\ \cfgnt{r}\ \cfgnt{f}\ \cfgnt{C}\rp  \rightarrow_I 
      \lp \cfgnt{L}\ \cfgnt{R}\ \phi_g\ \cfgnt{r}\ \cfgnt{f}\ \cfgnt{C}\rp 
	}
\end{mathpar}
\end{center}
\caption{The initialization machine, $s ::= \lp\cfgnt{L}\ \cfgnt{R}\ \phi_g\ \cfgnt{r}\ \cfgnt{f}\rp$, with $s \rightarrow_I^* s^\prime$ indicating stepping the machine until the state does not change.}
\label{fig:lazyInit}
\end{figure*}

\begin{figure*}[t]
\begin{center}
\mprset{flushleft}
\begin{mathpar}
	\inferrule[Summarize]{
	\Lambda = \mathbb{UN}\lp \cfgnt{L}, \cfgnt{R}, \cfgnt{r}, \cfgnt{f}\rp \\
      \Lambda \neq \emptyset \\
      \lp\phi_x\ \cfgnt{l}_x\rp = \mathrm{min}_l\lp \Lambda\rp\\
      \cfgnt{r}_f = \mathrm{init}_r\lp \rp \\
      l_f  = \mathrm{fresh}_l\lp \mathrm{C}\rp\\\\
      \rho = \{ \lp \cfgnt{r}_a\ \phi_a\ l_a\rp  \mid \mathrm{isInit}\lp \cfgnt{r}_a\rp  \wedge\cfgnt{r}_a = \mathrm{min}_r\lp \cfgnt{R}^{-1}[l_a]\rp \wedge \lp \phi_a\ l_a\rp  \in \cfgnt{L}\lp \cfgnt{r}_a\rp \wedge \mathrm{type}\lp l_a\rp  = \mathrm{C} \} \\\\
      \theta_\mathit{null} = \{ \lp \phi\ l_\mathit{null}\rp  \mid \phi = \lp \phi_x \wedge \cfgnt{r}_f = \cfgnt{r}_\mathit{null} \rp  \} \\\\
      \theta_\mathit{new} = \{\lp \phi\ l_f\rp  \mid \phi = \lp \phi_x \wedge \cfgnt{r}_f \neq \cfgnt{r}_\mathit{null} \wedge \lp \wedge_{\lp \cfgnt{r}^\prime_a,\ \phi^\prime_a,\ l^\prime_a\rp  \in \rho} \cfgnt{r}_f \ne \cfgnt{r}^\prime_a\rp \rp \}\\\\
      \theta_\mathit{alias} = \{ \lp \phi\ l_a\rp  \mid \exists\cfgnt{r}_a\ \lp\exists \phi_a\ \lp\lp\cfgnt{r}_a\ \phi_a\ l_a\rp  \in \rho \wedge \phi = \lp \phi_x \wedge \phi_a \wedge \cfgnt{r}_f \neq \cfgnt{r}_\mathit{null} \wedge \cfgnt{r}_f = \cfgnt{r}_a \wedge \lp \wedge_{\lp \cfgnt{r}^{\prime}_a\ \phi^{\prime}_a\ l^{\prime}_a\rp  \in \rho\ \lp \cfgnt{r}^\prime_a \neq \cfgnt{r}_a\rp } \cfgnt{r}_f \neq \cfgnt{r}^{\prime}_a \rp \rp \rp \rp \} \\\\
      \theta_\mathit{orig} = \{\lp\phi\ \cfgnt{l}_\mathit{orig}\rp \mid \exists \phi_\mathit{orig} \lp \lp\phi_\mathit{orig}\ \cfgnt{l}_\mathit{orig}\rp \in \cfgnt{L}\lp\cfgnt{R}\lp\cfgnt{l}_x,\cfgnt{f}\rp\rp \wedge \phi = \lp\neg\phi_x \wedge \phi_\mathit{orig}\rp\}\\\\ 
      \theta = \theta_\mathit{alias} \cup \theta_\mathit{new} \cup \theta_\mathit{null} \cup \theta_\mathit{old} \\
\cfgnt{R}^\prime = \cfgnt{R}[\forall \cfgnt{f} \in \mathit{fields}\lp \mathrm{C}\rp \ \lp \lp l_f\ \cfgnt{f}\rp  \mapsto \cfgnt{r}_\mathit{un} \rp ]
    }{
      \lp \cfgnt{L}\ \cfgnt{R}\ \cfgnt{r}\ \cfgnt{f}\ \cfgnt{C}\rp \rightarrow_S 
      \lp \cfgnt{L}[\cfgnt{r}_f \mapsto \theta]\ \cfgnt{R}^{\prime}[ \lp l_x,\cfgnt{f}\rp  \mapsto \cfgnt{r}_f ]\ \cfgnt{r}\ \cfgnt{f}\ \cfgnt{C}\rp
	}
\and
	\inferrule[Summarize (end)]{
	  \Lambda = \{ l \mid \exists \phi\ \lp \lp \phi\ l\rp  \in \cfgnt{L}\lp \cfgnt{r}\rp  \wedge  \cfgnt{R}\lp l,\cfgnt{f}\rp  = \bot\ \rp\}\\
      \Lambda = \emptyset
    }{
      \lp \cfgnt{L}\ \cfgnt{R}\ \cfgnt{r}\ \cfgnt{f}\ \cfgnt{C}\rp  \rightarrow_S
      \lp \cfgnt{L}\ \cfgnt{R}\ \cfgnt{r}\ \cfgnt{f}\ \cfgnt{C}\rp 
	}
\end{mathpar}
\end{center}
\caption{The summary machine, $s ::= \lp\cfgnt{L}\ \cfgnt{R}\ \cfgnt{r}\ \cfgnt{f}\ \cfgnt{C}\rp$, with $s\rightarrow_S^*s^\prime$ indicating stepping the machine until the state does not change.}
\label{fig:symInit}
\end{figure*}



The initialization rules are invoked when an uninitialized field in a
symbolic reference is accessed. The function $\mathbb{UN}(\cfgnt{L},
\cfgnt{R}, \cfgnt{r}, \cfgnt{f}) = \{\cfgnt{l}\ ...\}$ returns
constraint-location pairs in which the field $\cfgnt{f}$ is
uninitialized:
\[
\begin{array}{rcl}
\mathbb{UN}(\cfgnt{L}, \cfgnt{R}, \cfgnt{r}, \cfgnt{f}) & = &\{ \lp\phi\ \cfgnt{l}\rp \mid \lp \phi\ \cfgnt{l}\rp  \in \cfgnt{L}\lp \cfgnt{r}\rp  \wedge \\
& & \ \ \ \ \exists \phi^\prime \lp \lp \phi^\prime\ \cfgnt{l}_\mathit{un}\rp  \in \cfgnt{L}\lp \cfgnt{R}\lp l,\cfgnt{f}\rp\rp \wedge \\
& & \ \ \ \ \ \ \ \ \mathbb{S}\lp \phi \wedge \phi^\prime \rp\rp\}\\
\end{array}
\]
where $\mathbb{S}(\phi)$ returns true if $\phi$ is
satisfiable. Intutively, for the reference, $\cfgnt{r}$, it constructs
the set, $\theta$, that contains all contraint-location pairs that
point to the field $\cfgnt{f}$ and $\cfgnt{f}$ points to
$\cfgnt{l}_\mathit{un}$. The cardinality of the set, $\theta$ is never
greater than one in GSE and the constraint is always satisfiable
because all constraints are constant. This property is relaxed in GSE
with heap summaries.

The rules in~\figref{fig:lazyInit} present the rewrite rules for the
concrete initialization of symbolic heap objects.  These rules are
invoked until a fix pointed is reached. 

The initialize (null) rewrite rule in~\figref{fig:lazyInit} first
checks that the field, $\cfgnt{r}$ is uninitialized. The fresh method
returns a new input heap reference from the partition 


\section{Accessing and Writing to Field References}

\section{Equality and InEquality of References}

\begin{figure*}[t]
\begin{center}
\mprset{flushleft}
\begin{mathpar}
	\inferrule[Field Access]{
      \{\lp\phi\ l\rp\} = \cfgnt{L}\lp\cfgnt{r}\rp\\
      l \neq \cfgnt{l}_\mathit{null}\\
      \cfgnt{C} = \mathrm{type}\lp\cfgnt{l},\cfgnt{f}\rp\\\\
      \lp \cfgnt{L}\ \cfgnt{R}\ \cfgnt{r}\ \cfgnt{f}\ \cfgnt{C}\rp \rinit^*
      \lp \cfgnt{L}^\prime\ \cfgnt{R}^\prime\ \cfgnt{r}\ \cfgnt{f}\  \cfgnt{C}\rp \\\\ 
      \{\lp\phi^\prime\ l^\prime\rp\} = \cfgnt{L}^\prime\lp\cfgnt{R}^\prime\lp l,\cfgnt{f}\rp\rp \\
      \cfgnt{r}^\prime = \mathrm{stack}_r\lp\rp \\
    }{
      \lp \cfgnt{L}\ \cfgnt{R}\ \phi_g\ \eta\ \cfgnt{r}\ \lp \cfgt{*}\ \cfgt{\$}\ \cfgnt{f} \rightarrow \cfgnt{k}\rp \rp  \rightarrow_\ell \\\\
      \lp \cfgnt{L}^\prime[\cfgnt{r}^\prime \mapsto \lp\phi^\prime\ l^\prime\rp]\ \cfgnt{R}^\prime\ \phi_g^\prime\ \eta\ \cfgnt{r}^\prime\ \cfgnt{k}\rp 
	}
\and
	\inferrule[Field Write]{
      \cfgnt{r}_x = \eta\lp \cfgnt{x}\rp\\ 
      \theta = \{\lp\phi\ l\rp\} = \cfgnt{L}\lp\cfgnt{r}_x\rp \\\\
      l \neq \cfgnt{l}_\mathit{null}\\
      \cfgnt{r}^\prime = \mathrm{fresh}_r\lp\rp\\
    }{
      \lp \cfgnt{L}\ \cfgnt{R}\ \phi_g\ \eta\ \cfgnt{r}\ \lp \cfgnt{x}\ \cfgt{\$}\ \cfgnt{f}\ \cfgt{:=}\ \cfgt{*}\ \rightarrow\ \cfgnt{k}\rp \rp  \rightarrow_\ell \\\\
      \lp \cfgnt{L}[\cfgnt{r}^\prime \mapsto \theta]\ \cfgnt{R}[\lp l\ \cfgnt{f}\rp  \mapsto \cfgnt{r}^\prime]\ \phi_g\ \eta\ \cfgnt{r}\ \cfgnt{k}\rp 
	}
\and
  \inferrule[Equals (reference-true)]{
    \cfgnt{L}\lp \cfgnt{r}_0\rp = \cfgnt{L}\lp \cfgnt{r}_1\rp\\
    \phi^\prime = \lp\phi \wedge r_0 = r_1\rp
    }{
    \lp \cfgnt{L}\ \cfgnt{R}\ \phi\ \eta\ \cfgnt{r}_0\ \lp \cfgnt{r}_1\ \cfgt{=}\ \cfgt{*} \rightarrow \cfgnt{k}\rp \rp  \rightarrow_\ell \\\\
    \lp \cfgnt{L}\ \cfgnt{R}\ \phi^\prime\ \eta\ \cfgt{true}\ \cfgnt{k}\rp 
    }
\and
    \inferrule[Equals (reference-false)]{
    \cfgnt{L}\lp \cfgnt{r}_0\rp \neq \cfgnt{L}\lp \cfgnt{r}_1\rp\\
    \phi^\prime = \lp\phi \wedge r_0 \neq r_1\rp
   }{
    \lp \cfgnt{L}\ \cfgnt{R}\ \phi\ \eta\ \cfgnt{r}_0\ \lp \cfgnt{r}_1\ \cfgt{=}\ \cfgt{*} \rightarrow \cfgnt{k}\rp \rp  \rightarrow_\ell \\\\
    \lp \cfgnt{L}\ \cfgnt{R}\ \phi^\prime\ \eta\ \cfgt{false}\ \cfgnt{k}\rp 
    }	
\end{mathpar}
\end{center}
\caption{GSE with lazy initialization indicated by $\rgse = \rightarrow_\ell \cup \rcom$.}
\label{fig:lazy}
\end{figure*}



	

\newsavebox{\boxPFAFW}
\savebox{\boxPFAFW}{
%\begin{figure}[t]
%\begin{center}
\mprset{flushleft}
\begin{mathpar}
	\inferrule[Field Access]{
      \exists \lp \phi\ l\rp \in \cfgnt{L}\lp \cfgnt{r}\rp\ \lp l \neq l_{\mathit{null}} \wedge \mathbb{S}\lp \phi \wedge \phi_g\rp \rp \\\\
      \theta = \{ \phi \mid \lp \phi\ l_\mathit{null} \rp \wedge \mathbb{S}\lp \phi \wedge \phi_g\rp \} \\\\
      \phi_g^\prime = \phi_g \wedge (\wedge_{\phi \in \theta} \neg \phi) \\\\
      \{\cfgnt{C}\} = \{\cfgnt{C} \mid \exists \lp \phi\ l\rp  \in \cfgnt{L}\lp \cfgnt{r}\rp\ \lp\cfgnt{C} = \mathrm{type}\lp \cfgnt{l},\cfgnt{f}\rp\rp\} \\\\
      \lp \cfgnt{L}\ \cfgnt{R}\ \cfgnt{r}\ \cfgnt{f}\ \cfgnt{C}\rp \rsum^* \lp \cfgnt{L}^\prime\ \cfgnt{R}^\prime\ \cfgnt{r}\ \cfgnt{f}\ \cfgnt{C}\rp \\
      \cfgnt{r}^\prime = \mathrm{stack}_r\lp \rp
    }{
      \lp \cfgnt{L}\ \cfgnt{R}\ \phi_g\ \eta\ \cfgnt{r}\ \lp \cfgt{*}\ \cfgt{\$}\ \cfgnt{f} \rightarrow \cfgnt{k}\rp \rp  \rightarrow_\mathit{A}
      \lp \cfgnt{L}^\prime[\cfgnt{r}^\prime \mapsto \mathbb{VS}\lp \cfgnt{L}^\prime,\cfgnt{R}^\prime,\cfgnt{r},\cfgnt{f},\phi_g^\prime\rp ]\ \cfgnt{R}^\prime\ \phi_g^\prime\ \eta\ \cfgnt{r}^\prime\ \cfgnt{k}\rp 
	}
\and
	\inferrule[Field Access (NULL)]{
      \exists \lp \phi\ l\rp \in \cfgnt{L}\lp \cfgnt{r}\rp\ \lp l = l_{\mathit{null}} \wedge \mathbb{S}\lp \phi \wedge \phi_g\rp \rp
    }{
      \lp \cfgnt{L}\ \cfgnt{R}\ \phi_g\ \eta\ \cfgnt{r}\ \lp \cfgt{*}\ \cfgt{\$}\ \cfgnt{f} \rightarrow \cfgnt{k}\rp \rp  \rightarrow_\mathit{A}
      \lp \cfgnt{L}\ \cfgnt{R}\ \phi_g\ \eta\ \cfgt{error}\ \cfgt{end} \rp
	}
\and
	\inferrule[Field Write]{
      \cfgnt{r}_x = \eta\lp\cfgnt{x}\rp \\
      \exists \lp \phi\ l\rp \in \cfgnt{L}\lp \cfgnt{r}_x\rp\ \lp l \neq l_{\mathit{null}} \wedge \mathbb{S}\lp \phi \wedge \phi_g\rp \rp \\\\
      \theta = \{ \phi \mid \lp \phi\ l_\mathit{null} \rp \wedge \mathbb{S}\lp \phi \wedge \phi_g\rp \} \\\\
      \phi_g^\prime = \phi_g \wedge (\wedge_{\phi \in \theta} \neg \phi) \\\\
      \Psi_x =\{\lp \phi\ l\ \cfgnt{r}_\mathit{cur} \rp  \mid \lp \phi\ \cfgnt{l}\rp  \in \cfgnt{L}\lp \cfgnt{r}_x\rp  \wedge \cfgnt{r}_\mathit{cur} = \cfgnt{R}\lp l,\cfgnt{f}\rp  \}\\\\
      X = \{ \lp l\ \theta \rp  \mid \exists \phi\ \lp \lp \phi\ l\ \cfgnt{r}_\mathit{cur} \rp \in \Psi_x \wedge \theta = \mathbb{ST}\lp \cfgnt{L},\cfgnt{r},\phi,\phi_g^\prime\rp  \cup \mathbb{ST}\lp \cfgnt{L},\cfgnt{r}_\mathit{cur},\neg\phi,\phi_g^\prime\rp \rp  \}\\\\
      \cfgnt{R}^{\prime} = \cfgnt{R}[\forall \lp l\ \theta \rp  \in X\ \lp \lp l\ \cfgnt{f}\rp  \mapsto \mathrm{fresh}_r\lp \rp \rp ]\\\\
      \cfgnt{L}^{\prime} = \cfgnt{L}[\forall \lp l\ \theta \rp  \in X\ \lp \exists \cfgnt{r}_\mathit{targ}\ \lp \cfgnt{r}_\mathit{targ} = \cfgnt{R}^\prime\lp l,\cfgnt{f}\rp \wedge \lp\cfgnt{r}_\mathit{targ} \mapsto \theta\rp  \rp \rp ]
    }{
      \lp \cfgnt{L}\ \cfgnt{R}\ \phi_g\ \eta\ \cfgnt{r}\ \lp \cfgnt{x}\ \cfgt{\$}\ \cfgnt{f}\ \cfgt{:=}\ \cfgt{*}\ \rightarrow\ \cfgnt{k}\rp \rp  \rightarrow_\mathit{FW}
      \lp \cfgnt{L}^{\prime}\ \cfgnt{R}^{\prime}\ \phi_g^\prime\ \eta\ \cfgnt{r}\ \cfgnt{k}\rp 
	}	
\and
	\inferrule[Field Write (NULL)]{
      \cfgnt{r}_x = \eta\lp \cfgnt{x}\rp \\
      \exists \lp \phi\ l\rp \in \cfgnt{L}\lp \cfgnt{r}_x\rp\ \lp l \neq l_{\mathit{null}} \wedge \mathbb{S}\lp \phi \wedge \phi_g\rp \rp
    }{
      \lp \cfgnt{L}\ \cfgnt{R}\ \phi_g\ \eta\ \cfgnt{r}\ \lp \cfgnt{x}\ \cfgt{\$}\ \cfgnt{f}\ \cfgt{:=}\ \cfgt{*}\ \rightarrow\ \cfgnt{k}\rp \rp  \rightarrow_\mathit{FW}
      \lp \cfgnt{L}\ \cfgnt{R}\ \phi_g\ \eta\ \cfgt{error}\ \cfgt{end}\rp
	}	
\end{mathpar}}
%\end{center}
%\caption{Precise symbolic heap summaries from symbolic execution indicated by $\rsym = \rightarrow_\mathit{FA} \cup \rightarrow_\mathit{FW} \cup \rightarrow_\mathit{EQ} \cup \rcom$.}
%\label{fig:symfield}
%\end{figure}



\begin{figure*}[t]
\begin{center}
\mprset{flushleft}
\begin{mathpar}
	\inferrule[Summarize]{
	\Lambda = \mathbb{UN}\lp \cfgnt{L}, \cfgnt{R}, \cfgnt{r}, \cfgnt{f}\rp \\
      \Lambda \neq \emptyset \\
      \lp\phi_x\ \cfgnt{l}_x\rp = \mathrm{min}_l\lp \Lambda\rp\\
      \cfgnt{r}_f = \mathrm{init}_r\lp \rp \\
      l_f  = \mathrm{fresh}_l\lp \mathrm{C}\rp\\\\
      \rho = \{ \lp \cfgnt{r}_a\ \phi_a\ l_a\rp  \mid \mathrm{isInit}\lp \cfgnt{r}_a\rp  \wedge\cfgnt{r}_a = \mathrm{min}_r\lp \cfgnt{R}^{-1}[l_a]\rp \wedge \lp \phi_a\ l_a\rp  \in \cfgnt{L}\lp \cfgnt{r}_a\rp \wedge \mathrm{type}\lp l_a\rp  = \mathrm{C} \} \\\\
      \theta_\mathit{null} = \{ \lp \phi\ l_\mathit{null}\rp  \mid \phi = \lp \phi_x \wedge \cfgnt{r}_f = \cfgnt{r}_\mathit{null} \rp  \} \\\\
      \theta_\mathit{new} = \{\lp \phi\ l_f\rp  \mid \phi = \lp \phi_x \wedge \cfgnt{r}_f \neq \cfgnt{r}_\mathit{null} \wedge \lp \wedge_{\lp \cfgnt{r}^\prime_a,\ \phi^\prime_a,\ l^\prime_a\rp  \in \rho} \cfgnt{r}_f \ne \cfgnt{r}^\prime_a\rp \rp \}\\\\
      \theta_\mathit{alias} = \{ \lp \phi\ l_a\rp  \mid \exists\cfgnt{r}_a\ \lp\exists \phi_a\ \lp\lp\cfgnt{r}_a\ \phi_a\ l_a\rp  \in \rho \wedge \phi = \lp \phi_x \wedge \phi_a \wedge \cfgnt{r}_f \neq \cfgnt{r}_\mathit{null} \wedge \cfgnt{r}_f = \cfgnt{r}_a \wedge \lp \wedge_{\lp \cfgnt{r}^{\prime}_a\ \phi^{\prime}_a\ l^{\prime}_a\rp  \in \rho\ \lp \cfgnt{r}^\prime_a \neq \cfgnt{r}_a\rp } \cfgnt{r}_f \neq \cfgnt{r}^{\prime}_a \rp \rp \rp \rp \} \\\\
      \theta_\mathit{orig} = \{\lp\phi\ \cfgnt{l}_\mathit{orig}\rp \mid \exists \phi_\mathit{orig} \lp \lp\phi_\mathit{orig}\ \cfgnt{l}_\mathit{orig}\rp \in \cfgnt{L}\lp\cfgnt{R}\lp\cfgnt{l}_x,\cfgnt{f}\rp\rp \wedge \phi = \lp\neg\phi_x \wedge \phi_\mathit{orig}\rp\}\\\\ 
      \theta = \theta_\mathit{alias} \cup \theta_\mathit{new} \cup \theta_\mathit{null} \cup \theta_\mathit{old} \\
\cfgnt{R}^\prime = \cfgnt{R}[\forall \cfgnt{f} \in \mathit{fields}\lp \mathrm{C}\rp \ \lp \lp l_f\ \cfgnt{f}\rp  \mapsto \cfgnt{r}_\mathit{un} \rp ]
    }{
      \lp \cfgnt{L}\ \cfgnt{R}\ \cfgnt{r}\ \cfgnt{f}\ \cfgnt{C}\rp \rightarrow_S 
      \lp \cfgnt{L}[\cfgnt{r}_f \mapsto \theta]\ \cfgnt{R}^{\prime}[ \lp l_x,\cfgnt{f}\rp  \mapsto \cfgnt{r}_f ]\ \cfgnt{r}\ \cfgnt{f}\ \cfgnt{C}\rp
	}
\and
	\inferrule[Summarize (end)]{
	  \Lambda = \{ l \mid \exists \phi\ \lp \lp \phi\ l\rp  \in \cfgnt{L}\lp \cfgnt{r}\rp  \wedge  \cfgnt{R}\lp l,\cfgnt{f}\rp  = \bot\ \rp\}\\
      \Lambda = \emptyset
    }{
      \lp \cfgnt{L}\ \cfgnt{R}\ \cfgnt{r}\ \cfgnt{f}\ \cfgnt{C}\rp  \rightarrow_S
      \lp \cfgnt{L}\ \cfgnt{R}\ \cfgnt{r}\ \cfgnt{f}\ \cfgnt{C}\rp 
	}
\end{mathpar}
\end{center}
\caption{The summary machine, $s ::= \lp\cfgnt{L}\ \cfgnt{R}\ \cfgnt{r}\ \cfgnt{f}\ \cfgnt{C}\rp$, with $s\rightarrow_S^*s^\prime$ indicating stepping the machine until the state does not change.}
\label{fig:symInit}
\end{figure*}


\section{Symbolic Heap Algorithm}


%\begin{figure*}[t]
%\begin{center}
%\setlength{\tabcolsep}{60pt}
%\hspace*{-35pt}
%\begin{tabular}[c]{cc}
%\scalebox{1.0}{\usebox{\boxPFAFW}} & 
%%\scalebox{0.91}{\input{faYHeap.pdf_t}} &
%\scalebox{0.91}{\input{fwXHeap.pdf_t}} \\ \\
%(a) & (b)
%\end{tabular}
%\end{center}
%\caption{Field read and write relations with an example heap. (a) Field-access, $\rsym^\mathit{A}$, and field-write, $\rsym^\mathit{W}$, rewrite rules for the $\rsym$ relation. (b) The final heap after $\lp\cfgt{this}\  \cfgt{\$}\ \cfgnt{x}\ \cfgt{:=}\ \lp\cfgt{this}\  \cfgt{\$}\ \cfgnt{y}\rp\rp$.}
%\label{fig:fHeap}
%\end{figure*}

%\begin{figure*}
%\begin{tabular}[c]{l}
%\scalebox{1.0}{\usebox{\boxPI}} \\
%\end{tabular}
%\caption{The summary machine, $s ::= \lp\cfgnt{L}\ \cfgnt{R}\ \cfgnt{r}\ \cfgnt{f}\ \cfgnt{C}\rp$, with $s\rsum^*s^\prime =  s \rsum \cdots \rsum s^\prime \rsum s^\prime$.}
%\label{fig:symInit}
%\end{figure*}

The rewrite rules for field access or field write when a null reference is possible leads to an error state.
$$
\begin{array}{c}
	\inferrule[Field Access (NULL)]{
      \exists \lp \phi\ l\rp \in \cfgnt{L}\lp \cfgnt{r}\rp\ \lp l = l_{\mathit{null}} \wedge \mathbb{S}\lp \phi \wedge \phi_g\rp \rp
    }{
      \lp \cfgnt{L}\ \cfgnt{R}\ \phi_g\ \eta\ \cfgnt{r}\ \lp \cfgt{*}\ \cfgt{\$}\ \cfgnt{f} \rightarrow \cfgnt{k}\rp \rp  \rsym^\mathit{A^\prime}
      \lp \cfgnt{L}\ \cfgnt{R}\ \phi_g\ \eta\ \cfgt{error}\ \cfgt{end} \rp
	} \\
\\
	\inferrule[Field Write (NULL)]{
      \cfgnt{r}_x = \eta\lp \cfgnt{x}\rp \\
      \exists \lp \phi\ l\rp \in \cfgnt{L}\lp \cfgnt{r}_x\rp\ \lp l \neq l_{\mathit{null}} \wedge \mathbb{S}\lp \phi \wedge \phi_g\rp \rp
    }{
      \lp \cfgnt{L}\ \cfgnt{R}\ \phi_g\ \eta\ \cfgnt{r}\ \lp \cfgnt{x}\ \cfgt{\$}\ \cfgnt{f}\ \cfgt{:=}\ \cfgt{*}\ \rightarrow\ \cfgnt{k}\rp \rp  \rsym^\mathit{F^\prime}
      \lp \cfgnt{L}\ \cfgnt{R}\ \phi_g\ \eta\ \cfgt{error}\ \cfgt{end}\rp
	}	
\end{array}
$$ The rules reflect a branch in the control flow for the symbolic
heap algorithm: one branch being the feasible null outcome (shown
here), and the other branch being the potential non-null outcome
(shown in the main paper). In the non-null case, the path constraint
is updated to restrict out all possible null instances.

%\subsection{Definitions}

\begin{definition}
\label{def:state}
The set of \textbf{states} $\mathcal{S}$ is defined as
\end{definition}

\begin{definition}
\label{def:initstate}
$\mathcal{S}_0$ is defined as the set of \textbf{initial states}. An initial state is a state meeting the following conditions:  The range of L has exactly three locations: $l_{null}$, $l_{un}$, and $l_0$, the function R is defined only for location $l_0$, and for any field $f$, $R(l_0,f)$ returns $r_{un}$. 
\end{definition}

\begin{definition}
The set of \textbf{references} $\mathcal{R}$ is defined as the set of natural numbers
 $$\mathcal{R} = \mathbb{N}$$
\end{definition}

The total number of references in a summary state and a lazy state that it represents are generally not the same. However, the number of references on the stack in either state is always the same. In order to make the distinction between different types of references, we partition the set of natural numbers using modular arithmetic.

\begin{definition}
The set of \textbf{stack references} $\mathcal{R}_t$ is defined as
 $$\mathcal{R}_t =\{i \in \mathbb{N} \mid ( i\ \bmod\ 3 ) = 0\}$$. 
\end{definition}

\begin{definition}
The set of \textbf{input heap references} $\mathcal{R}_h$ is defined as
 $$\mathcal{R}_h =\{i \in \mathbb{N} \mid ( i\ \bmod\ 3 ) = 1\}$$. 
\end{definition}

\begin{definition}
The set of \textbf{new heap references} $\mathcal{R}_f$ is defined as
 $$\mathcal{R}_n =\{i \in \mathbb{N} \mid ( i\ \bmod\ 3 ) = 2\}$$. 
\end{definition}

\begin{definition}
For a given function $f:A \mapsto B$, the \textbf{image} $f^\rightarrow$ and \textbf{preimage} $f^\leftarrow$ are defined as
\begin{align}
 f^\rightarrow &= \{ f(a) \mid a \in A\}\\
 f^\leftarrow &= \{ a \mid f(a) \in B \}
 \end{align}
 The bracket notation $ f^\rightarrow [C] $ is used to denote that the image is drawn from a specific subset:
 \begin{align}
 f^\rightarrow [C] &= \{ f(a) \mid a \in C\}\\
 f^\leftarrow [D] &= \{ a \mid f(a) \in D \}
 \end{align}
 Where $C \subset A$ and $D \subset B$
\end{definition}

\begin{definition}
A \textbf{state transition function} $\rightarrow_{\Phi}$ is a mapping $\rightarrow_{\Phi} : s \mapsto s$, which takes one machine state and transforms it into another machine state. Two important state transition functions are the \textbf{lazy state transition function} $\rightarrow_\ell$ and the \textbf{summary state transition function} $\rightarrow_s$.
\end{definition}

\begin{definition}
A \textbf{state sequence} is a sequence of states denoted as $\Pi_n = s_0,s_1,...,s_n$. A \textbf{feasible state sequence}, $\Pi_n^\phi = s_0,s_1,...,s_n$ is consistent with the transition: $\forall i\ (0 \leq i < n \Rightarrow s_i \rightarrow_{\Phi} s_{i+1})$, where $s_0\in \mathcal{S}_0$.
\end{definition}

%For 
\begin{comment}
\begin{definition}
A \textbf{feasible state sequence} is defined as a sequence of states resulting from repeated application of the state transition relation to some initial state $s_0\in \mathcal{S}_0$: $$\Pi_n = s_0,s_1,...,s_n$$ where the relation $s_i \rightarrow_{\Phi} s_{i+1}$ holds for all $i \in \{ i | 0 \leq i < n \}$
\end{definition}
\end{comment}

%For these definitions, we should bring in the notion of 
\begin{definition}
The set of \textbf{lazy states} $\mathcal{S}_\ell$ is defined as
\begin{align}
\mathcal{S}_\ell = \{s_n \mid \exists \Pi_n^\ell\ (\Pi_n^\ell = s_0, \ldots, s_n)\}
\end{align}
\end{definition}

\begin{definition}
The set of \textbf{summary states} $\mathcal{S}_s$ is defined as
\begin{align}
\mathcal{S}_s = \{s_n \mid \exists \Pi_n^s\ (\Pi_n^s = s_0, \ldots, s_n)\}
\end{align}
\end{definition}

\begin{definition}
The sets $\mathcal{FA}$, $\mathcal{FW}$, $\mathcal{RC}$, and  $\mathcal{NW}$ are defined as the sets of states having the forms $ \lp \cfgnt{L}\ \cfgnt{R}\ \phi_g\ \eta\ \cfgnt{r}\ \lp \cfgt{*}\ \cfgt{\$}\ \cfgnt{f} \rightarrow \cfgnt{k}\rp \rp$,  $\lp \cfgnt{L}\ \cfgnt{R}\ \phi_g\ \eta\ \cfgnt{r}\ \lp \cfgnt{x}\ \cfgt{\$}\ \cfgnt{f}\ \cfgt{:=}\ \cfgt{*}\ \rightarrow\ \cfgnt{k}\rp \rp$, $\lp \cfgnt{L}\ \cfgnt{R}\ \phi_g\ \eta\ \cfgnt{r}_0\ \lp \cfgnt{r}_1\; \cfgt{=}\; \cfgt{*} \rightarrow \cfgnt{k}\rp \rp$, and $\lp \cfgnt{L}\ \cfgnt{R}\ \phi_g\ \eta\ \lp \cfgt{new}\ \cfgnt{C}\rp \ \cfgnt{k}\rp$ respectively.
\end{definition}

\begin{definition}
\label{def:interstate}
\textbf{Intermediate states} are imaginary placeholder states used when reasoning about complex transition rules in terms of simpler sub-rules. For example, the transition $s_x \rightarrow_\phi s_y$ may be equivalent to a sequence of simpler transitions $s_x \rightarrow_\alpha s_a \rightarrow_\beta s_b \rightarrow_\gamma s_y$.  When reasoning about this equivalent transition sequence, it can be useful to discuss the notional intermediate states $s_a$ and $s_b$. However, it is important to remember that $s_a$ and $s_b$ are not technically involved in the transition $s_x \rightarrow_\phi s_y$, and indeed may not be part of any feasible state sequence under transition relation $\rightarrow_\phi$.
\end{definition}

\begin{definition}
Given a sequence of states $$\Pi_n = s_0,s_1,...,s_n$$ where $$s_i = ( \mu_i\ \cfgnt{L}_i\ \cfgnt{R}_i\ \phi_i\ \eta_i\ \cfgnt{e}_i\ \cfgnt{k}_i )$$ the \textbf{control flow sequence} of $\Pi_n$ is the defined as the sequence of tuples $$ \pi_n = \mathbb{CF}(\Pi_n) = (\eta_0\ \cfgnt{e}_0\ \cfgnt{k}_0),(\eta_1\ \cfgnt{e}_1\ \cfgnt{k}_1),...,(\eta_n\ \cfgnt{e}_n\ \cfgnt{k}_n)$$
\end{definition}

%\begin{definition}
%A \textbf{field access descriptor} $\gamma_i$ is a tuple 
%$$ \gamma_i = (\cfgnt{r}_i\ \phi_i\ \cfgnt{l}_i\ \cfgnt{f}_i)$$
%\end{definition}
%
%\begin{definition}
%An \textbf{access path} $\Gamma_n$ is a sequence of field access descriptors
%$ \Gamma_n = \gamma_0,\gamma_1,...,\gamma_n $.
%\end{definition}
%
%\begin{definition}
%For a given access path $\Gamma_n$ the \textbf{access path constraint} $\mathbb{PC}(\Gamma_n)$ is defined as
%$$\mathbb{PC}(\Gamma_n) =  \bigwedge \{\phi \mid \exists \gamma \in \Gamma_n \ (\gamma = (\cfgnt{r}\ \phi\ \cfgnt{e}\ \cfgnt{f}))\}$$  
%\end{definition}
%
%\begin{definition}
%For a given state $s= ( \cfgnt{L}_s\ \cfgnt{R}_s\ \phi_s\ \eta_s\ \cfgnt{e}_s\ \cfgnt{k}_s )$, a \textbf{valid access path} $\Gamma_n^s = \gamma_0,\gamma_1,...,\gamma_n$ satisfies the properties
%\begin{align*}
%\cfgnt{r}_0 &\in \mathit{refs}(\eta_s) \\
%\mathbb{S}(&\phi_s \wedge \mathbb{PC}(\Gamma_n^s)) \\
%\forall i & \in \mathbb{N}\ ( 0 \leq i < n \Leftrightarrow \gamma_i \in \Gamma_n^s\  \wedge\\
%&\ \ \ \  \gamma_{i+1} \in \Gamma_n^s\  \wedge\\
%&\ \ \ \  (\phi_i\ \cfgnt{l}_i)\in \cfgnt{L}_s(\cfgnt{r}_i)\ \wedge \\
%&\ \ \ \  \cfgnt{r}_{i+1} = \cfgnt{R}_s(\cfgnt{l}_i,\cfgnt{f}_i)\ \wedge\\
%&\ \ \ \ (\phi_{i+1}\ \cfgnt{l}_{i+1}) = \cfgnt{L}_s(\cfgnt{r}_{i+1}) )
%\end{align*}
%where $\gamma_i = (\cfgnt{r}_i\ \phi_i\ \cfgnt{l}_i\ \cfgnt{f}_i)$
%\end{definition}

\begin{definition}
\label{def:homomorphism}
A \textbf{homomorphism} $s_x \rightharpoonup_{h} s_y$, from state $s_x = ( \cfgnt{L}_x\ \cfgnt{R}_x\ \phi_x\ \eta_x\ \cfgnt{e}_x\ \cfgnt{k}_x )$ to state $s_y = ( \cfgnt{L}_y\ \cfgnt{R}_y\ \phi_y\ \eta_y\ \cfgnt{e}_y\ \cfgnt{k}_y )$, is defined as follows: 
$$
\begin{array}{l}
 s_x \rightharpoonup_{h} s_y \Leftrightarrow \\
\ \ \ \ \exists h: \mathcal{L} \mapsto \mathcal{L}\ (\forall \cfgnt{l}_\alpha \in \mathcal{L}\ (\forall \cfgnt{l}_\beta \in \mathcal{L}\ ( \forall f \in \mathcal{F}( \exists \phi_\alpha \in \Phi(\exists \phi_\beta \in \Phi( \\ 
\ \ \ \ \ \ \ \ \ \ \ \ (\phi_a\ \cfgnt{l}_\alpha) \in \cfgnt{L}_x(\cfgnt{R}_x (\cfgnt{l}_\beta,f )) \Rightarrow (\phi_b\ h(\cfgnt{l}_\alpha))\in \cfgnt{L}_y(\cfgnt{R}_y (h(\cfgnt{l}_\beta),f ))\ \\
\ \ \ \ \ \ \ \ \ \ \ \ \ \ \ \  )) ) ) ) )
\end{array}
$$
\end{definition}

\begin{definition}
\label{def:hc}
Given homomorphism $s_x \rightharpoonup_{h} s_y$, the \textbf{homomorphism constraint} $\mathbb{HC}(s_x \rightharpoonup_{h} s_y)$ is defined as:
\begin{align*}
\mathbb{HC}(s_x \rightharpoonup_{h} s_y) &= \\
 \bigwedge \{ \phi_b\ | \exists (\phi_a\ l) \in \cfgnt{L}_x^\rightarrow ( (\phi_b\ h(l)) \in \cfgnt{L}^\rightarrow)\} 
\end{align*}
\end{definition}

\begin{definition}
\label{representation}
The \textbf{representation relation} is defined as follows: given state $s_y = ( \cfgnt{L}_y\ \cfgnt{R}_y\ \phi_y\ \eta_y\ \cfgnt{e}_y\ \cfgnt{k}_y )$ and state $s_x = ( \cfgnt{L}_x\ \cfgnt{R}_x\ \phi_x\ \eta_x\ \cfgnt{e}_x\ \cfgnt{k}_x )$, $s_y \sqsubset s_x $ if and only if $\eta_y = \eta_x ,\ \cfgnt{e}_y = \cfgnt{e}_x ,\ \cfgnt{k}_y = \cfgnt{k}_x$, and there exists a homomorphism $\ s_y \rightharpoonup_{h} s_x $ such that 
\begin{equation}
\label{eqn:valid}
 \mathbb{S}( \phi_x \wedge \mathbb{HC}(s_y \rightharpoonup_{h} s_x) ) 
\end{equation}
\end{definition}

\begin{definition}
\label{bisimulation}
A state relation $\mathcal{R}$ is a \textbf{bisimulation} if and only if for every state $s_x$ and $s_y$ such that $s_x\ \mathcal{R}\ s_y$, the following two properties hold: 
\begin{equation}
\label{eqn:BisimulationForwards}
\forall s_x^\prime ( s_x \rightarrow_x s_x^\prime \Rightarrow \exists s_y^\prime( (s_y \rightarrow_y s_y^\prime )\wedge (s_x^\prime\ \mathcal{R}\ s_y^\prime ))  )
\end{equation}
\begin{equation}
\label{eqn:BisimulationBackwards}
\forall s_y^\prime ( s_y \rightarrow_y s_y^\prime\Rightarrow \exists s_x^\prime( (s_x \rightarrow_x s_x^\prime )\wedge (s_x^\prime\ \mathcal{R}\ s_y^\prime ))  )
\end{equation}
\end{definition}

Note that in the literature it is customary to define bisimulation in terms of a single labeled transition system, whereas for the purposes of this paper the definition of bisimulation refers to a pair of transition relations $\rightarrow_x$ and $\rightarrow_y$ defined by reduction rules. Since it is possible to create a union of the two rule systems $\rightarrow_x \cup \rightarrow_y$, and since none of the transitions in the reduction rules in this paper are labeled, this definition is sufficient for all of the customary properties of bisimulation to apply. For a more detailed treatment on the application of bisimulation to reduction rule systems see \cite{GSE:barbedbisimulation}.

\begin{definition}
Given a state transition function $\rightarrow_{\Phi}$, an initial state $s_0$ and a control flow sequence $\pi_n$, the \textbf{feasible state set}, $\mathbb{FS}(\rightarrow_{\Phi},s_0,\pi_n)$, is defined as
 $$
\begin{array}{l}
\mathbb{FS}(\rightarrow_{\Phi},s_0,\pi_n) = \\
\ \ \ \ \ \ \{s \mid \exists \Pi_n^\Phi\ (\pi_n = \mathbb{CF}(\Pi_n^\Phi) \wedge s \sqsubset \mathit{last}(\Pi_n^\Phi))\} 
\end{array}
$$
where $\mathit{last}(\Pi_n^\Phi)$ returns the last state on the feasible sequence.
\end{definition}

\begin{definition}
\label{equivalent}
A summary state $s_s$ is \textbf{equivalent} to a set of lazy states $P$ if and only if $s_s$ represents every state in $P$ and represents no other state: 
$$s_s \cong P \Leftrightarrow (\forall s_i \in \mathcal{S}\ (s_i \in P \Leftrightarrow s_i \sqsubset s_s) )$$
\end{definition}

\begin{definition}
\label{sound}
A state $s_s$ is \textbf{sound} with respect to a transition relation, $\rightarrow_\phi$, initial state, $s_0$, and control flow path, $\pi_n$, if and only if 
$$ \forall s_\ell \in \mathcal{S}_\ell\ (s_\ell \sqsubset s_s \Rightarrow s_\ell \in \mathbb{FS}(\rightarrow_{\phi},s_0,\pi_n) ) $$
\end{definition}

\begin{definition}
\label{complete}
A state $s_s$ is \textbf{complete} with respect to a transition relation, $\rightarrow_\phi$, initial state, $s_0$, and control flow path, $\pi_n$, if and only if 
$$ \forall s_\ell \in \mathcal{S}_\ell\ ( s_\ell \in \mathbb{FS}(\rightarrow_{\phi},s_0,\pi_n)\Rightarrow s_\ell \sqsubset s_s ) $$
\end{definition}

\begin{definition}
\label{exact}
A state $s$ is \textbf{exact} with respect to a transition relation, $\rightarrow_\phi$, initial state, $s_0$, and control flow path, $\pi_n$, if and only if it is both sound and complete:
$$ s \cong \mathbb{FS}(\rightarrow_{\phi},s_0,\pi_n)$$
\end{definition}




