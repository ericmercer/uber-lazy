\section{Introduction}

In recent years symbolic execution has provided the basis for various
software testing and analysis techniques. Symbolic execution explores
the program by treating input values as symbolic, and for each
explored path, it computes a set of feasible constraints on the
symbolic inputs to create a \emph{path condition}. Path conditions
computed by symbolic execution characterize the observed program
execution behaviors and have been used as an enabling technology for
various applications, e.g., regression
analysis~\cite{backes:2012,Godefroid:SAS11,Person:FSE08,person:pldi2011,Ramos:2011,Yang:ISSTA12},
data structure repair~\cite{KhurshidETAL05RepairingStructurally},
dynamic discovery of
invariants~\cite{CsallnerETAL08DySy,Zhang:ISSTA14}, and
debugging~\cite{Ma:2011}.

The path conditions computed by symbolic execution have two important
features: 1) they characterize the set of feasible execution paths,
i.e., infeasible paths are omitted, and 2) each path condition encodes
the set of concrete program inputs that would cause execution to
follow the same control flow path. Because path conditions are encoded
as constraints over program inputs, the reasoning capabilities of
symbolic execution are limited by the underlying constraint
solver. Recent work has explored the strengths and limitations of
tools which leverage constraint solvers, e.g., concolic execution,
identifying symbolic dereferencing and updating operations as areas
for potential improvement~\cite{Chen:2013,Qu:2011}.  Precisely
modeling these operations, however, is challenging because it requires
formulating logical predicates over an input domain that contains a
potentially unbounded number of references.

The most direct approach to address these challenges is to approximate
the results of a symbolic deference by substituting in a valid
concrete value, as is done in dynamic symbolic execution
(DSE)~\cite{Godefroid:POPL07,Godefroid:2005,Sen:2005,Tillmann:2008}. These
approaches partition the universe of heaps in the symbolic
representation but often under-approximate the partitions or require
array theories in the constraint
solver~\cite{Chen:2013,Elkarablieh:2009}. A more effective approach is
to partition the reachable heaps by using value summary
representations, where multiple values for a reference may be
represented along a single execution
path~\cite{Dillig:2011,Elkarablieh:2009,Sen:2014,Torlak:2014,Xie:2005,Yorsh:2008}. Systems
using these representations perform analyses that are more complete
than standard DSE methods, but their associated dereferencing
operations require placing further restrictions on the input heap,
such as limiting the number of nodes~\cite{Elkarablieh:2009}, the
nature of aliasing relationships~\cite{Babic:2007,Xie:2005}, or the
the use of recursive data structures~\cite{Dillig:2011}.

An alternative approach which does not under-approximate the heap is
Generalized Symbolic Execution
(GSE)~\cite{Cadar:2008,KiasanKunit,GSE03,Rosner:2015}.  On symbolic
dereferencing operations, GSE partitions the input space between the
various possible aliasing configurations on-the-fly and also creates
non-deterministic choice points on the current path in order to
explore each heap configuration. This approach, however, is
inefficient because it does not treat heap allocated data as symbolic,
but instead concretizes all possible heap configurations.

The contribution of this work is a non-trivial adaptation of concepts
in abstract interpretation to the domain of symbolic execution. The
result of which, unlike GSE, is the ability to efficiently generate
and update \emph{symbolic heaps} to produce a single path constraint
that is sound and complete up to the point of execution. This is
accomplished by representing the symbolic heap with a field sensitive
points-to graph with edges annotated with disjunctive formulas
describing aliasing relationships. Where GSE branches the heap at every
de-reference, the symbolic heap in this work splits only at reference
comparison operations or field operations that may throw exceptions.

The correctness of the symbolic heap and its on-the-fly update rules
for symbolic execution is shown by proving the existence of a
bisimulation between symbolic heap states and GSE states. From that
result comes the soundness and completeness of the path constraint up
to the point of execution.  The efficiency of the representation is
established with an exploratory empirical evaluation within the Java
PathFinder framework over a classical set of symbolic execution
benchmarks. The evaluation demonstrates how the symbolic heap scales
compared to state-of-art GSE algorithms.

This new symbolic heap and update rules for symbolic execution is a
novel approach for de-referencing and updating heap allocated data
that brings together the best features of previous work while removing
the limitations of that same work. Although it draws inspiration from
abstract interpretation by defining a symbolic heap,
\begin{compactitem}
  \item it is an exact representation, without over- or under-approximation;
  \item it imposes no restrictions on the set of heaps that can be defined;
  \item it does not require the creation of an initial set of input heaps;
  \item it is not tied to specific constraint solver theories, e.g., arrays; and
  \item it does not require global constraint rewriting for on-the-fly updates.
\end{compactitem}
These several properties set this work apart from other related works.

\begin{comment}
\section{Introduction}

% SymExe is cool because for reasons x,y, and z

In recent years symbolic execution has provided the basis for various
software testing and analysis techniques. Symbolic execution
systematically explores the program execution space representing input
values symbolically, and for each explored path, it computes
constraints on the symbolic inputs to create a \emph{ path condition}.
Path conditions computed by symbolic execution characterize the
observed program execution behaviors and have been used as an enabling
technology for various applications, e.g., regression
analysis~\cite{backes:2012,Godefroid:SAS11,Person:FSE08,person:pldi2011,Ramos:2011,Yang:ISSTA12},
data structure repair~\cite{KhurshidETAL05RepairingStructurally},
dynamic discovery of
invariants~\cite{CsallnerETAL08DySy,Zhang:ISSTA14}, and
debugging~\cite{Ma:2011}.


The path conditions computed by symbolic execution have two important
characteristics: 1) they characterize the set of feasible execution
paths, i.e., infeasible paths are omitted, and 2) each path condition
represents the set of concrete program inputs that would cause
execution to follow that program path. Because path conditions are
encoded as constraints over program inputs, the reasoning capabilities
of symbolic execution are limited by the underlying constraint solver.
Thus, extending the capabilities of symbolic execution to reason about
new theories is an area of active research. Of particular interest are
symbolic dereferencing and updating
operations~\cite{Chen:2013,Qu:2011}. Precisely modeling these
operations is challenging, because it requires formulating logical
predicates over an input domain containing a potentially unbounded
number of references.

Current strategies for symbolic dereferencing require making tradeoffs
between the completeness of the analysis and the completeness of the
path condition. For example, one simple approach is to terminate
execution upon attempting to dereference any symbolic
value~\cite{Clarke:2004}. While this preserves the integrity of the
path condition, it comes at the cost of an incomplete analysis. To
enable a more complete analysis, dynamic symbolic execution (DSE),
approximates the results of a symbolic dereferencing by substituting
in a valid concrete
value~\cite{Godefroid:POPL07,Godefroid:2005,Sen:2005,Tillmann:2008}. This
allows execution to continue, but the approximation renders the path
condition incomplete~\cite{Chen:2013,Elkarablieh:2009}, by effectively
partitioning the space of input heaps into those where the result of
the dereferencing matches the concrete execution. Larger partitions
can be achieved by using value summary representations, where multiple
values for a reference may be represented along a single execution
path~\cite{Dillig:2011,Elkarablieh:2009,Sen:2014,Torlak:2014,Xie:2005,Yorsh:2008}. Systems
using these representations perform analyses that are more complete
than standard DSE methods, but their associated dereferencing
operations require placing further restrictions on the input heap,
such as limiting the number of nodes~\cite{Elkarablieh:2009}, the
nature of aliasing relationships~\cite{Babic:2007,Xie:2005}, or the
shape of the heap itself~\cite{Dillig:2011}. Finally, at the opposite
end of the spectrum lies generalized symbolic execution
(GSE)~\cite{Cadar:2008,KiasanKunit,GSE03,Rosner:2015}, which
accomplishes a fully complete analysis at the cost of an incomplete
path condition \sjp{not clear what an incomplete path condition
  is}. On symbolic dereferencing operations, GSE partitions the input
space between the various possible aliasing configurations
\sjp{so?}. Thus, none of the current techniques provide the option of
maintaining both a complete analysis and a complete path
condition. This is in stark contrast to symbolic execution over
primitive types, which enjoys both properties simultaneously.

\sjp{Need to explain that the techniques still operate within some bound for loops, etc.}

\sjp{Notion of on-the-fly updates is not here}

\sjp{Nowhere do we refer to the technique as a summary heap framework or algorithm,
so the next section with its heading just jumps out of nowhere.}

This paper introduces a technique that avoids these problems by
preserving the following properties of classical symbolic execution:
(i) Inputs are represented by unbounded and unconstrained input
symbols. This lack of predetermined constraints is key to avoiding
under-approximation, which in turn is required for performing a
complete analysis. (ii) Points of nondeterminism are only introduced
at control flow instructions, which is critical for maintaining a
complete path condition for a given control flow path.

\noindent{This paper makes the following contributions:}

\begin{compactdesc}

\item\textbf{-} The first sound and complete system for reasoning symbolically about the 
set of input heaps along any valid program path. 

\item\textbf{-} A bisimulation proof establishing the soundness and 
completeness of the heap summary approach with respect to
properties provable by GSE.

\item\textbf{-} A proof-of-concept implementation and empirical study 
demonstrating the scalability of the summary heap approach
compared to other GSE approaches.

\end{compactdesc}
\end{comment}
