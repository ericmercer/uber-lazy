\section{Generating Symbolic Heap Summaries}

\subsection{Initialization of Symbolic References}

In this section we present the Javalite rewrite rules for the
initialization of references in the symbolic heap summary. The
initialization rules are defined on the bi-partite graph consisting of
references and locations. In contrast, the heap summarization approach
creates a single branch that contains the summarization for all the
initialization in a single bi-partitate graph.


\begin{comment}
NOTE: It may be required to add back into the constraint on new the
conditions under which the original location was created. In other
words, look at the preimage on the location used for new, find the
least reference (the one that attached to the first creation), pull
the constraint off of that reference, and add it to the constraint for
nw. This is required for the isClone() to work on the example that
tests equivalence.
\end{comment}

\newsavebox{\boxPI}
\savebox{\boxPI}{
\mprset{flushleft}
\begin{mathpar}
	\inferrule[Summarize]{
	\Lambda = \mathbb{UN}\lp \cfgnt{L}, \cfgnt{R}, \cfgnt{r}, \cfgnt{f}\rp \\
      \Lambda \neq \emptyset \\
      \lp\phi_x\ \cfgnt{l}_x\rp = \mathrm{min}_l\lp \Lambda\rp\\
      \cfgnt{r}_f = \mathrm{init}_r\lp \rp \\
      l_f  = \mathrm{fresh}_l\lp \mathrm{C}\rp\\\\
      \rho = \{ \lp \cfgnt{r}_a\ l_a\rp  \mid \mathrm{isInit}\lp \cfgnt{r}_a\rp  \wedge\cfgnt{r}_a = \mathrm{min}_r\lp \cfgnt{R}^{\leftarrow}[l_a]\rp \wedge \mathrm{type}\lp l_a\rp  = \mathrm{C} \} \\\\
      \theta_\mathit{null} = \{ \lp \phi\ l_\mathit{null}\rp  \mid \phi = \lp \phi_x \wedge \cfgnt{r}_f = \cfgnt{r}_\mathit{null} \rp  \} \\\\
      \theta_\mathit{new} = \{\lp \phi\ l_f\rp  \mid \phi = \lp \phi_x \wedge \cfgnt{r}_f \neq \cfgnt{r}_\mathit{null} \wedge \lp \wedge_{\lp \cfgnt{r}^\prime_a\ l^\prime_a\rp  \in \rho} \cfgnt{r}_f \ne \cfgnt{r}^\prime_a\rp \rp \}\\\\
      \theta_\mathit{alias} = \{ \lp \phi\ l_a\rp  \mid \exists\cfgnt{r}_a\ \lp\lp\cfgnt{r}_a\ l_a\rp  \in \rho \wedge \phi = \lp \phi_x \wedge \cfgnt{r}_f \neq \cfgnt{r}_\mathit{null} \wedge \cfgnt{r}_f = \cfgnt{r}_a \wedge \lp \wedge_{\lp \cfgnt{r}^{\prime}_a\ l^{\prime}_a\rp  \in \rho\ \lp \cfgnt{r}^\prime_a < \cfgnt{r}_a\rp } \cfgnt{r}_f \neq \cfgnt{r}^{\prime}_a \rp \rp \rp \} \\\\
      \theta_\mathit{orig} = \{\lp\phi\ \cfgnt{l}_\mathit{orig}\rp \mid \exists \phi_\mathit{orig} \lp \lp\phi_\mathit{orig}\ \cfgnt{l}_\mathit{orig}\rp \in \cfgnt{L}\lp\cfgnt{R}\lp\cfgnt{l}_x,\cfgnt{f}\rp\rp \wedge \phi = \lp\neg\phi_x \wedge \phi_\mathit{orig}\rp\}\\\\ 
      \theta = \theta_\mathit{null} \cup \theta_\mathit{new} \cup \theta_\mathit{alias} \cup \theta_\mathit{orig} \\
\cfgnt{R}^\prime = \cfgnt{R}[\forall \cfgnt{f} \in \mathit{fields}\lp \mathrm{C}\rp \ \lp \lp l_f\ \cfgnt{f}\rp  \mapsto \cfgnt{r}_\mathit{un} \rp ]
    }{
      \lp \cfgnt{L}\ \cfgnt{R}\ \cfgnt{r}\ \cfgnt{f}\ \cfgnt{C}\rp \rsum 
      \lp \cfgnt{L}[\cfgnt{r}_f \mapsto \theta]\ \cfgnt{R}^{\prime}[ \lp l_x,\cfgnt{f}\rp  \mapsto \cfgnt{r}_f ]\ \cfgnt{r}\ \cfgnt{f}\ \cfgnt{C}\rp
	}
\and
	\inferrule[Summarize-end]{
	  \Lambda = \mathbb{UN}\lp \cfgnt{L}, \cfgnt{R}, \cfgnt{r}, \cfgnt{f}\rp \\
      \Lambda = \emptyset
    }{
      \lp \cfgnt{L}\ \cfgnt{R}\ \cfgnt{r}\ \cfgnt{f}\ \cfgnt{C}\rp  \rsum
      \lp \cfgnt{L}\ \cfgnt{R}\ \cfgnt{r}\ \cfgnt{f}\ \cfgnt{C}\rp 
	}
\end{mathpar}}
%\end{center}
%\caption{The summary machine, $s ::= \lp\cfgnt{L}\ \cfgnt{R}\ \cfgnt{r}\ \cfgnt{f}\ \cfgnt{C}\rp$, with $s\rsum^*s^\prime =  s \rsum \cdots \rsum s^\prime \rsum s^\prime$.}
%\label{fig:symInit}
%\end{figure*}


\begin{figure*}
\begin{tabular}[c]{l}
\scalebox{0.81}{\usebox{\boxPI}} \\
\end{tabular}
\caption{The summary machine, $s ::= \lp\cfgnt{L}\ \cfgnt{R}\ \cfgnt{r}\ \cfgnt{f}\ \cfgnt{C}\rp$, with $s\rsum^*s^\prime =  s \rsum \cdots \rsum s^\prime \rsum s^\prime$.}
\label{fig:symInit}
\end{figure*}

The initialization rules are invoked when an uninitialized field in a
symbolic reference is accessed. The function $\mathbb{UN}(\cfgnt{L},
\cfgnt{R}, \cfgnt{r}, \cfgnt{f}) = \{\cfgnt{l}\ ...\}$ returns
constraint-location pairs in which the field $\cfgnt{f}$ is
uninitialized:
\[
\begin{array}{rcl}
\mathbb{UN}(\cfgnt{L}, \cfgnt{R}, \cfgnt{r}, \cfgnt{f}) & = &\{ \lp\phi\ \cfgnt{l}\rp \mid \lp \phi\ \cfgnt{l}\rp  \in \cfgnt{L}\lp \cfgnt{r}\rp  \wedge \\
& & \ \ \ \ \exists \phi^\prime \lp \lp \phi^\prime\ \cfgnt{l}_\mathit{un}\rp  \in \cfgnt{L}\lp \cfgnt{R}\lp l,\cfgnt{f}\rp\rp \wedge \\
& & \ \ \ \ \ \ \ \ \mathbb{S}\lp \phi \wedge \phi^\prime \rp\rp\}\\
\end{array}
\]
where $\mathbb{S}(\phi)$ returns true if $\phi$ is
satisfiable. Intutively, for the reference, $\cfgnt{r}$, it constructs
the set, $\theta$, that contains all contraint-location pairs that
point to the field $\cfgnt{f}$ and $\cfgnt{f}$ points to
$\cfgnt{l}_\mathit{un}$.

% The cardinality of the set, $\theta$ is never
%greater than one in GSE and the constraint is always satisfiable
%because all constraints are constant. This property is relaxed in GSE
%with heap summaries.

The rewrite rule to initialize a heap in the symbolic summarization
technique is shown in~\figref{fig:symInit}. Given the unitialized set
$\mathbb{UN}$ for field $f$, the $\mathit{min}_l$ function returns the
constraint location pair based on a lexical ordering of locations. The
init$_r$ method returns a new input reference $r_f$ while the
fresh$_f$ returns a new location of type $C$. The set $\rho$ contains
the set of reference location, $(r_a, l_a)$, pairs that the current
field may alias. The pair $(r_a, l_a)$ is contained in the set $ \rho
$ if the following conditions hold: (i) $r_a$ is part of the input
references initialized in the summary heap, (ii) $r_a$ is in the
pre-image of $l_a$, (iii) and the type of $l_a$ is the same as the
type of the field being initialized.

There are three constraint-location pair sets that correspond to
non-deterministic choices in GSE: (a) $\theta_\mathit{null}$ that
creates a constraint that states $r_f$ is null under which it points
to $l_\mathit{null}$, (b) $\theta_\mathit{new}$ the constraint that
$r_f$ is not null and it does not alias another reference in set
$\rho$ under which it points to the newly created location $l_f$, (c)
and $\theta_\mathit{alias}$ where for each element in the set $(r_a,
l_a) \in \rho$ it creates a constraints that $r_f$ is not null, points
to $r_a$, and does not point to any reference contained in
$(r_a^\prime, l_a^\prime) \in \rho$ such that $r_a \neq
r_a^\prime$. Additionally $\theta_\mathit{orig}$ allows for the
possibility that the field continues to remain uninitialized since the
heap is symbolic. We refer to this initialization in the summary
machine as conditional initialization. Finally, the fields of the
reference $r_f$ are marked as uninitialized.

In the post condition of the rule the reference $r_f$ points to
$\theta$ which contains the union of all the constraint location pairs
sets, and the field $f$ points to the new reference $r_f$ in the
summary heap.  Note that the summarize rule invoked repeatedly until
the set of unitialized constraint location pairs for field $f$ is
empty. In other words the set $\lambda$ is empty. This consititutes
the summarize end rule in~\figref{fig:symInit}.

\begin{figure*}[t]
\begin{center}
\begin{tabular}[c]{c|c|c|c}
\begin{tabular}[c]{c}
\scalebox{0.81}{\input{origHeap.pdf_t}} \\
\end{tabular} &
\begin{tabular}[c]{c}
\scalebox{0.81}{\input{summarizeXHeap.pdf_t}} \\
\end{tabular} &
\begin{tabular}[c]{c}
\scalebox{0.81}{\input{summarizeYHeap.pdf_t}} \\
\end{tabular} &
\begin{tabular}[c]{l}
$\rho := \{ (r_1^i, r_1^i \neq r_\mathit{null}, l_1 \}$ \\
$\theta_\mathit{null} := \{ ( r_2^i = r_\mathit{null}, l_\mathit{null}) \}$\\
$\theta_\mathit{new} := \{ ( r_2^i \neq r_\mathit{null} \wedge r_2^i \neq r_1^i, l_2) \}$\\
$\theta_\mathit{alias} := \{ ( r_1^i \neq r_\mathit{null} \wedge r_2^i \neq r_\mathit{null} \wedge r_2^i = r_1^i, l_1) \}$\\
$\theta_\mathit{orig} := \{ \}$ \\
$\phi_{\mathit{1a}} := r_1^i = r_\mathit{null} $ \\
$\phi_{\mathit{1b}} := r_1^i \neq r_\mathit{null} $  \\
$\phi_{\mathit{2a}} := r_2^i = r_\mathit{null}$ \\
$\phi_{\mathit{2b}} := r_2^i \neq r_\mathit{null} \wedge r_2^i \neq r_1^i$ \\
$\phi_{\mathit{2c}} :=  r_2^i \neq r_\mathit{null} \wedge r_2^i = r_1^i $ \\
\end{tabular} \\
(a) & (b) & (c) & (d) \\
\end{tabular}
\end{center}
\caption{initialize this.x and this.y}
\label{fig:initHeap}
\end{figure*}

We visualize the initialization process in the symbolic summarization
technique in~\figref{fig:initHeap}. The heap
in~\figref{fig:initHeap}(a) represents the initial heap. The
references with the superscript $s$ indicates that it is a local
reference. In~\figref{fig:initHeap} $r_0^s$ represents the reference
for the $\mathit{this}$ instance which has two fields $x$ and $y$ of
the same type. The reference $r_0^s$ points to the location
$l_0$. Note that when no constraint is specified, then there is a
implicit $\mathit{true}$ constraint. In~\figref{fig:initHeap} $r_0^s$
points to $l_0$ on $\mathit{true}$. The fields $x$ and $y$ point to
the uninitialized reference $r_\mathit{un}$. The reference
$r_\mathit{un}$ points to the uninitialized location $l_\mathit{un}$
on the $\mathit{true}$ constraint.

The graph in~\figref{fig:initHeap}(b) represents the summary heap
after the initialization of the $\mathit{this}.x$ field while the
graph in~\figref{fig:initHeap}(c) represents the summary heap after
the initialization of the $\mathit{this}.y$ field following the
initialization of $\mathit{this}.x$. The list
in~\figref{fig:initHeap}(d) represents the various sets constructed in
the rewrite system when the initialization takes places. We also
define the constraints of references to their corresponding labels in
the graph.

The access on $l_0.x$ creates a new input reference $r_1^i$ in the
summary heap shown in~\figref{fig:initHeap}(b). The reference $r_1^1$
points to the $l_\mathit{null}$ location under the constraint that
$r_1^i$ is null: $\phi_{1a} := r_1^i = r_\mathit{null}$. The reference
$r_1^i$ points to the location $l_1$ under the constraint that $r_1^i$
is not $\mathit{null}$ $\phi_{1b} :=r_1^i \neq r_\mathit{null}$. The
location $l_1$ represents a fresh location of type $C$ is created on
the heap such that $C$ is type of $\mathit{this}.x$.

Following the initialization to the $\mathit{this}.x$, when we
initialize $\mathit{this}.y$ the resulting summary heap is shown
in~\figref{fig:initHeap}(c). The set $\rho$ shown
in~\figref{fig:initHeap}(d) contains the potential aliases for
$\mathit{this}.y$ which is location $l_1$ ($\mathit{this}.x$). The
constraint location $\theta$ sets represents the initialization
choices for $\mathit{this}.y$. The new input reference $r_2^i$ for
$this.y$ points to (a) location $l_\mathit{null}$ under the constraint
$\phi_{2a} := r_2^i = r_\mathit{null}$, (b) location $l_2$ under the
constraint $\phi_{2b} := r_2^i \neq r_\mathit{null} \wedge r_2^i \neq
r_1^i$ that states that $r_2^i$ is not null and does not alias
$r_1^i$, and (c) location $l_1$ under the constraint $\phi_{2c} :=
r_2^i \neq r_\mathit{null} \wedge r_2^i = r_1^i$ that states $r_2^i$
is not null and \emph{does} alias $r_1^i$.


\subsection{Accessing a Field Reference}
\newsavebox{\boxPFAFW}
\savebox{\boxPFAFW}{
%\begin{figure}[t]
%\begin{center}
\mprset{flushleft}
\begin{mathpar}
	\inferrule[Field Access]{
      \exists \lp \phi\ l\rp \in \cfgnt{L}\lp \cfgnt{r}\rp\ \lp l \neq l_{\mathit{null}} \wedge \mathbb{S}\lp \phi \wedge \phi_g\rp \rp \\\\
      \theta = \{ \phi \mid \lp \phi\ l_\mathit{null} \rp \wedge \mathbb{S}\lp \phi \wedge \phi_g\rp \} \\\\
      \phi_g^\prime = \phi_g \wedge (\wedge_{\phi \in \theta} \neg \phi) \\\\
      \{\cfgnt{C}\} = \{\cfgnt{C} \mid \exists \lp \phi\ l\rp  \in \cfgnt{L}\lp \cfgnt{r}\rp\ \lp\cfgnt{C} = \mathrm{type}\lp \cfgnt{l},\cfgnt{f}\rp\rp\} \\\\
      \lp \cfgnt{L}\ \cfgnt{R}\ \cfgnt{r}\ \cfgnt{f}\ \cfgnt{C}\rp \rsum^* \lp \cfgnt{L}^\prime\ \cfgnt{R}^\prime\ \cfgnt{r}\ \cfgnt{f}\ \cfgnt{C}\rp \\
      \cfgnt{r}^\prime = \mathrm{stack}_r\lp \rp
    }{
      \lp \cfgnt{L}\ \cfgnt{R}\ \phi_g\ \eta\ \cfgnt{r}\ \lp \cfgt{*}\ \cfgt{\$}\ \cfgnt{f} \rightarrow \cfgnt{k}\rp \rp  \rsym^\mathit{A}
      \lp \cfgnt{L}^\prime[\cfgnt{r}^\prime \mapsto \mathbb{VS}\lp \cfgnt{L}^\prime,\cfgnt{R}^\prime,\cfgnt{r},\cfgnt{f},\phi_g^\prime\rp ]\ \cfgnt{R}^\prime\ \phi_g^\prime\ \eta\ \cfgnt{r}^\prime\ \cfgnt{k}\rp 
	}
\and
	\inferrule[Field Write]{
      \cfgnt{r}_x = \eta\lp\cfgnt{x}\rp \\
      \exists \lp \phi\ l\rp \in \cfgnt{L}\lp \cfgnt{r}_x\rp\ \lp l \neq l_{\mathit{null}} \wedge \mathbb{S}\lp \phi \wedge \phi_g\rp \rp \\\\
      \theta = \{ \phi \mid \lp \phi\ l_\mathit{null} \rp \wedge \mathbb{S}\lp \phi \wedge \phi_g\rp \} \\\\
      \phi_g^\prime = \phi_g \wedge (\wedge_{\phi \in \theta} \neg \phi) \\\\
      \Psi_x =\{\lp \phi\ l\ \cfgnt{r}_\mathit{cur} \rp  \mid \lp \phi\ \cfgnt{l}\rp  \in \cfgnt{L}\lp \cfgnt{r}_x\rp  \wedge \cfgnt{r}_\mathit{cur} = \cfgnt{R}\lp l,\cfgnt{f}\rp  \}\\\\
      X = \{ \lp l\ \theta \rp  \mid \exists \phi\ \lp \lp \phi\ l\ \cfgnt{r}_\mathit{cur} \rp \in \Psi_x \wedge \theta = \mathbb{ST}\lp \cfgnt{L},\cfgnt{r},\phi,\phi_g^\prime\rp  \cup \mathbb{ST}\lp \cfgnt{L},\cfgnt{r}_\mathit{cur},\neg\phi,\phi_g^\prime\rp \rp  \}\\\\
      \cfgnt{R}^{\prime} = \cfgnt{R}[\forall \lp l\ \theta \rp  \in X\ \lp \lp l\ \cfgnt{f}\rp  \mapsto \mathrm{fresh}_r\lp \rp \rp ]\\\\
      \cfgnt{L}^{\prime} = \cfgnt{L}[\forall \lp l\ \theta \rp  \in X\ \lp \exists \cfgnt{r}_\mathit{targ}\ \lp \cfgnt{r}_\mathit{targ} = \cfgnt{R}^\prime\lp l,\cfgnt{f}\rp \wedge \lp\cfgnt{r}_\mathit{targ} \mapsto \theta\rp  \rp \rp ]
    }{
      \lp \cfgnt{L}\ \cfgnt{R}\ \phi_g\ \eta\ \cfgnt{r}\ \lp \cfgnt{x}\ \cfgt{\$}\ \cfgnt{f}\ \cfgt{:=}\ \cfgt{*}\ \rightarrow\ \cfgnt{k}\rp \rp  \rsym^\mathit{W}
      \lp \cfgnt{L}^{\prime}\ \cfgnt{R}^{\prime}\ \phi_g^\prime\ \eta\ \cfgnt{r}\ \cfgnt{k}\rp 
	}	
\end{mathpar}}
%\end{center}
%\caption{Precise symbolic heap summaries from symbolic execution indicated by $\rsym = \rightarrow_\mathit{FA} \cup \rightarrow_\mathit{FW} \cup \rightarrow_\mathit{EQ} \cup \rcom$.}
%\label{fig:symfield}
%\end{figure}



\begin{figure*}[t]
\begin{center}
\setlength{\tabcolsep}{50pt}
\begin{tabular}[c]{cc}
\scalebox{0.81}{\usebox{\boxPFAFW}} & 
%\scalebox{0.91}{\input{faYHeap.pdf_t}} &
\scalebox{0.91}{\input{fwXHeap.pdf_t}} \\ \\
(a) & (b)
\end{tabular}
\end{center}
\caption{field access for this.y and field write for this.x = this.y}
\label{fig:fHeap}
\end{figure*}

There are two rewrite rules in~\figref{fig:fHeap}(a), one for field
access and the other for field write. Let us consider the rewrite rule
for field access. The first check we perform is whether there exists a
constraint location pair for the $r$ being accessed such that the
location is not null and the constraint when conjuncted with the
global constraint is satisfiable. Next we extract all possible
constraints under which $r$ points to a null location such that the
constraint is satisfiable under the current global constraint,
$\phi_g$. The negation of these constraints are added to the global
constraint to create a new global constraint $\phi_g^\prime$. The
update to the global constraint ensures that access of the field $f$
happens only on non-null locations. The type $C$ of the field is
extracted and passed to rewrite rule that performs the initialization
of the symbolic heap. The initialization rules are described earlier
in this section. 

Recall that during the initialization the rewrite rules
in~\figref{fig:symHeap} add input references that map to fields being
initialized.  Once the initialization is complete, however, we create
a new local reference $r^\prime$. An important property of the
references in the bi-partiate graph is that they are
\emph{immutable}. Hence we de-reference the initialized input
reference, assign its value to the the new local reference, and return
the local reference. In order to de-reference a field $r.f$ we define
a helper function which is called the value set.

\begin{definition}
\label{def:VS}
The $\mathbb{VS}$ function constructs the value set given a
heap, reference, and desired field where $\mathbb{S}(\phi)$ returns true if $\phi$ is satisfiable.
\[
\begin{array}{rcl}
  \mathbb{VS}(\cfgnt{L},\cfgnt{R},\phi_g,\cfgnt{r},\cfgnt{f}) & = &
  \{(\phi\wedge\phi^\prime\ \cfgnt{l}^\prime) \mid \exists
  \cfgnt{l}\ ((\phi\ l) \in L(r)\ \wedge \\ & &
  \ \ \ \ \ \ \ \ \exists \cfgnt{r}^\prime ( \cfgnt{r}^\prime =
  R(\cfgnt{l},\cfgnt{f})\ \wedge (\phi^\prime\ l^\prime) \in
  \cfgnt{L}(\cfgnt{r}^\prime)\ \wedge
  \mathbb{S}(\phi\wedge\phi^\prime\wedge \phi_g)))\}
\end{array}
\]
\end{definition}


In the post-condition of the rewrite rule we assign
the value set of input reference $r$ to the local reference $r^\prime$
and return the local reference $r^\prime$ in the next state.

Consider the graph shown in~\figref{fig:fHeap}(b) the reference
$r_2^i$ is created during the initialization of $\mathit{this}.y$. The
reference that is returned during the access of the field, however, is
$r_3^s$. The reference $r_3^s$ points to the value set of $r_2^i$
which are: $(\phi_{2a}, l_\mathit{null})$, $(\phi_{2b}, l_2)$, and
$(\phi_{2c}, l_1)$. The values of the constraints $\phi_{2a}$,
$\phi_{2b}$, and $\phi_{2c}$ are defined in~\figref{fig:initHeap}(d).

\subsection{Writing to a Field}

The reference $r_x$ is the base pointer whose field is being written
to while $r$ is the target reference. We look up the value of the base
reference in the environment $\eta(x)$. The set $\Psi_x$ is the set of
tuples of constraints, locations, and references which provide the
reference chains leading from $r_x$ to the reference of the field,
$r_\mathit{curr}$, being written to through $\phi$ and $l$.The goal is
to overwrite the $r_\mathit{curr}$ references with the target
references. Since the target of the write is $r$, we first check that the
location constraint pairs of $L(r)$ are satisfiable when accessed
through the $r_x$ chain. This is accomplished by the strengthing
function.

\begin{definition}
\label{def:ST}
The strengthen function $\mathbb{ST}(\cfgnt{L},\cfgnt{r},\phi,\phi_g)$ strengthens every
constraint from the reference $\cfgnt{r}$ with $\phi$ and keeps only location-constraint
pairs that are satisfiable after this strengthening with the inclusion of the global heap constraint $\phi_g$:
\[
\begin{array}{rcl} 
\mathbb{ST}(\cfgnt{L},\cfgnt{r},\phi,\phi_g) & = & \{ (\phi\wedge\phi^\prime\ \cfgnt{l}^\prime) \mid  \\
& & \ \ \ \ (\phi^\prime\ \cfgnt{l}^\prime)\in \cfgnt{L}(\cfgnt{r})\wedge\mathbb{S}(\phi\wedge\phi^\prime\wedge\phi_g) \}
\end{array}
\]
\end{definition}


Additionally, we also check for conditions where the write is not possible 

\begin{comment}
\begin{figure}[t]
\begin{center}
\begin{tabular}[c]{l}
$\Psi_x = \{ (true, l_0, r_1^i) \}$\\
$ST (L, r_3^s, \phi, \phi_g)$ \\
$\theta = \{ (\phi_{2a}\; l_\mathit{null} ) (\phi_{2b}\; l_2) (\phi_{2c}\; l_1) \}$\\
$ST(L, r_0, \phi, \phi_g)$\\
$\theta = \{ \}$\\
\end{tabular}
\end{center}
\caption{field write for this.x = this.y sets}
\label{fig:faHeapSets}
\end{figure}
\end{comment}

\subsection{Equality and InEquality of References}

\newsavebox{\boxPEQ}
\savebox{\boxPEQ}{
%\begin{figure}[t]
%\begin{center}
\mprset{flushleft}
\begin{tabular}[c]{c}
\begin{mathpar}
 \inferrule[Equals (references-true)]{
    \Phi_\alpha = \{\lp\phi_0 \wedge \phi_1\rp \mid \exists l\ \lp \lp \phi_0\ l\rp  \in \cfgnt{L}\lp \cfgnt{r}_0\rp  \wedge \lp \phi_1\ l\rp  \in \cfgnt{L}\lp \cfgnt{r}_1\rp \rp \} \\\\
    \Phi_0 = \{\phi_0 \mid \exists l_0\ \lp \lp \phi_0\ l_0\rp  \in \cfgnt{L}\lp \cfgnt{r}_0\rp  \wedge \forall \lp \phi_1\ l_1\rp  \in \cfgnt{L}\lp \cfgnt{r}_1\rp \ \lp l_0 \neq l_1\rp \rp \} \\\\
    \Phi_1 = \{\phi_1 \mid \exists l_1\ \lp \lp \phi_1\ l_1\rp  \in \cfgnt{L}\lp \cfgnt{r}_1\rp  \wedge \forall \lp \phi_0\ l_0\rp  \in \cfgnt{L}\lp \cfgnt{r}_0\rp \ \lp l_0 \neq l_1\rp \rp \} \\\\
    \phi^\prime =  \phi \wedge \lp \vee_{\phi_\alpha\in\Phi_\alpha}\phi_\alpha\rp \wedge\lp \wedge_{\phi_0 \in \Phi_0} \neg \phi_0\rp \wedge\lp \wedge_{\phi_1
    \in \Phi_1} \neg \phi_1\rp \\\\ 
    \mathbb{S}(\phi^\prime)}{
    \lp \cfgnt{L}\ \cfgnt{R}\ \phi\ \eta\ \cfgnt{r}_0\ \lp \cfgnt{r}_1\; \cfgt{=}\; \cfgt{*} \rightarrow \cfgnt{k}\rp \rp  \rightarrow_\mathit{EQ}^T
    \lp \cfgnt{L}\ \cfgnt{R}\ \phi^\prime\ \eta\ \cfgt{true}\ \cfgnt{k}\rp 
    }
\end{mathpar}
\end{tabular}}
%\end{center}
%\caption{FIX THIS CAPTION AND MOVE $\rsym$ DEFINITION (MAY NOT BE NEEDED BECAUSE OF \defref{def:meta}: Precise symbolic heap summaries from symbolic execution indicated by $\rsym = \rightarrow_\mathit{FA} \cup \rightarrow_\mathit{FW} \cup \rightarrow_\mathit{EQ}^T \cup \rightarrow_\mathit{EQ}^F \cup \rcom$.}
%\label{fig:symeq}
%\end{figure}


\newsavebox{\boxPEX}
\savebox{\boxPEX}{
\begin{tabular}[c]{l}
$L(r_1^i) = \{ (\phi_{1a}\; l_\mathit{null})\; (\phi_{1b}\; l_1) \}$ \\
$L(r_2^i) = \{ (\phi_{2a}\; l_\mathit{null}),\; (\phi_{2b}\; l_2),\; (\phi_{2c}\; l_1) \} $\\
  $\theta_0 = \{ \} $\\
$\theta_1 = \{ \phi_{2b}\} $\\ \hline
Equals true \\
$\theta_\alpha = \{ (\phi_{1a}\; \wedge\; \phi_{2a} ) (\phi_{1b}\; \wedge\; \phi_{2c} ) \}$\\
$\phi^\prime = \mathit{true} \wedge [ (\phi_{1a}\; \wedge\; \phi_{2a} )\vee (\phi_{1b}\; \wedge\; \phi_{2c} ) ] \wedge \neg\phi_{2b} $\\ \hline
Equals false \\
$\theta_\alpha = \{ (\phi_{1a}\; \implies\; \neg\phi_{2a} ) (\phi_{1b}\; \implies\; \neg\phi_{2c} ) \}$\\
$\phi^\prime = \mathit{true} \wedge  (\phi_{1a}\; \implies\; \neg\phi_{2a} )\wedge (\phi_{1b}\; \implies\; \neg\phi_{2c} )  \wedge \phi_{2b} $\\ \hline
\end{tabular}}

\begin{figure}
\begin{tabular}[c]{l}
\scalebox{0.81}{\usebox{\boxPEQ}} \\
% & \usebox{\boxPEX} \\ \\
%(a) & (b) \\
\end{tabular}
\caption{equals true for this.x == this.y}
\label{fig:eqs}
\end{figure}


The rewrite rules for check the equals true and equals false when
comparing two references in the symbolic summary heap is shown
in~\figref{fig:eqs}(a). First the equals reference true rewrite rule
returns true if two references $r_0$ and $r_1$ are equal. In GSE this
is a simple comparision of object refrences. In the symbolic summary
heap, however, there we campare sets of constraint location pairs for
each reference. We construct three sets of constraints. The first set,
$\Phi_\alpha$, we check whether under some constraints $\phi_0$ and
$\phi_1$ whether references $r_0$ and $r_1$ respectively point to the
same location $l$.  For all such constraints we create conjunctions of
the constraints $\phi_0 \wedge \phi_1$ and add them to set
$\Phi_\alpha$. The next set of constraints si $\Phi_0$ which contains
the set of constraints from $r_0$ that lead to $l$ such that there is
no constraints under which $r_1$ can point to that $l$.
