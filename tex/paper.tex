
\documentclass[pldi]{sigplanconf-pldi15}

\usepackage{amsmath}
\usepackage{comment}
\usepackage{mathpartir}
\usepackage{amssymb}
\usepackage{amsfonts}

\hyphenation{op-tical net-works semi-conduc-tor}

\input{cfg-commands}
\newcommand{\figref}[1]{Figure~\ref{#1}}
\newcommand{\secref}[1]{Section~\ref{#1}}
\newcommand{\thref}[1]{Theorem~\ref{#1}}
\newcommand{\lemref}[1]{Lemma~\ref{#1}}
\newcommand{\defref}[1]{Definition~\ref{#1}}

\newcommand{\sym}{\ensuremath{\varsigma}}
\newcommand{\gse}{\ensuremath{g}}
\newcommand{\rsym}{\ensuremath{\rightarrow_\sym}}
\newcommand{\rssym}{\ensuremath{\leadsto_\sym}}
\newcommand{\rgse}{\ensuremath{\rightarrow_\gse}}
\newcommand{\rsgse}{\ensuremath{\leadsto_\gse}}
\newcommand{\rsum}{\ensuremath{\rightarrow_S}}
\newcommand{\rinit}{\ensuremath{\rightarrow_I}}
\newcommand{\com}{\ensuremath{J}}
\newcommand{\rcom}{\ensuremath{\rightarrow_\com}}

\newcommand{\symtxt}{GSESH}
\newcommand{\gsetxt}{GSE}

\usepackage{url}
\usepackage{listings}
\usepackage{color}
\usepackage[T1]{fontenc}
%\usepackage{SIunits}            % typset units correctly
\usepackage{courier}            % standard fixed width font
\usepackage[scaled]{helvet} % see www.ctan.org/get/macros/latex/required/psnfss/psnfss2e.pdf

\newcommand{\doi}[1]{doi:~\href{http://dx.doi.org/#1}{\Hurl{#1}}}   % print a hyperlinked DOI

\definecolor{sjpcolor}{rgb}{1.0,0.0,0.722}
\newcommand*{\sjp}[1]%
{\textcolor{sjpcolor}{\noindent\textbf{[sjp:~}\textit{#1}]}}

\definecolor{nsrcolor}{rgb}{0.7,0.0,1.0}
\newcommand*{\nsr}[1]%
{\textcolor{nsrcolor}{\noindent\textbf{[nsr:~}\textit{#1}]}}

\definecolor{dkgreen}{rgb}{0,0.6,0}
\definecolor{gray}{rgb}{0.5,0.5,0.5}
\definecolor{mauve}{rgb}{0.58,0,0.82}

\lstset{frame=tb,
  language=Java,
  aboveskip=3mm,
  belowskip=3mm,
  showstringspaces=false,
  columns=flexible,
  basicstyle={\small\ttfamily},
  numbers=none,
  numberstyle=\tiny\color{gray},
  keywordstyle=\color{blue},
  commentstyle=\color{dkgreen},
  stringstyle=\color{mauve},
  breaklines=true,
  breakatwhitespace=true
  tabsize=3
}


\usepackage{graphicx}
\usepackage{graphics}
\usepackage{epsfig}
\usepackage{comment}

% used for inline lists
\usepackage{paralist}

%% This enables the xfig overlays to use the same font family as the document
%% (i.e., font family and size is the same in the figure as it is
%% in the text).

\gdef\SetFigFont#1#2#3#4#5{}

\usepackage{multirow}
\usepackage{algorithm} 
\RequirePackage[noend]{algorithmic}
\renewenvironment{algorithm}[1][\textwidth]%  
{\begin{minipage}[t][\totalheight][c]{#1}\begin{algorithmic}[1]}  %%% change [1] to [0] to turn off line numbers
{\end{algorithmic}\end{minipage}}

%% Use for each in ``FOR'' constructs

\renewcommand{\algorithmicfor}{\textbf{for each}}

%% All comments are in italics

\renewcommand{\algorithmiccomment}[1]{\textit{${/\ast}$~#1~${\ast/}$}}

%% Use ``procedure'' instead of ``Algorithm'' for off set

\renewcommand{\algorithmicensure}{\textbf{procedure}}

\newcommand{\algoname}[1]{\ENSURE #1}
\newcommand*{\algobox}[1]{\framebox{#1}}

\newcommand{\bo}[1]{\textbf{#1}}
\newcommand{\cc}[1]{\cellcolor[gray]{.6}{#1}}
\newcommand{\negspace}{\hspace{-.40cm}}

%define theorem environments
\usepackage{mathtools}
\usepackage{amsthm}
\newtheorem{theorem}{Theorem}
\newtheorem{lemma}[theorem]{Lemma}
\newtheorem{proposition}[theorem]{Proposition}
\newtheorem{corollary}[theorem]{Corollary}

\newtheorem{definition}{Definition}
%\newenvironment{definition}[1][Definition]{\begin{trivlist}
%\item[\hskip \labelsep {\bfseries #1}]}{\end{trivlist}}

%% This enables the xfig overlays to use the same font family as the document
%% (i.e., font family and size is the same in the figure as it is
%% in the text).

\gdef\SetFigFont#1#2#3#4#5{}




\begin{document}



%\title{Contraints-based Reasoning of Heaps in Symbolic Execution}

%\title{Using Constraints to Characterize Heaps in Symbolic Execution}

%\title{Symbolic Execution with precise heap constraints}

%\title{Precise Heap Summaries from Symbolic Execution}

\title{Exact Heap Summaries from Symbolic Execution}


\maketitle

\begin{abstract}
A recent trend in symbolic execution of object-oriented programs is the modeling of references as sets of guarded values, enabling multiple heap shapes to be represented in a single state. A fundamental problem with using these guarded value sets is the creation of test inputs for programs accepting symbolic reference input parameters. Although several solutions have been proposed, none have been proven to be sound and complete with respect to the properties provable by generalized symbolic execution (GSE). This work presents a method for initializing reference inputs in a black-box symbolic input heap that exactly preserves GSE semantics. A correctness proof for the initialization scheme is provided, as well as the results of an empirical evaluation of a proof-of-concept implementation. The initialization technique can be used to ensure that guarded value set based symbolic execution engines operate in a provably correct manner with regards to symbolic references.


%A fundamental challenge of using symbolic execution for software analysis is the treatment of dynamically allocated data. Existing techniques either underapproximate the space of possible inputs or are computationally infeasible. For example, dynamic symbolic execution (DSE) handles symbolic dereferencing by substituting in a value from a valid concrete execution. Generalized symbolic execution (GSE) initiates a new search path for every possible aliasing configuration. This paper introduces a method for de-referencing and manipulating values in a true block-box symbolic input heap that overcomes the limitations of previous methods. The symbolic heap supports arbitrary recursive data structures and captures all possible heaps that follow a common control flow path. Computation of complex preconditions and postconditions is supported, as well as automatic generation of test inputs. An evaluation of a proof-of-concept implementation in the Java Pathfinder framework is presented to demonstrate the computational feasibility of the approach over several classical symbolic execution benchmarks.


\end{abstract}


\section{Introduction}

% SymExe is cool because for reasons x,y, and z

In recent years symbolic execution has provided the basis for various
software testing and analysis techniques. Symbolic execution
systematically explores the program execution space represeing input values symbolically, 
and for each explored path, it computes constraints on the
symbolic inputs to create a \emph{ path condition}.  Path
conditions computed by symbolic execution characterize the observed
program execution behaviors and have been used as an enabling
technology for various applications, e.g., regression
analysis~\cite{backes:2012,Godefroid:SAS11,Person:FSE08,person:pldi2011,Ramos:2011,Yang:ISSTA12},
data structure repair~\cite{KhurshidETAL05RepairingStructurally},
dynamic discovery of
invariants~\cite{CsallnerETAL08DySy,Zhang:ISSTA14}, and
debugging~\cite{Ma:2011}.

%
A major reason that path conditions are so useful is that each one represents exactly the set of concrete program inputs that would result in a given execution path. For any given program, a concrete execution will follow the same path as a symbolic execution if and only if the concrete inputs satisfy the path condition. Furthermore, because path conditions are logical predicates, they allow precise reasoning over potentially unbounded sets of program inputs. 

Symbolic execution's reliance on path conditions is both it's greatest strength, and a significant challenge. Since the path condition must be encoded in the form of an SMT problem, symbolic execution is limited by the capabilities of the underlying constraint solver. If the solver does not include a theory for reasoning about a given program operation, then the symbolic execution cannot proceed. Thus, extending the capabilities of symbolic execution to reason about new theories is an area of active research. 

A primary interest in the research is how to dereference symbolic
input references in an unknown heap. Despite the many proposed solutions for reasoning
about symbolic input references and heaps (ADD CITATIONSS), to date, none has
been able to preserve symbolic execution's most desirable property:
the ability to produce a path condition that exactly represents all
possible input heaps for a given execution path.

% A couple of symExe�s big problems are path explosion and references. 

%Two of the main challenges facing current symbolic execution techniques 
%are path explosion and programs accepting references as inputs~\cite{Qu:2011,Chen:2013}.
% The path explosion problem stems from the way that 
%symbolic execution explores a program execution on a per-path basis. For 
%those programs for which there is an exponential number of possible 
%program paths, symbolic execution can be extremely inefficient. 
%References are a problem because symbolic execution requires that the 
%program state be represented by predicates formulated in terms of the 
%program inputs. Formulating such predicates for referencing operations 
%over potentially unbounded inputs has until now remained an open 
%question. 


% You can solve dereferencing by concretization, but that�s incomplete. 
%Here is an example of why this stinks.
One possible way to cope with the reference problem is to use a technique 
known as dynamic symbolic execution (DSE). When a DSE tool encounters 
an operation that is beyond the capabilities of its associated constraint 
solvers, such as dereferencing a symbolic reference, it substitutes a valid 
concrete result in place of the symbolic expression. However, this process 
of concretization introduces approximation into the formerly exact path condition,
and may even cause the analysis to miss valid program behaviors. 

%\begin{figure}
%void example( container a,b,c,d,x,y){
%a.f = x;
%b.f = c;
%c.f = d;
%d.f = a;
%if(x.f == y.f.f)
%     abort();\\error!
%}
%\caption{completeness example}
%\label{fig:DSEtest}
%\end{figure}
%
%Consider the example in \ref{fig:DSEtest}. During DSE, both symbolic and 
%concrete executions occur in parallel. Suppose that for the first pass, the 
%concrete execution picks the following points-to relationships: $a\rightarrow 
%loc_a, b\rightarrow loc_b, c\rightarrow loc_c, d\rightarrow loc_d ,x
%\rightarrow loc_a,  y\rightarrow loc_b$. When the program reaches the if() 
%statement, symbolic dereferencing of x and y are required. In order to 
%dereference symbol x to get x.f, x is concretized to point to $loc_a$, and 
%symbol x from field f is returned. Dereferencing y to find y.f requires 
%concretizing y to point to $loc_b$, From there, dereferencing b.f gives us 
%the reference d as the final value for y.f.f . Assuming the �false" branch is 
%executed first, the branch condition (x.f != y.f.f) reduces to x!=d, which 
%evaluates to �true" and DSE proceeds until program termination.
%After following the false branch, DSE will attempt to follow the true branch, 
%but the branch condition x==d will evaluate to false, and so DSE will 
%attempt a second pass starting from the beginning with a modified path 
%condition. For the second pass, DSE adds x==d to the path condition from 
%the beginning, to try to get the �true� branch. This time, the concrete 
%execution will choose x to point to $loc_d$. When evaluating the branch 
%condition, x.f will evaluate to a. Since the value for y is unchanged, the read 
%from y.f.f evaluates to d. The branch condition reduces to a==d, which 
%evaluates to false, so the �true� branch cannot be followed. In this instance, 
%DSE will fail to find a path to the error condition. 

% You can model references by forking the system state, but that makes 
%path explosion worse.
An alternative to concretization is to construct the input heap in a lazy manner,
deferring materialization of objects on the concrete heap until they
are needed for the analysis to proceed. The materialization creates
additional non-deterministic choice points in the symbolic execution
tree by representing the feasible heap configurations as (i) null,
(ii) an instance of a new reference of a compatible class, and (iii)
an alias to a previously initialized symbolic reference.  Symbolic
execution then follows concrete program semantics for materialized
heap locations. Although this approach enables the analysis of heap
manipulating programs, a large number of feasible concrete heap
configurations are created. \emph{Since each configuration requires a separate
path} (what does this mean), GSE induces a path explosion problem while at the 
same time splitting the symbolic input space and decreasing the 
utility of the path condition. (The last sentence is difficult to undersatnd. Consdier a rewrite).

%Attempt to argue for completeness:
In searching for a solution to the path explosion problem, it is tempting to consider one of the several techniques developed for representing multiple heap shapes within a single predicate. However, in contrast to GSE, these techniques tend to impose constraints on the symbolic input heap. These constraints extend to the path condition, such that it no longer forms a complete representation of all possible input heaps.

A common constraint is a requirement to specify some arbitrary bound before starting the analysis. For example, the analysis might require that the maximum number of nodes in the heap be specified. This creates a problem, because it's difficult to determine beforehand the right bound to set to ensure completeness, and the analysis may provide no completeness information upon termination. Other limitations prevalent with summary-type heap representations include prohibiting recursive data structures, data structures with loops, or disallowing any type of aliasing whatsoever.

Since these constraints are frequently built in to the core of the underlying representation, it is not often obvious how to remove them. For example, methods that require setting bounds on the node count are usually predicated on a fixed address space model. In a fixed model, symbolic pointer dereferencing is accomplished by first conducting a search for potential aliases of compatible type, then producing an expression based on that set of aliases. Since number of terms in the expression depends on the size of the address space, it seems that dereferencing a pointer in a space with an unbounded number of aliases would require an expression with an unbounded number of terms.

% You can solve path explosion by bounding the execution and rolling 
%everything into one big equation, but then you�re incomplete. Here is an 
%example why that stinks.
%-maybe point out that finding the right bounds is a hard problem
%One solution to the path explosion problem is to model the input heap as a 
%predicate over a bounded set of locations. This allows the analysis to 
%maintain a single representation for a set of possible heap configurations 
%on each execution path, without case splitting for dereferencing operations. 
%However, by placing an arbitrary bounds on the size of the input heap, neither
%the path condition, nor the analysis performed by these techniques is complete.

% This paper introduces a technique that is, to the best knowledge of the 
%authors, the first sound and complete heap for SymEXE
This paper introduces a technique for modeling references and their
associated operations that is sound and complete with respect to the
properties provable by symbolic execution. The technique captures in a
summary representation all heaps that follow a specific program path
up to a given point of execution. To the knowledge of the authors, it
is the first technique to do so.

The technique begins with the notion of an unconstrained input
heap. As a symblic integer represents all integers at the beginning of
execution, so does the symbolic input heap.

THIS NEXT BIT STILL NEEDS SOME WORK:

% Creating this technique required the following key insights:
These advances were enabled by the following key insights: 

First, that a complete analysis requires an unconstrained input heap. 
However, it is unclear from prior work how this might be accomplished.

Second, that GSE techniques needlessly split during dereferencing 
operations. If GSE was performing a concretization, as in DSE, then case 
splitting makes more sense. However, lazy initialization is not a 
concretization, and is in fact only a partitioning of the input space. 

Third, that we can combine lazy initialization with a compositional heap 
abstraction to form constraints on a potentially unbounded input heap.


% This paper makes the following contributions.


\noindent{This paper makes the following contributions:}

\begin{compactdesc}

\item\textbf{-} The first sound and complete system for reasoning symbolically about the 
set of input heaps along any valid program path. 

\item\textbf{-} A bisimulation proof establishing the soundness and 
completeness of the heap summary approach with respect to
properties provable by GSE.

\item\textbf{-} A proof-of-concept implementation and empirical study 
demonstrating the scalability of the summary heap approach
compared to other GSE approaches.

\end{compactdesc}


%Initial work on symbolic execution largely focused on checking
%properties of programs with primitive
%types~\cite{clarke76TSE,King:76}.  With the advent of object-oriented
%languages, recent work has generalized the core ideas of symbolic
%execution to enable analysis of programs containing complex data
%structures with unbounded domains, i.e., data stored on the
%heap~\cite{Kiasan06,Kiasan07,GSE03}.  These \emph{Generalized 
%Symbolic
%  Execution} (GSE) techniques construct the heap in a lazy manner,..

%The goal of this work is to mitigate the path explosion problem in GSE
%by grouping multiple heaps together and only partitioning the heaps at
%points of divergence in the control flow graph. Our inspiration is
%found in the domain of static analysis that uses sets of constraints
%over heap locations to encode multiple heaps in a single
%representation. These sets, sometimes known as \emph{value sets} or
%\emph{value summaries}, allow multiple heaps to be represented
%simultaneously with a higher degree of precision than afforded by
%traditional techniques for shape analysis. Some of these previous
%attempts, however, are unable to handle aliasing in heaps due to a
%recursive definition of objects~\cite{Xie:2005}, while others require
%a set of heaps as input and are unable to initialize heaps
%on-the-fly~\cite{Dillig:2011,Tillmann:2008}.  As is typical in static
%analysis, over approximation of the heap representations alleviates
%some of these limitations but often leads to infeasible heaps. Also,
%most static analysis techniques for heap updates require a rewrite of
%the constraint system making it prohibitive for use in the context of
%symbolic execution.

%In this work, to effectively represent multiple heaps simultaneously
%in the context of symbolic execution, we define a novel heap
%representation and an algorithm to initialize and update the heap
%on-the-fly. In essence, our summary heap approach supports aliasing,
%does not require constraint rewriting on heap updates, and reduces
%non-determinism in the search space during symbolic execution.
%
%We represent the heap as a bipartite graph consisting of references
%and locations. Each reference is able to point to multiple locations,
%where each edge is predicated on constraints over aliasing between
%references. Each location points to a single reference for a given
%field. The use of a bipartite graph affords two key advantages: (i) it
%allows for a non-recursive definition of objects which enables us to
%support aliasing, and (ii) it does not introduce auxiliary variables
%or require rewriting of non-local constraints during updates to the
%heap.
%
%The summary heap algorithm defines how the bipartite graph is updated
%during lazy initialization. Unlike GSE, however, the summary heap
%algorithm introduces non-determinism in the search only at points of
%divergence in the control flow graph. These points of non-determinism
%are at field accesses that lead to null-pointer exceptions and at
%comparisons of references. The former represents a divergence due to
%exceptional control flow while the latter is due to program structure.
%The combination of the heap representation and the summary heap
%algorithm enables us to improve the efficiency of symbolic execution
%of heap manipulating programs over state-of-the-art techniques.
%
%The summary heap algorithm is sound and complete with respect to
%properties that are provable using GSE. This proof is accomplished by
%showing the existence of a bisimulation between states in GSE and
%states in the summary heap algorithm.  A preliminary implementation in
%Java Pathfinder (JPF) shows that in general, the heap summary
%algorithm improves over other state-of-the-art techniques, and in some
%instances, the improvement is remarkable: two-orders of magnitude
%reduction in running time. More importantly though, the heap summary
%enables the initialization of larger more complex heaps than
%previously possible as shown in our results.


%\section{Background and Motivation}
\label{motivations}

In this section we present the background on state of the art
techniques that have been developed to handle data non-determinism
arising from complex data structures. We present an overview of lazy
initialization and lazier\# initialization. We also present a brief
description of the two bounding strategies used in symbolic execution
in heap manipulating programs. Next we present a motivating examples
where current concrete initialization of the heap structures struggle
to scale to medium sized program due to non-determinism introduced in
the symbolic execution tree. We use this example to motivate the need
for a more truly symbolic and compact representation of the heap in a
manner similar to that of primitive types.

Generalized symbolic execution technique generates a concrete
representation of connected memory structures using only the implicit
information from the program itself.  In the original lazy
initialization algorithm, symbolic execution explores different heap
shapes by concretizing the heap at the first memory access (read) to
an un-initialized symbolic object. At this point, a non-deterministic
choice point of concreate heap locations is created that includes: (a)
null, (b) an access to a new instance of the object, and (c) aliases
to other type-compatible symbolic objects that have been concretized
along the same execution path~\cite{GSE:TACAS2003}.  The number of
choices explored in lazy initialization greatly increases the
non-determinism and often makes the exploration of the program state
space intractable.

%Various improvements have been proposed to reduce the amount of
%nondeterminism from lazy initialization. One such optimization,
%introduced in \cite{Kiasan06} , is called lazier\#
%initialization.
The Lazier\# algorithm is an improvement of the lazy initialization
and it pushes the non-deterministic choices further into the execution
tree. In the case of a memory access to an uninitialized reference
location, by default, no choice point is created. Instead, the read
returns a unique symbolic reference representing the contents of the
location. The reference may assume any one of three states:
uninitialized, non-null, or initialized. The reference is returned in
an uninitialized state, and only in a subsequent memory access is the
reference concretely initialized.

\section{Javalite}
The algorithm for generalized symbolic execution with symbolic heaps
is presented using Javalite. Javalite is a syntactic machine defined
as rewrites on a string \cite{saints-MS}. The semantics are small-step
using a CESK model with a (C)ontrol string representing the expression
being evaluated, an (E)nvironment for local variables, a (S)tore for
the heap, and a (K)ontinuation for what is to be executed next. The
semantics of the original Javalite machine, and its machine syntax,
are changed to reflect generalized symbolic execution and then
generalized execution with symbolic heaps.

The surface syntax (input) for Javalite is in
\figref{fig:surface-syntax} with the machine syntax (state) in
\figref{fig:machine-syntax}. Terminals are in bold face while
non-terminals are italicized. Ellipses indicate zero or more
repetitions. Tuples omit the commas for compactness. The language
itself uses s-expressions for convenience.

\begin{figure}
\begin{center}
\cfgstart
\cfgrule{P}{\lp $\mu$ \lp \cfgnt{C} \cfgnt{m}\rp\rp}
\cfgrule{$\mu$}{(\cfgnt{CL} ...)}
\cfgrule{T}{\cfgt{bool} \cfgor \cfgnt{C}}
\cfgrule{CL}{\lp\cfgt{class} \cfgnt{C} \lp\lb\cfgnt{T} \cfgnt{f}\rb ...\rp \lp\cfgnt{M} ...\rp}
\cfgrule{M}{\lp\cfgnt{T} \cfgnt{m} \lb\cfgnt{T} \cfgnt{x}\rb\  e\rp}
\cfgrule{e}{\cfgnt{x}
\cfgor{\lp\cfgt{new} \cfgnt{C}\rp}
\cfgor{\lp\cfgnt{e} \cfgt{\$} \cfgnt{f}\rp}
\cfgor{\lp\cfgnt{x} \cfgt{\$} \cfgnt{f} \cfgt{:=} \cfgnt{e}\rp}
\cfgor{\lp\cfgnt{e} \cfgt{=} \cfgnt{e}\rp}}
\cfgorline{\lp\cfgt{if} \cfgnt{e} \cfgnt{e} \cfgt{else} \cfgnt{e}\rp 
\cfgor {\lp\cfgt{var} \cfgnt{T} \cfgnt{x} \cfgt{:=} \cfgnt{e} \cfgt{in} \cfgnt{e}\rp}
\cfgor {\lp\cfgnt{e} \cfgt{@} \cfgnt{m} \cfgnt{e} \rp}}
\cfgorline{\lp\cfgnt{x} \cfgt{:=} \cfgnt{e}\rp
\cfgor{\lp\cfgt{begin} \cfgnt{e} ...\rp}
\cfgor{\cfgnt{v}}}
\cfgrule{x}{\cfgt{this} \cfgor \cfgnt{id}}
\cfgrule{f,m,C}{\cfgnt{id}}
%\cfgrule{m}{\cfgnt{id}}
%\cfgrule{C}{\cfgnt{id}}
\cfgrule{v}{\cfgnt{r} \cfgor \cfgt{null} \cfgor \cfgt{true} \cfgor \cfgt{false} \cfgor \cfgt{error}}
\cfgrule{r}{\cfgt{number}}
\cfgrule{id}{\cfgt{variable-not-otherwise-mentioned}}
\cfgend
\end{center}
\caption{The Javalite surface syntax.}
\label{fig:surface-syntax}
\end{figure}

\begin{figure}
\begin{center}
\cfgstart
\cfgrule{e}{\lp .... \cfgor \lp\cfgt{raw} \cfgnt{v} \cfgt{@} \cfgnt{m} \cfgnt{v} \rp\rp}
%    ;; eval syntax
%  (object (C [f loc] ...))
%  (hv v
%      object)
\cfgrule{$\phi$}{\cfgnt{constraint}}
\cfgrule{l}{\cfgt{number}}
%  (h mt
%     (h [loc -> hv]))
%  (η mt
%     (η [x -> loc]))
\cfgrule{s}{\lp$\mu$ \cfgnt{L} \cfgnt{R} $\eta$ \cfgnt{e} \cfgnt{k}\rp}
\cfgrule{k}{\cfgt{end}}
\cfgorline{\lp \cfgt{*} \cfgt{\$} \cfgnt{f} $\rightarrow$ \cfgnt{k}\rp}
%     (* @ m (e ...) -> k)
%     (v @ m (v ...) * (e ...) -> k)
%     (* == e -> k)
%     (v == * -> k)
%     (x := * -> k)
%     (x $ f := * -> k)
%     (if * e else e -> k)
%     (var T x := * in e -> k)
%     (begin * (e ...) -> k)
%     (pop η k))
\cfgend
\end{center}
\caption{The machine syntax for Javalite.}
\label{fig:machine-syntax}
\end{figure}


A Javalite program, $\cfgnt{P}$, is a registry of classes, $\mu$, with
a tuple indicating a class, $\cfgnt{C}$, and a method, $\cfgnt{m}$,
where execution starts. The only primitive type in Javalite is
Boolean. Classes have fields, $([\cfgnt{T}\ \cfgnt{f}]\ ...)$, and
methods, $(\cfgnt{M}\ ...)$. Methods only take a single
parameter. Expressions, $\cfgnt{e}$, are somewhat typical with
$\cfgt{:=}$ indicating assignment, $\cfgt{=}$ indicating comparison,
$\cfgt{\$}$ as the dot-operator for field access, and $\cfgt{@}$ as
the dot-operator for method invocation. There is no explicit return
statement in Javalite; rather, the value of the last expression is
used as a return value. A variable is always indicated by \cfgnt{x}
and a value by \cfgnt{v}. A value can be a reference in the heap,
$\cfgnt{r}$, or any of the special values. It is assumed that only
type correct programs are given as input to the machine.

The state of the Javalite machine (\figref{fig:machine-syntax}),
$\cfgnt{s}$, includes the program registry\footnote{The registry $\mu$
  is implied in the rewrite rules rather that states for
  compactness.}, a pair of functions defining the heap
$\lp\cfgnt{L}\ \cfgnt{R}\rp$, which is the store, a constraint,
$\phi$, defined over references in the heap, the environment, $\eta$,
for local variables, and the continuation $\cfgnt{k}$. The more
complex structures such as the environment are defined as lists which
start with empty, \cfgnt{mt}. The rewrite rules that define the
semantics treat these lists as partial functions. As such,
$\eta(\cfgnt{x}) = \cfgnt{r}$ is the reference mapped to the variable
$\cfgnt{x}$. The notation $\eta^\prime = \eta[\cfgnt{x} \mapsto
  \cfgnt{v}]$ defines a new partial function $\eta^\prime$ that is
just like $\eta$ only the variable $\cfgnt{x}$ now maps to
$\cfgnt{v}$.

The continuation $\cfgnt{k}$ helps accomplish the small-step semantics
by indicating where the hole, $\cfgt{*}$, is located, by storing
temporary computations, and by keeping track of the next
computation. For example, the continuation
$\lp\cfgt{*}\ \cfgt{\$}\ \cfgnt{f} \rightarrow \cfgnt{k}\rp$ indicates
that the machine is currently computing the expression for the object
reference on which the field $\cfgnt{f}$ is going to be accessed. Once
the field access is complete, the machine continues with $\cfgnt{k}$.





\subsection{Heaps as Bipartite Graphs}
An important aspect, and a novel contribution, of symbolic heaps its
representation as a labeled bipartite graph consisting of
references, $\cfgnt{r}$, and constraint location pairs,
$\lp\phi\ \cfgnt{l}\rp$. The heap structure conveys that the actual
location pointed to by a reference is conditional on aliasing
relationships within the heap. In this way, the graph represents sets
of heaps with the conditions under which those heaps exist.

The machine syntax in \figref{fig:machine-syntax} defines that graph
in $\cfgnt{L}$, the location map, and $\cfgnt{R}$, the reference
map. As done with the environment, $\cfgnt{L}$ and $\cfgnt{R}$ are
treated as partial functions where $\cfgnt{L}(r) =
\{(\phi\ \cfgnt{l})\ ...\}$ is the set of location-constraint pairs in
the heap associated with the given reference, and
$\cfgnt{R}(\cfgnt{l},\cfgnt{f}) = \cfgnt{r}$ is the reference
associated with the given location-field pair in the heap. Predicate
calculus is used to describe the more complex updates to the heap in
the semantics.

There are several properties of the heap that are invariant and
important to the correctness of the algorithm.
\begin{compactitem}
\item ADD REFERENCES ARE IMMUTABLE
\end{compactitem}

The location $\cfgnt{l}_\mathit{null}$ is a special location in the
heap to represent null. It has a companion reference
$\cfgnt{r}_\mathit{null}$. The initial heap for the machine is defined
such that $\cfgnt{L}(\cfgnt{r}_\mathit{null}) =
\{(\cfgt{true}\ \cfgnt{l}_\mathit{null})\}$

DEFINE $\rsym$ and $\rgse$.

%\section{Initialization of Symbolic References}

In this section we present the Javalite rewrite rules for the concrete
as well as summary initialization of symbolic references. The
initialization rules are defined on the bi-partite graph consisting of
references and locations. The lazy initialization of symbolic
references consists of three key points of non-determinism where each
symbolic reference can be initialized non-deterministically to null, a
new instance of the symbolic reference, or aliases to symbolic
references of the same type previously initialized. The initialization
in GSE consists of creating branches in the execution tree for all the
non-deterministic choices. On the other hand, the heap summarization
approach creates a single branch that contains the summarization for
all the initialization in a single bi-partitate graph.

\begin{figure*}[t]
\begin{center}
\mprset{flushleft}
\begin{mathpar}
	\inferrule[Initialize (null)]{
	  \Lambda = \mathbb{UN}\lp \cfgnt{L}, \cfgnt{R}, \cfgnt{r}, \cfgnt{f}\rp \\
      \Lambda \neq \emptyset\\\\
      \cfgnt{r}^\prime = \mathrm{fresh}_r\lp \rp\\ 
      \theta_\mathit{null} = \{ \lp \cfgt{true}\ l_\mathit{null}\rp \} \\\\
      l_x = \mathrm{min}_l\lp \Lambda\rp \\\\
      \phi_g^\prime = \lp\phi_g \wedge \cfgnt{r}^\prime = \cfgnt{r}_\mathit{null}\rp
    }{
      \lp \cfgnt{L}\ \cfgnt{R}\ \phi_g\ \cfgnt{r}\ \cfgnt{f}\ \cfgnt{C}\rp  \rightarrow_I 
      \lp \cfgnt{L}[\cfgnt{r}^\prime \mapsto \theta_\mathit{null}]\ \cfgnt{R}[ \lp l_x,\cfgnt{f}\rp  \mapsto \cfgnt{r}^\prime]\ \phi_g^\prime\ \cfgnt{r}\ \cfgnt{f}\ \cfgnt{C}\rp 
	}
\and
	\inferrule[Initialize (new)]{
	  \Lambda = \mathbb{UN}\lp \cfgnt{L}, \cfgnt{R}, \cfgnt{r}, \cfgnt{f}\rp \\
      \Lambda \neq \emptyset\\
      \lp\phi_x\ \cfgnt{l}_x\rp = \mathrm{min}_l\lp \Lambda\rp\\\\
      \cfgnt{r}_f = \mathrm{init}_r\lp \rp\\
      l_f = \mathrm{fresh}_l\lp \cfgnt{C}\rp \\\\
      \rho = \{ \lp\cfgnt{r}_a\ l_a\rp \mid \mathrm{isInit}\lp \cfgnt{r}_a\rp  \wedge \cfgnt{r}_a = \mathrm{min}_r\lp \cfgnt{R}^{-1}[l_a]\rp \wedge \mathrm{type}\lp l_a\rp  = \cfgnt{C} \}\\\\
      \theta_\mathit{new} = \{\lp \cfgt{true}\ l_f\rp \} \\\\
      \cfgnt{R}^\prime = \cfgnt{R}[\forall \cfgnt{f} \in \mathit{fields}\lp \mathrm{C}\rp \ \lp \lp l_f\ \cfgnt{f}\rp  \mapsto \cfgnt{r}_\mathit{un} \rp ] \\\\
      \phi_g^\prime = \lp\phi_g \wedge \cfgnt{r}_f \neq \cfgnt{r}_\mathit{null} \wedge \lp \wedge_{\lp\cfgnt{r}_a\ l_a\rp \in \rho} \cfgnt{r}_f \ne \cfgnt{r}_a\rp\rp
    }{
      \lp \cfgnt{L}\ \cfgnt{R}\ \phi_g\ \cfgnt{r}\ \cfgnt{f}\ \cfgnt{C}\rp  \rightarrow_I 
      \lp \cfgnt{L}[\cfgnt{r}_f \mapsto \theta_\mathit{new}]\ \cfgnt{R}^\prime[ \lp l_x,\cfgnt{f}\rp  \mapsto \cfgnt{r}_f ]\ \phi_g^\prime\ \cfgnt{r}\ \cfgnt{f}\ \cfgnt{C}\rp 
	}
\and
	\inferrule[Initialize (alias)]{
  	  \Lambda = \mathbb{UN}\lp \cfgnt{L}, \cfgnt{R}, \cfgnt{r}, \cfgnt{f}\rp \\
      \Lambda \neq \emptyset\\
      \lp\phi_x\ \cfgnt{l}_x\rp = \mathrm{min}_l\lp \Lambda\rp\\\\
      \cfgnt{r}^\prime = \mathrm{fresh}_r\lp \rp\\\\
      \rho = \{ \lp\cfgnt{r}_a\ l_a\rp \mid \mathrm{isInit}\lp \cfgnt{r}_a\rp  \wedge \cfgnt{r}_a = \mathrm{min}_r\lp \cfgnt{R}^{-1}[l_a]\rp \wedge \mathrm{type}\lp l_a\rp  = \cfgnt{C} \}\\\\
      \lp\cfgnt{r}_a\ l_a\rp \in \rho \\
      \theta_\mathit{alias} = \{ \lp \cfgt{true}\ l_a\rp\}\\\\
      \phi^\prime_g = \lp\phi_g \wedge \cfgnt{r}^\prime \neq \cfgnt{r}_\mathit{null} \wedge \cfgnt{r}^\prime = \cfgnt{r}_a \wedge \lp \wedge_{\lp \cfgnt{r}^{\prime}_a\ l_a\rp  \in \rho\ \lp \cfgnt{r}^{\prime}_a \neq \cfgnt{r}_a\rp } \cfgnt{r}^\prime \neq \cfgnt{r}^{\prime}_a \rp\rp
    }{
      \lp \cfgnt{L}\ \cfgnt{R}\ \phi_g\ \cfgnt{r}\ \cfgnt{f}\ \cfgnt{C}\rp  \rightarrow_I 
      \lp \cfgnt{L}[\cfgnt{r}^\prime \mapsto \theta_\mathit{alias}]\ \cfgnt{R}[ \lp l_x,\cfgnt{f}\rp  \mapsto \cfgnt{r}^\prime ]\ \phi_g^\prime\ \cfgnt{r}\ \cfgnt{f}\ \cfgnt{C}\rp 
	}
\and
	\inferrule[Initialize (end)]{
	  \Lambda = \mathbb{UN}\lp \cfgnt{L}, \cfgnt{R}, \cfgnt{r}, \cfgnt{f}\rp \\
      \Lambda = \emptyset
    }{
      \lp \cfgnt{L}\ \cfgnt{R}\ \phi_g\ \cfgnt{r}\ \cfgnt{f}\ \cfgnt{C}\rp  \rightarrow_I 
      \lp \cfgnt{L}\ \cfgnt{R}\ \phi_g\ \cfgnt{r}\ \cfgnt{f}\ \cfgnt{C}\rp 
	}
\end{mathpar}
\end{center}
\caption{The initialization machine, $s ::= \lp\cfgnt{L}\ \cfgnt{R}\ \phi_g\ \cfgnt{r}\ \cfgnt{f}\rp$, with $s \rightarrow_I^* s^\prime$ indicating stepping the machine until the state does not change.}
\label{fig:lazyInit}
\end{figure*}

\begin{figure*}[t]
\begin{center}
\mprset{flushleft}
\begin{mathpar}
	\inferrule[Summarize]{
	\Lambda = \mathbb{UN}\lp \cfgnt{L}, \cfgnt{R}, \cfgnt{r}, \cfgnt{f}\rp \\
      \Lambda \neq \emptyset \\
      \lp\phi_x\ \cfgnt{l}_x\rp = \mathrm{min}_l\lp \Lambda\rp\\
      \cfgnt{r}_f = \mathrm{init}_r\lp \rp \\
      l_f  = \mathrm{fresh}_l\lp \mathrm{C}\rp\\\\
      \rho = \{ \lp \cfgnt{r}_a\ \phi_a\ l_a\rp  \mid \mathrm{isInit}\lp \cfgnt{r}_a\rp  \wedge\cfgnt{r}_a = \mathrm{min}_r\lp \cfgnt{R}^{-1}[l_a]\rp \wedge \lp \phi_a\ l_a\rp  \in \cfgnt{L}\lp \cfgnt{r}_a\rp \wedge \mathrm{type}\lp l_a\rp  = \mathrm{C} \} \\\\
      \theta_\mathit{null} = \{ \lp \phi\ l_\mathit{null}\rp  \mid \phi = \lp \phi_x \wedge \cfgnt{r}_f = \cfgnt{r}_\mathit{null} \rp  \} \\\\
      \theta_\mathit{new} = \{\lp \phi\ l_f\rp  \mid \phi = \lp \phi_x \wedge \cfgnt{r}_f \neq \cfgnt{r}_\mathit{null} \wedge \lp \wedge_{\lp \cfgnt{r}^\prime_a,\ \phi^\prime_a,\ l^\prime_a\rp  \in \rho} \cfgnt{r}_f \ne \cfgnt{r}^\prime_a\rp \rp \}\\\\
      \theta_\mathit{alias} = \{ \lp \phi\ l_a\rp  \mid \exists\cfgnt{r}_a\ \lp\exists \phi_a\ \lp\lp\cfgnt{r}_a\ \phi_a\ l_a\rp  \in \rho \wedge \phi = \lp \phi_x \wedge \phi_a \wedge \cfgnt{r}_f \neq \cfgnt{r}_\mathit{null} \wedge \cfgnt{r}_f = \cfgnt{r}_a \wedge \lp \wedge_{\lp \cfgnt{r}^{\prime}_a\ \phi^{\prime}_a\ l^{\prime}_a\rp  \in \rho\ \lp \cfgnt{r}^\prime_a \neq \cfgnt{r}_a\rp } \cfgnt{r}_f \neq \cfgnt{r}^{\prime}_a \rp \rp \rp \rp \} \\\\
      \theta_\mathit{orig} = \{\lp\phi\ \cfgnt{l}_\mathit{orig}\rp \mid \exists \phi_\mathit{orig} \lp \lp\phi_\mathit{orig}\ \cfgnt{l}_\mathit{orig}\rp \in \cfgnt{L}\lp\cfgnt{R}\lp\cfgnt{l}_x,\cfgnt{f}\rp\rp \wedge \phi = \lp\neg\phi_x \wedge \phi_\mathit{orig}\rp\}\\\\ 
      \theta = \theta_\mathit{alias} \cup \theta_\mathit{new} \cup \theta_\mathit{null} \cup \theta_\mathit{old} \\
\cfgnt{R}^\prime = \cfgnt{R}[\forall \cfgnt{f} \in \mathit{fields}\lp \mathrm{C}\rp \ \lp \lp l_f\ \cfgnt{f}\rp  \mapsto \cfgnt{r}_\mathit{un} \rp ]
    }{
      \lp \cfgnt{L}\ \cfgnt{R}\ \cfgnt{r}\ \cfgnt{f}\ \cfgnt{C}\rp \rightarrow_S 
      \lp \cfgnt{L}[\cfgnt{r}_f \mapsto \theta]\ \cfgnt{R}^{\prime}[ \lp l_x,\cfgnt{f}\rp  \mapsto \cfgnt{r}_f ]\ \cfgnt{r}\ \cfgnt{f}\ \cfgnt{C}\rp
	}
\and
	\inferrule[Summarize (end)]{
	  \Lambda = \{ l \mid \exists \phi\ \lp \lp \phi\ l\rp  \in \cfgnt{L}\lp \cfgnt{r}\rp  \wedge  \cfgnt{R}\lp l,\cfgnt{f}\rp  = \bot\ \rp\}\\
      \Lambda = \emptyset
    }{
      \lp \cfgnt{L}\ \cfgnt{R}\ \cfgnt{r}\ \cfgnt{f}\ \cfgnt{C}\rp  \rightarrow_S
      \lp \cfgnt{L}\ \cfgnt{R}\ \cfgnt{r}\ \cfgnt{f}\ \cfgnt{C}\rp 
	}
\end{mathpar}
\end{center}
\caption{The summary machine, $s ::= \lp\cfgnt{L}\ \cfgnt{R}\ \cfgnt{r}\ \cfgnt{f}\ \cfgnt{C}\rp$, with $s\rightarrow_S^*s^\prime$ indicating stepping the machine until the state does not change.}
\label{fig:symInit}
\end{figure*}



The initialization rules are invoked when an uninitialized field in a
symbolic reference is accessed. The function $\mathbb{UN}(\cfgnt{L},
\cfgnt{R}, \cfgnt{r}, \cfgnt{f}) = \{\cfgnt{l}\ ...\}$ returns
constraint-location pairs in which the field $\cfgnt{f}$ is
uninitialized:
\[
\begin{array}{rcl}
\mathbb{UN}(\cfgnt{L}, \cfgnt{R}, \cfgnt{r}, \cfgnt{f}) & = &\{ \lp\phi\ \cfgnt{l}\rp \mid \lp \phi\ \cfgnt{l}\rp  \in \cfgnt{L}\lp \cfgnt{r}\rp  \wedge \\
& & \ \ \ \ \exists \phi^\prime \lp \lp \phi^\prime\ \cfgnt{l}_\mathit{un}\rp  \in \cfgnt{L}\lp \cfgnt{R}\lp l,\cfgnt{f}\rp\rp \wedge \\
& & \ \ \ \ \ \ \ \ \mathbb{S}\lp \phi \wedge \phi^\prime \rp\rp\}\\
\end{array}
\]
where $\mathbb{S}(\phi)$ returns true if $\phi$ is
satisfiable. Intutively, for the reference, $\cfgnt{r}$, it constructs
the set, $\theta$, that contains all contraint-location pairs that
point to the field $\cfgnt{f}$ and $\cfgnt{f}$ points to
$\cfgnt{l}_\mathit{un}$. The cardinality of the set, $\theta$ is never
greater than one in GSE and the constraint is always satisfiable
because all constraints are constant. This property is relaxed in GSE
with heap summaries.

The rules in~\figref{fig:lazyInit} present the rewrite rules for the
concrete initialization of symbolic heap objects.  These rules are
invoked until a fix pointed is reached. 

The initialize (null) rewrite rule in~\figref{fig:lazyInit} first
checks that the field, $\cfgnt{r}$ is uninitialized. The fresh method
returns a new input heap reference from the partition 


\section{Accessing and Writing to Field References}

\section{Equality and InEquality of References}

\begin{figure*}[t]
\begin{center}
\mprset{flushleft}
\begin{mathpar}
	\inferrule[Field Access]{
      \{\lp\phi\ l\rp\} = \cfgnt{L}\lp\cfgnt{r}\rp\\
      l \neq \cfgnt{l}_\mathit{null}\\
      \cfgnt{C} = \mathrm{type}\lp\cfgnt{l},\cfgnt{f}\rp\\\\
      \lp \cfgnt{L}\ \cfgnt{R}\ \cfgnt{r}\ \cfgnt{f}\ \cfgnt{C}\rp \rinit^*
      \lp \cfgnt{L}^\prime\ \cfgnt{R}^\prime\ \cfgnt{r}\ \cfgnt{f}\  \cfgnt{C}\rp \\\\ 
      \{\lp\phi^\prime\ l^\prime\rp\} = \cfgnt{L}^\prime\lp\cfgnt{R}^\prime\lp l,\cfgnt{f}\rp\rp \\
      \cfgnt{r}^\prime = \mathrm{stack}_r\lp\rp \\
    }{
      \lp \cfgnt{L}\ \cfgnt{R}\ \phi_g\ \eta\ \cfgnt{r}\ \lp \cfgt{*}\ \cfgt{\$}\ \cfgnt{f} \rightarrow \cfgnt{k}\rp \rp  \rightarrow_\ell \\\\
      \lp \cfgnt{L}^\prime[\cfgnt{r}^\prime \mapsto \lp\phi^\prime\ l^\prime\rp]\ \cfgnt{R}^\prime\ \phi_g^\prime\ \eta\ \cfgnt{r}^\prime\ \cfgnt{k}\rp 
	}
\and
	\inferrule[Field Write]{
      \cfgnt{r}_x = \eta\lp \cfgnt{x}\rp\\ 
      \theta = \{\lp\phi\ l\rp\} = \cfgnt{L}\lp\cfgnt{r}_x\rp \\\\
      l \neq \cfgnt{l}_\mathit{null}\\
      \cfgnt{r}^\prime = \mathrm{fresh}_r\lp\rp\\
    }{
      \lp \cfgnt{L}\ \cfgnt{R}\ \phi_g\ \eta\ \cfgnt{r}\ \lp \cfgnt{x}\ \cfgt{\$}\ \cfgnt{f}\ \cfgt{:=}\ \cfgt{*}\ \rightarrow\ \cfgnt{k}\rp \rp  \rightarrow_\ell \\\\
      \lp \cfgnt{L}[\cfgnt{r}^\prime \mapsto \theta]\ \cfgnt{R}[\lp l\ \cfgnt{f}\rp  \mapsto \cfgnt{r}^\prime]\ \phi_g\ \eta\ \cfgnt{r}\ \cfgnt{k}\rp 
	}
\and
  \inferrule[Equals (reference-true)]{
    \cfgnt{L}\lp \cfgnt{r}_0\rp = \cfgnt{L}\lp \cfgnt{r}_1\rp\\
    \phi^\prime = \lp\phi \wedge r_0 = r_1\rp
    }{
    \lp \cfgnt{L}\ \cfgnt{R}\ \phi\ \eta\ \cfgnt{r}_0\ \lp \cfgnt{r}_1\ \cfgt{=}\ \cfgt{*} \rightarrow \cfgnt{k}\rp \rp  \rightarrow_\ell \\\\
    \lp \cfgnt{L}\ \cfgnt{R}\ \phi^\prime\ \eta\ \cfgt{true}\ \cfgnt{k}\rp 
    }
\and
    \inferrule[Equals (reference-false)]{
    \cfgnt{L}\lp \cfgnt{r}_0\rp \neq \cfgnt{L}\lp \cfgnt{r}_1\rp\\
    \phi^\prime = \lp\phi \wedge r_0 \neq r_1\rp
   }{
    \lp \cfgnt{L}\ \cfgnt{R}\ \phi\ \eta\ \cfgnt{r}_0\ \lp \cfgnt{r}_1\ \cfgt{=}\ \cfgt{*} \rightarrow \cfgnt{k}\rp \rp  \rightarrow_\ell \\\\
    \lp \cfgnt{L}\ \cfgnt{R}\ \phi^\prime\ \eta\ \cfgt{false}\ \cfgnt{k}\rp 
    }	
\end{mathpar}
\end{center}
\caption{GSE with lazy initialization indicated by $\rgse = \rightarrow_\ell \cup \rcom$.}
\label{fig:lazy}
\end{figure*}




\subsection{Accessing a Field Reference}



\newsavebox{\boxPFAFW}
\savebox{\boxPFAFW}{
%\begin{figure}[t]
%\begin{center}
\mprset{flushleft}
\begin{mathpar}
	\inferrule[Field Access]{
      \exists \lp \phi\ l\rp \in \cfgnt{L}\lp \cfgnt{r}\rp\ \lp l \neq l_{\mathit{null}} \wedge \mathbb{S}\lp \phi \wedge \phi_g\rp \rp \\\\
      \theta = \{ \phi \mid \lp \phi\ l_\mathit{null} \rp \wedge \mathbb{S}\lp \phi \wedge \phi_g\rp \} \\\\
      \phi_g^\prime = \phi_g \wedge (\wedge_{\phi \in \theta} \neg \phi) \\\\
      \{\cfgnt{C}\} = \{\cfgnt{C} \mid \exists \lp \phi\ l\rp  \in \cfgnt{L}\lp \cfgnt{r}\rp\ \lp\cfgnt{C} = \mathrm{type}\lp \cfgnt{l},\cfgnt{f}\rp\rp\} \\\\
      \lp \cfgnt{L}\ \cfgnt{R}\ \cfgnt{r}\ \cfgnt{f}\ \cfgnt{C}\rp \rsum^* \lp \cfgnt{L}^\prime\ \cfgnt{R}^\prime\ \cfgnt{r}\ \cfgnt{f}\ \cfgnt{C}\rp \\
      \cfgnt{r}^\prime = \mathrm{stack}_r\lp \rp
    }{
      \lp \cfgnt{L}\ \cfgnt{R}\ \phi_g\ \eta\ \cfgnt{r}\ \lp \cfgt{*}\ \cfgt{\$}\ \cfgnt{f} \rightarrow \cfgnt{k}\rp \rp  \rightarrow_\mathit{A}
      \lp \cfgnt{L}^\prime[\cfgnt{r}^\prime \mapsto \mathbb{VS}\lp \cfgnt{L}^\prime,\cfgnt{R}^\prime,\cfgnt{r},\cfgnt{f},\phi_g^\prime\rp ]\ \cfgnt{R}^\prime\ \phi_g^\prime\ \eta\ \cfgnt{r}^\prime\ \cfgnt{k}\rp 
	}
\and
	\inferrule[Field Access (NULL)]{
      \exists \lp \phi\ l\rp \in \cfgnt{L}\lp \cfgnt{r}\rp\ \lp l = l_{\mathit{null}} \wedge \mathbb{S}\lp \phi \wedge \phi_g\rp \rp
    }{
      \lp \cfgnt{L}\ \cfgnt{R}\ \phi_g\ \eta\ \cfgnt{r}\ \lp \cfgt{*}\ \cfgt{\$}\ \cfgnt{f} \rightarrow \cfgnt{k}\rp \rp  \rightarrow_\mathit{A}
      \lp \cfgnt{L}\ \cfgnt{R}\ \phi_g\ \eta\ \cfgt{error}\ \cfgt{end} \rp
	}
\and
	\inferrule[Field Write]{
      \cfgnt{r}_x = \eta\lp\cfgnt{x}\rp \\
      \exists \lp \phi\ l\rp \in \cfgnt{L}\lp \cfgnt{r}_x\rp\ \lp l \neq l_{\mathit{null}} \wedge \mathbb{S}\lp \phi \wedge \phi_g\rp \rp \\\\
      \theta = \{ \phi \mid \lp \phi\ l_\mathit{null} \rp \wedge \mathbb{S}\lp \phi \wedge \phi_g\rp \} \\\\
      \phi_g^\prime = \phi_g \wedge (\wedge_{\phi \in \theta} \neg \phi) \\\\
      \Psi_x =\{\lp \phi\ l\ \cfgnt{r}_\mathit{cur} \rp  \mid \lp \phi\ \cfgnt{l}\rp  \in \cfgnt{L}\lp \cfgnt{r}_x\rp  \wedge \cfgnt{r}_\mathit{cur} = \cfgnt{R}\lp l,\cfgnt{f}\rp  \}\\\\
      X = \{ \lp l\ \theta \rp  \mid \exists \phi\ \lp \lp \phi\ l\ \cfgnt{r}_\mathit{cur} \rp \in \Psi_x \wedge \theta = \mathbb{ST}\lp \cfgnt{L},\cfgnt{r},\phi,\phi_g^\prime\rp  \cup \mathbb{ST}\lp \cfgnt{L},\cfgnt{r}_\mathit{cur},\neg\phi,\phi_g^\prime\rp \rp  \}\\\\
      \cfgnt{R}^{\prime} = \cfgnt{R}[\forall \lp l\ \theta \rp  \in X\ \lp \lp l\ \cfgnt{f}\rp  \mapsto \mathrm{fresh}_r\lp \rp \rp ]\\\\
      \cfgnt{L}^{\prime} = \cfgnt{L}[\forall \lp l\ \theta \rp  \in X\ \lp \exists \cfgnt{r}_\mathit{targ}\ \lp \cfgnt{r}_\mathit{targ} = \cfgnt{R}^\prime\lp l,\cfgnt{f}\rp \wedge \lp\cfgnt{r}_\mathit{targ} \mapsto \theta\rp  \rp \rp ]
    }{
      \lp \cfgnt{L}\ \cfgnt{R}\ \phi_g\ \eta\ \cfgnt{r}\ \lp \cfgnt{x}\ \cfgt{\$}\ \cfgnt{f}\ \cfgt{:=}\ \cfgt{*}\ \rightarrow\ \cfgnt{k}\rp \rp  \rightarrow_\mathit{FW}
      \lp \cfgnt{L}^{\prime}\ \cfgnt{R}^{\prime}\ \phi_g^\prime\ \eta\ \cfgnt{r}\ \cfgnt{k}\rp 
	}	
\and
	\inferrule[Field Write (NULL)]{
      \cfgnt{r}_x = \eta\lp \cfgnt{x}\rp \\
      \exists \lp \phi\ l\rp \in \cfgnt{L}\lp \cfgnt{r}_x\rp\ \lp l \neq l_{\mathit{null}} \wedge \mathbb{S}\lp \phi \wedge \phi_g\rp \rp
    }{
      \lp \cfgnt{L}\ \cfgnt{R}\ \phi_g\ \eta\ \cfgnt{r}\ \lp \cfgnt{x}\ \cfgt{\$}\ \cfgnt{f}\ \cfgt{:=}\ \cfgt{*}\ \rightarrow\ \cfgnt{k}\rp \rp  \rightarrow_\mathit{FW}
      \lp \cfgnt{L}\ \cfgnt{R}\ \phi_g\ \eta\ \cfgt{error}\ \cfgt{end}\rp
	}	
\end{mathpar}}
%\end{center}
%\caption{Precise symbolic heap summaries from symbolic execution indicated by $\rsym = \rightarrow_\mathit{FA} \cup \rightarrow_\mathit{FW} \cup \rightarrow_\mathit{EQ} \cup \rcom$.}
%\label{fig:symfield}
%\end{figure}




\begin{figure*}[t]
\begin{center}
\setlength{\tabcolsep}{50pt}
\begin{tabular}[c]{cc}
\usebox{\boxPFAFW} & 
%\scalebox{0.91}{\input{faYHeap.pdf_t}} &
\scalebox{0.91}{\input{fwXHeap.pdf_t}} \\ \\
(a) & (b)
\end{tabular}
\end{center}
\caption{field access for this.y and field write for this.x = this.y}
\label{fig:fHeap}
\end{figure*}

There are two rewrite rules in~\figref{fig:fHeap}(a), one for field
access and the other for field write. Let us consider the rewrite rule
for field access. The first check we perform is whether there exists a
constraint location pair for the $r$ being accessed such that the
location is not null and the constraint when conjuncted with the
global constraint is satisfiable. Next we extract all possible
constraints under which $r$ points to a null location such that the
constraint is satisfiable under the current global constraint,
$\phi_g$. The negation of these constraints are added to the global
constraint to create a new global constraint $\phi_g^\prime$. The
update to the global constraint ensures that access of the field $f$
happens only on non-null locations. The type $C$ of the field is
extracted and passed to rewrite rule that performs the initialization
of the symbolic heap. The initialization rules are described earlier
in this section. 

Recall that during the initialization the rewrite rules
in~\figref{fig:symHeap} add input references that map to fields being
initialized.  Once the initialization is complete, however, we create
a new local reference $r^\prime$. An important property of the
references in the bi-partiate graph is that they are
\emph{immutable}. Hence we de-reference the initialized input
reference, assign its value to the the new local reference, and return
the local reference. In order to de-reference a field $r.f$ we define
a helper function which is called the value set.

\begin{definition}
\label{def:VS}
The function $\mathbb{VS}(L,R,\phi_g,r,f)$ constructs the value-set given a
heap, reference, and desired field:
\[
\begin{array}{rcl}
  \mathbb{VS}(\cfgnt{L},\cfgnt{R},\phi_g,\cfgnt{r},\cfgnt{f}) & = & \{(\phi\wedge\phi^\prime\ \cfgnt{l}^\prime) \mid \\
  & & \ \ \ \ \exists \cfgnt{l}\ ((\phi\ l) \in L(r)\ \wedge \\
  & & \ \ \ \ \ \ \ \ \exists \cfgnt{r}^\prime ( \cfgnt{r}^\prime = R(\cfgnt{l},\cfgnt{f})\ \wedge \\
  & & \ \ \ \ \ \ \ \ \ \ \ \ (\phi^\prime\ l^\prime) \in \cfgnt{L}(\cfgnt{r}^\prime)\ \wedge\\
  & & \ \ \ \ \ \ \ \ \ \ \ \ \mathbb{S}(\phi\wedge\phi^\prime\wedge \phi_g)))\}
\end{array}
\]
where $\mathbb{S}(\phi)$ returns true if $\phi$ is satisfiable.
\end{definition}


In the post-condition of the rewrite rule we assign
the value set of input reference $r$ to the local reference $r^\prime$
and return the local reference $r^\prime$ in the next state.

Consider the graph shown in~\figref{fig:fHeap}(b) the reference
$r_2^i$ is created during the initialization of $\mathit{this}.y$. The
reference that is returned during the access of the field, however, is
$r_3^s$. The reference $r_3^s$ points to the value set of $r_2^i$
which are: $(\phi_{2a}, l_\mathit{null})$, $(\phi_{2b}, l_2)$, and
$(\phi_{2c}, l_1)$. The values of the constraints $\phi_{2a}$,
$\phi_{2b}$, and $\phi_{2c}$ are defined in~\figref{fig:initHeap}(d).

\subsection{Writing to a Field}

The reference $r_x$ is the base pointer whose field is being written
to while $r$ is the target reference. We look up the value of the base
reference in the environment $\eta(x)$. The set $\Psi_x$ is the set of
tuples of constraints, locations, and references which provide the
reference chains leading from $r_x$ to the reference of the field,
$r_\mathit{curr}$, being written to through $\phi$ and $l$.The goal is
to overwrite the $r_\mathit{curr}$ references with the target
references. Since the target of the write is $r$, we first check that the
location constraint pairs of $L(r)$ are satisfiable when accessed
through the $r_x$ chain. This is accomplished by the strengthing
function.

\begin{definition}
\label{def:ST}
The strengthen function $\mathbb{ST}(\cfgnt{L},\cfgnt{r},\phi,\phi_g)$ strengthens every
constraint from the reference $\cfgnt{r}$ with $\phi$ and keeps only location-constraint
pairs that are satisfiable after this strengthening with the inclusion of the global heap constraint $\phi_g$:
\[
\begin{array}{rcl} 
\mathbb{ST}(\cfgnt{L},\cfgnt{r},\phi,\phi_g) & = & \{ (\phi\wedge\phi^\prime\ \cfgnt{l}^\prime) \mid  \\
& & \ \ \ \ (\phi^\prime\ \cfgnt{l}^\prime)\in \cfgnt{L}(\cfgnt{r})\wedge\mathbb{S}(\phi\wedge\phi^\prime\wedge\phi_g) \}
\end{array}
\]
\end{definition}

Additionally, we also check for conditions where the write is not possible 

\begin{comment}
\begin{figure}[t]
\begin{center}
\begin{tabular}[c]{l}
$\Psi_x = \{ (true, l_0, r_1^i) \}$\\
$ST (L, r_3^s, \phi, \phi_g)$ \\
$\theta = \{ (\phi_{2a}\; l_\mathit{null} ) (\phi_{2b}\; l_2) (\phi_{2c}\; l_1) \}$\\
$ST(L, r_0, \phi, \phi_g)$\\
$\theta = \{ \}$\\
\end{tabular}
\end{center}
\caption{field write for this.x = this.y sets}
\label{fig:faHeapSets}
\end{figure}
\end{comment}

\subsection{Equality and InEquality of References}

\newsavebox{\boxPEQ}
\savebox{\boxPEQ}{
%\begin{figure}[t]
%\begin{center}
\mprset{flushleft}
\begin{tabular}[c]{c}
\begin{mathpar}
    \inferrule[Equals (references-true)]{
    \Phi_\alpha = \{\lp\phi_0 \wedge \phi_1\rp \mid \exists l\ \lp \lp \phi_0\ l\rp  \in \cfgnt{L}\lp \cfgnt{r}_0\rp  \wedge \lp \phi_1\ l\rp  \in \cfgnt{L}\lp \cfgnt{r}_1\rp \rp \} \\\\
    \Phi_0 = \{\phi_0 \mid \exists l_0\ \lp \lp \phi_0\ l_0\rp  \in \cfgnt{L}\lp \cfgnt{r}_0\rp  \wedge \forall \lp \phi_1\ l_1\rp  \in \cfgnt{L}\lp \cfgnt{r}_1\rp \ \lp l_0 \neq l_1\rp \rp \} \\\\
    \Phi_1 = \{\phi_1 \mid \exists l_1\ \lp \lp \phi_1\ l_1\rp  \in \cfgnt{L}\lp \cfgnt{r}_1\rp  \wedge \forall \lp \phi_0\ l_0\rp  \in \cfgnt{L}\lp \cfgnt{r}_0\rp \ \lp l_0 \neq l_1\rp \rp \} \\\\
    \phi^\prime =  \phi \wedge \lp \vee_{\phi_\alpha\in\Phi_\alpha}\phi_\alpha\rp \wedge\lp \wedge_{\phi_0 \in \Phi_0} \neg \phi_0\rp \wedge\lp \wedge_{\phi_1
    \in \Phi_1} \neg \phi_1\rp \\\\ 
    \mathbb{S}(\phi^\prime)}{
    \lp \cfgnt{L}\ \cfgnt{R}\ \phi\ \eta\ \cfgnt{r}_0\ \lp \cfgnt{r}_1\; \cfgt{=}\; \cfgt{*} \rightarrow \cfgnt{k}\rp \rp  \rsym^\mathit{E}
    \lp \cfgnt{L}\ \cfgnt{R}\ \phi^\prime\ \eta\ \cfgt{true}\ \cfgnt{k}\rp 
    }
%\and
%    \inferrule[Equals (references-false)]{
%    \Phi_\alpha = \{\lp\phi_0 \Rightarrow \neg \phi_1\rp \mid \exists l\ \lp \lp \phi_0\ l\rp  \in \cfgnt{L}\lp \cfgnt{r}_0\rp  \wedge \lp \phi_1\ l\rp  \in \cfgnt{L}\lp \cfgnt{r}_1\rp \rp \} \\\\
%    \Phi_0 = \{\phi_0 \mid \exists l_0\ \lp \lp \phi_0\ l_0\rp  \in \cfgnt{L}\lp \cfgnt{r}_0\rp  \wedge \forall \lp \phi_1\ l_1\rp  \in \cfgnt{L}\lp \cfgnt{r}_1\rp \ \lp l_0 \neq l_1\rp \rp \} \\\\
%    \Phi_1 = \{\phi_1 \mid \exists l_1\ \lp \lp \phi_1\ l_1\rp  \in \cfgnt{L}\lp \cfgnt{r}_1\rp  \wedge \forall \lp \phi_0\ l_0\rp  \in \cfgnt{L}\lp \cfgnt{r}_0\rp \ \lp l_0 \neq l_1\rp \rp \} \\\\
%    \phi^\prime = \phi \wedge \lp \wedge_{\phi_\alpha\in\theta_\alpha}\phi_\alpha\rp \vee\lp \lp \vee_{\phi_0 \in \theta_0} \phi_0\rp   \vee\lp \vee_{\phi_1
%    \in \theta_1} \phi_1\rp \rp  \\\\ 
%    \mathbb{S}(\phi^\prime)}{
%    \lp \cfgnt{L}\ \cfgnt{R}\ \phi\ \eta\ \cfgnt{r}_0\ \lp \cfgnt{r}_1\; \cfgt{%=}\; \cfgt{*} \rightarrow \cfgnt{k}\rp \rp  \rsym^\mathit{E^\prime}
%    \lp \cfgnt{L}\ \cfgnt{R}\ \phi^\prime\ \eta\ \cfgt{false}\ \cfgnt{k}\rp 
%    }
\end{mathpar}
\end{tabular}}
%\end{center}
%\caption{FIX THIS CAPTION AND MOVE $\rsym$ DEFINITION (MAY NOT BE NEEDED BECAUSE OF \defref{def:meta}: Precise symbolic heap summaries from symbolic execution indicated by $\rsym = \rightarrow_\mathit{FA} \cup \rightarrow_\mathit{FW} \cup \rightarrow_\mathit{EQ}^T \cup \rightarrow_\mathit{EQ}^F \cup \rcom$.}
%\label{fig:symeq}
%\end{figure}


\newsavebox{\boxPEX}
\savebox{\boxPEX}{
\begin{tabular}[c]{l}
$L(r_1^i) = \{ (\phi_{1a}\; l_\mathit{null})\; (\phi_{1b}\; l_1) \}$ \\
$L(r_2^i) = \{ (\phi_{2a}\; l_\mathit{null}),\; (\phi_{2b}\; l_2),\; (\phi_{2c}\; l_1) \} $\\
  $\theta_0 = \{ \} $\\
$\theta_1 = \{ \phi_{2b}\} $\\ \hline
Equals true \\
$\theta_\alpha = \{ (\phi_{1a}\; \wedge\; \phi_{2a} ) (\phi_{1b}\; \wedge\; \phi_{2c} ) \}$\\
$\phi^\prime = \mathit{true} \wedge [ (\phi_{1a}\; \wedge\; \phi_{2a} )\vee (\phi_{1b}\; \wedge\; \phi_{2c} ) ] \wedge \neg\phi_{2b} $\\ \hline
Equals false \\
$\theta_\alpha = \{ (\phi_{1a}\; \implies\; \neg\phi_{2a} ) (\phi_{1b}\; \implies\; \neg\phi_{2c} ) \}$\\
$\phi^\prime = \mathit{true} \wedge  (\phi_{1a}\; \implies\; \neg\phi_{2a} )\wedge (\phi_{1b}\; \implies\; \neg\phi_{2c} )  \wedge \phi_{2b} $\\ \hline
\end{tabular}}

\begin{figure*}
\begin{tabular}[c]{cl}
\usebox{\boxPEQ} & \usebox{\boxPEX} \\ \\
(a) & (b) \\
\end{tabular}
\caption{equals true for this.x == this.y}
\label{fig:eqs}
\end{figure*}


The rewrite rules for check the equals true and equals false when
comparing two references in the symbolic summary heap is shown
in~\figref{fig:eqs}(a). First the equals reference true rewrite rule
returns true if two references $r_0$ and $r_1$ are equal. In GSE this
is a simple comparision of object refrences. In the symbolic summary
heap, however, there we campare sets of constraint location pairs for
each reference. We construct three sets of constraints (i) In order to check whether 
there are locations, $l$ in the heap such that under some constraints $\phi_0$ references $r_0$ and $r_1$
$\Phi_\alpha$ constains combinations of constraints ($\phi_0 \wedge
\phi_1$) from $r_0$ and $r_1$

\section{Bisimilarity Proof}
\label{sec:bisim}


The bisimulation proof in this section establishes the soundness and
completeness of the symbolic heap approach with respect to Generalized
Symbolic Execution (\gsetxt{}). Intuitively, any properties proven
with~\gsetxt{} can also be proved using the symbolic heap
algorithm.~\gsetxt{} concretizes the possible heap configurations
along with various aliasing relationships when an uninitialized field
is dereferenced. The rules for initializing and updating the fields in
the machine's semantics for \gsetxt{} as well as
the the complete set of rules for the symbolic heap algorithm can
be found in ~\cite{Hillery:2015}. 

The proof shows the existence of a
bisimulation between sets of states related by~\gsetxt{} ($p \rgse
p^\prime$) and states related by the symbolic heap and update rules in this paper ($q \rsym
q^\prime$). The relations are on the universe of \emph{well-formed}
states $S$, which have the properties in \secref{sec:sh} with the
constraint that the states are \emph{feasible}: successors exist
unless at \cfgt{end}. Let $p \rgse p^\prime$ be a union over relations
for~\gsetxt{} (see~\cite{Hillery:2015}): $\rgse = \rgse^\mathit{A} \cup
\rgse^\mathit{A^\prime} \cup \rgse^\mathit{W} \cup
\rgse^\mathit{W^\prime} \cup \rgse^\mathit{E} \cup
\rgse^\mathit{E^\prime} \cup \rcom$, where \emph{A} is a field access
after evaluating the expression for the base reference $\lp
\cfgt{*}\ \cfgt{\$}\ \cfgnt{f} \rightarrow \cfgnt{k}\rp$, \emph{W} is
a field write after evaluating the expression for the right operand
$\lp \cfgnt{x}\ \cfgt{\$}\ \cfgnt{f}\ \cfgt{:=}\ \cfgnt{*} \rightarrow
\cfgnt{k}\rp$, \emph{E} is a reference compare after evaluating the
left and right operands $\lp \cfgnt{v}\ \cfgt{=}\ \cfgnt{*}
\rightarrow \cfgnt{k}\rp$, the prime symbol indicates a null reference
in the operation or a false outcome, and \emph{\com} is everything
else in the language. Any state relation, say $\rightarrow_x$, is
trivially extended to sets of states as
$$
P \hookrightarrow_x P^\prime \Longleftrightarrow \forall p \in P\ (\forall p^\prime\ (p \rightarrow_x p^\prime \Leftrightarrow p^\prime \in P^\prime))
$$
Let $\hookrightarrow_\gse$ be the extension of $\rgse$ to sets of
states. From this extension, a new meta transition relation is defined
over sets of states as
$$
\rsgse = \hookrightarrow_\gse^\mathit{A}
\cup \hookrightarrow_\gse^\mathit{A^\prime} \cup \hookrightarrow_\gse^\mathit{W} \cup
\hookrightarrow_\gse^\mathit{W^\prime} \cup \hookrightarrow_\gse^\mathit{E} \cup \hookrightarrow_\gse^\mathit{E^\prime}
\cup \hookrightarrow_\com
$$
The relation captures the notion of splitting groups of heaps at
certain operations. For example, suppose we have a set $P$ containing a single state $p$
with all references uninitialized. If $p$ is a field access state, it has two
potential successors in~\gsetxt{}: a non-null reference and a null reference. 
Thus, the $\rsgse$ relation has two successors and divides $P$ into the two outcomes.

The functional equivalence between heaps in states $p$
and $q$ requires both a mapping to relate the two heaps and a
constraint on the feasibility of that mapping in the presence of a
path constraint from symbolic execution. Subscripts indicate state tuple
members as in $p = (
\cfgnt{L}_p\ \cfgnt{R}_p\ \phi_p\ \eta_p\ \cfgnt{e}_p\ \cfgnt{k}_p )$.
\begin{definition}
\label{def:homomorphism}
A \textbf{homomorphism}, given the universe of field indices $\mathcal{F}$ and the universe of locations $\mathcal{L}$, is 
$$
\begin{array}{l}
 s_p \rightharpoonup_{h} s_q \Leftrightarrow 
\exists h: \mathcal{L} \mapsto \mathcal{L}\ (\forall \cfgnt{l}_\alpha\ (\forall \cfgnt{l}_\beta\ (\forall f \in \mathcal{F}\ ( \forall \phi_\alpha\ (\\ 
\ \ \ \ \ \ \ \ (\phi_\alpha\ \cfgnt{l}_\alpha) \in \cfgnt{L}_p(\cfgnt{R}_p (\cfgnt{l}_\beta,f )) \Rightarrow \\
\ \ \ \ \ \ \ \ \exists \phi_\beta\ ( (\phi_\beta\ h(\cfgnt{l}_\alpha))\in \cfgnt{L}_q(\cfgnt{R}_q (h(\cfgnt{l}_\beta),f ))\ 
 )) ) ) ) )
\end{array}
$$
\end{definition}
\begin{definition}
\label{def:hc}
The \textbf{homomorphism constraint} is
%\begin{align*}
\[
\mathbb{HC}(p \rightharpoonup_{h} q) = 
\bigwedge \{ \phi_b\ | \exists (\phi_a\ l) \in \cfgnt{L}_p^\rightarrow ( (\phi_b\ h(l)) \in \cfgnt{L}_q^\rightarrow)\}
\]
%\end{align*}
\end{definition}
Functional equivalence asserts a common structure in the two heaps under certain
conditions. It is used to relate states in $p \rgse
p^\prime$ to states in $q \rsym q^\prime$.
\begin{definition}
\label{representation}
States $(p\ q)$ are in the \textbf{representation relation}, $p \sqsubset q$, if and only if, $\eta_p = \eta_q ,\ \cfgnt{e}_p =
\cfgnt{e}_q ,\ \cfgnt{k}_p = \cfgnt{k}_q$, and there exists a
homomorphism $p \rightharpoonup_{h} q$
such that $\mathbb{S}( \phi_q \wedge \mathbb{HC}(s_p \rightharpoonup_{h} s_q) )$.
The represented relation is extended to sets of states $P$ and a single state $q$ as
$P \sqsubset q \Longleftrightarrow \forall p\ (p \sqsubset q \Leftrightarrow p \in P)$.
\end{definition}
The statement $p \sqsubset q$ ensures that a functionally equivalent
heap to the one in $p$ is present, by the homomorphism, and valid, by
the heap constraint and path constraint, in $q$. As the states in $P$
are only differentiated by heaps and those states are only
differentiated from $q$ by both the heap and path constraint in $q$,
$P \sqsubset q$ implies that $q$ is representative of all the states
in $P$ up to the given point of execution expressed in $\phi_q$.
\begin{definition}
\label{bisimulation}
The \textbf{functional associated to bisimulation} applied to $\sqsubset$, denoted as $F_\sim(\sqsubset)$, is the set of all pairs
$(P\ q)$ such that
\begin{equation}
\label{eqn:BisimulationForwards}
\forall P^\prime\ ( P \rsgse P^\prime \Rightarrow \exists q^\prime\ ( q \rsym q^\prime \wedge P^\prime\ \sqsubset\ q^\prime))
\end{equation}
\begin{equation}
\label{eqn:BisimulationBackwards}
\forall q^\prime\ ( q \rsym q^\prime\Rightarrow \exists P^\prime\ ( P \rsgse P^\prime \wedge P^\prime\ \sqsubset\ q^\prime))
\end{equation}
If $\sqsubset$ is a bisimulation, then the greatest fixed point of $F_\sim(\sqsubset)$ is the bisimilarity relation denoted by $\sim$.
\end{definition}
Although the functional reasons over a forward and backward
simulation as typical~ \cite{Sangiorgi:2011}, its use of a meta-relation, $\rsgse$, is unique. 

In the lemma, $S$ is the universe of well-formed states, and the
set $S_A \subseteq S$ is states at a field access continuation having computed the base reference.
\begin{lemma}[\textrm{F{\footnotesize IELD}} \textrm{A{\footnotesize CCESS}} preserves $\sqsubset\ \subseteq F_\sim(\sqsubset)$]
If $P \in 2^{S_\mathit{FA}}$ and $q \in S$ are such that $P \sqsubset q$, then $(P\ q)$ is in the functional associated to bisimulation.
\label{lem:access}
$$
\forall P \in 2^{S_\mathit{FA}}\ (P \sqsubset q \Rightarrow (P\ q) \in F_\sim(\sqsubset))
$$
\end{lemma}

\begin{proof}
Proof by contradiction: assume $P \sqsubset q \wedge (P\ q) \not\in F_\sim(\sqsubset)$.

\noindent\textbf{Sketch}: Choose any $P$ and $q$ at the field access continuation such
that $P \sqsubset q$. In the forward simulation
\eqref{eqn:BisimulationForwards}, for each $P^\prime$ such that $P
\rsgse P^\prime$, pick some $p^\prime \in P^\prime$. By definition $p \rgse
p^\prime$, $p \in P$, and $p \sqsubset q$. The proof shows the
existence of $q^\prime$ such that $q \rsym q^\prime$ and $p^\prime
\sqsubset q^\prime$ using the existing homomorphism in $p \sqsubset q$.

In the backward simulation \eqref{eqn:BisimulationBackwards}, for each
$q^\prime$ such that $q \rsym q^\prime$, the proof posits the existence of some 
$p^\prime$ such that $p^\prime \sqsubset q^\prime$. It then uses the
homomorphism in $p^\prime \sqsubset q^\prime$ to derive a $p$ such
that $p \rgse p^\prime$ and $p \in P$. The existence of $P^\prime$ is established by the existence of $p^\prime$, and $P \sqsubset q$ makes $P^\prime$ the actual successor of $P$.

As the forward and backward simulations hold for any $p \in P$, $P
\sqsubset q$ must be in the functional, $F_\sim(\sqsubset)$, which is
a contradiction.
\end{proof}

Similar lemmas are proved for field write and equals reference. These
require additional lemmas on the summarize machine
$\rightarrow_S$: that it preserves determinism, the
homomorphism, and the satisfiability of the homomorphism constraint

\begin{theorem}
\label{th:bisim}
The relation $\sqsubset$ is a bisimulation: $\sqsubset\ \subseteq\ \sim$
\end{theorem}
\begin{proof}
By definition of bisimilarity. All of the common rules in $\rcom$
make no changes to the heap, so $P \sqsubset q$ is included in the
functional. The other rules that affect the heap are supported by lemmas
such as \lemref{lem:access}.
\end{proof}

The notation $p \stackrel{n}{\rgse} p^\prime$ denotes that $p^\prime$
is the $n$-step successor of $p$, when it exists, and $q
\stackrel{n}{\rsym} q^\prime$ is similarly defined. Completeness and
soundness fall out of \thref{th:bisim} by inducting
over $n$. 

\begin{corollary}[$\rsym$ is complete]
If $P \in 2^{S_\mathit{A}}$ and $q \in S$ are such that $P \sqsubset q$ then for any $p \in P$
$\forall p^\prime\ (p \stackrel{n}{\rgse} p^\prime \Rightarrow \exists q^\prime\ (q \stackrel{n}{\rsym} q^\prime \wedge p^\prime \sqsubset q^\prime))$
\end{corollary}

\begin{corollary}[$\rsym$ is sound]
If $P \in 2^{S_\mathit{A}}$ and $q \in S$ are such that $P \sqsubset q$ then
$\forall q^\prime\ (q \stackrel{n}{\rsym} q^\prime \Rightarrow \exists p \in P\ (\exists p^\prime\ (p \stackrel{n}{\rgse} p^\prime \wedge p^\prime \sqsubset q^\prime)))$
\end{corollary}

The relation, $P \sqsubset q$, is readily established in the initial
state of a given program as there is single initial state, $P_o =
\{p_o\}$, in any valid program. The initial state for $\rsym$ is
then defined as $q_o = p_o$ so $P_o \sqsubset q_o$ trivially holds.

%\section{Proofs}

\subsection{Definitions}

\begin{definition}
A feasible symbolic heap is defined as...
\end{definition}

\begin{definition}
A subheap is defined as...
\end{definition}

\begin{definition}
A symbolic system state represents a set of lazy system states. To denote the representation relationship, we use the expression $\mathcal{L}\sqsubset \mathcal{S} $ to say that lazy system state $\mathcal{L}:\lp \mu_{\mathcal{L}}\ \cfgnt{L}_{\mathcal{L}}\ \cfgnt{R}_{\mathcal{L}}\ \phi_{\mathcal{L}}\ \eta_{\mathcal{L}}\ \cfgnt{e}_{\mathcal{L}}\ \cfgnt{k}_{\mathcal{L}}\rp$ is represented by symbolic state $\mathcal{S}:\lp \mu_{\mathcal{S}}\ \cfgnt{L}_{\mathcal{S}}\ \cfgnt{R}_{\mathcal{S}}\ \phi_{\mathcal{S}}\ \eta_{\mathcal{S}}\ \cfgnt{e}_{\mathcal{S}}\ \cfgnt{k}_{\mathcal{S}}\rp$ . The \textbf{representation relation} is defined as follows: $\mathcal{L}\sqsubset \mathcal{S} $ if and only if 
$$\eta_{\mathcal{L}} = \eta_{\mathcal{S}} ,\ \cfgnt{e}_{\mathcal{L}} = \cfgnt{e}_{\mathcal{S}} ,\ \cfgnt{k}_{\mathcal{L}} = \cfgnt{k}_{\mathcal{S}}$$
and there exists functions $g:\cfgnt{r} \mapsto \cfgnt{r}$ and $h:\cfgnt{l} \mapsto \cfgnt{l}$ such that for any reference $r \in \mathcal{L}$, location $l \in \mathcal{L}$, and field $f$, $$ l = L_{\mathcal{L}}(r) \leftrightarrow h(l)\in L_{\mathcal{S}}(g(r))$$ and $$ r = R_{\mathcal{L}}(\cfgnt{l},f) \leftrightarrow g(r) = R_{\mathcal{S}}(h(l),f)$$ and the subheap of state $\mathcal{S}$ formed by selecting the image of $g$ and $h$ is feasible.

\end{definition}

\begin{definition}
A symbolic state $\mathcal{S}$ is \textbf{exact} if and only if it represents the set of all feasible lazy states on the current execution path, and represents no infeasible state.
\end{definition}

\subsection{Theorems}

\begin{lemma}
If symbolic state $\mathcal{S}$ is exact prior to executing the Field Access rule, then the state $\mathcal{S}^\prime $ is also exact.
\end{lemma}

\begin{proof}
Suppose we have some symbolic state  $\mathcal{S} =  \lp \cfgnt{L}_{\mathcal{S}}\ \cfgnt{R}_{\mathcal{S}}\ \phi_g\ \eta\ \cfgnt{r}\ \lp \cfgt{*}\ \cfgt{\$}\ \cfgnt{f} \rightarrow \cfgnt{k}\rp \rp $ , and that all relevant fields are initialized. Take an arbitrary lazy state $\mathcal{L} \sqsubset \mathcal{S}$. Since $\mathcal{S}$ is exact,  $\mathcal{L} = \lp \cfgnt{L}_{\mathcal{L}}\ \cfgnt{R}_{\mathcal{L}}\ \phi_L\ \eta\ \cfgnt{r}\ \lp \cfgt{*}\ \cfgt{\$}\ \cfgnt{f} \rightarrow \cfgnt{k} \rp \rp$. If we apply the lazy field access rule to $\mathcal{L}$, we achieve state $\mathcal{L}^\prime = \lp \cfgnt{L}_{\mathcal{L}} [\cfgnt{r}^\prime \mapsto \lp\phi^\prime\ l^\prime\rp]\ \cfgnt{R}_{\mathcal{L}}\ \phi_L\ \eta\ \cfgnt{r}^\prime\ \cfgnt{k}\rp $. Likewise, by applying the symbolic field access rule to $\mathcal{S}$, we obtain state $\mathcal{S}^\prime = \lp \cfgnt{L}_{\mathcal{S}}[\cfgnt{r}^\prime \mapsto \mathbb{VS}\lp \cfgnt{L}_{\mathcal{S}},\cfgnt{R}_{\mathcal{S}},\cfgnt{r},\cfgnt{f},\phi_g\rp ]\ \cfgnt{R}_{\mathcal{S}}\ \phi_g\ \eta\ \cfgnt{r}^\prime\ \cfgnt{k}\rp $.

We now show that state $\mathcal{S}^\prime$ represents state $\mathcal{L}^\prime $. Since $\eta$, $e$, and $k$ are identical between $\mathcal{S}^\prime$ and $\mathcal{L}^\prime $, the first condition is met by default. Constructing the reference and location mappings is only slightly less trivial: $$g^\prime = g$$ $$h^\prime = h[ r^\prime \mapsto r^\prime]$$ 

It only remains to show that the image of $\mathcal{L}^\prime$ in $\mathcal{S}^\prime$ is feasible, WHICH WILL BE EASY ONCE WE DEFINE WHAT THAT MEANS.

Now, we show that no infeasible lazy states are allowed by $\mathcal{S}^\prime$. NEED TO FILL IN THIS BIT, TOO
\end{proof}

\section{Empirical Evaluation}
\symtxt{} creates fewer execution paths than \gsetxt{}, but it does so with increased path constraint complexity. Determining the impact of this tradeoff in real-world applications is an important research question. 

Since the number of paths in \gsetxt{} is exponential in the number of lazy initializations, it is expected that programs for which the number of lazy initializations is proportional to program complexity will evaluate more quickly using \symtxt{}. On the other hand, since \symtxt{} places a greater burden on the constraint solver, it is expected that programs where the lazy initialization count is constant or slowly increasing with respect to program complexity will evaluate more quickly using \symtxt{}.

\subsection{JPF Implementation}
The Java Pathfinder (JPF) symbolic execution engine contains an implementation of \gsetxt{}, which we extended with our own implementation of \symtxt{}. 

\subsection{Bounding}
In order to perform a practical evaluation of programs involving recursive data structures, it is generally necessary to impose some sort of bounding to eliminate the number of lazy initializations to a manageable level. While these bounds may be asserted as invariants within the program in-situ, it is more practical to apply bounding via external means. There are several bounding techniques that may be applied to both \gsetxt{} and \symtxt{}, including execution path length bounding, k-bounding and n-bounding. k-bounding and n-bounding both operate by imposing limits on the size of the symbolic input heap, but differ in how such bounds are defined. n-bounding limits the number of objects created by lazy initialization, while k-bounding limits the length of initialized reference chains. k-bounding has the property of exhaustively exploring the full breadth of a given data structure, while n-bounding tends to explore deeper heap shapes while being somewhat less thorough in breadth. In addition, k-bounding and n-bounding may be combined to adjust the desired level of approximation between depth and breadth.

While k-bounding and n-bounding are applicable to both \gsetxt{} and \symtxt{}, they are not both equally suited for comparing the two methods. The main reason for this lies in how how each bounding technique impacts the functional equivalence of the respective methods.  k-bounding preserves the functional equivalence of \gsetxt{} and \symtxt{}, a fact which can be established by induction over the length of reference chains from a given initial state. In contrast, \gsetxt{} and \symtxt{} are not generally functionally equivalent under n-bounding. The reason is that since a \symtxt{} state must represent all possible \gsetxt{} states on an execution path, it generally includes more locations than any single \gsetxt{} state it represents. This difference in state counts makes it difficult to externally assert n-bounding in a manner that preserves functional equivalence between \gsetxt{} and \symtxt{}. Functional equivalence may be preserved in the case where n-bounds are asserted via in-situ invariants, but the significant increase in constraint complexity makes this approach impractical. For this reason, n-bounding is was not selected for comparing \gsetxt{} to \symtxt{} in the tests performed in this paper.

In contrast to the heap-based bounds provided by k-bounding and n-bounding, execution path-length bounding works by limiting the number along any given execution path. As with other bounding techniques, the usefulness of path-length bounding hinges on whether it preserves functional equivalence. While path-length bounding preserves functional equivalence for \gsetxt{} and \symtxt{} as formulated in this paper, it does not preserve it for the algorithms as they are implemented in JPF. The reason for the discrepancy lies in how states are defined in JPF. In JPF, the notion of states is closely tied to the concept of nondeterminism. New states are created in JPF only when a nondeterministic choice is made, meaning the number of states on an execution path is tied to the number of nondeterministic choices. A discrepancy in the length of the respective execution paths may be caused by any difference in nondeterminism between the two methods. This is significant, since the nondeterminism in \symtxt{} and \gsetxt{} differs in both field reads and in reference compares. Field reads in \symtxt{} are deterministic, whereas in \gsetxt{} they may initiate a nondeterministic lazy initialization. Conversely, in \gsetxt{} address compares are deterministic, but in \symtxt{} address compares require a nondeterministic assertion over the truth of the branch condition. For this reason, path-length bounding was not used for this experimental evaluation. For future tests, we are considering inserting artificial points of nondeterminism in order to establish functional equivalence of path-length bounds between \gsetxt{} and \symtxt{}. 

\subsection{Evaluation} 
Experimental set-up: TO BE ADDED

Test Cases

In order to evaluate the performance of \symtxt{}, the implementation was tested against three examples: LinkedList, BinarySearchTree, TreeMap.
The LinkedList example tests performance on linked list data structures. The test program begins by assuming an object invariant before initiating a sequence of contains() method calls. This is a challenging example for \gsetxt{} because each call to the contains() method contains a new lazy initialization, which in turn creates an exponential path expansion throughout program execution. While the constraints for each path are easily solved, the sheer number of paths quickly becomes overwhelming.

The BinarySearchTree example test program consists simply of asserting the object invariant for a binary search tree. This example is interesting because it demonstrates nondeterminism from both aliasing constraints created during lazy initialization, and linear constraints from evaluating conditionals over numeric symbols. 

Like the BinarySearchTree test, the TreeMap test is simply an assertion of an object invariant, in this case for a red/black tree. This example serves as a stress test for \symtxt{}, since most of the nondeterminism comes from linear constraints over the shape of the tree, rather than aliasing constraints created during lazy initialization. In this case, the path advantage for \symtxt{} is minimized.

For all tests, program complexity was adjusted via k-bounding. For the LinkedList test, the number of contains() method calls was added as an additional parameter. After the conclusion of each test, the test run-time, state count, and path counts were recorded. 

The results of the experiments are shown in table \ref{tab:results}.

\begin{table} [h]
  \centering
  \begin{tabular}{| c | c | c | c | c | c | c | c |}
  \hline
   \multirow{2}{*}{Method }&\multirow{2}{*}{ $k$ }
   &\multicolumn{2}{|c|}{Time} &\multicolumn{2}{|c|}{ Paths }\\
								&	&\gsetxt{}	&SH	&\gsetxt{} & SH\\
   \hline
    \multirow{3}{*}{LinkedList }&3	& 1 & 1  &1656 & 25		 \\
   		 				& 4	& 3 & 2	&17485  & 39 \\
   						& 5	& 20 & 3	&232743 & 56\\
						& 6	& 338 & 8		&3731094 & 76\\
    \hline
    \multirow{3}{*}{BinarySearchTree }&1	& 0 & 1	& 4	 & 4\\
   		 				& 2	& 1 & 1 	& 26 & 17\\
   						& 3	& 12 & 7	& 305 & 118\\
    \hline
      \multirow{3}{*}{TreeMap}&1	& 0 & 1 	&9 & 9 \\
   		 				&2	& 1 & 3		& 100 & 46 \\
   						&3	&21 & 206	& 3026 & 547 \\
						
    \hline
  \end{tabular}
  \caption{test results}
  \label{tab:results}
\end{table}

\subsection{Analysis}

LinkedList
As expected, the repeated use of lazy initialization throughout the program causes an exponential increase in the number of paths explored by \gsetxt{}. By combining many of the \gsetxt{} paths, SH avoids the exponential path explosion, demonstrating dramatically reduced, though still exponential, execution time growth. Remarkably, path growth in the k-bound appears to be sub-linear.

BST
The BinarySearchTree example is somewhat of a mixed bag. 

TreeMap


\section{Related Work}
\label{related}

In this section we discuss some closely related works, including
techniques for scaling symbolic execution of programs with complex
data types and heap summary techniques. Note that there is a huge body
of work on analyzing programs with heaps in the domain of static
analysis, testing, and verification.

Initial work on symbolic execution largely focused on checking
properties of imperative programs with primitive primitive
types~\cite{Clarke:76,King:76}. Several recent projects generalized
the core idea of symbolic execution which enabled it to be applied to
programs with more general types, including references and
arrays~\cite{GSE03,Godefroid:PLDI05,Sen:FSE05,CadarEngler05EXE,KiasanKunit}.
Our work leverages the core idea in generalized symbolic execution
with lazy initialization to construct the heap on-the-fly during
symbolic execution, but avoids the nondeterminism introduced by GSE by
constructing a summary heap and will generally create fewer paths than
GSE techniques.

Pex is a tool for dynamic symbolic execution for .NET programs. In
order to handle data structures Pex uses a theory of arrays and does
not initialization~\cite{Tillmann:2008}.  It is required to have a
constraint solver that handles the theory for unbounded arrays for
this approach to be tractable. It would be an interesting comparision
to perform in future work.

A number of static analysis techniques have been proposed to support
reasoning about input heap structures. The technique most closely
related to (and which served as inspiration for) our approach is the
procedure summary proposed by Dillig et al. to support modular
reasoning~\cite{Dillig:2011}.  While their method improves path
sensitivity over standard static analysis methods, it remains limited
by a statically-generated symbolic input heap and it cannot adapt to
path-sensitive heap properties of the input heap.

Other techniques have been proposed to model references using sets of
guarded
locations~\cite{Xie:2005,Cherem:2007,Dillig:2011,Sen:2014}. These
sets, sometimes known as value sets or value summaries, allow multiple
heaps to be represented simultaneously with a higher degree of
precision than afforded by shape analysis. However, some of these
previous attempts~\cite{Xie:2005,Cherem:2007} utilize non-local
recursive definitions for objects.  In our heap representation, object descriptions are
entirely local, thus avoiding these pitfalls. Other methods
\cite{Dillig:2011,Sen:2014}, while having local morphological
representations, do not provide for dynamic exploration of the initial
heap state.

There is also a large body of work on generating tests for
object-oriented programs that use a variety of techniques to generate
heap structures to test object-oriented
programs~\cite{xie2005symstra,xie2004rostra,boyapati2002korat,artzi2006finding}.
Another line of work in heap representation lies in identifying
isomorphic heap structures in order to avoid exploring redundant
paths~\cite{milicevic2007korat}.  These methods are generally focused
on the morphology of the heap apart from the external constraint
problems encountered during program execution, and as such, are
largely orthogonal to this work. Recent work shows that the need for
reasoning about heap structures is an important
problem~\cite{barr2013collecting}. Delta execution is another work
that combines states for object-oriented programs~\cite{d2008delta}.


%The impact of such an approach is two-fold.  First, these techniques
%must use approximations to resolve reference inequalities, resulting
%in false positives in the results. Second, these techniques are
%incapable of reasoning about heaps with recursive data structures due
%to the potential for infinite recursion during dereferencing. This
%means that either a large class of data structures must be avoided,
%or a large class of behaviors must be ignored.

\section{Conclusion}

\subsection{Future Work}
While our representation reduces paths significantly compared to \gsetxt{}, there remains room for improvement. Another way to reduce path explosion is through state merging. We are investigating ways of merging states at join points in the program, perhaps involving a combination of heap merging and / or heap equivalence testing.

A key advantage of many symbolic execution techniques is the ability to generate test inputs to exercise every possible control flow path. Our technique could be adapted for test input generation by maintaining an explicit input heap in addition to the actual heap which is mutated as the program progresses. 

We are also contemplating a variety of performance optimizations. Due to the structure of the constraints used in the heap representation, our representation is amenable to sophisticated results-caching schemes such as Green. Likewise, a considerable amount of solver time is spent calculating object invariants during program initialization. In order to improve responsiveness during debugging operations, it should be possible to cache object invariants between test runs in to obtain a substantial speed-up.


%\section{Uber-lazy Operational Semantics}
\label{semantics}

The operational semantics for Uber-lazy
symbolic execution are specified using the Javalite 
language~\cite{saints-MS}. Javalite is an imperative model of Java
that includes many features of the Java language. In this work, we 
present only the salient features of the Javalite
language relevant to understanding the Uber-lazy symbolic execution
algorithm. A detailed explanation of Javalite is available in~\cite{saints-MS}.

Javalite is an imperative model of the Java language
developed to facilitate rapid prototyping 
of model checking algorithms and proofs about the
algorithms. It is specified with a Java-like syntax and a set
of reduction rules for syntacticly executing Javalite programs.
Javalite is based on a variant of the CEKS syntactic 
machine~\cite{Felleisen:1987}. The operational semantics are 
defined using the structure (syntax)
of the language and a set of reduction rules for syntactically
executing Javalite programs. In a CEKS machine, the
(C)ontrol string represents a program, command, or instruction to be
evaluated; it is initialized to a string representing the entire progam. 
An (E)nvironment  maps (local) variables to their values. The (K)ontinuation 
specifies what is to be executed next, and the (S)tore is used to store 
dynamically allocated data, i.e., the heap. 

\subsection{Symbolic Store}

At the core of the Uber-lazy symbolic execution algorithm
is a fully symbolic heap, i.e., a heap in which objects
are never materialized as concrete objects during symbolic
execution, but instead are represented using constraints
charactering feasible heap shapes. To represent the heap
store ($S$), our algorithm defines a labeled bi-partite 
graph where $S = (R, L, E)$.  $R$ contains the set of nodes 
representing program \emph{references}.
$L$ contains the set of nodes representing \emph{locations} in the store. 
The store is initialized with two special locations: $null$ and $\bot$ representing a null object
and an uninitialized location respectively. Each edge in the set of labeled
edges, $E$, is uni-directional. An edge from a reference $r \in R$ to a location $l \in L$ is
labeled with a constraint $\phi$ indicating the conditions under which $r$ references
i.e., points to, that location in the store. This paper uses a standard definition
of constraints $\phi \in \Phi$ assuming all of the usual relation operators and connectives.
Reference nodes collect the the
feasible points-to relations for a given program execution path during symbolic execution.
Each edge from a location $l \in L$ to a reference $r \in R$ is labeled with the 
name of a field $f \in F$. 

More details here including we assume the input program is typesafe
or if there is a type and a mismatch occurs, then the machine halts.

The machine also halts if an exception is thrown.

\subsection{Syntax}

Javalite programs and expressions are written in a syntax specified by the 
grammar in~\figref{fig:machine-syntax}. The production rules correspond
to the various features supported by Javalite such as classes, fields,
methods, and expressions. 

\figref{fig:machine-syntax} specifies
the machine syntax for Javalite. The expression $e$ is equivalent to
a CEKS machine's control string. The 

\subsection{Reduction Rules}

We first present the reduction rules 


\subsubsection{Basic Reduction Rules}

Most rules presented here - short discussion

\subsubsection{Store Update Rules}

Interesting rules presented here; more elaborate discussion

Start off with helper functions for rules that manipulate the store...

The function $\mathbb{VS}(L,R,r,f)$ constructs the value-set given a
heap, reference, and desired field such that
$(l^\prime\ \phi^\prime\wedge\phi) \in \mathbb{VS}(L,R,r,f)$ if and
only if
\[
\begin{array}{l}
  \exists (l\ \phi) \in L(r) ( \\
  \ \ \ \ \ \ \ \ \ \exists r^\prime \in R(l,f) ( \\
  \ \ \ \ \ \ \ \ \ \ \ \ \ \ \ \ \ \ \exists (l^\prime\ \phi^\prime) \in L(r^\prime) (\mathbb{S}(\phi\wedge\phi^\prime))))
\end{array}
\]
where $\mathbb{S}(\phi)$ returns true if $\phi$ is satisfiable.

The strengthen function $\mathbb{ST}(L,r,\phi^\prime)$ strengthens every
constraint from the reference $r$ and keeps only location-constraint
pairs that are satisfiable after strengthening. Formally,
$(l\ \phi\wedge\phi^\prime)\in\mathbb{ST}(L,r,\phi^\prime)$ if and
only if $\exists (l\ \phi)\in
L(r)\wedge\mathbb{S}(\phi\wedge\phi^\prime)$

The empty-reference function $\mathbb{ER}(L,\phi^\prime) = \{r\ |\ L(r) \neq
\emptyset \wedge \forall(l,\phi) \in L(r)(\neg \mathbb{S}(\phi \wedge
\phi^\prime))\}$ searches the heap for references that become
disconnected from all their locations after strengthening. These
references, if reachable, imply the heap is no longer valid on the
current search path. As such, the symbolic execution algorithm should
backtrack. This check is similar to a feasibility check in classic
symbolic execution with only primitive data types.

The consistency function is critical to the soundness of the algorithm
as it detects when a symbolic heap becomes invalid along a path,
similar to a feasibility check when doing classical symbolic execution
with just primitives. As the constraints in the heap are strengthened
with different aliasing requirements, it is possible to reach a point
where the heap is no longer connected. Meaning, a valid reference is
live, either in the local environment or the continuation, the reaches
another reference that is no longer connected to any locations due to
strengthening. The function relies on the empty-reference function to
identify disconnected references. The function in essence checks every
reference in the local environment and every reference found in the
continuation, as these are all considered live. This operation similar
to garbage collection where the local environment and stack are
inspected to find the roots of the heap for the scan.


The consistency function relies on two auxiliary functions which are
informally defined. The function $\mathrm{ref}(\eta,\cfgnt{k})$
inspects the local environment and continuation for all live
references, and it returns those references in a set. The function
$\mathrm{reach}(L, R, r, r^\prime)$ returns true if $r^\prime$ is
reachable from $r$ in the heap and false otherwise. The consistency function $\mathbb{C}(L,R,X,\eta,k)$ is now defined as
\[
 \left\{ \begin{array}{rl} 
        0 & \exists r \in \mathrm{ref}(\eta, \cfgnt{k})\ (\exists r^\prime \in X\ (\mathrm{reach}(L, R, r,r^\prime))) \\ 
        1 & \mbox{otherwise}\end{array}\right .
\]

%\section{System State}
The program state is represented using a path condition, a program location, a stack, a symbolic heap, and a symbolic store. The path condition is a collection of predicates over the program inputs that indicates constraints on the values of those inputs at the present location. The program location indicates the instruction to be executed presently. The symbolic heap is a mapping from heap locations symbolic objects. 
\paragraph{Heap Symbols}
Heap symbols are symbols which are created and destroyed in the process of performing heap operations. The dynamic nature of heap symbols distinguishes them from statically-created symbols like tbose found in the path condition. Heap symbols may be symbolic primitives, symbolic references, symbolic locations, or symbolic types.
\paragraph{Symbolic References}
Symbolic references are symbolic points-to relations. The symbolic reference has a state parameter that indicates whether the reference is uninitialized, non-null, or initialized. The uninitialized and non-null states indicate that the location pointed to by the reference has yet to be resolved. The non-null state reflects the additional constraint that the reference does not point to the null location. References in the initialized state are assosciated with constraints reflecting which location the reference points to and under what condition it points to that location. References point to one and only one location at a time, so the conditions must be mutually exclusive, yet collectively exhaustive.
\paragraph{Symbolic Locations}
Symbolic locations represent the locations of slots in the heap structure that may hold symbolic objects. Each symbolic location represents a unique slot on the symbolic heap. The constraints assosciated with symbolic locations reflect the uniqueness of each location. Each slot on the heap may contain a symbolic object. 
\paragraph{Symbolic Objects}
A symbolic object represents the contents of a heap slot. A symbolic object contains a symbolic type paired with a mapping from fields to heap symbols.
\paragraph{Symbolic Store}
The symbolic store is a mapping from heap symbols to symbolic value sets. The symbolic value set represents the values that may be assigned to the symbol, along with the constraints assosciated with each value. Taken together, the constraints in the abstract store represent a set of unique heaps common to the current program execution path. Compare this to the path condition, which contains constraints common to all the heaps in the abstract store.
\paragraph{Heap}
The symbolic heap is a mapping from symbolic locations to symbolic objects. Like the path condition, the symbolic heap contains heap state which is common to all heaps on the current execution path. Unlike the path condition, the symbolic heap does not contain any logical constraints.
\paragraph{Stack}
The Java Virtual Machine uses a system stack.
%\section{Semantic Model}
Program execution proceeds an in standard symbolic execution, with additional rules to handle heap constructs. The Java Virtual Macine (JVM) implements five classes of semantic rules, a categorization which we find relevant to symbolic execution: 1) load/store instructions, 2) arithmetic instructions, 3) object creation / manipulation instructions, 4) control transfer instructions, and 5) assume/assert instructions. We will deal with each class of instruction in sequence.
\subsection{Reference Target Resolution}
Load and store instructions take symbolic references as arguments. Since an uninitialized symbolic reference may point to any one of a number of symbolic locations, we need to be able to resolve which locations are feasible targets of a particular references.
Location resolution begins with an empty list of heap locations, and then adds locations by comparing locations on the symbolic heap to the constraints on the symbolic reference obtained from the abstract store. Locations are checked in the following order:
\begin{compactenum}
\item If the null location is feasible, return a null pointer exception and terminate execution.
\item If the non-null location is feasible, create a new symbolic location of the proper type and add it to the symbolic heap. Search the heap for type-compatible objects and add those to the symbolic value set for the reference, and remove the non-null location from the symbolic value set.
\item Check the symbolic heap for type-compatible objects, comparing those against the constraints in the symbolic value set. Add any feasible objects to the feasible location set.
\end{compactenum}

\subsection{Read}
The load instruction has two arguments: a reference $r$ and a field index $f$. If $r$ is a symbolic location, then load simply accesses the field and returns the value contained there. If $r$ is a symbolic reference, then the following steps are taken, in order.
\begin{compactenum}
\item Resolve a list of feasible targets as described above.
\item Access the fields of the feasible targets, gathering a set of the symbolic values contained therein. 
\item Form a new symbolic value, and create a new entry in the symbolic store mapping the value to the symbolic value set. Return the new symbolic value.
\end{compactenum}

\subsection{Write}
The write instruction takes three arguments: a reference $r$, a field $f$, and a value $v$. If $r$ is a symbolic location, then the target field is written with the value directly. If $r$ is a symbolic reference, write proceeds as follows:
\begin{compactenum}
\item Resolve a list of feasible targets.
\item Access the fields of the feasible targets, performing a conditional write. The conditional write works by modifying the constraint equation as follows:
\newline  ((A==h_n) => F==V) and (~(A==h_n) => Eqn_o_l_d)
\end{compactenum}

\subsection{Arithmetic Instructions}
\subsection{Object Creation/Manipulation Instructions}
\subsection{Control Transfer Instructions}
\subsection{Assume / Assert Instructions}
%\input{examples}
%\section{Bytecodes}

PUTFIELD needs to remove path constraints from PC that enforcing equality between references.

\noindent \textbf{Assume}: all symbolic locations are concertized lazily. Although the algorithm is not specific to any particular initialization strategy, the presentation assumes a lazy initialization. Extending to lazier initialization may be non-trivial even though this reduction is orthogonal to the lazier reduction (i.e., this reduction should further improve the performance of the lazier algorithm).

\noindent \textbf{Assume}: all variables, symbolic or otherwise, are non-primitive (i.e. objects). \textit{Must relax this assumption because you need to do some interesting things with primitives as they relate to getfield}.

This papers uses subsumption, which is expressed as a subtyping relation $\leq$ over types $T$. For classes $C$ and $D$, $C \leq D$ iff either $C = D$ or the class declaration for $C$ is \texttt{class C extends B $\{\ldots\}$} for some $B \leq D$. For example, in $A \leq B \leq C \leq D$, $D$ is the supertype, and if you have something that is an instance of $A$ but currently viewed as $B$, then you can move it toward $D$ in a typecast (up the hierarchy).

Heap locations range over positive natural numbers $H \subseteq \mathbb{N}_{\geq 0}$. Every heap location has a special variable $T_h$ used in constraints over the type of the object stored in the heap location. The varable $\mathit{SH}$ is the set of heap locations created when concretizing symbolic variables. The set of constraints over the type stored in the heap location is given by the function $\mathtt{C}(h)$. Constraints are of the form $T_h \sim T$ or $T \sim T$ where $\sim\ \in \{\leq, =, \not =\}$. The initial type hierarchy for the program is expressed as a set of constraints $C_\mathrm{init}$. The set contains all relationships needed to describe the entire hierarchy.

For a set of constraints $C$, the function $\mathtt{SAT}(C) \mapsto \{0,1\}$ returns 1 if the constraints are satisfiable and 0 otherwise. The usual Boolean connectives are used as expected. The function $\mathtt{Type}(h)$ returns the actual type of the object in the heap location and the function $\mathtt{Obj}(h)$ returns the object at the location.

Each variable $v$ is associated with a set of heap locations $H(v) = \{h_0, h_1, \ldots, h_n\}$ that represents an equivalence class (i.e., each heap location yields the same execution path and behavior up to the current point of execution).  The representative object for a given variable (i.e., the one that is currently being used by the variable) is given by the function $I(v) \mapsto H(v)$. The set of heap locations and the representative location are part of the meta-deta for the variable. This meta-data follows the variable through the program execution and is appropriately copied on assignment to other variables such that each variable has its own copy of the meta-data that is separate from other copies.

Finally, there is a global variable $\mathit{PC}$ that represents the path constraint along the current path of exploration. This path constraint is used to track relationships between symbolic variables such as equality. Properties of symbolic variables are represented in the path constraint by creating special variables for the representative and the set of represented objects. For a variable $v$, the special variable $I_v$ is the representative heap opbject and the special variable $H_v$ is the set of associated heap locations. It is assumed the $v$ is alpha-renamed to be unique in the path constraint. Finally, we use the special value \texttt{SYM} to denote a symbolic variable that is yet to be initialized.

\subsection{Reference}
\noindent \textbf{\texttt{GETFIELD}}: the bytecode behavior depends on the field operand: concrete, concrete though initialized from symbolic, and symbolic. Each case is enumerated:
\begin{compactenum}
\item Referencing a concrete field: the bytecode has default behavior returning the field.
\item Referencing an initialized field from a symbolic variable (i.e., the base type for the field is initialized from a symbolic object): the bytecode may have multiple outcomes; it partitions the equiavalence class to group heap locations with objects that have common values for the field.
\item Referencing a symbolic variable that has yet to be initialized: the bytecode has two outcomes: one that returns \texttt{null} and another that builds the potential equivalence class, chooses a representative location, and returns that location.
\end{compactenum}
Consider the code
\begin{lstlisting}
// The declared type of f is F
T t = b.f;
\end{lstlisting}
For the case where \texttt{b.f} is an already initialized symbolic variable, $h = \mathtt{I}(\mathtt{b})$, $\mathtt{H}(\mathtt{b})$ is partitioned into disjoint sets, $S_0, P_1, \ldots, P_n$, with $n+1$ partitions, or choices. The first set $S_0$ is a special set that includes $h$, the representative object for \texttt{b}, and any members of the equivalence class that either have the same field value for \texttt{b.f} or the field value is a symbolic variable that has yet to be initialized:
\begin{eqnarray*}
S_0 &=& \{h_i \mid h_i \in \mathtt{H}(\mathtt{b})\ \wedge \\
    & & (\mathtt{Obj}(h).\mathtt{f} = \mathtt{Obj}(h_i).\mathtt{f}\ \vee\ \mathtt{Obj}(h_i).\mathtt{f} = \mathtt{SYM})\}
\end{eqnarray*}
For this special case of $S_0$,  $\mathtt{H}_0(\mathtt{b}) = S_0$, and $I_0(\mathtt{b}) = h$ where the subscript indicates the choice number in the choice generator (i.e., the partition size may change but not the representative heap location). The other partitions group common values of the field such that
\begin{compactitem}
\item $\forall h_i, h_j \in P_i,\ \mathtt{Obj}(h_i).f = \mathtt{Obj}(h_j).f$ 
\item $\mathtt{H}_i(\mathtt{b}) = P_i$ 
\item $\exists h \in P_i,\ \mathtt{I}_i(\mathtt{b}) = h$
\end{compactitem}
The partitions are maximal and represent a unique value that has been created thus far in the program execution. The non-initialized symbolic members of the partition all belong to $S_0$ as the original representative heap location $h$ captures that these other aliases were intended to have the same field value for field \texttt{f} before the split (i.e., the value is assigned programatically but the change was only reflected in the representative heap location). Once the choice generator is created over the different partitions, the bytecode returns the requested field value of the representative as expected.

Returning to the third behavior of the bytecode, the case in which the accessed field has yet to be initialized, then the bytecode follows lazy initialization creating a \texttt{null} instance, a new instance that is the representative, and the alias set. When creating the alias set, the new instance should be included in the set, as well as any prior object created in concretization of symbolic variables that is type compatible with the new instance. Recall that $\mathit{SH}$ is set of locations in the symbolic heap and $C_\mathrm{init}$ is the set of constraints describing the type hierarchy, assuming $h$ is the heap location of the new instance of the type, then 
\begin{compactitem}
\item $\mathtt{I}(\mathtt{b.f}) = h$
\item $\forall h_i \in \mathit{SH}, \mathtt{SAT}(C_\mathrm{init} \cup \{T_h \leq \mathtt{Type}(h_i)\}) \rightarrow h_i \in \mathtt{H}(\mathtt{b.f})$  
\item $\mathtt{C}(\mathtt{b.f}) = C_\mathrm{init} \cup \{T_h \leq \mathtt{Type}(h)\}$  
\end{compactitem}
The $C_\mathrm{init}$ set constains relationships in the class hierarchy with the correct sub-types and super-types as they relate to the delcared type of the object.

\noindent \textbf{\texttt{GETSTATIC}}: the bytecode is handled similarly to \texttt{GETFIELD}. 

\noindent \textbf{\texttt{ALOAD}}: the bytecode is handled similarly to \texttt{GETFIELD}. 

\subsection{Comparison}

\noindent \textbf{\texttt{IF\_ACMPEQ}}: the bytecode may return both the \texttt{true} and \texttt{false} values, and it must possibly refine the set of represented concretizations and mutate the heap location of the object involved in the bytecode according to the returned outcome. Consider the code
\begin{lstlisting}
if (a == b) {
   // code...
}
\end{lstlisting}
There are two cases that need to be considered to determine the outcome of the bytecode:
\begin{compactenum}
\item $\mathtt{H}(a) \cap \mathtt{H}(b) = \emptyset$: the bytecode returns \texttt{false} and nothing further is requred.
\item $\mathtt{H}(a) \cap \mathtt{H}(b) \not = \emptyset$ $\wedge$ $\mathtt{SAT}(PC \cup \{I_a = I_b, H_a = H_b\})$: the bytecode may return either \texttt{true} or \texttt{false} and a choice generator needs to be created.
\end{compactenum}
The choice generator for the compare bytecode is more complex than for other bytecodes because it must create representative sets without enumerating all possible outcomes using the path constraint. For the case \texttt{true} outcome
\begin{compactitem}
\item $\mathit{PC} = \mathit{PC} \cup \{\mathtt{I}(a) = \mathtt{I}(b),\mathtt{H}(a) = \mathtt{H}(b)\}$
\item $\mathtt{H}(a) = \mathtt{H}(b) = \mathtt{H}(a) \cap \mathtt{H}(b)$
\item $\mathtt{I}(a) \in \mathtt{H}(a) \cap \mathtt{H}(b) \rightarrow \mathtt{I}(b) = \mathtt{I}(a)$ $\vee$ $\exists h \in \mathtt{H}(a) \cap \mathtt{H}(b)\ .\ \mathtt{I}(b) = \mathtt{I}(a) = h$
\end{compactitem}
In essence, in the case where two variable reference the same object, the path constraint and sets are modified to represent the new restriction. The \texttt{false} outcome is handled similarly with a few notable exceptions on the path constraint and the represented set.
\begin{compactitem}
\item $\mathit{PC} = \mathit{PC} \cup \{\mathtt{I}(a) \not = \mathtt{I}(b)\}$
\item $\mathtt{I}(a) = \mathtt{I}(b) \rightarrow \exists h \in \mathtt{H}(b)\ .\ h \not = \mathtt{I}(a) \wedge \mathtt{I}(b) = h$
\end{compactitem}

\noindent \textbf{\texttt{IF\_ACMPNE}}: the bytecode is handled similarly to \texttt{IF\_ACMPEQ}. 

\subsection{Invocation}
\textbf{\texttt{INVOKEVIRTUAL}}

When we come to an invoke virtual you have to look for all the specialized implementations
of the method, creating choices with symbolic locations of various "actual types". The number
of choices will be equal to the number of specialized implementations of the method. When you create a choice on a specialization, you need to update the "actual type" field in the symbolic location. The "current cast" does not need to change. The number of types that the symbolic location cannot be will also be updated according to the "actual type" field. The number of types that the symbolic location cannot be will be updated with the types of the other specializations since invoking a specialization associated with a type implies that the object cannot be the types containing the other specializations.

\subsection{Checking Types and Casting}
\noindent\textbf{\texttt{INSTANCEOF}}: the bytecode may return both the \texttt{true} and \texttt{false} values when dealing with initialized symbolic variables, and it must possibly refine the equivalence class for the represented object referenced by the variable and mutate the contents of the heap location of the object involved in the bytecode according to the returned outcome. The bytecode implements the default bahvior when the operand is concrete and not an initialized symbolic variable. For the rest of the discussion, assume the operand is an initialized symbolic variable.
Consider the code
\begin{lstlisting}
if (a instanceof C) {
   // code...
}
\end{lstlisting}
There are two cases that need to be considered to determine the outcome of the bytecode where $h = \mathtt{I}(a)$ is the representative object of the equivalence class:
\begin{compactenum}
\item $\mathtt{Type}(h) = C$: the bytecode returns \texttt{true} and nothing further is required as the type stored in the heap location is $C$.
\item $\neg \mathtt{SAT}(\mathtt{C}(h) \cup \{T_h \leq C\})$: the bytecode returns \texttt{false} and nothing further is requred as the current constraints on what is in the heap location restrict it from being of type $C$.
\item $\mathtt{SAT}(\mathtt{C}(h) \cup \{T_h \leq C\})$: the bytecode can return either \texttt{true} or \texttt{false} requiring a choice generator.
\end{compactenum}
% // C <= B <= A
% A a; // C(a) = {(T_a <= A)}
%
%if (a instance of C) {
%    ** TRUE **
%    (T_h \leq C)
%    ...
%}
% ** FALSE **
% (C \leq T_h) \wedge (T_h \not = C)
% ** TRUE **
% (T_h \leq C)
\noindent The \texttt{true} outcome for the choice generator in clause (3) is
\begin{compactitem}
    \item $\mathtt{C}(h) = \mathtt{C}(h) \cup \{T_h \leq C\}$
    \item $\mathtt{Type}(h) = C$
    \item $\mathtt{H}(a) = H^\prime$ where $H^\prime = \{h_i \mid h_i \in \mathtt{H}(a) \wedge \mathtt{SAT}(\mathtt{C}(h_i) \cup \{T_{h_i} \leq C\})\}$ 
\end{compactitem}
The second statement indicates that the actual type in the heap location $h$ needs to change. As such, the object is mutated to be an instance of $C$. This mutation retains all fields and values from the previous object and only adds new fields for type $C$. The last statement refines the equivalence class to exclude any heap locations that cannot be considered something of type $C$. 

For the \texttt{false} outcome of the generator, $\mathtt{C}(h) = \mathtt{C}(h) \cup \{C \leq T_h, T_h \not = C\}$. Unlike the \texttt{true} outcome, the \texttt{false} outcome retains the entire equivalence class and does not need to mutate any heap entries.

\noindent\textbf{\texttt{CHECKCAST}}: the bytecode is syntactic sugar for 
\begin{lstlisting}
if (! (obj == null  ||  obj instanceof <class>)) {
    throw new ClassCastException();
}
// if this point is reached, then object 
// is either null, or an instance of <class> 
// or one of its superclasses.
\end{lstlisting}
Please see the \texttt{IFNULL} and \texttt{INSTANCEOF} bytecodes for details. If the exception is thrown, then JPF will catch the unhandled exception as per its normal behavior.

\subsection{Programs to consider}
\begin{compactitem}
\item \texttt{TestGetfieldSplit.java}: checks alias equivalence classes when assigning to initialized values.
\end{compactitem}
%\section{Conclusion}

\acks

Acknowledgments, if needed.

% We recommend abbrvnat bibliography style.
\bibliographystyle{abbrvnat}

\bibliography{../bib/paper}


\end{document}

