\subsection{Heaps as Bipartite Graphs}
An important aspect, and a novel contribution, of symbolic heaps its
representation as a labeled bipartite graph consisting of
references, $\cfgnt{r}$, and constraint location pairs,
$\lp\phi\ \cfgnt{l}\rp$. The heap structure conveys that the actual
location pointed to by a reference is conditional on aliasing
relationships within the heap. In this way, the graph represents sets
of heaps with the conditions under which those heaps exist.

The machine syntax in \figref{fig:machine-syntax} defines that graph
in $\cfgnt{L}$, the location map, and $\cfgnt{R}$, the reference
map. As done with the environment, $\cfgnt{L}$ and $\cfgnt{R}$ are
treated as partial functions where $\cfgnt{L}(r) =
\{(\phi\ \cfgnt{l})\ ...\}$ is the set of location-constraint pairs in
the heap associated with the given reference, and
$\cfgnt{R}(\cfgnt{l},\cfgnt{f}) = \cfgnt{r}$ is the reference
associated with the given location-field pair in the heap. Predicate
calculus is used to describe the more complex updates to the heap in
the semantics.

There are several properties of the heap that are invariant and
important to the correctness of the algorithm.
\begin{compactitem}
\item ADD REFERENCES ARE IMMUTABLE
\end{compactitem}

The location $\cfgnt{l}_\mathit{null}$ is a special location in the
heap to represent null. It has a companion reference
$\cfgnt{r}_\mathit{null}$. The initial heap for the machine is defined
such that $\cfgnt{L}(\cfgnt{r}_\mathit{null}) =
\{(\cfgt{true}\ \cfgnt{l}_\mathit{null})\}$

DEFINE $\rsym$ and $\rgse$.
