\section{Bisimularity Relation}
Let $p \rgse p^\prime$ be the binary relation over the universe of states, $S$,
according to generalized symbolic execution with lazy initialization
(see the supplemental text for a precise definition of that
relation). Similarly, let $q \rsym q^\prime$ be the relation for the GSESH
algorithm in this paper. The principle theoretical result is the
existence of a bisimularity relation between states related on $p \rgse p^\prime$
and the states related on $q \rsym q^\prime$. A corollary is that $\rsym$ is
both sound and complete with respect to $\rgse$. Only a summary of the
proof is given due to space.

The inspiration for $\rsym$ came from a desire to create
a new way to initialize heap objects that avoided initializing heaps
that traverse the same control flow paths. The thought was to group together heaps
into equivalence classes: all the heaps that followed the same control
flow path to a given point of execution would be in the same
equivalence class at that point of execution. An equivalence class
splits at branches, where the program actually differentiates heaps.

The proof only reasons about well-formed states: heaps are
deterministic, type consistent, with no dangling references, and it
knows about any references in the environment, expression, or
continuation. Well-formed states always have a successor unless it is
the end of the program execution.

Let $p \rgse p^\prime$ be defined as a union over state relations:
$$
\rgse = \rgse^\mathit{A} \cup  \rgse^\mathit{A^\prime} \cup \rgse^\mathit{W} \cup \rgse^\mathit{W^\prime} \cup \rgse^\mathit{E} \cup \rgse^\mathit{E\cdot} \cup \rgse^\com
$$
where \emph{A} is a field access $\lp \cfgt{*}\ \cfgt{\$}\ \cfgnt{f}
\rightarrow \cfgnt{k}\rp$, \emph{W} is a field write $\lp
\cfgnt{x}\ \cfgt{\$}\ \cfgnt{f}\ \cfgt{:=}\ \cfgnt{*} \rightarrow
\cfgnt{k}\rp$, \emph{E} is a reference compare $\lp
\cfgnt{v}\ \cfgt{=}\ \cfgnt{*} \rightarrow \cfgnt{k}\rp$, $\cdot$
indicates a null reference in the operation or a false outcome, and
\emph{\com} is everything else in the language. Any state relation, say $\rightarrow_x$, is trivially extended to sets of states as
$$
P \hookrightarrow_x P^\prime \Longleftrightarrow \forall p \in P\ (\forall p^\prime\ (p \rightarrow_x p^\prime \Rightarrow p^\prime \in P^\prime))
$$
Let $\hookrightarrow_\gse$ be the extension of $\rgse$ to sets of states.
$$
\rsgse = \hookrightarrow_\gse^\mathit{A}
\cup \hookrightarrow_\gse^\mathit{A\cdot} \cup \hookrightarrow_\gse^\mathit{W} \cup
\hookrightarrow_\gse^\mathit{W\cdot} \cup \hookrightarrow_\gse^\mathit{E} \cup \hookrightarrow_\gse^\mathit{E\cdot}
\cup \hookrightarrow_\gse^\com
$$
The definition embodies the notion of splitting equivalence classes at
certain operations. The set $P$ at a field access for example has two
potential successors: a non-null and null reference. Respectively, the
set $P \rgse^\com P^\prime$ only has a single successor since those expressions
are not related to control flow.

The functional equivalence between heaps in states $p$
and $q$ requires both a mapping to relate the two heaps and a
constraint on the feasibility of that mapping in the presence of a
path constraint from symbolic execution. Subscripts indicates state
members as in $p = (
\cfgnt{L}_p\ \cfgnt{R}_p\ \phi_p\ \eta_p\ \cfgnt{e}_p\ \cfgnt{k}_p )$.
\begin{definition}
\label{def:homomorphism}
A \textbf{homomorphism} given the universe of field names $\mathcal{F}$ is 
$$
\begin{array}{l}
 s_p \rightharpoonup_{h} s_q \Leftrightarrow \\
\ \ \ \ \exists h: \mathcal{L} \mapsto \mathcal{L}\ (\forall \cfgnt{l}_\alpha\ (\forall \cfgnt{l}_\beta\ (\forall f \in \mathcal{F}\ ( \exists \phi_\alpha\ (\exists \phi_\beta\ ( \\ 
\ \ \ \ \ \ \ \ \ \ \ \ (\phi_\alpha\ \cfgnt{l}_\alpha) \in \cfgnt{L}_p(\cfgnt{R}_p (\cfgnt{l}_\beta,f )) \Rightarrow \\
\ \ \ \ \ \ \ \ \ \ \ \ \ \ \ \ (\phi_\beta\ h(\cfgnt{l}_\alpha))\in \cfgnt{L}_q(\cfgnt{R}_q (h(\cfgnt{l}_\beta),f ))\ \\
\ \ \ \ \ \ \ \ \ \ \ \ \ \ \ \  )) ) ) ) )
\end{array}
$$
\end{definition}

\begin{definition}
\label{def:hc}
The \textbf{homomorphism constraint} is
\begin{align*}
\mathbb{HC}(p \rightharpoonup_{h} q) &= \\
 \bigwedge \{ \phi_b\ | \exists (\phi_a\ l) \in \cfgnt{L}_p^\rightarrow ( (\phi_b\ h(l)) \in \cfgnt{L}_q^\rightarrow)\} 
\end{align*}
\end{definition}
The functional equivalence between heaps on a homomorphism and
a homomorphism constraint asserts a common structures in the two heaps under certain
conditions. The equivalence is used to related states in $p \rgse
p^\prime$ to states in $q \rsym q^\prime$.
\begin{definition}
\label{representation}
States $(p\ q)$ are in the \textbf{representation relation}, $p \sqsubset q$, if and only if, $\eta_p = \eta_q ,\ \cfgnt{e}_p =
\cfgnt{e}_q ,\ \cfgnt{k}_p = \cfgnt{k}_q$, and there exists a
homomorphism $p \rightharpoonup_{h}
q$ such that
\begin{equation}
\label{eqn:valid}
 \mathbb{S}( \phi_q \wedge \mathbb{HC}(s_p \rightharpoonup_{h} s_q) ) 
\end{equation}
The represented relation is extended to sets of states $P$ and a single state $q$ as\footnote{The full proof argues $P \sqsubset Q$ but that $Q$ is
  always a singleton set containing $q$.}
$$
P \sqsubset q \Longleftrightarrow \forall p\ (p \sqsubset q \Rightarrow p \in P)
$$
\end{definition}
The statement $p \sqsubset q$ ensures that a functionally equivalent
heap to the one in $p$ is present, by the homomorphism, and valid, by
the heap constraint and path constraint, in $q$. As the states in $P$
are only differentiated by heaps and those states are only
differentiated from $q$ by both the heap and path constraint in $q$,
$P \sqsubset q$ implies that $q$ is representative of all the states
in $P$ up to the given point of execution expressed in $\phi_q$.
\begin{definition}
\label{bisimulation}
The \textbf{functional associated to bisimulation} applied to $\sqsubset$, denoted as $F_\sim(\sqsubset)$, is the set of all pairs
$(P\ q)$ such that
\begin{equation}
\label{eqn:BisimulationForwards}
\forall P^\prime\ ( P \rsgse P^\prime \Rightarrow \exists q^\prime\ ( q \rsym q^\prime \wedge P^\prime\ \sqsubset\ q^\prime))
\end{equation}
\begin{equation}
\label{eqn:BisimulationBackwards}
\forall q^\prime\ ( q \rsym q^\prime\Rightarrow \exists P^\prime\ ( P \rsgse P^\prime \wedge P^\prime\ \sqsubset\ q^\prime))
\end{equation}
If $\sqsubset$ is a bisimulation, then the greatest fixed point of $F_\sim(\sqsubset)$ is the bisimilarity relation denoted by $\sim$.
\end{definition}
As is typical, the functional reasons over a forward and backward
simulation, while the application to dissimilar states using a
meta-relation, $\rsgse$, is somewhat non-standard; though convenient
in this instance since it captures the notion of $q$ as a set of
states \cite{GSE:barbedbisimulation}. 

\begin{lemma}[\textrm{F{\footnotesize IELD}} \textrm{A{\footnotesize CCESS}} preserves $\sqsubset\ \subseteq F_\sim(\sqsubset)$]
If $P \in 2^{S_\mathit{FA}}$ and $q \in S$ are such that $P \sqsubset q$, then $(P\ q)$ is in the functional associated to bisumlation.
\label{lem:access}
$$
\forall P \in 2^{S_\mathit{FA}}\ (P \sqsubset q \Rightarrow (P\ q) \in F_\sim(\sqsubset))
$$
\end{lemma}

\begin{proof}
Proof by contradiction: assume $P \sqsubset q \wedge (P\ q) \not\in F_\sim(\sqsubset)$.

\noindent\textbf{Sketch}: Choose any $P$ and $q$ at the field access continuation such
that $P \sqsubset q$. In the forward simulation
\eqref{eqn:BisimulationForwards}, for each $P^\prime$ such that $P
\rsgse P^\prime$, pick some $p^\prime \in P^\prime$. By definition $p \rgse
p^\prime$, $p \in P$, and $p \sqsubset q$. The proof shows the
existence of $q^\prime$ such that $q \rsym q^\prime$ and $p^\prime
\sqsubset q^\prime$ using the existing homomorphism in $p \sqsubset q$.

In the backward simulation \eqref{eqn:BisimulationBackwards}, for each
$q^\prime$ such that $q \rsym q^\prime$, the proof posits the
existences of some $p^\prime \sqsubset q^\prime$ and uses the
homomorphism in $p^\prime \sqsubset q^\prime$ to derive a $p$ such
that $p \rgse p^\prime$ and $p \in P$.

As the forward and backward simulations hold for any $p \in P$, $P
\sqsubset q$ must be in the functional, $F_\sim(\sqsubset)$, which is
a contradiction.
\end{proof}
Similar lemmas are proved for field write and equals reference.

\begin{theorem}
\label{th:bisim}
The relation $\sqsubset$ is a bisimulation: $\sqsubset\ \subseteq\ \sim$
\end{theorem}
\begin{proof}
By definition of the bisimilarity relation. All of the common rules
make no changes to the heap, so $P \sqsubset q$ is included in the
functional. The other transition relations are supported by lemmas
such as \lemref{lem:access}.
\end{proof}

The notation $p \stackrel{n}{\rgse} p^\prime$ denotes that $p^\prime$ is
the $n$-step successor of $p$, if it exists, and $q
\stackrel{n}{\rsym} q^\prime$ is similarly defined. Completeness and
soundness are readily established by \thref{th:bisim} and inducting
over $n$.

\begin{corollary}[$\rsym$ is complete]
If $P \in 2^{S_\mathit{FA}}$ and $q \in S$ are such that $P \sqsubset q$ then for any $p \in P$
$$
\forall p^\prime\ (p \stackrel{n}{\rgse} p^\prime \Rightarrow \exists q^\prime\ (q \stackrel{n}{\rsym} q^\prime \wedge p^\prime \sqsubset q^\prime))
$$
\end{corollary}

\begin{corollary}[$\rsym$ is sound]
If $P \in 2^{S_\mathit{FA}}$ and $q \in S$ are such that $P \sqsubset q$ then
$$
\forall q^\prime\ (q \stackrel{n}{\rsym} q^\prime \Rightarrow \exists p \in P\ (\exists p^\prime\ (p \stackrel{n}{\rgse} p^\prime \wedge p^\prime \sqsubset q^\prime)))
$$
\end{corollary}

The relation, $P \sqsubset q$, is readily established in the initial
state of a given program, as there is single initial state, $P_o =
\{p_o\}$, in any valid Javalite program. The initial state $\rsym$ is
then defined as $q_o = p_o$; $P_o \sqsubset q_o$ is trivially shown.
