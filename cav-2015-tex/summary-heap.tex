\section{Summary Heap Framework}
The summary heap algorithm is presented using Javalite, an
executable formalization of a Java-like language~\cite{saints-MS}. Javalite retains 
the essence of Java and other object-oriented languages but with formalizations 
in both PLT Redex and Coq so that the outcome of any program has a precise
and deterministic meaning.

Javalite is a syntactic machine defined as rewrites on a string.  The
semantics are small-step using the CESK architecture with a (C)ontrol string
representing the expression being evaluated, an (E)nvironment for
local variables, a (S)tore for the heap, and a (K)ontinuation for what
is to be executed next.  In this work we change the semantics of the original
 Javalite machine and its machine syntax to represent (i) GSE~\cite{GSE03}, and (ii) the summary heap algorithm
presented in this work.

The surface syntax (input) for Javalite is shown 
in~\figref{fig:surface-syntax}, and the machine syntax (state)
in~\figref{fig:machine-syntax}. Terminals are in bold face while
non-terminals are italicized. Ellipses indicate zero or more
repetitions. Tuples omit the commas for compactness. The language
itself uses s-expressions for convenience.

\begin{figure}
\begin{center}
\cfgstart
\cfgrule{P}{\lp $\mu$ \lp \cfgnt{C} \cfgnt{m}\rp\rp}
\cfgrule{$\mu$}{(\cfgnt{CL} ...)}
\cfgrule{T}{\cfgt{bool} \cfgor \cfgnt{C}}
\cfgrule{CL}{\lp\cfgt{class} \cfgnt{C} \lp\lb\cfgnt{T} \cfgnt{f}\rb ...\rp \lp\cfgnt{M} ...\rp}
\cfgrule{M}{\lp\cfgnt{T} \cfgnt{m} \lb\cfgnt{T} \cfgnt{x}\rb\  e\rp}
\cfgrule{e}{\cfgnt{x}
\cfgor{\lp\cfgt{new} \cfgnt{C}\rp}
\cfgor{\lp\cfgnt{e} \cfgt{\$} \cfgnt{f}\rp}
\cfgor{\lp\cfgnt{x} \cfgt{\$} \cfgnt{f} \cfgt{:=} \cfgnt{e}\rp}
\cfgor{\lp\cfgnt{e} \cfgt{=} \cfgnt{e}\rp}}
\cfgorline{\lp\cfgt{if} \cfgnt{e} \cfgnt{e} \cfgt{else} \cfgnt{e}\rp 
\cfgor {\lp\cfgt{var} \cfgnt{T} \cfgnt{x} \cfgt{:=} \cfgnt{e} \cfgt{in} \cfgnt{e}\rp}
\cfgor {\lp\cfgnt{e} \cfgt{@} \cfgnt{m} \cfgnt{e} \rp}}
\cfgorline{\lp\cfgnt{x} \cfgt{:=} \cfgnt{e}\rp
\cfgor{\lp\cfgt{begin} \cfgnt{e} ...\rp}
\cfgor{\cfgnt{v}}}
\cfgrule{x}{\cfgt{this} \cfgor \cfgnt{id}}
\cfgrule{f,m,C}{\cfgnt{id}}
%\cfgrule{m}{\cfgnt{id}}
%\cfgrule{C}{\cfgnt{id}}
\cfgrule{v}{\cfgnt{r} \cfgor \cfgt{null} \cfgor \cfgt{true} \cfgor \cfgt{false} \cfgor \cfgt{error}}
\cfgrule{r}{\cfgt{number}}
\cfgrule{id}{\cfgt{variable-not-otherwise-mentioned}}
\cfgend
\end{center}
\caption{The Javalite surface syntax.}
\label{fig:surface-syntax}
\end{figure}

\begin{figure}
\begin{center}
\cfgstart
\cfgrule{e}{\lp ... \cfgor \lp \cfgnt{v} \cfgt{@} \cfgnt{m} \cfgnt{v} \rp\rp}
%\cfgrule{object}{ (\cfgnt{C} [ \cfgnt{f} \cfgnt{loc} ] ...) }
%\cfgrule{hv}{ (\cfgnt{v} \cfgnt{object})}
\cfgrule{$\phi$}{\cfgnt{constraint}}
\cfgrule{l}{\cfgt{number}}
%\cfgrule{h}{(\cfgnt{mt}\ (\cfgnt{h}\ [\cfgnt{loc} $\rightarrow$ \cfgnt{hv}]) )}
\cfgrule{$\eta$}{(\cfgnt{mt}\ ($\eta$ [\cfgnt{x} $\rightarrow$ \cfgnt{loc}]))}
\cfgrule{s}{\lp$\mu$ \cfgnt{L} \cfgnt{R} \cfgnt{g} $\eta$ \cfgnt{e} \cfgnt{k}\rp}
\cfgrule{k}{\cfgt{end}}
\cfgorline{\lp \cfgt{*} \cfgt{\$} \cfgnt{f} $\rightarrow$ \cfgnt{k}\rp}
\cfgorline{\lp \cfgt{*} \cfgt{@} \cfgnt{m} \lp \cfgnt{e} ... \rp $\rightarrow$ \cfgnt{k} \rp}
\cfgorline{\lp \cfgnt{v} \cfgt{@} \cfgnt{m} \cfgnt{v} \cfgt{*} \lp \cfgnt{e} ... \rp $\rightarrow$ \cfgnt{k} \rp}
\cfgorline{\lp \cfgt{*} \cfgt{=} \cfgnt{e} $\rightarrow$ \cfgnt{k}\rp}
\cfgorline{\lp \cfgt{v} \cfgt{=} \cfgnt{*} $\rightarrow$ \cfgnt{k}\rp}
\cfgorline{\lp \cfgt{x} \cfgt{:=} \cfgnt{*} $\rightarrow$ \cfgnt{k}\rp}
\cfgorline{\lp \cfgt{x} \cfgt{\$} \cfgnt{f}  \cfgt{:=} \cfgnt{*} $\rightarrow$ \cfgnt{k}\rp}
\cfgorline{\lp \cfgt{if} \cfgnt{*} \cfgnt{e} \cfgt{else} \cfgnt{e} $\rightarrow$ \cfgnt{k} \rp}
\cfgorline{\lp\cfgt{var} \cfgnt{T} \cfgnt{x} \cfgt{:=} \cfgnt{*} \cfgt{in} \cfgnt{e}  $\rightarrow$ \cfgnt{k} \rp}
\cfgorline{\lp\cfgt{begin}  \cfgnt{*} \lp \cfgnt{e} ...\rp $\rightarrow$ \cfgnt{k} \rp}
\cfgorline{\lp\cfgt{pop} $\eta$ \cfgnt{k}\rp}
\cfgend
\end{center}
\caption{The machine syntax for Javalite.}
\label{fig:machine-syntax}
\end{figure}


A Javalite program, $\cfgnt{P}$, is a registry of classes, $\mu$, with
a tuple indicating a class, $\cfgnt{C}$, and a method, $\cfgnt{m}$,
where execution starts. The only primitive type in Javalite is
Boolean. Classes have non-primitive fields, $([\cfgnt{C}\ \cfgnt{f}]\ ...)$, and
methods, $(\cfgnt{M}\ ...)$.\footnote{Summary heaps are able to support non-primitive fields (see \secref{sec:eval}). They are omitted to focus and simplify the presentation.} Methods are limited to a single
parameter in our Javalite machine. Expressions, $\cfgnt{e}$,  
include statements, and they
 use '$\cfgt{:=}$' to indicate assignment and '$\cfgt{=}$' to indicate comparison.
The dot-operator for field access is replaced by '$\cfgt{\$}$', and the dot-operator
for method invocation is replaced by '$\cfgt{@}$' since '.' is reserved in PLT Redex. There is no explicit return
statement in Javalite; rather, the value of the last expression is
used as the return value. A variable is always indicated by \cfgnt{x}
and a value by \cfgnt{v}. A value can be a reference in the heap, $\cfgnt{r}$, or any of the special values shown in~\figref{fig:surface-syntax}. 
We assume that only
type correct programs are given as input to the machine.

The state of the Javalite machine (\figref{fig:machine-syntax}),
$\cfgnt{s}$, includes the program registry, $\mu$, a pair of functions defining the heap
$\lp\cfgnt{L}\ \cfgnt{R}\rp$, a constraint,
$\phi$, defined over references in the heap, the environment, $\eta$,
for local variables, and the continuation $\cfgnt{k}$.\footnote{The registry $\mu$ never changes so it is omitted from the state tuple in the rest of the presentation for compactness.}  The more
complex structures, such as the environment and heap, are defined as lists that
start with empty, \cfgnt{mt}. The rewrite rules that define the
semantics treat these lists as partial functions. As such,
$\eta(\cfgnt{x}) = \cfgnt{r}$ is the reference $\cfgnt{r}$ associated with variable
$\cfgnt{x}$. The notation $\eta^\prime = \eta[\cfgnt{x} \mapsto
  \cfgnt{v}]$ defines a new partial function $\eta^\prime$ that is
the same as $\eta$ except that the variable $\cfgnt{x}$ now maps to
$\cfgnt{v}$.

The continuation $\cfgnt{k}$ helps accomplish the small-step semantics
by indicating where the hole, $\cfgt{*}$, is located, by storing
temporary computations, and by keeping track of the next
computation. For example, the continuation
$\lp\cfgt{*}\ \cfgt{\$}\ \cfgnt{f} \rightarrow \cfgnt{k}\rp$ indicates
that the machine is currently computing the expression for the object
reference on which the field $\cfgnt{f}$ is going to be accessed. Once
the field access is complete, the machine continues with $\cfgnt{k}$.

\subsection{Summary Heaps as Bipartite Graphs}
A novel contribution of summary heaps for symbolic execution is the
labeled bipartite graph representation of the heap which enables a
summary heap to group heaps together that follow the same control flow
to a given point of execution. The graph structure is able to capture
group membership and compute value sets using entirely non-recursive
local lookup. Additionally, updates on the heap do not require
rewriting the path condition.

The bipartite graph itself consists of references, $\cfgnt{r}$, and
constraint-location pairs, $\lp\phi\ \cfgnt{l}\rp$. Locations may be 
altered to perform updates, whereas references are immutable
once they are added to the graph. The graph is
expressed in the location map, $\cfgnt{L}$, and the reference map,
$\cfgnt{R}$. As done with the environment, $\cfgnt{L}$ and $\cfgnt{R}$
are treated as partial functions where $\cfgnt{L}(r) =
\{(\phi\ \cfgnt{l})\ ...\}$ is the set of constraint-location pairs in
the heap associated with the given reference, and
$\cfgnt{R}(\cfgnt{l},\cfgnt{f}) = \cfgnt{r}$ is the reference
associated with the given location-field pair in the heap.  

There are several properties of the summary heap enforced by the
rewrite rules that are invariant and important to the correctness
proofs in \secref{sec:bisim}: \emph{reference partitions},
\emph{immutability}, \emph{determinism}, \emph{type
  consistency}, and \emph{null and uninitialized}.

\emph{Reference partitions}: new references drawn from
  three distinct partitions (i) $\mathrm{init}_\cfgnt{r}()$ for references for the \emph{input heap}, (ii) $\mathrm{fresh}_\cfgnt{r}()$ for references for \emph{auxiliary literals}, and
  (iii) $\mathrm{stack}_\cfgnt{r}()$ for references for \emph{stack literals} in the expressions and
  continuations. The references strictly increase in value on each
  call, and modular arithmetic is used to determine the partition to
  which a reference belongs.
\begin{compactdesc}
\item[Input heap:] these references have two
  interpretations: one as a literal and another as a variable. As a literal it is a node in
  the summary heap. As a variable it is a term in any number of constraints that label
  edges in the summary heap. Only input heap references and the
  special reference $\cfgnt{r}_\mathit{null}$ appear in constraints
\item[Auxiliary literals:] these references are only used to define additional structure in
  the bipartite graph and are not part of any constraint and do not appear
  in any environment, expression, or continuation.
\item[Stack references:] these references are also used to define additional structure in the bipartite graph but may also appear in the environment, expression, or continuation. These are not part of any constraint.
\end{compactdesc}
Reference partitions are a critical piece to the existence proof of a
bisimilation between the summary heap algorithm and generalized
symbolic execution (see \secref{sec:bisim}).

\emph{Immutability} means that a reference does not change in the
location map. It always points to the same set of constraint location
pairs. This property ensures that constraints never need to be
rewritten as the program evolves its state.

\emph{Determinism} means that a reference in $(\cfgnt{L}\ \cfgnt{R})$ cannot point to multiple locations simultaneously. In the definition $\cfgnt{L}^\leftarrow$ is the pre-image of the location map.
$$
\begin{array}{l}
\forall \cfgnt{r} \in \cfgnt{L}^\leftarrow\ (\forall (\phi\ \cfgnt{l}),(\phi^\prime\ \cfgnt{l}^\prime) \in \cfgnt{L}(r)\ (\\
\ \ \ \ (\cfgnt{l} \neq \cfgnt{l}^\prime \vee \phi \neq \phi^\prime) \Rightarrow (\phi \wedge \phi^\prime = \cfgt{false}))
\end{array}
$$
Informally, any two constraint location pairs connected to a
common reference form a contradiction: only one can be true in any
satisfying assignment. The property ensures that any satisfying assignment
resolving aliasing relationships results in a deterministic heap: a
field access returns a single location under that assignment.

\emph{Type consistency} implies that all locations associated with a reference have the same type.
\[
\begin{array}{l}
\forall \cfgnt{r} \in \cfgnt{L}^\leftarrow\ (\forall (\phi\ \cfgnt{l}),(\phi^\prime\ \cfgnt{l}^\prime) \in \cfgnt{L}(r)\ (\\
\ \ \ \ (\mathrm{Type}\lp\cfgnt{l}\rp = \mathrm{Type}\lp\cfgnt{l}^\prime\rp)))
\end{array}
\]

\emph{Null and Uninitialized} means that every summary heap contains 
a special location for null ($\cfgnt{l}_\mathit{null}$), and a special
location for uninitialized
  ($\cfgnt{l}_\mathit{un}$), with corresponding references
  $\cfgnt{r}_\mathit{null}$ and
  $\cfgnt{r}_\mathit{un}$, respectively. $\cfgnt{L}(\cfgnt{r}_\mathit{null}) =
  \{(\cfgt{true}\ \cfgnt{l}_\mathit{null})\}$ and
  $\cfgnt{L}(\cfgnt{r}_\mathit{un}) =
  \{(\cfgt{true}\ \cfgnt{l}_\mathit{un})\}$. These special references can be used in constraints.

To summarize, the heap summary algorithm creates
many references that are auxiliary literals or stack literals to
define the structure of the bipartite graph.  It only uses input heap
references for constraints expressing potential aliasing. A solver
reasons over the aliasing relationships, and a satisfying assignment
uniquely identifies active edges in the graph. Those active edges
describe a concrete heap and its shape.

For example, if the solver assigns input heap variables $\cfgnt{r}$ and $\cfgnt{r}^\prime$
such that $\cfgnt{r} = \cfgnt{r}^\prime$ is true, then there exists
some location $\cfgnt{l}$ and constraints $\phi$ and $\phi^\prime$ such
that
\begin{compactitem}
\item[-] $\lp\phi\ \cfgnt{l}\rp \in \cfgnt{L}(\cfgnt{r})$; and
\item[-] $\lp\phi^\prime\ \cfgnt{l}\rp \in \cfgnt{L}(\cfgnt{r}^\prime)$; and
\item[-] $\phi$ and $\phi^\prime$ hold in the aliasing assignments from the solver.
\end{compactitem}
In this way, the graph groups heaps with the conditions under which
those heaps exist. This representation separates the constraint
problem from the morphology problem: it is possible to update the heap
without having to rewrite constraints. Without this separation, it
would not be possible to lazily initialize heap locations along a
program path.

\subsection{State Transition Relation}

The initial state of a Javalite program is constructed from its
surface syntax on the $\cfgnt{P}$ production:
$\lp\mu\ \lp\cfgnt{C}\ \cfgnt{m}\rp\rp$. Assuming that
$\lp\cfgnt{L}\ \cfgnt{R}\rp$ is an empty heap with no references
or locations and $\eta$ is an empty environment with no
variables, the initial state is
$$
\begin{array}{l}
\lp\mu 
\ \cfgnt{L}[\cfgnt{r}_\mathit{null} \mapsto \{\lp\cfgt{true}\ \cfgnt{l}_\mathit{null})\}\rp] 
           [\cfgnt{r}_\mathit{un} \mapsto \{\lp\cfgt{true}\ \cfgnt{l}_\mathit{un}\rp\}] \\
\ \cfgnt{R}
\ \cfgt{true}\ \eta\  \lp\lp\cfgt{new}\ \cfgnt{C}\rp\ \cfgt{@}\ \cfgnt{m}\ \cfgt{true}\rp\rp\ \cfgt{end}\rp
\end{array}
$$
The first expression in the initial state is to construct a new object
in the summary heap of type $\cfgnt{C}$. Note that the resulting object is not referenced
by the input heap since it comes from the new-expression.  Once an
instance of $\cfgnt{C}$ exists in the summary heap, the method
$\cfgnt{m}$ of the class is invoked.

In general, the semantics of Javalite are expressed as
rewrites on strings using pattern matching. Consider the rewrite rule
for the beginning of a field access instruction in the Javalite
machine:
$$
\mprset{flushleft}
	\inferrule[Field Access(eval)]{}{
      \lp \cfgnt{L}\ \cfgnt{R}\ \phi\ \eta\ \lp \cfgnt{e}\ \cfgt{\$}\ \cfgnt{f}\rp \ \cfgnt{k}\rp  \rcom 
      \lp \cfgnt{L}\ \cfgnt{R}\ \phi\ \eta\ \cfgnt{e}\ \lp \cfgt{*}\ \cfgt{\$}\ \cfgnt{f} \rightarrow \cfgnt{k}\rp \rp 
	}
$$
If the string representing the current state matches the left side, then it
creates the new string on the right. In this example, the new string
on the right is now evaluating the expression $\cfgnt{e}$ in the field
access, and it includes the continuation indicating that it still
needs to complete the actual field access once the expression is
evaluated.

Other more complex rules, such as one to create a new instance of a
class, define constraints on the rewrites and more complex
transformations on the heap.
$$
\mprset{flushleft}
	\inferrule[New]{
      \cfgnt{r} = \mathrm{stack}_r\lp \rp\\
      l = \mathrm{fresh}_l\lp \cfgnt{C}\rp\\\\
      \cfgnt{R}^\prime = \cfgnt{R}[\forall \cfgnt{f} \in \mathit{fields}\lp \mathrm{C}\rp \ \lp \lp l\ \cfgnt{f}\rp  \mapsto \cfgnt{r}_\mathit{null} \rp ] \\\\
      \cfgnt{L}^\prime = \cfgnt{L}[\cfgnt{r} \mapsto \{\lp \cfgt{true}\ l\rp \}]
    }{
      \lp \cfgnt{L}\ \cfgnt{R}\ \phi\ \eta\ \lp \cfgt{new}\ \cfgnt{C}\rp \ \cfgnt{k}\rp  \rcom
      \lp \cfgnt{L}^\prime\ \cfgnt{R}^\prime\ \phi\ \eta\ \cfgnt{r}\ \cfgnt{k}\rp 
	}
$$
In this rule, when the string matches the new-expression, it is rewritten to use
a new heap location where all of the fields for the new object point to
$\cfgnt{r}_\mathit{null}$
and the location map points a new stack reference to that new object.

To show the heap summary algorithm is sound and complete with respect
to properties proved by GSE, we define a transition relation that
relates states according GSE and refer to it as $\rgse$ or $p \rgse
p^\prime$. Similarly we define for the new summary heap algorithm
$\rsym$ or $q \rsym q^\prime$. For simplicity, both use lazy
initialization as opposed to lazier or lazier\#.

The full operational semantics for $\rgse$ are in the supplemental
document. In the next section, we present the rewrite rules specific
to $\rsym$.

