\section{Proofs}

\subsection{Definitions}

\begin{definition}
A \textbf{state transition function} $\Phi$ is a mapping $\Phi : s \mapsto s$ , which takes one machine state and transforms it into another machine state.
\end{definition}

\begin{definition}
Given a sequence of states $$\Sigma_n = s_0,s_1,...,s_n$$ where $$s_i = \lp \mu_i\ \cfgnt{L}_i\ \cfgnt{R}_i\ \phi_i\ \eta_i\ \cfgnt{e}_i\ \cfgnt{k}_i\rp$$ the \textbf{control flow sequence} of $\Sigma_n$ is the defined as the sequence of tuples $$ \sigma_n = \mathbb{CF}(\Sigma_n) = (\eta_0,\cfgnt{e}_0,\cfgnt{k}_0),(\eta_1,\cfgnt{e}_1,\cfgnt{k}_1),...,(\eta_n,\cfgnt{e}_n,\cfgnt{k}_n)$$
\end{definition}

\begin{definition}
A \textbf{feasible state sequence} is defined as a sequence of states resulting from repeated application of the state transition relation: $$ \Sigma_n = s_0, \Phi (s_0) ,..., \Phi_{n-1} (s_0) $$
\end{definition}

\begin{definition}
Given a state transition function $\Phi$, an initial state $s_0$ and a control flow sequence $\sigma_n$, the \textbf{feasible state set} $\zeta = \mathbb{FS}(\Phi,s_0,sigma)$ is defined as
 $$\zeta = \{ \forall s | \sigma_n = \mathbb{CF}(\Sigma_n) \wedge s = max_s(\Sigma) \wedge s = \Phi_{n-1}(s_o)\} $$
\end{definition}

\begin{definition}
A symbolic system state represents a set of lazy system states. To denote the representation relationship, we use the expression $\mathcal{L}\sqsubset \mathcal{S} $ to say that lazy system state $\mathcal{L}:\lp \mu_{\mathcal{L}}\ \cfgnt{L}_{\mathcal{L}}\ \cfgnt{R}_{\mathcal{L}}\ \phi_{\mathcal{L}}\ \eta_{\mathcal{L}}\ \cfgnt{e}_{\mathcal{L}}\ \cfgnt{k}_{\mathcal{L}}\rp$ is represented by symbolic state $\mathcal{S}:\lp \mu_{\mathcal{S}}\ \cfgnt{L}_{\mathcal{S}}\ \cfgnt{R}_{\mathcal{S}}\ \phi_{\mathcal{S}}\ \eta_{\mathcal{S}}\ \cfgnt{e}_{\mathcal{S}}\ \cfgnt{k}_{\mathcal{S}}\rp$ . The \textbf{representation relation} is defined as follows: $\mathcal{L}\sqsubset \mathcal{S} $ if and only if 
$$\eta_{\mathcal{L}} = \eta_{\mathcal{S}} ,\ \cfgnt{e}_{\mathcal{L}} = \cfgnt{e}_{\mathcal{S}} ,\ \cfgnt{k}_{\mathcal{L}} = \cfgnt{k}_{\mathcal{S}}$$
and there exists functions $g:\cfgnt{r} \mapsto \cfgnt{r}$ and $h:\cfgnt{l} \mapsto \cfgnt{l}$ such that for any reference $r \in \mathcal{L}$, location $l \in \mathcal{L}$, and field $f$, $$ l = L_{\mathcal{L}}(r) \leftrightarrow h(l)\in L_{\mathcal{S}}(g(r))$$ and $$ r = R_{\mathcal{L}}(\cfgnt{l},f) \leftrightarrow g(r) = R_{\mathcal{S}}(h(l),f)$$ and the subheap of state $\mathcal{S}$ formed by selecting the image of $g$ and $h$ is feasible.

\end{definition}

\begin{definition}
A symbolic state $\mathcal{S}$ is \textbf{exact} if and only if it represents the set of all feasible lazy states on the current execution path, and represents no infeasible state.
\[
 \forall (R_S^i\ L_S^i) \in \mathit{Heaps}(S_s)\ (\exists (R_L\ S_L)\ ((R_L\ S_L) = (R_S^i\ L_S^i))) \\ \\
\]

\end{definition}

\subsection{Theorems}

\begin{lemma}
If symbolic state $\mathcal{S}$ is exact prior to executing the Field Access rule, then the state $\mathcal{S}^\prime $ is also exact.
\end{lemma}

\begin{proof}
Suppose we have some symbolic state  $\mathcal{S} =  \lp \cfgnt{L}_{\mathcal{S}}\ \cfgnt{R}_{\mathcal{S}}\ \phi_g\ \eta\ \cfgnt{r}\ \lp \cfgt{*}\ \cfgt{\$}\ \cfgnt{f} \rightarrow \cfgnt{k}\rp \rp $ , and that all relevant fields are initialized. Take an arbitrary lazy state $\mathcal{L} \sqsubset \mathcal{S}$. Since $\mathcal{S}$ is exact,  $\mathcal{L} = \lp \cfgnt{L}_{\mathcal{L}}\ \cfgnt{R}_{\mathcal{L}}\ \phi_L\ \eta\ \cfgnt{r}\ \lp \cfgt{*}\ \cfgt{\$}\ \cfgnt{f} \rightarrow \cfgnt{k} \rp \rp$. If we apply the lazy field access rule to $\mathcal{L}$, we achieve state $\mathcal{L}^\prime = \lp \cfgnt{L}_{\mathcal{L}} [\cfgnt{r}^\prime \mapsto \lp\phi^\prime\ l^\prime\rp]\ \cfgnt{R}_{\mathcal{L}}\ \phi_L\ \eta\ \cfgnt{r}^\prime\ \cfgnt{k}\rp $. Likewise, by applying the symbolic field access rule to $\mathcal{S}$, we obtain state $\mathcal{S}^\prime = \lp \cfgnt{L}_{\mathcal{S}}[\cfgnt{r}^\prime \mapsto \mathbb{VS}\lp \cfgnt{L}_{\mathcal{S}},\cfgnt{R}_{\mathcal{S}},\cfgnt{r},\cfgnt{f},\phi_g\rp ]\ \cfgnt{R}_{\mathcal{S}}\ \phi_g\ \eta\ \cfgnt{r}^\prime\ \cfgnt{k}\rp $.

We now show that state $\mathcal{S}^\prime$ represents state $\mathcal{L}^\prime $. Since $\eta$, $e$, and $k$ are identical between $\mathcal{S}^\prime$ and $\mathcal{L}^\prime $, the first condition is met by default. Constructing the reference and location mappings is only slightly less trivial: $$g^\prime = g$$ $$h^\prime = h[ r^\prime \mapsto r^\prime]$$ 

It only remains to show that the image of $\mathcal{L}^\prime$ in $\mathcal{S}^\prime$ is feasible, WHICH WILL BE EASY ONCE WE DEFINE WHAT THAT MEANS.

Now, we show that no infeasible lazy states are allowed by $\mathcal{S}^\prime$. NEED TO FILL IN THIS BIT, TOO

$\mathbb{S}(\bigwedge_{(\phi l) \in L_s^i} \phi)) \wedge \forall l \in L_L (\exists l \in L_s^i)$
\end{proof}


