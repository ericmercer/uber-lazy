\section{Javalite}
\figref{fig:surface-syntax} defines the surface syntax for the
Javalite language \cite{saints-MS}. \figref{fig:machine-syntax} is the
machine syntax. Javalite is syntactic machine defined as rewrites on a
string. The semantics use a CEKS model with a (C)ontrol string
representing the expression being evaluated, an (E)nvironment for
local variables, a (K)ontinuation for what is to be executed next, and
a (S)tore for the heap.

\begin{figure}
\begin{center}
\cfgstart
\cfgrule{P}{\lp $\mu$ \lp \cfgnt{C} \cfgnt{m}\rp\rp}
\cfgrule{$\mu$}{(\cfgnt{CL} ...)}
\cfgrule{T}{\cfgt{bool} \cfgor \cfgnt{C}}
\cfgrule{CL}{\lp\cfgt{class} \cfgnt{C} \lp\lb\cfgnt{T} \cfgnt{f}\rb ...\rp \lp\cfgnt{M} ...\rp}
\cfgrule{M}{\lp\cfgnt{T} \cfgnt{m} \lb\cfgnt{T} \cfgnt{x}\rb\  e\rp}
\cfgrule{e}{\cfgnt{x}
\cfgor{\lp\cfgt{new} \cfgnt{C}\rp}
\cfgor{\lp\cfgnt{e} \cfgt{\$} \cfgnt{f}\rp}
\cfgor{\lp\cfgnt{x} \cfgt{\$} \cfgnt{f} \cfgt{:=} \cfgnt{e}\rp}
\cfgor{\lp\cfgnt{e} \cfgt{=} \cfgnt{e}\rp}}
\cfgorline{\lp\cfgt{if} \cfgnt{e} \cfgnt{e} \cfgt{else} \cfgnt{e}\rp 
\cfgor {\lp\cfgt{var} \cfgnt{T} \cfgnt{x} \cfgt{:=} \cfgnt{e} \cfgt{in} \cfgnt{e}\rp}
\cfgor {\lp\cfgnt{e} \cfgt{@} \cfgnt{m} \cfgnt{e} \rp}}
\cfgorline{\lp\cfgnt{x} \cfgt{:=} \cfgnt{e}\rp
\cfgor{\lp\cfgt{begin} \cfgnt{e} ...\rp}
\cfgor{\cfgnt{v}}}
\cfgrule{x}{\cfgt{this} \cfgor \cfgnt{id}}
\cfgrule{f,m,C}{\cfgnt{id}}
%\cfgrule{m}{\cfgnt{id}}
%\cfgrule{C}{\cfgnt{id}}
\cfgrule{v}{\cfgnt{r} \cfgor \cfgt{null} \cfgor \cfgt{true} \cfgor \cfgt{false} \cfgor \cfgt{error}}
\cfgrule{r}{\cfgt{number}}
\cfgrule{id}{\cfgt{variable-not-otherwise-mentioned}}
\cfgend
\end{center}
\caption{The Javalite surface syntax.}
\label{fig:surface-syntax}
\end{figure}

\begin{figure}
\begin{center}
\cfgstart
\cfgrule{e}{\lp ... \cfgor \lp \cfgnt{v} \cfgt{@} \cfgnt{m} \cfgnt{v} \rp\rp}
%\cfgrule{object}{ (\cfgnt{C} [ \cfgnt{f} \cfgnt{loc} ] ...) }
%\cfgrule{hv}{ (\cfgnt{v} \cfgnt{object})}
\cfgrule{$\phi$}{\cfgnt{constraint}}
\cfgrule{l}{\cfgt{number}}
%\cfgrule{h}{(\cfgnt{mt}\ (\cfgnt{h}\ [\cfgnt{loc} $\rightarrow$ \cfgnt{hv}]) )}
\cfgrule{$\eta$}{(\cfgnt{mt}\ ($\eta$ [\cfgnt{x} $\rightarrow$ \cfgnt{loc}]))}
\cfgrule{s}{\lp$\mu$ \cfgnt{L} \cfgnt{R} \cfgnt{g} $\eta$ \cfgnt{e} \cfgnt{k}\rp}
\cfgrule{k}{\cfgt{end}}
\cfgorline{\lp \cfgt{*} \cfgt{\$} \cfgnt{f} $\rightarrow$ \cfgnt{k}\rp}
\cfgorline{\lp \cfgt{*} \cfgt{@} \cfgnt{m} \lp \cfgnt{e} ... \rp $\rightarrow$ \cfgnt{k} \rp}
\cfgorline{\lp \cfgnt{v} \cfgt{@} \cfgnt{m} \cfgnt{v} \cfgt{*} \lp \cfgnt{e} ... \rp $\rightarrow$ \cfgnt{k} \rp}
\cfgorline{\lp \cfgt{*} \cfgt{=} \cfgnt{e} $\rightarrow$ \cfgnt{k}\rp}
\cfgorline{\lp \cfgt{v} \cfgt{=} \cfgnt{*} $\rightarrow$ \cfgnt{k}\rp}
\cfgorline{\lp \cfgt{x} \cfgt{:=} \cfgnt{*} $\rightarrow$ \cfgnt{k}\rp}
\cfgorline{\lp \cfgt{x} \cfgt{\$} \cfgnt{f}  \cfgt{:=} \cfgnt{*} $\rightarrow$ \cfgnt{k}\rp}
\cfgorline{\lp \cfgt{if} \cfgnt{*} \cfgnt{e} \cfgt{else} \cfgnt{e} $\rightarrow$ \cfgnt{k} \rp}
\cfgorline{\lp\cfgt{var} \cfgnt{T} \cfgnt{x} \cfgt{:=} \cfgnt{*} \cfgt{in} \cfgnt{e}  $\rightarrow$ \cfgnt{k} \rp}
\cfgorline{\lp\cfgt{begin}  \cfgnt{*} \lp \cfgnt{e} ...\rp $\rightarrow$ \cfgnt{k} \rp}
\cfgorline{\lp\cfgt{pop} $\eta$ \cfgnt{k}\rp}
\cfgend
\end{center}
\caption{The machine syntax for Javalite.}
\label{fig:machine-syntax}
\end{figure}


The environment, $\eta$, associates a variable $\cfgnt{x}$ with a
value $\cfgnt{v}$. The value can be a reference, $\cfgnt{r}$ or one of
the special values $\cfgt{null}$, $\cfgt{true}$, or
$\cfgt{false}$. Although the Javalite machine is purely syntactic, for
clarity and brevity in the presentation, the more complex structures
such as the environment are treated as partial functions. As such,
$\eta(\cfgnt{x}) = \cfgnt{r}$ is the reference mapped to the variable
in the environment. The notation $\eta^\prime = \eta[\cfgnt{x} \mapsto
  \cfgnt{v}]$ defines a new partial function $\eta^\prime$ that is
just like $\eta$ only the variable $\cfgnt{x}$ now maps to
$\cfgnt{v}$.

The heap is a labeled bipartite graph consisting of references,
$\cfgnt{r}$, and locations, $\cfgnt{l}$. The machine syntax in
\figref{fig:machine-syntax} defines that graph in $\cfgnt{L}$, the
location map, and $\cfgnt{R}$, the reference map. As done with the
environment, $\cfgnt{L}$ and $\cfgnt{R}$ are treated as partial
functions where $\cfgnt{L}(r) = \{(\phi\ \cfgnt{l})\ ...\}$ is the set
of location-constraint pairs in the heap associated with the given
reference, and$\cfgnt{R}(\cfgnt{l},\cfgnt{f}) = \cfgnt{r}$ is the
reference associated with the given location-field pair in the
heap. 

As the updates to $\cfgnt{L}$ and $\cfgnt{R}$ are complex in the
machine semantics, predicate calculus is used to describe updates to
the functions. Consider the following example where $\cfgnt{l}$ is
some location and$\rho$ is a set of references.
\[
\cfgnt{L}^\prime = \cfgnt{L}[\cfgnt{r} \mapsto \{\lp \cfgt{true}\ l\rp \}][\forall \cfgnt{r}^\prime \in \rho\ \lp \cfgnt{r}^\prime \mapsto \lp \cfgt{true}\ l_\mathit{null}\rp \rp ]
\]
The new function $\cfgnt{L}^\prime$ is just like $\cfgnt{L}$ only it
remaps $\cfgnt{r}$, and it remaps all the references in $\rho$.

The location $\cfgnt{l}_\mathit{null}$ is a special location in the
heap to represent null. It has a companion reference
$\cfgnt{r}_\mathit{null}$. The initial heap for the machine is defined such that $\cfgnt{L}(\cfgnt{r}_\mathit{null}) = \{(\cfgt{true}\ \cfgnt{l}_\mathit{null})}$

$\cfgnt{R}(\cfgnt{l},\cfgnt{f}) = \cfgnt{r}$ is the
reference associated with the given location-field pair in the heap,
and $\cfgnt{L}(r) = \{(\cfgnt{l}\ \phi)\ ...\}$

\begin{figure*}[t]
\begin{center}
\mprset{flushleft}
\begin{mathpar}
	\inferrule[Variable lookup]{}{
      \lp \cfgnt{L}\ \cfgnt{R}\ \phi\ \eta\ \cfgnt{x}\ \cfgnt{k}\rp  \rcom \\\\
      \lp \cfgnt{L}\ \cfgnt{R}\ \phi\ \eta\ \eta\lp \cfgnt{x}\rp \ \cfgnt{k}\rp 
	}
\and
	\inferrule[New]{
      \cfgnt{r} = \mathrm{stack}_r\lp \rp\\
      l = \mathrm{fresh}_l\lp \cfgnt{C}\rp\\\\
      \cfgnt{R}^\prime = \cfgnt{R}[\forall \cfgnt{f} \in \mathit{fields}\lp \mathrm{C}\rp \ \lp \lp l\ \cfgnt{f}\rp  \mapsto \cfgnt{r}_\mathit{null} \rp ] \\\\
      \cfgnt{L}^\prime = \cfgnt{L}[\cfgnt{r} \mapsto \{\lp \cfgt{true}\ l\rp \}]
    }{
      \lp \cfgnt{L}\ \cfgnt{R}\ \phi\ \eta\ \lp \cfgt{new}\ \cfgnt{C}\rp \ \cfgnt{k}\rp  \rcom
      \lp \cfgnt{L}^\prime\ \cfgnt{R}^\prime\ \phi\ \eta\ \cfgnt{r}\ \cfgnt{k}\rp 
	}
\and
	\inferrule[Field Access(eval)]{}{
      \lp \cfgnt{L}\ \cfgnt{R}\ \phi\ \eta\ \lp \cfgnt{e}\ \cfgt{\$}\ \cfgnt{f}\rp \ \cfgnt{k}\rp  \rcom \\\\
      \lp \cfgnt{L}\ \cfgnt{R}\ \phi\ \eta\ \cfgnt{e}\ \lp \cfgt{*}\ \cfgt{\$}\ \cfgnt{f} \rightarrow \cfgnt{k}\rp \rp 
	}
\and
	\inferrule[Field Write (eval)]{}{
       \lp \cfgnt{L}\ \cfgnt{R}\ \phi\ \eta\ \lp \cfgnt{x}\ \cfgt{\$}\ \cfgnt{f}\ \cfgt{:=}\ \cfgnt{e}\rp \ \cfgnt{k}\rp  \rcom \\\\
       \lp \cfgnt{L}\ \cfgnt{R}\ \phi\ \eta\ \cfgnt{e}\ \lp \cfgnt{x}\ \cfgt{\$}\ \cfgnt{f}\ \cfgt{:=}\ \cfgt{*}\ \rightarrow\ \cfgnt{k}\rp \rp 
	}
\and
    \inferrule[Equals (l-operand eval)]{}{
      \lp \cfgnt{L}\ \cfgnt{R}\ \phi\ \eta\ \lp \cfgnt{e}_0\ \cfgt{=}\ \cfgnt{e}\rp  \ \cfgnt{k}\rp  \rcom \\\\
      \lp \cfgnt{L}\ \cfgnt{R}\ \phi\ \eta\ \cfgnt{e}_0\ \lp \cfgt{*}\ \cfgt{=}\; \cfgnt{e} \rightarrow \cfgnt{k}\rp \rp 
    }
\and
    \inferrule[Equals (r-operand eval)]{}{
    \lp \cfgnt{L}\ \cfgnt{R}\ \phi\ \eta\ \cfgnt{v}\ \lp \cfgt{*}\; \cfgt{=}\; \cfgnt{e} \rightarrow \cfgnt{k}\rp \rp  \rcom \\\\
    \lp \cfgnt{L}\ \cfgnt{R}\ \phi\ \eta\ \cfgnt{e}\ \lp \cfgnt{v}\; \cfgt{=}\; \cfgt{*} \rightarrow \cfgnt{k}\rp \rp 
    }
\and
    \inferrule[Equals (bool)]{
    \cfgnt{v}_0 \in \{\cfgt{true}, \cfgt{false}\} \\
    \cfgnt{v}_1 \in \{\cfgt{true}, \cfgt{false}\} \\\\
    \cfgnt{v}_r = \mathrm{eq?}\lp \cfgnt{v}_0, \cfgnt{v}_1\rp}{
    \lp \cfgnt{L}\ \cfgnt{R}\ \phi\ \eta\ \cfgnt{v}_0\ \lp \cfgnt{v}_1\; \cfgt{=}\; \cfgt{*} \rightarrow \cfgnt{k}\rp \rp  \rcom \\\\
    \lp \cfgnt{L}\ \cfgnt{R}\ \phi\ \eta\ \cfgnt{v}_r\ \cfgnt{k}\rp 
    }
\and
    \inferrule[If-then-else (eval)]{}{
      \lp \cfgnt{L}\ \cfgnt{R}\ \phi\ \eta\ \lp \cfgt{if}\ \cfgnt{e}_0\ \cfgnt{e}_1\ \cfgt{else}\ \cfgnt{e}_2\rp \ \cfgnt{k}\rp  \rcom \\\\
      \lp \cfgnt{L}\ \cfgnt{R}\ \phi\ \eta\ \cfgnt{e}_0\ \lp \cfgt{if}\ \cfgt{*}\ \cfgnt{e}_1\ \cfgt{else}\ \cfgnt{e}_2\rp  \rightarrow \cfgnt{k}\rp 
	}
\and
	\inferrule[If-then-else (true) ]{}{
       \lp \cfgnt{L}\ \cfgnt{R}\ \phi\ \eta\ \cfgt{true}\ \lp \cfgt{if}\ \cfgt{*}\ \cfgnt{e}_1\ \cfgt{else}\ \cfgnt{e}_2\rp  \rcom \cfgnt{k}\rp  \rightarrow \\\\
       \lp \cfgnt{L}\ \cfgnt{R}\ \phi\ \eta\ \cfgnt{e}_1\  \cfgnt{k}\rp 
	}
\and
	\inferrule[If-then-else (false)]{}{
       \lp \cfgnt{L}\ \cfgnt{R}\ \phi\ \eta\ \cfgt{false}\ \lp \cfgt{if}\ \cfgt{*}\ \cfgnt{e}_1\ \cfgt{else}\ \cfgnt{e}_2\rp  \rcom \cfgnt{k}\rp  \rightarrow \\\\
       \lp \cfgnt{L}\ \cfgnt{R}\ \phi\ \eta\ \cfgnt{e}_2\  \cfgnt{k}\rp 
	}
\and
   \inferrule[Variable Declaration (eval)]{}{
    \lp \cfgnt{L}\ \cfgnt{R}\ \phi\ \eta\ \lp\cfgt{var}\ \cfgnt{T}\ \cfgnt{x}\ \cfgt{:=}\ \cfgnt{e}_0\ \cfgt{in}\ \cfgnt{e}_1\rp\ \cfgnt{k}\rp  \rcom \\\\
    \lp \cfgnt{L}\ \cfgnt{R}\ \phi\ \eta\ \cfgnt{e}_0\ \lp\cfgt{var}\ \cfgnt{T}\ \cfgnt{x}\ \cfgt{:=}\ \cfgt{*}\ \cfgt{in}\ \cfgnt{e}_1 \rightarrow \cfgnt{k}\rp\rp 
   }	
\and
   \inferrule[Variable Declaration]{}{
    \lp \cfgnt{L}\ \cfgnt{R}\ \phi\ \eta\ \cfgnt{v}\ \lp\cfgt{var}\ \cfgnt{T}\ \cfgnt{x}\ \cfgt{*}\ \cfgt{:=}\ \cfgt{in}\ \cfgnt{e}_1 \rightarrow \cfgnt{k}\rp\rp  \rcom \\\\
    \lp \cfgnt{L}\ \cfgnt{R}\ \phi\ \eta[x \mapsto \cfgnt{v}]\ \cfgnt{e}_1\ \lp \cfgt{pop}\ \eta\ \cfgnt{k}\rp \rp 
   }	
\and
   \inferrule[Method Invocation (object eval)]{}{
    \lp \cfgnt{L}\ \cfgnt{R}\ \phi\ \eta\ \lp\cfgnt{e}_0\ \cfgt{@}\ \cfgnt{m}\ \cfgnt{e}_1\rp\ \cfgnt{k}\rp  \rcom \\\\
    \lp \cfgnt{L}\ \cfgnt{R}\ \phi\ \eta\ \cfgnt{e}_0\ \lp \cfgt{*}\ \cfgt{@}\ \cfgnt{m}\ \cfgnt{e}_1\ \rightarrow \cfgnt{k}\rp \rp 
   }
\and
   \inferrule[Method Invocation (arg eval)]{}{
    \lp \cfgnt{L}\ \cfgnt{R}\ \phi\ \eta\ \cfgnt{v}_0\ \lp \cfgt{*}\ \cfgt{@}\ \cfgnt{m}\ \cfgnt{e}_1\ \rightarrow \cfgnt{k}\rp \rp  \rcom \\\\
    \lp \cfgnt{L}\ \cfgnt{R}\ \phi\ \eta\ \cfgnt{e}_1\ \lp \cfgnt{v}_0\ \cfgt{@}\ \cfgnt{m}\ \cfgt{*}\ \rightarrow \cfgnt{k}\rp \rp 
   }
\and
   \inferrule[Method Invocation]{
    \cfgnt{C} = \mathrm{type}\lp\cfgnt{r}\rp \\\\
    \lp\cfgnt{T}_o\ \cfgnt{m}\ \lb\cfgnt{T}_1\ \cfgnt{x}\rb\ \ \cfgnt{e}_m\rp = \mathrm{lookup}\lp \cfgnt{C},\cfgnt{m}\rp\\\\
    \eta_m = \eta[\cfgt{this} \mapsto \cfgnt{r}][\cfgnt{x} \mapsto \cfgnt{v}]}{
    \lp \cfgnt{L}\ \cfgnt{R}\ \phi\ \eta\ \cfgnt{v}\ \lp \cfgnt{r}\ \cfgt{@}\ \cfgnt{m}\ \cfgt{*}\ \rightarrow \cfgnt{k}\rp \rp  \rcom \\\\
    \lp \cfgnt{L}\ \cfgnt{R}\ \phi\ \eta_m\ \cfgnt{e}_m\ \lp \cfgt{pop}\ \eta\ \cfgnt{k}\rp \rp 
   }
\and
   \inferrule[Variable Assignment (eval)]{}{
    \lp \cfgnt{L}\ \cfgnt{R}\ \phi\ \eta\ \lp \cfgnt{x}\ \cfgt{:=}\ \cfgnt{e}\rp \ \cfgnt{k}\rp  \rcom \\\\
    \lp \cfgnt{L}\ \cfgnt{R}\ \phi\ \eta\ \cfgnt{e}\ \lp \cfgnt{x}\ \cfgt{:=}\ \cfgt{*} \rightarrow\ \cfgnt{k}\rp \rp 
   }	
\and
   \inferrule[Variable Assignment]{}{
    \lp \cfgnt{L}\ \cfgnt{R}\ \phi\ \eta\ \cfgnt{v}\ \lp \cfgnt{x}\ \cfgt{:=}\ \cfgt{*} \rightarrow\ \cfgnt{k}\rp \rp  \rcom \\\\
    \lp \cfgnt{L}\ \cfgnt{R}\ \phi\ \eta[\cfgnt{x} \mapsto \cfgnt{v}]\ \cfgnt{v}\ \cfgnt{k}\rp 
   }	
\and
   \inferrule[Begin (no args)]{}{
    \lp \cfgnt{L}\ \cfgnt{R}\ \phi\ \eta\ \lp \cfgt{begin}\rp \ \cfgnt{k}\rp  \rightarrow \\\\
    \lp \cfgnt{L}\ \cfgnt{R}\ \phi\ \eta\ \cfgnt{k}\rp 
   }
\and
   \inferrule[Begin (arg0 eval)]{}{
    \lp \cfgnt{L}\ \cfgnt{R}\ \phi\ \eta\ \lp \cfgt{begin}\ \cfgnt{e}_0\ \cfgnt{e}_1\ ...\rp \ \cfgnt{k}\rp  \rcom \\\\
    \lp \cfgnt{L}\ \cfgnt{R}\ \phi\ \eta\ \cfgnt{e}_0\ \lp \cfgt{begin}\ \cfgt{*}\ \lp\cfgnt{e}_1\ ...\rp \rightarrow \cfgnt{k}\rp \rp 
   }
\and
   \inferrule[Begin (argi eval)]{}{
    \lp \cfgnt{L}\ \cfgnt{R}\ \phi\ \eta\ \cfgnt{v}\ \lp \cfgt{begin}\ \cfgt{*}\ \lp\cfgnt{e}_i\ \cfgnt{e}_{i+1}\ ...\rp \rightarrow \cfgnt{k}\rp \rp  \rcom \\\\
    \lp \cfgnt{L}\ \cfgnt{R}\ \phi\ \eta\ \cfgnt{e}_i\ \lp \cfgt{begin}\ \cfgt{*}\ \lp\cfgnt{e}_{i+1}\ ...\rp \rightarrow \cfgnt{k}\rp \rp 
   }
\and
   \inferrule[Begin (argN eval)]{}{
    \lp \cfgnt{L}\ \cfgnt{R}\ \phi\ \eta\ \cfgnt{v}\ \lp \cfgt{begin}\ \cfgt{*}\ \lp\cfgnt{e}_{n}\rp \rightarrow \cfgnt{k}\rp \rp  \rcom \\\\
    \lp \cfgnt{L}\ \cfgnt{R}\ \phi\ \eta\ \cfgnt{e}_n\ \cfgnt{k}\rp 
   }
\and
	\inferrule[NULL]{}{
      \lp \cfgnt{L}\ \cfgnt{R}\ \phi\ \eta\ \cfgt{null}\ \cfgnt{k}\rp \rcom \\\\ 
      \lp \cfgnt{L}\ \cfgnt{R}\ \phi\ \eta\ \cfgnt{r}_\mathit{null}\ \cfgnt{k}\rp 
	}
\and
   \inferrule[Pop]{}{
    \lp \cfgnt{L}\ \cfgnt{R}\ \phi\ \eta\ \cfgnt{v}\ \lp \cfgt{pop}\ \eta_0\ \cfgnt{k}\rp \rp  \rcom \\\\
    \lp \cfgnt{L}\ \cfgnt{R}\ \phi\ \eta_0\  \cfgnt{v}\ \cfgnt{k}\rp 
   }
\end{mathpar}
\end{center}
\caption{Javalite rewrite rules, indicated by $\rcom$, that are common to generalized symbolic execution and precise heap summaries.}
\label{fig:javalite-common}
\end{figure*}


The rewrite rules defining the semantics update those functions with the following notation: 

The rewrite rules that define the Javalite semantics are in
\figref{fig:general}.

