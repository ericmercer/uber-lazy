\subsection{Reading and Writing References}
\newsavebox{\boxPFAFW}
\savebox{\boxPFAFW}{
%\begin{figure}[t]
%\begin{center}
\mprset{flushleft}
\begin{mathpar}
	\inferrule[Field Access]{
      \exists \lp \phi\ l\rp \in \cfgnt{L}\lp \cfgnt{r}\rp\ \lp l \neq l_{\mathit{null}} \wedge \mathbb{S}\lp \phi \wedge \phi_g\rp \rp \\\\
      \theta = \{ \phi \mid \lp \phi\ l_\mathit{null} \rp \wedge \mathbb{S}\lp \phi \wedge \phi_g\rp \} \\\\
      \phi_g^\prime = \phi_g \wedge (\wedge_{\phi \in \theta} \neg \phi) \\\\
      \{\cfgnt{C}\} = \{\cfgnt{C} \mid \exists \lp \phi\ l\rp  \in \cfgnt{L}\lp \cfgnt{r}\rp\ \lp\cfgnt{C} = \mathrm{type}\lp \cfgnt{l},\cfgnt{f}\rp\rp\} \\\\
      \lp \cfgnt{L}\ \cfgnt{R}\ \cfgnt{r}\ \cfgnt{f}\ \cfgnt{C}\rp \rsum^* \lp \cfgnt{L}^\prime\ \cfgnt{R}^\prime\ \cfgnt{r}\ \cfgnt{f}\ \cfgnt{C}\rp \\
      \cfgnt{r}^\prime = \mathrm{stack}_r\lp \rp
    }{
      \lp \cfgnt{L}\ \cfgnt{R}\ \phi_g\ \eta\ \cfgnt{r}\ \lp \cfgt{*}\ \cfgt{\$}\ \cfgnt{f} \rightarrow \cfgnt{k}\rp \rp  \rsym^\mathit{A}
      \lp \cfgnt{L}^\prime[\cfgnt{r}^\prime \mapsto \mathbb{VS}\lp \cfgnt{L}^\prime,\cfgnt{R}^\prime,\cfgnt{r},\cfgnt{f},\phi_g^\prime\rp ]\ \cfgnt{R}^\prime\ \phi_g^\prime\ \eta\ \cfgnt{r}^\prime\ \cfgnt{k}\rp 
	}
\and
	\inferrule[Field Write]{
      \cfgnt{r}_x = \eta\lp\cfgnt{x}\rp \\
      \exists \lp \phi\ l\rp \in \cfgnt{L}\lp \cfgnt{r}_x\rp\ \lp l \neq l_{\mathit{null}} \wedge \mathbb{S}\lp \phi \wedge \phi_g\rp \rp \\\\
      \theta = \{ \phi \mid \lp \phi\ l_\mathit{null} \rp \wedge \mathbb{S}\lp \phi \wedge \phi_g\rp \} \\\\
      \phi_g^\prime = \phi_g \wedge (\wedge_{\phi \in \theta} \neg \phi) \\\\
      \Psi_x =\{\lp \phi\ l\ \cfgnt{r}_\mathit{cur} \rp  \mid \lp \phi\ \cfgnt{l}\rp  \in \cfgnt{L}\lp \cfgnt{r}_x\rp  \wedge \cfgnt{r}_\mathit{cur} = \cfgnt{R}\lp l,\cfgnt{f}\rp  \}\\\\
      X = \{ \lp l\ \theta \rp  \mid \exists \phi\ \lp \lp \phi\ l\ \cfgnt{r}_\mathit{cur} \rp \in \Psi_x \wedge \theta = \mathbb{ST}\lp \cfgnt{L},\cfgnt{r},\phi,\phi_g^\prime\rp  \cup \mathbb{ST}\lp \cfgnt{L},\cfgnt{r}_\mathit{cur},\neg\phi,\phi_g^\prime\rp \rp  \}\\\\
      \cfgnt{R}^{\prime} = \cfgnt{R}[\forall \lp l\ \theta \rp  \in X\ \lp \lp l\ \cfgnt{f}\rp  \mapsto \mathrm{fresh}_r\lp \rp \rp ]\\\\
      \cfgnt{L}^{\prime} = \cfgnt{L}[\forall \lp l\ \theta \rp  \in X\ \lp \exists \cfgnt{r}_\mathit{targ}\ \lp \cfgnt{r}_\mathit{targ} = \cfgnt{R}^\prime\lp l,\cfgnt{f}\rp \wedge \lp\cfgnt{r}_\mathit{targ} \mapsto \theta\rp  \rp \rp ]
    }{
      \lp \cfgnt{L}\ \cfgnt{R}\ \phi_g\ \eta\ \cfgnt{r}\ \lp \cfgnt{x}\ \cfgt{\$}\ \cfgnt{f}\ \cfgt{:=}\ \cfgt{*}\ \rightarrow\ \cfgnt{k}\rp \rp  \rsym^\mathit{W}
      \lp \cfgnt{L}^{\prime}\ \cfgnt{R}^{\prime}\ \phi_g^\prime\ \eta\ \cfgnt{r}\ \cfgnt{k}\rp 
	}	
\end{mathpar}}
%\end{center}
%\caption{Precise symbolic heap summaries from symbolic execution indicated by $\rsym = \rightarrow_\mathit{FA} \cup \rightarrow_\mathit{FW} \cup \rightarrow_\mathit{EQ} \cup \rcom$.}
%\label{fig:symfield}
%\end{figure}



\begin{figure*}[t]
\begin{center}
\setlength{\tabcolsep}{60pt}
\hspace*{-35pt}
\begin{tabular}[c]{cc}
\scalebox{1.0}{\usebox{\boxPFAFW}} & 
%\scalebox{0.91}{\input{faYHeap.pdf_t}} &
\scalebox{0.91}{\input{fwXHeap.pdf_t}} \\ \\
(a) & (b)
\end{tabular}
\end{center}
\caption{Field read and write relations with an example heap. (a) Field-access, $\rsym^\mathit{A}$, and field-write, $\rsym^\mathit{W}$, rewrite rules for the $\rsym$ relation. (b) The final heap after $\lp\cfgt{this}\  \cfgt{\$}\ \cfgnt{x}\ \cfgt{:=}\ \lp\cfgt{this}\  \cfgt{\$}\ \cfgnt{y}\rp\rp$.}
\label{fig:fHeap}
\end{figure*}

There are two rewrite rules in~\figref{fig:fHeap}(a), one for reading
the value of a field (field-access) and the other when we write to a
field (field-write). Both rules first check that the operations can be
performed on a non-null location. If such is the case, then the global
constraint $\phi_g$ is updated to disallow any viable null locations
from this point forward: future accesses are non-null.
%perform is whether there exists a constraint location pair for the $r$
%being accessed such that the location is not null and the constraint
%when conjuncted with the global constraint is satisfiable. Next we
%extract all possible constraints under which $r$ points to a null
%location such that the constraint is satisfiable under the current
%global constraint, $\phi_g$. The negation of these constraints are
%added to the global constraint to create a new global constraint
%$\phi_g^\prime$.
%The update to the global constraint ensures that
%access or write of the field $f$ happens only on non-null
%locations. 
When there is a feasible null location in the summary heap on the
field access operations, the machine moves to an error state. The
rewrite rules for null locations are in the supplemental document.

In the field-access rewrite rule in~\figref{fig:fHeap}(a), after the
non-nullness check we extract the The type $C$ of the field is
extracted and invoke the summarize rewrite rule
in~\figref{fig:symInit} that performs the initialization of the field
if it points to uninitialized locations. Once the initialization is
complete, we create a new local reference $r^\prime$. An important
property of the references in the bi-partiate graph is that they are
\emph{immutable}. Hence we create a new local reference and assign it
to the initialized reference, and return the local reference. In order
to de-reference a field $r.f$ we define a helper function which is
called the value set.

\begin{definition}
\label{def:VS}
The $\mathbb{VS}$ function constructs the value set given a
heap, reference, and desired field where $\mathbb{S}(\phi)$ returns true if $\phi$ is satisfiable.
\[
\begin{array}{rcl}
  \mathbb{VS}(\cfgnt{L},\cfgnt{R},\phi_g,\cfgnt{r},\cfgnt{f}) & = &
  \{(\phi\wedge\phi^\prime\ \cfgnt{l}^\prime) \mid \exists
  \cfgnt{l}\ ((\phi\ l) \in L(r)\ \wedge \\ & &
  \ \ \ \ \ \ \ \ \exists \cfgnt{r}^\prime ( \cfgnt{r}^\prime =
  R(\cfgnt{l},\cfgnt{f})\ \wedge (\phi^\prime\ l^\prime) \in
  \cfgnt{L}(\cfgnt{r}^\prime)\ \wedge
  \mathbb{S}(\phi\wedge\phi^\prime\wedge \phi_g)))\}
\end{array}
\]
\end{definition}


In the post-condition of the rewrite rule we assign
the value set of input reference $r$ to the local reference $r^\prime$
and return the local reference $r^\prime$ in the next state.

Consider the graph shown in~\figref{fig:fHeap}(b) the reference
$\cfgnt{r}_2^i$ is created during the initialization of $\lp\cfgt{this}\  \cfgt{\$}\ \cfgnt{y}\rp$. The
reference that is returned during the access of the field, however, is
$\cfgnt{r}_3^s$. The reference $\cfgnt{r}_3^s$ points to the value set of $\cfgnt{r}_2^i$
which are: $(\phi_{2a}\ \cfgnt{l}_\mathit{null})$, $(\phi_{2b}\ \cfgnt{l}_2)$, and
$(\phi_{2c}\ \cfgnt{l}_1)$. The values of the constraints $\phi_{2a}$,
$\phi_{2b}$, and $\phi_{2c}$ are defined in~\figref{fig:initHeap}(d).

The field-write rewrite rule in~\figref{fig:fHeap}(a) after the
non-nullness check we We look up the value of the base reference in
the environment $\eta(x)$.  The reference $\cfgnt{r}_x$ is the base pointer
whose field, $\cfgnt{r}_\mathit{curr}$ is being written to to while the
reference $r$ is the target reference. The set $\Psi_x$ contains
tuples $(\phi\ l\ \cfgnt{r}_\mathit{curr})$ of constraints, locations, and
references. These tuples represent access chains leading from $\cfgnt{r}_x$ to
the reference of the field, $\cfgnt{r}_\mathit{curr}$. The goal is to
overwrite the $\cfgnt{r}_\mathit{curr}$ references with the target
references. Since the target of the write is $r$, we first check that
the location constraint pairs of $\cfgnt{L}(r)$ are satisfiable when accessed
through the $\cfgnt{r}_x$ chain. This is accomplished by the strengthening
function.

\begin{definition}
\label{def:ST}
The strengthen function $\mathbb{ST}(\cfgnt{L},\cfgnt{r},\phi,\phi_g)$ strengthens every
constraint from the reference $\cfgnt{r}$ with $\phi$ and keeps only location-constraint
pairs that are satisfiable after this strengthening with the inclusion of the global heap constraint $\phi_g$:
\[
\begin{array}{rcl} 
\mathbb{ST}(\cfgnt{L},\cfgnt{r},\phi,\phi_g) & = & \{ (\phi\wedge\phi^\prime\ \cfgnt{l}^\prime) \mid  \\
& & \ \ \ \ (\phi^\prime\ \cfgnt{l}^\prime)\in \cfgnt{L}(\cfgnt{r})\wedge\mathbb{S}(\phi\wedge\phi^\prime\wedge\phi_g) \}
\end{array}
\]
\end{definition}


Note that the write is conditional. In the case that $\phi$ is true
then $\cfgnt{r}_\mathit{curr}$ will point to the constraint location pairs of
$\cfgnt{L}(\cfgnt{r})$ while if $\phi$ is false then $\cfgnt{r}_\mathit{curr}$ will continue
to point to the constraint location pair it is currently pointing to.
Again since the references are immutable we create a new reference for
each $\cfgnt{r}_\mathit{curr}$ and point them to the target constraint
location pairs.

Consider the example shown in~\figref{fig:fHeap}(b) where we assign
$\lp\cfgt{this}\  \cfgt{\$}\ \cfgnt{x}\ \cfgt{:=}\ \lp\cfgt{this}\  \cfgt{\$}\ \cfgnt{y}\rp\rp$. Note that
in~\figref{fig:initHeap}(c) $\lp\cfgt{this}\  \cfgt{\$}\ \cfgnt{x}\rp$ is represented by
$\cfgnt{r}_1^i$. After applying the field-write rule the reference $\cfgnt{r}_1^i$ is
replaced by the fresh reference $\cfgnt{r}_4$ which now points to target
 $\lp\cfgt{this}\  \cfgt{\$}\ \cfgnt{y}\rp$ represented by $\cfgnt{L}(\cfgnt{r}_2^i)$.


\begin{comment}
\begin{figure}[t]
\begin{center}
\begin{tabular}[c]{l}
$\Psi_x = \{ (true\ \cfgnt{l}_0\ \cfgnt{r}_1^i) \}$\\
$ST (\cfgnt{L}, \cfgnt{r}_3^s, \phi, \phi_g)$ \\
$\theta = \{ (\phi_{2a}\; \cfgnt{l}_\mathit{null} ) (\phi_{2b}\; \cfgnt{l}_2) (\phi_{2c}\; \cfgnt{l}_1) \}$\\
$ST(\cfgnt{L}, \cfgnt{r}_0, \phi, \phi_g)$\\
$\theta = \{ \}$\\
\end{tabular}
\end{center}
\caption{FIXME: When will I get a caption}
\label{fig:faHeapSets}
\end{figure}
\end{comment}



\subsection{Equality and InEquality of References}
\begin{figure*}[t]
\begin{center}
\mprset{flushleft}
\begin{mathpar}
    \inferrule[Equals (references-true)]{
    \theta_\alpha = \{\lp\phi_0 \wedge \phi_1\rp \mid \exists l\ \lp \lp \phi_0\ l\rp  \in \cfgnt{L}\lp \cfgnt{r}_0\rp  \wedge \lp \phi_1\ l\rp  \in \cfgnt{L}\lp \cfgnt{r}_1\rp \rp \} \\\\
    \theta_0 = \{\phi_0 \mid \exists l_0\ \lp \lp \phi_0\ l_0\rp  \in \cfgnt{L}\lp \cfgnt{r}_0\rp  \wedge \forall \lp \phi_1\ l_1\rp  \in \cfgnt{L}\lp \cfgnt{r}_1\rp \ \lp l_0 \neq l_1\rp \rp \} \\\\
    \theta_1 = \{\phi_1 \mid \exists l_1\ \lp \lp \phi_1\ l_1\rp  \in \cfgnt{L}\lp \cfgnt{r}_1\rp  \wedge \forall \lp \phi_0\ l_0\rp  \in \cfgnt{L}\lp \cfgnt{r}_0\rp \ \lp l_0 \neq l_1\rp \rp \} \\\\
    \phi^\prime =  \phi \wedge \lp \vee_{\phi_\alpha\in\theta_\alpha}\phi_\alpha\rp \wedge\lp \wedge_{\phi_0 \in \theta_0} \neg \phi_0\rp \wedge\lp \wedge_{\phi_1
    \in \theta_1} \neg \phi_1\rp \\\\ 
    \mathbb{S}(\phi^\prime)}{
    \lp \cfgnt{L}\ \cfgnt{R}\ \phi\ \eta\ \cfgnt{r}_0\ \lp \cfgnt{r}_1\; \cfgt{=}\; \cfgt{*} \rightarrow \cfgnt{k}\rp \rp  \rightarrow_\mathit{EQ}
    \lp \cfgnt{L}\ \cfgnt{R}\ \phi^\prime\ \eta\ \cfgt{true}\ \cfgnt{k}\rp 
    }
\and
    \inferrule[Equals (references-false)]{
    \theta_\alpha = \{\lp\phi_0 \Rightarrow \neg \phi_1\rp \mid \exists l\ \lp \lp \phi_0\ l\rp  \in \cfgnt{L}\lp \cfgnt{r}_0\rp  \wedge \lp \phi_1\ l\rp  \in \cfgnt{L}\lp \cfgnt{r}_1\rp \rp \} \\\\
    \theta_0 = \{\phi_0 \mid \exists l_0\ \lp \lp \phi_0\ l_0\rp  \in \cfgnt{L}\lp \cfgnt{r}_0\rp  \wedge \forall \lp \phi_1\ l_1\rp  \in \cfgnt{L}\lp \cfgnt{r}_1\rp \ \lp l_0 \neq l_1\rp \rp \} \\\\
    \theta_1 = \{\phi_1 \mid \exists l_1\ \lp \lp \phi_1\ l_1\rp  \in \cfgnt{L}\lp \cfgnt{r}_1\rp  \wedge \forall \lp \phi_0\ l_0\rp  \in \cfgnt{L}\lp \cfgnt{r}_0\rp \ \lp l_0 \neq l_1\rp \rp \} \\\\
    \phi^\prime = \phi \wedge \lp \wedge_{\phi_\alpha\in\theta_\alpha}\phi_\alpha\rp \vee\lp \lp \vee_{\phi_0 \in \theta_0} \phi_0\rp   \vee\lp \vee_{\phi_1
    \in \theta_1} \phi_1\rp \rp  \\\\ 
    \mathbb{S}(\phi^\prime)}{
    \lp \cfgnt{L}\ \cfgnt{R}\ \phi\ \eta\ \cfgnt{r}_0\ \lp \cfgnt{r}_1\; \cfgt{=}\; \cfgt{*} \rightarrow \cfgnt{k}\rp \rp  \rightarrow_\mathit{EQ}
    \lp \cfgnt{L}\ \cfgnt{R}\ \phi^\prime\ \eta\ \cfgt{false}\ \cfgnt{k}\rp 
    }
\end{mathpar}
\end{center}
\caption{Precise symbolic heap summaries from symbolic execution indicated by $\rsym = \rightarrow_\mathit{FA} \cup \rightarrow_\mathit{FW} \cup \rightarrow_\mathit{EQ} \cup \rcom$.}
\label{fig:symeq}
\end{figure*}


\begin{figure*}
\begin{center}
\begin{tabular}[c]{c}
\scalebox{1.0}{\usebox{\boxPEQ}} \\
% & \usebox{\boxPEX} \\ \\
%(a) & (b) \\
\end{tabular}
\end{center}
\caption{The reference compare rewrite rule for both the true, $\rsym^\mathit{E}$, and false, $\rsym^\mathit{E^\prime}$ outcomes.}
\label{fig:eqs}
\end{figure*}



\newsavebox{\boxPEX}
\savebox{\boxPEX}{
\begin{tabular}[c]{l}
$\cfgnt{L}(\cfgnt{r}_1^i) = \{ (\phi_{1a}\; \cfgnt{l}_\mathit{null})\; (\phi_{1b}\; \cfgnt{l}_1) \}$ \\
$\cfgnt{L}(\cfgnt{r}_2^i) = \{ (\phi_{2a}\; \cfgnt{l}_\mathit{null}),\; (\phi_{2b}\; \cfgnt{l}_2),\; (\phi_{2c}\; \cfgnt{l}_1) \} $\\
  $\theta_0 = \{ \} $\\
$\theta_1 = \{ \phi_{2b}\} $\\ \hline
Equals true \\
$\theta_\alpha = \{ (\phi_{1a}\; \wedge\; \phi_{2a} ) (\phi_{1b}\; \wedge\; \phi_{2c} ) \}$\\
$\phi^\prime = \mathit{true} \wedge [ (\phi_{1a}\; \wedge\; \phi_{2a} )\vee (\phi_{1b}\; \wedge\; \phi_{2c} ) ] \wedge \neg\phi_{2b} $\\ \hline
Equals false \\
$\theta_\alpha = \{ (\phi_{1a}\; \implies\; \neg\phi_{2a} ) (\phi_{1b}\; \implies\; \neg\phi_{2c} ) \}$\\
$\phi^\prime = \mathit{true} \wedge  (\phi_{1a}\; \implies\; \neg\phi_{2a} )\wedge (\phi_{1b}\; \implies\; \neg\phi_{2c} )  \wedge \phi_{2b} $\\ \hline
\end{tabular}}


The rewrite rules to compare two references in the symbolic summary
heap are shown in~\figref{fig:eqs}. The equals references-true rewrite
rule returns true when two references $\cfgnt{r}_0$ and $\cfgnt{r}_1$ \emph{can} be
equal. In GSE semantics checking equality of references is a simple
comparision of two concrete object references. In the symbolic summary
heap, however, we compare sets of constraint location pairs pointed to
by each reference to determine if they could be equal. 

Consider the equals reference-true rewrite rule
in~\figref{fig:eqs}. In order to check equality we construct three
sets of constraints. For all constraint location pairs $(\phi_0\ \cfgnt{l}_0)
\in \cfgnt{L}(\cfgnt{r}_0)$ and $(\phi_1\ \cfgnt{l}_1) \in \cfgnt{L}(\cfgnt{r}_1)$ such that $\cfgnt{l}_0$ and $\cfgnt{l}_1$
are the same ($\cfgnt{l}_0 = \cfgnt{l}_1$) we create conjunctions of the constraints
$\phi_0 \wedge \phi_1$ and add them to set $\Phi_\alpha$. Intutively,
$\Phi_\alpha$ contains all constraints under which $\cfgnt{r}_0$ and $\cfgnt{r}_1$ may
point to the same location in the symbolic summary heap. The second
set, $\Phi_0$, contains constraints under which the reference $\cfgnt{r}_0$
points corresponding locations such that the reference $\cfgnt{r}_1$
\emph{does not} point to those locations under any
constraint. Finally, the set, $\Phi_1$, contains constraints under
which $\cfgnt{r}_1$ points to corresponding locations and $\cfgnt{r}_0$ \emph{does
  not} point to those locations under any constraint.

We use the three sets of constraints to update the current global heap
constraint $\phi_g$ and create a new global heap constraint
$\phi_g^\prime$. We add to $\phi_g$ the disjunction of the constraints
in $\Phi_\alpha$ to indicate that if any of the constraints are
satisfiable, then references $\cfgnt{r}_0$ and $\cfgnt{r}_1$ can be
equal. Furthermore, we had to the global heap constraint,
$\phi_g^\prime$, the conjunctions of negations to the constraints in
$\Phi_0$ and $\Phi_1$. This indicates for locations that are not
common to the references, the negations of their constraints are
satisfiable. Before the rewrite rule returns true we check the
satisfiability of the updated global heap constraint.

Consider the example in~\figref{fig:initHeap}(c). In order to to check
if $\cfgnt{r}_1^i$ and $\cfgnt{r}_2^i$ are equal we first get the constraint location
pairs associated with each of the references:
\[
\cfgnt{L}(\cfgnt{r}_1^i) = \{ (\phi_{1a}\; \cfgnt{l}_\mathit{null})\; (\phi_{1b}\; \cfgnt{l}_1) \} 
\]
\[
\cfgnt{L}(\cfgnt{r}_2^i) = \{ (\phi_{2a}\; \cfgnt{l}_\mathit{null})\; (\phi_{2b}\; \cfgnt{l}_2)\; (\phi_{2c}\; \cfgnt{l}_1) \} \\
\]
\noindent{The three constraint sets are:} 
\[
\Phi_\alpha = \{ (\phi_{1a}\; \wedge\; \phi_{2a} ) (\phi_{1b}\; \wedge\; \phi_{2c} ) \}\;
\Phi_0 = \{ \}\; \Phi_1 = \{ \phi_{2b}\} \\
\]
\noindent{Finally the global constraint is} 
\[
\phi^\prime = \mathit{true} \wedge [ (\phi_{1a}\; \wedge\; \phi_{2a} )\vee (\phi_{1b}\; \wedge\; \phi_{2c} ) ] \wedge \neg\phi_{2b} 
\]
\noindent{The equals references-false is the logical dual of the
  reference-true rewrite rule. The references-false is shown
  in~\figref{fig:eqs}.}
