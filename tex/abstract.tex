A fundamental challenge of using symbolic execution for software
analysis has been the treatment of dynamically allocated
data. Generalized symbolic execution (GSE) techniques have addressed
this challenge by materializing a concrete heap lazily during field
accesses.  These techniques however, exacerbate the path explosion
problem.  In this work, we present a heap summary approach which takes
inspiration from the static analysis domain and defines a bipartiate
graph to encode multiple heaps in single representation, which, unlike
state-of-the-art static analysis approaches, (i) supports aliasing,
(ii) does not require rewrite of constraints on heap updates, and
(iii) can initialize the heap-on-the-fly. We also define a heap
summary algorithm to perform heap updates during symbolic execution on
this representation in order to mitigate the path explosion problem in
GSE techniques. Our approach is sound and complete with respect to the
properties provable by GSE. A preliminary evaluation of our
implementation in Java Pathfinder framework shows the summary heap
enables the initialization of larger, more complex heaps than
previously possible.

\begin{comment}
A fundamental challenge of using symbolic execution for software
analysis has been the treatment of dynamically allocated data.
State-of-the-art techniques have addressed this challenge through
either (a) the use of summaries that over-approximate possible heaps, or
(b) by materializing a concrete heap lazily during generalized
symbolic execution. In this work, we present a novel heap
initialization and analysis technique which takes inspiration from
both approaches and constructs precise heap summaries lazily during
symbolic execution.  Our approach is 1) \emph{scalable}: it reduces the
points of non-determinism compared to generalized symbolic execution
and explores each control-flow path only once for any given set of
isomorphic heaps, 2) \emph{precise}: at any given point during symbolic
execution, the symbolic heap represents the exact set of feasible
concrete heap structures for the program under analysis, and 3)
\emph{expressive}: the symbolic heap can represent recursive data
structures. We demonstrate the precision and scalability of our
approach by implementing it as an extension to the Symbolic PathFinder
framework for analyzing Java programs.
\end{comment}

%and by using 
%materializing objects on the concrete heap ``as needed'' and
%using non-deterministic choice points to explore each feasible
%concrete heap configuration. 

%Because analysis of the materialized
%heap locations relies on concrete program semantics, the lazy
%initialization approach exacerbates the state space explosion
%problem that limits the scalability of symbolic execution.
%In this work we present a novel approach for lazy symbolic execution
%of heap manipulating software which utilizes 
%a fully symbolic heap constructed on-the-fly during symbolic execution.
%Our approach is 1) \emph{scalable} -- it does not
%create the additional points of non-determinism
%introduced by existing lazy initialization techniques and it explores
%each execution path only once for any given set of isomorphic heaps,
%2) \emph{precise} -- at any given point during symbolic execution, 
%the symbolic heap represents the exact set of feasible
%concrete heap structures for the program under analysis, and
%3) \emph{expressive} -- the symbolic heap can represent recursive data structures 
%and heaps resulting from loops and recursive control structures in the code. 
%We report on a case-study of an implementation of our technique in the
%Symbolic PathFinder tool to illustrate its scalability, precision
%and expressiveness. We also discuss how test case generation -- a common
 %use for symbolic execution results -- can benefit from symbolic
 %execution which uses a fully symbolic heap.
