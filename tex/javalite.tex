\section{Preliminaries}

\figref{fig:surface-syntax} defines the surface syntax for the
Javalite language \cite{saints-MS}. \figref{fig:machine-syntax} is the
machine syntax. Javalite is syntactic machine defined as rewrites on a
string. The semantics use a CEKS model with a (C)ontrol string
representing the expression being evaluated, an (E)nvironment for
local variables, a (K)ontinuation for what is to be executed next, and
a (S)tore for the heap.

\begin{figure}
\begin{center}
\cfgstart
\cfgrule{P}{\lp $\mu$ \lp \cfgnt{C} \cfgnt{m}\rp\rp}
\cfgrule{$\mu$}{(\cfgnt{CL} ...)}
\cfgrule{T}{\cfgt{bool} \cfgor \cfgnt{C}}
\cfgrule{CL}{\lp\cfgt{class} \cfgnt{C} \lp\lb\cfgnt{T} \cfgnt{f}\rb ...\rp \lp\cfgnt{M} ...\rp}
\cfgrule{M}{\lp\cfgnt{T} \cfgnt{m} \lb\cfgnt{T} \cfgnt{x}\rb\  e\rp}
\cfgrule{e}{\cfgnt{x}
\cfgor{\lp\cfgt{new} \cfgnt{C}\rp}
\cfgor{\lp\cfgnt{e} \cfgt{\$} \cfgnt{f}\rp}
\cfgor{\lp\cfgnt{x} \cfgt{\$} \cfgnt{f} \cfgt{:=} \cfgnt{e}\rp}
\cfgor{\lp\cfgnt{e} \cfgt{=} \cfgnt{e}\rp}}
\cfgorline{\lp\cfgt{if} \cfgnt{e} \cfgnt{e} \cfgt{else} \cfgnt{e}\rp 
\cfgor {\lp\cfgt{var} \cfgnt{T} \cfgnt{x} \cfgt{:=} \cfgnt{e} \cfgt{in} \cfgnt{e}\rp}
\cfgor {\lp\cfgnt{e} \cfgt{@} \cfgnt{m} \cfgnt{e} \rp}}
\cfgorline{\lp\cfgnt{x} \cfgt{:=} \cfgnt{e}\rp
\cfgor{\lp\cfgt{begin} \cfgnt{e} ...\rp}
\cfgor{\cfgnt{v}}}
\cfgrule{x}{\cfgt{this} \cfgor \cfgnt{id}}
\cfgrule{f,m,C}{\cfgnt{id}}
%\cfgrule{m}{\cfgnt{id}}
%\cfgrule{C}{\cfgnt{id}}
\cfgrule{v}{\cfgnt{r} \cfgor \cfgt{null} \cfgor \cfgt{true} \cfgor \cfgt{false} \cfgor \cfgt{error}}
\cfgrule{r}{\cfgt{number}}
\cfgrule{id}{\cfgt{variable-not-otherwise-mentioned}}
\cfgend
\end{center}
\caption{The Javalite surface syntax.}
\label{fig:surface-syntax}
\end{figure}


\subsection{Environment}
The environment, $\eta$, associates a variable $\cfgnt{x}$ with a
value $\cfgnt{v}$. The value can be a reference, $\cfgnt{r}$ or one of
the special values $\cfgt{null}$, $\cfgt{true}$, or
$\cfgt{false}$. Although the Javalite machine is purely syntactic, for
clarity and brevity in the presentation, the more complex structures
such as the environment are treated as partial functions. As such,
$\eta(\cfgnt{x}) = \cfgnt{r}$ is the reference mapped to the variable
in the environment. The notation $\eta^\prime = \eta[\cfgnt{x} \mapsto
  \cfgnt{v}]$ defines a new partial function $\eta^\prime$ that is
just like $\eta$ only the variable $\cfgnt{x}$ now maps to
$\cfgnt{v}$.



